% Edit window test corpus - 4KB LaTeX for SIMD testing
% This file tests SIMD performance with typical LaTeX constructs

\documentclass{article}
\usepackage{amsmath,amsthm,amssymb}
\usepackage{graphicx}
\usepackage{hyperref}

\title{SIMD Lexer Performance Test Document}
\author{LaTeX Perfectionist v25}
\date{\today}

\begin{document}

\maketitle

\section{Introduction}

This document contains typical \LaTeX{} constructs to test SIMD lexer performance. 
We include various token types: \textbf{commands}, \textit{arguments}, $mathematics$, 
and \verb|verbatim| text.

\subsection{Mathematical Content}

The quadratic formula is:
\begin{equation}
    x = \frac{-b \pm \sqrt{b^2 - 4ac}}{2a}
\end{equation}

Some inline math: $E = mc^2$ and $\sum_{i=1}^{n} i = \frac{n(n+1)}{2}$.

Here are more complex expressions:
\begin{align}
    \nabla \cdot \mathbf{E} &= \frac{\rho}{\epsilon_0} \\
    \nabla \cdot \mathbf{B} &= 0 \\
    \nabla \times \mathbf{E} &= -\frac{\partial \mathbf{B}}{\partial t} \\
    \nabla \times \mathbf{B} &= \mu_0 \mathbf{J} + \mu_0 \epsilon_0 \frac{\partial \mathbf{E}}{\partial t}
\end{align}

\section{Code and Verbatim}

Here is some code:
\begin{verbatim}
def fibonacci(n):
    if n <= 1:
        return n
    return fibonacci(n-1) + fibonacci(n-2)
\end{verbatim}

And inline code: \verb|grep -E "\\\\[a-zA-Z]+\{" file.tex|

\section{Lists and Environments}

\begin{itemize}
    \item First item with \emph{emphasis}
    \item Second item with \textbf{bold text}
    \item Third item with $math = \alpha + \beta$
    \begin{enumerate}
        \item Nested numbered list
        \item Another nested item
        \item With more \textsc{small caps} text
    \end{enumerate}
\end{itemize}

\section{References and Citations}

See Section~\ref{sec:conclusion} for conclusions. Also refer to~\cite{knuth1986texbook} 
for more information about \TeX{} and \LaTeX{}.

\begin{theorem}
    For any positive integer $n$, we have:
    \[ \sum_{k=1}^n k^2 = \frac{n(n+1)(2n+1)}{6} \]
\end{theorem}

\begin{proof}
    We proceed by mathematical induction. The base case $n=1$ gives $1^2 = 1$ and 
    $\frac{1 \cdot 2 \cdot 3}{6} = 1$, so the formula holds.
    
    Assume the formula holds for some $n = m$. Then for $n = m + 1$:
    \begin{align}
        \sum_{k=1}^{m+1} k^2 &= \sum_{k=1}^m k^2 + (m+1)^2 \\
        &= \frac{m(m+1)(2m+1)}{6} + (m+1)^2 \\
        &= \frac{(m+1)[m(2m+1) + 6(m+1)]}{6} \\
        &= \frac{(m+1)(2m^2 + 7m + 6)}{6} \\
        &= \frac{(m+1)(m+2)(2m+3)}{6}
    \end{align}
    which is the desired formula for $n = m + 1$.
\end{proof}

\section{Tables and Figures}

\begin{table}[h]
\centering
\begin{tabular}{|c|c|c|}
\hline
Algorithm & Time Complexity & Space Complexity \\
\hline
Bubble Sort & $O(n^2)$ & $O(1)$ \\
Merge Sort & $O(n \log n)$ & $O(n)$ \\
Quick Sort & $O(n \log n)$ & $O(\log n)$ \\
\hline
\end{tabular}
\caption{Comparison of sorting algorithms}
\label{tab:sorting}
\end{table}

\section{Advanced Constructs}

Some advanced LaTeX constructs for comprehensive testing:

\newcommand{\customcmd}[2]{\textbf{#1}: \textit{#2}}
\customcmd{Definition}{A precise statement of meaning}

\def\mydef#1#2{\texttt{#1} $\rightarrow$ #2}
\mydef{input}{output}

\section{Unicode and Special Characters}

Testing Unicode support: α, β, γ, δ, ε, ζ, η, θ, ι, κ, λ, μ, ν, ξ, ο, π, ρ, σ, τ, υ, φ, χ, ψ, ω

And some special characters: © ® ™ § ¶ † ‡ • … ‰ ′ ″ ‴ ⁗ ⁰ ¹ ² ³ ⁴ ⁵ ⁶ ⁷ ⁸ ⁹

\section{Conclusion} \label{sec:conclusion}

This document contains approximately 4KB of typical \LaTeX{} content designed to test 
SIMD lexer performance with realistic token distributions.

\end{document}