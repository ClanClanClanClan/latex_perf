% LaTeX Perfectionist v25 - Edit Window Performance Corpus
% edit_window_4kb.tex - 4KB edit-window simulation
% Target: ~4096 bytes for edit-window performance testing (≤1ms P95)

\documentclass[11pt]{article}
\usepackage{amsmath,amsfonts,amssymb}
\usepackage{mathtools}

\begin{document}

% Typical editing scenario: mixed content with formulas, text, and commands
\section{Performance Analysis}

Consider the computational complexity of lexical analysis over LaTeX documents. For a document of length $n$ with $k$ macro expansions, the time complexity is bounded by:

\begin{equation}
T(n,k) = O(n \log k) + O(k \cdot m_{\text{avg}})
\end{equation}

where $m_{\text{avg}}$ represents the average macro expansion length. In practice, this reduces to linear time $O(n)$ for most documents since $k \ll n$ and macro expansions are typically constant-sized.

\subsection{Tokenization Patterns}

The lexer processes several token categories:
\begin{itemize}
    \item \textbf{Control sequences}: Commands like \texttt{\textbackslash alpha}, \texttt{\textbackslash sum}, \texttt{\textbackslash int}
    \item \textbf{Mathematics}: Inline $x^2 + y^2 = r^2$ and display formulas
    \item \textbf{Text content}: Regular paragraph text with various punctuation
    \item \textbf{Special characters}: \&, \%, \$, \#, \_, \{, \}, \textbackslash
\end{itemize}

\begin{align}
\nabla \cdot \mathbf{E} &= \frac{\rho}{\epsilon_0} \\
\nabla \cdot \mathbf{B} &= 0 \\
\nabla \times \mathbf{E} &= -\frac{\partial \mathbf{B}}{\partial t} \\
\nabla \times \mathbf{B} &= \mu_0 \left( \mathbf{J} + \epsilon_0 \frac{\partial \mathbf{E}}{\partial t} \right)
\end{align}

The performance characteristics of arena-based allocation show significant improvement over traditional malloc/free patterns. With pre-allocated buffers of size $A = 3n$ bytes, we eliminate per-token allocation overhead:

\begin{equation}
\text{Speedup} = \frac{T_{\text{malloc}}}{T_{\text{arena}}} = \frac{n \cdot t_{\text{alloc}} + n \cdot t_{\text{proc}}}{n \cdot t_{\text{proc}}} = 1 + \frac{t_{\text{alloc}}}{t_{\text{proc}}}
\end{equation}

For typical values where $t_{\text{alloc}} \approx 15\text{ns}$ and $t_{\text{proc}} \approx 5\text{ns}$, this yields a theoretical speedup of $4\times$. Empirical measurements confirm speedups between $2.5\times$ and $5.75\times$ depending on document characteristics.

\subsection{Memory Layout Optimization}

Token packing using 32-bit integers provides optimal cache utilization:

\begin{verbatim}
struct token_packed {
    uint32_t data : 26;  // Position or content ref
    uint32_t catcode : 6; // Category code (0-63)
}
\end{verbatim}

This representation fits exactly in L1 cache lines (64 bytes = 16 tokens), enabling efficient vectorized processing. The catcode space maps to standard TeX categories:

\begin{center}
\begin{tabular}{cl}
\hline
Code & Category \\
\hline
0 & Escape character (\textbackslash) \\
1 & Begin group (\{) \\
2 & End group (\}) \\
3 & Math shift (\$) \\
4 & Alignment tab (\&) \\
11 & Letter (a-z, A-Z) \\
12 & Other character \\
\hline
\end{tabular}
\end{center}

Modern SIMD instructions (AVX2, NEON) can process multiple tokens simultaneously, achieving throughput rates exceeding $20$ GB/s on contemporary hardware. The performance bottleneck shifts from tokenization to higher-level parsing operations.

\subsection{Benchmarking Methodology}

Performance measurements use hardware timestamp counters (RDTSC) for sub-nanosecond precision. Test runs execute 100 iterations with statistical analysis:

\begin{equation}
\bar{t} = \frac{1}{n} \sum_{i=1}^{n} t_i, \quad \sigma = \sqrt{\frac{1}{n-1} \sum_{i=1}^{n} (t_i - \bar{t})^2}
\end{equation}

The P95 latency represents the 95th percentile: $P_{95} = \min\{t : F(t) \geq 0.95\}$ where $F(t)$ is the cumulative distribution function of measured latencies. This metric captures tail behavior critical for interactive editing responsiveness.

Flambda2 compiler optimizations provide consistent $2\times$ performance improvement over standard OCaml compilation, primarily through aggressive inlining and closure optimization.

\end{document}