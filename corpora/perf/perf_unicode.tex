% Unicode-heavy LaTeX document for L0 lexer performance testing
% Target: ~300KB with mixed Unicode content per L0_LEXER_SPEC.md

\documentclass[12pt]{article}
\usepackage[utf8]{inputenc}
\usepackage[T1]{fontenc}
\usepackage{amsmath}
\usepackage{amssymb}
\usepackage{textgreek}

\begin{document}

\title{Unicode Mathematical Symbols and International Text}
\author{Performance Test Generator}
\date{\today}
\maketitle

\section{Greek Letters and Mathematical Symbols}

The Greek alphabet in mathematics: Α α (alpha), Β β (beta), Γ γ (gamma), Δ δ (delta), Ε ε (epsilon), Ζ ζ (zeta), Η η (eta), Θ θ (theta), Ι ι (iota), Κ κ (kappa), Λ λ (lambda), Μ μ (mu), Ν ν (nu), Ξ ξ (xi), Ο ο (omicron), Π π (pi), Ρ ρ (rho), Σ σ ς (sigma), Τ τ (tau), Υ υ (upsilon), Φ φ (phi), Χ χ (chi), Ψ ψ (psi), Ω ω (omega).

Mathematical operators: ± ∓ × ÷ ∞ ∂ ∇ ∆ ∑ ∏ ∫ ∮ √ ∛ ∜ ∝ ∀ ∃ ∄ ∈ ∉ ∋ ∌ ⊂ ⊃ ⊆ ⊇ ∪ ∩ ∧ ∨ ¬ ⇒ ⇔ ≡ ≠ ≤ ≥ ≪ ≫ ≈ ≅ ≃ ∼ ∝ ⊥ ∥ ∠ ∡ ∢ ⊕ ⊗ ⊙ ⌈ ⌉ ⌊ ⌋ ⟨ ⟩ ‖ ∅ ℵ ℶ ℷ ℸ

Arrows: ← → ↑ ↓ ↔ ↕ ↖ ↗ ↘ ↙ ↚ ↛ ↜ ↝ ↞ ↟ ↠ ↡ ↢ ↣ ↤ ↥ ↦ ↧ ↨ ↩ ↪ ↫ ↬ ↭ ↮ ↯ ↰ ↱ ↲ ↳ ↴ ↵ ↶ ↷ ↸ ↹ ↺ ↻ ⇀ ⇁ ⇂ ⇃ ⇄ ⇅ ⇆ ⇇ ⇈ ⇉ ⇊ ⇋ ⇌ ⇍ ⇎ ⇏ ⇐ ⇑ ⇒ ⇓ ⇔ ⇕ ⇖ ⇗ ⇘ ⇙ ⇚ ⇛ ⇜ ⇝ ⇞ ⇟ ⇠ ⇡ ⇢ ⇣ ⇤ ⇥ ⇦ ⇧ ⇨ ⇩ ⇪

\section{International Languages}

\subsection{European Languages}

French: La théorie des ensembles étudie les propriétés des collections d'objets mathématiques. Les ensembles peuvent être finis ou infinis, dénombrables ou non dénombrables. L'axiomatique de Zermelo-Fraenkel avec l'axiome du choix (ZFC) constitue la base moderne de cette théorie.

German: Die Mengenlehre untersucht die Eigenschaften von Kollektionen mathematischer Objekte. Mengen können endlich oder unendlich, abzählbar oder überabzählbar sein. Das Zermelo-Fraenkel-Axiomensystem mit dem Auswahlaxiom (ZFC) bildet die moderne Grundlage dieser Theorie.

Spanish: La teoría de conjuntos estudia las propiedades de las colecciones de objetos matemáticos. Los conjuntos pueden ser finitos o infinitos, numerables o no numerables. La axiomática de Zermelo-Fraenkel con el axioma de elección (ZFC) constituye la base moderna de esta teoría.

Italian: La teoria degli insiemi studia le proprietà delle collezioni di oggetti matematici. Gli insiemi possono essere finiti o infiniti, numerabili o non numerabili. L'assiomatica di Zermelo-Fraenkel con l'assioma della scelta (ZFC) costituisce la base moderna di questa teoria.

Portuguese: A teoria dos conjuntos estuda as propriedades das coleções de objetos matemáticos. Os conjuntos podem ser finitos ou infinitos, enumeráveis ou não enumeráveis. A axiomática de Zermelo-Fraenkel com o axioma da escolha (ZFC) constitui a base moderna desta teoria.

Dutch: De verzamelingenleer bestudeert de eigenschappen van collecties van wiskundige objecten. Verzamelingen kunnen eindig of oneindig, aftelbaar of overaftelbaar zijn. Het Zermelo-Fraenkel axiomenstelsel met het keuzeaxioma (ZFC) vormt de moderne basis van deze theorie.

\subsection{Nordic Languages}

Swedish: Mängdläran studerar egenskaperna hos kollektioner av matematiska objekt. Mängder kan vara ändliga eller oändliga, räknebara eller icke-räknebara. Zermelo-Fraenkels axiomatik med urvalsaxiomet (ZFC) utgör den moderna grunden för denna teori.

Norwegian: Mengdelære studerer egenskapene til kolleksjoner av matematiske objekter. Mengder kan være endelige eller uendelige, tellbare eller ikke-tellbare. Zermelo-Fraenkels aksiomatikk med utvalgsaksiomet (ZFC) utgjør det moderne grunnlaget for denne teorien.

Danish: Mængdelære studerer egenskaberne ved kollektioner af matematiske objekter. Mængder kan være endelige eller uendelige, tællelige eller ikke-tællelige. Zermelo-Fraenkels aksiomatik med valgaksiomet (ZFC) udgør det moderne grundlag for denne teori.

\subsection{Slavic Languages}

Russian: Теория множеств изучает свойства коллекций математических объектов. Множества могут быть конечными или бесконечными, счётными или несчётными. Аксиоматика Цермело-Френкеля с аксиомой выбора (ZFC) составляет современную основу этой теории.

Polish: Teoria mnogości bada właściwości kolekcji obiektów matematycznych. Zbiory mogą być skończone lub nieskończone, przeliczalne lub nieprzeliczalne. Aksjomatyka Zermelo-Fraenkela z aksjomatem wyboru (ZFC) stanowi współczesną podstawę tej teorii.

Czech: Teorie množin studuje vlastnosti kolekcí matematických objektů. Množiny mohou být konečné nebo nekonečné, spočetné nebo nespočetné. Zermelova-Fraenkelova axiomatika s axiomem výběru (ZFC) tvoří moderní základ této teorie.

\section{Mathematical Formulas with Unicode}

Set operations: A ∪ B = {x : x ∈ A ∨ x ∈ B}, A ∩ B = {x : x ∈ A ∧ x ∈ B}, A \ B = {x : x ∈ A ∧ x ∉ B}

Logic: ∀x ∈ ℝ, ∃y ∈ ℝ such that x < y. For any ε > 0, ∃δ > 0 such that |x - a| < δ ⟹ |f(x) - f(a)| < ε.

Number systems: ℕ ⊂ ℤ ⊂ ℚ ⊂ ℝ ⊂ ℂ, where ℕ = {0, 1, 2, 3, ...}, ℤ = {..., -2, -1, 0, 1, 2, ...}, ℚ = {p/q : p, q ∈ ℤ, q ≠ 0}.

Complex analysis: For z = x + iy ∈ ℂ, we have |z| = √(x² + y²), arg(z) = arctan(y/x), z* = x - iy.

Integrals: ∫₋∞^∞ e^(-x²) dx = √π, ∮_γ f(z) dz = 2πi ∑ Res(f, zₖ), ∫∫∫_V f(x,y,z) dx dy dz

Series: ∑_{n=1}^∞ 1/n² = π²/6, ∏_{p prime} (1 - p⁻ˢ)⁻¹ = ζ(s) for Re(s) > 1

\section{Special Characters and Symbols}

Currency: $ ¢ £ ¤ ¥ € ₹ ₽ ₿ ₡ ₢ ₣ ₤ ₥ ₦ ₧ ₨ ₩ ₪ ₫ € ₭ ₮ ₯ ₰ ₱ ₲ ₳ ₴ ₵ ₶ ₷ ₸ ₹ ₺ ₻ ₼ ₽ ₾ ₿

Punctuation: ‹ › « » " " ' ' ‚ „ ‟ … – — ‐ ‑ ‒ – ‖ ‗ ' ' ‚ ‛ " " „ ‟ † ‡ • ‰ ‱ ′ ″ ‴ ‵ ‶ ‷ ‸ ‹ › ‼ ‽ ‾ ‿

Diacritics: À Á Â Ã Ä Å Æ Ç È É Ê Ë Ì Í Î Ï Ð Ñ Ò Ó Ô Õ Ö Ø Ù Ú Û Ü Ý Þ ß à á â ã ä å æ ç è é ê ë ì í î ï ð ñ ò ó ô õ ö ø ù ú û ü ý þ ÿ

\section{Asian Scripts Sample}

Chinese: 数学是研究数量、结构、变化以及空间等概念的一门学科。从历史上看,数学的发展源于计算、测量和对物体形状及运动的描述。数学的基本要素是逻辑和直觉、分析和构造、一般性和个别性。

Japanese: 数学(すうがく)は、数量、構造、変化、空間を研究する学問である。数学は、算術、代数学、幾何学、解析学などの分野に分けられる。これらの分野は相互に関連し合っている。

Korean: 수학은 수, 양, 구조, 공간, 변화 등의 개념을 다루는 학문이다. 수학은 논리와 추상화를 바탕으로 한 체계적인 학문이며, 다른 과학 분야의 기초가 된다.

\section{More Mathematical Content}

Topology: A topological space (X, τ) consists of a set X and a topology τ ⊆ P(X) such that:
• ∅ ∈ τ and X ∈ τ
• ⋃_{i∈I} U_i ∈ τ for any family {U_i}_{i∈I} ⊆ τ
• ⋂_{i=1}^n U_i ∈ τ for any finite subfamily {U_i}_{i=1}^n ⊆ τ

Measure Theory: Let (Ω, ℱ, μ) be a measure space. For any measurable function f: Ω → ℝ̄, the integral is defined as:
∫_Ω f dμ = sup{∫_Ω s dμ : s simple, s ≤ f}

Category Theory: A category 𝒞 consists of:
• A class Ob(𝒞) of objects
• For each pair A, B ∈ Ob(𝒞), a set Hom_𝒞(A, B) of morphisms
• Composition ∘: Hom_𝒞(B, C) × Hom_𝒞(A, B) → Hom_𝒞(A, C)
• Identity morphisms id_A ∈ Hom_𝒞(A, A)

Satisfying associativity and identity laws.

\section{Physics Symbols}

Quantum mechanics: ℏ = h/(2π), where h is Planck's constant. The Schrödinger equation: iℏ ∂ψ/∂t = Ĥψ

Thermodynamics: ΔS ≥ 0 for isolated systems, PV = nRT for ideal gases, dU = δQ - δW

Electromagnetism: ∇ · E = ρ/ε₀, ∇ × B = μ₀J + μ₀ε₀ ∂E/∂t, F = qE + q(v × B)

Relativity: E = mc², ds² = -c²dt² + dx² + dy² + dz², Gμν = 8πG/c⁴ Tμν

\section{More Unicode Examples}

Fractions: ½ ⅓ ¼ ⅕ ⅙ ⅐ ⅛ ⅑ ⅒ ⅔ ¾ ⅖ ⅗ ⅘ ⅚ ⅝ ⅞ ⅟

Superscripts: ⁰ ¹ ² ³ ⁴ ⁵ ⁶ ⁷ ⁸ ⁹ ⁺ ⁻ ⁼ ⁽ ⁾ ⁿ ⁱ

Subscripts: ₀ ₁ ₂ ₃ ₄ ₅ ₆ ₇ ₈ ₉ ₊ ₋ ₌ ₍ ₎ ₐ ₑ ₒ ₓ ₕ ₖ ₗ ₘ ₙ ₚ ₛ ₜ

Geometric shapes: ○ ◯ ● ◦ ◉ ◎ ◍ □ ■ ▢ ▣ ▤ ▥ ▦ ▧ ▨ ▩ ◇ ◆ ◈ ◊ ◉ △ ▲ ▴ ▵ ▶ ▷ ▸ ▹ ► ▻ ▼ ▽ ▾ ▿ ◀ ◁ ◂ ◃ ◄ ◅ ♠ ♣ ♥ ♦ ♤ ♧ ♡ ♢

Musical symbols: ♩ ♪ ♫ ♬ ♭ ♮ ♯ 𝄞 𝄢 𝄡 𝄟 𝄠 𝄪 𝄫

Chess symbols: ♔ ♕ ♖ ♗ ♘ ♙ ♚ ♛ ♜ ♝ ♞ ♟

Zodiac: ♈ ♉ ♊ ♋ ♌ ♍ ♎ ♏ ♐ ♑ ♒ ♓

Weather: ☀ ☁ ☂ ☃ ☄ ★ ☆ ☇ ☈ ☉ ☊ ☋ ☌ ☍ ☎ ☏ ☐ ☑ ☒ ☓ ☔ ☕ ☖ ☗ ☘ ☙ ☚ ☛ ☜ ☝ ☞ ☟ ☠ ☡ ☢ ☣ ☤ ☥ ☦ ☧ ☨ ☩ ☪ ☫ ☬ ☭ ☮ ☯ ☰ ☱ ☲ ☳ ☴ ☵ ☶ ☷

\section{Extended Mathematical Examples}

Complex integrals around various contours with residue calculations:

∮_{|z|=1} z^n dz = {2πi if n = -1; 0 otherwise}

∮_{|z-a|=r} f(z)/(z-a)^{n+1} dz = 2πi/n! · f^{(n)}(a)

Laurent series expansions:
f(z) = ∑_{n=-∞}^∞ a_n (z-z₀)^n

Power series with radius of convergence:
R = 1/limsup_{n→∞} |a_n|^{1/n}

Fourier transforms:
F(ω) = ∫_{-∞}^∞ f(t) e^{-iωt} dt
f(t) = 1/(2π) ∫_{-∞}^∞ F(ω) e^{iωt} dω

Laplace transforms:
L[f(t)] = ∫_0^∞ f(t) e^{-st} dt

Special functions:
Γ(z) = ∫_0^∞ t^{z-1} e^{-t} dt
B(x,y) = ∫_0^1 t^{x-1} (1-t)^{y-1} dt = Γ(x)Γ(y)/Γ(x+y)
ζ(s) = ∑_{n=1}^∞ 1/n^s = ∏_p (1-p^{-s})^{-1}

\end{document}