%http://www.editorialmanager.com/bmms/Default.aspx
\documentclass[10pt]{article}
\usepackage{cite}
\usepackage{mathrsfs}
\usepackage{amsfonts}
\usepackage{amsmath}
\usepackage{amsfonts,amssymb}
\usepackage{dsfont}
\usepackage{curves}
\usepackage{mathrsfs}
\usepackage{pifont}
\usepackage{amssymb}
\allowdisplaybreaks
\newcommand{\bas}{\backslash}
\numberwithin{equation}{section}
\newtheorem{theorem}{Theorem}[section]
\newtheorem{proposition}{Proposition}[section]
\newtheorem{definition}{Definition}[section]
\newtheorem{lemma}{Lemma}[section]
\newtheorem{corollary}{Corollary}[section]
\newtheorem{remark}{Remark}[section]
\newtheorem{problem}{Problem}[section]
\newtheorem{conjecture}{Conjecture}[section]
\newtheorem{example}{Example}[section]
\newtheorem{observation}{Observation}[section]
\newcommand{\oversign}{\hfill\rule{0.5em}{0.809em}}
\newcommand{\qed}{\hfill\rule{0.5em}{0.809em}}
\date{}
\newcommand{\ppmod}[1]{\hspace{-3mm}\pmod{#1}}
%\def\emptyset{\mbox{{\rm \O}}}
\def\bar{\overline}
\newenvironment{proof}{
%\parskip
\noindent {\hspace*{0.7cm}\bf Proof.}\rm}
{\mbox{}\hfill\rule{0.5em}{0.809em}\par}
\renewcommand{\baselinestretch}{1.2}
\textwidth=15.5cm
%\textheight=22cm
\textheight=22cm
%\input{psfig}
%\font\tencyr=wncyr10

\def\BigRoman{\uppercase\expandafter{\romannumeral\number\count 255 }}
\def\Romannumeral{\afterassignment\BigRoman\count255=}
\def\qed{\hfill \rule{4pt}{7pt}}

\setlength{\topmargin}{-0.2in} \setlength{\oddsidemargin}{0.1in}
\begin{document}
\title{Spanning subgraphs and spectral radius in graphs
%\thanks{}
}
\author{\small  Sizhong Zhou\footnote{Corresponding
author. E-mail address: zsz\_cumt@163.com (S. Zhou)}\\
\small School of Science, Jiangsu University of Science and Technology,\\
\small Zhenjiang, Jiangsu 212100, China\\
}


\maketitle
\begin{abstract}
\noindent A spanning tree $T$ of a connected graph $G$ is a subgraph of $G$ that is a tree covers all vertices of $G$. The leaf distance of $T$ is
defined as the minimum of distances between any two leaves of $T$. A fractional matching of a graph $G$ is a function $h$ assigning every edge a
real number in $[0,1]$ so that $\sum\limits_{e\in E_G(v)}{h(e)}\leq1$ for any $v\in V(G)$, where $E_G(v)$ denotes the set of edges incident with $v$
in $G$. A fractional matching of $G$ is called a fractional perfect matching if $\sum\limits_{e\in E_G(v)}{h(e)}=1$ for any $v\in V(G)$. A graph $G$
with at least $2k+2$ vertices is said to be fractional $k$-extendable if every $k$-matching $M$ in $G$ is included in a fractional perfect matching
$h$ of $G$ such that $h(e)=1$ for any $e\in M$. This paper considers a lower bound on the spectral radius of $G$ to guarantee that $G$ has a spanning
tree with leaf distance at least $d$. At the same time, we obtain a lower bound on the spectral radius of $G$ to ensure that $G$ is fractional
$k$-extendable.
\\
\begin{flushleft}
{\em Keywords:} graph; spectral radius; spanning tree; fractional perfect matching; fractional $k$-extendable graph.

(2020) Mathematics Subject Classification: 05C50, 05C05, 05C70
\end{flushleft}
\end{abstract}

\section{Introduction}

In this paper, all graphs considered are assumed to simple and undirected. Let $G=(V(G),E(G))$ denote a graph with vertex set $V(G)$ and edge
set $E(G)$. The order and size of $G$ is denoted by $n$ and $e(G)$, respectively. That is, $n=|V(G)|$ and $e(G)=|E(G)|$. For $v\in V(G)$, the
set of vertices adjacent to $v$ in $G$ is called the neighborhood of $v$ and denoted by $N_G(v)$. We denote by $d_G(v)=|N_G(v)|$ the degree of
$v$ in $G$, and by $\delta(G)$ (or $\delta$ for short) the minimum degree of $G$. Let $\alpha(G)$ and $i(G)$ denote the independence number and
the number of isolated vertices in $G$, respectively. For any $S\subseteq V(G)$, let $G[S]$ denote the subgraph of $G$ induced by $S$, and write
$G-S=G[V(G)\setminus S]$. The complete graph of order $n$ is denoted by $K_n$. Let $c$ be a real number. Recall that $\lceil c\rceil$ is the
smallest integer satisfying $\lceil c\rceil\geq c$.

Let $G_1$ and $G_2$ be two vertex-disjoint graphs. The union of $G_1$ and $G_2$ is denoted by $G_1\cup G_2$, which is the graph with vertex set
$V(G_1)\cup V(G_2)$ and edge set $E(G_1)\cup E(G_2)$. The join $G_1\vee G_2$ is obtained from $G_1\cup G_2$ by adding all the edges joining a
vertex of $G_1$ to a vertex of $G_2$.

Given a graph $G$ with vertex set $V(G)=\{v_1,v_2,\ldots,v_n\}$, let $A(G)$ denote the adjacency matrix of $G$. The $(i,j)$-entry of $A(G)$ is 1
if $v_iv_j\in E(G)$, and 0 otherwise. The eigenvalues of $A(G)$ are called the eigenvalues of $G$. It is obvious that $A(G)$ is a real symmetric
nonnegative matrix. Consequently, its eigenvalues are real, which can be arranged in non-increasing order as
$\lambda_1(G)\geq\lambda_2(G)\geq\cdots\geq\lambda_n(G)$. Notice that the spectral radius of $G$, denoted by $\rho(G)$, is equal to $\lambda_1(G)$.

A spanning tree $T$ of a connected graph $G$ is a subgraph of $G$ that is a tree covers all vertices of $G$. For $v\in V(T)$, the vertex $v$ is
called a leaf of $T$ if $d_T(v)=1$. The leaf degree of a vertex $v\in V(T)$ is defined as the number of leaves adjacent to $v$ in $T$. The leaf
degree of $T$ is the maximum leaf degree among all the vertices of $T$. The leaf distance of $T$ is defined as the minimum of distances between
any two leaves of $T$. In fact, a tree with leaf degree 1 has leaf distance at least 3.

Kaneko \cite{Ks} presented some sufficient conditions for a connected graph to have a spanning tree with leaf distance at least $d=3$ and
conjectured that similar conditions suffice for larger $d$. Later, Kaneko, Kano and Suzuki \cite{KKS} claimed that Kaneko's conjecture is true
for $d=4$. For $d\geq4$, Erbes, Molla, Mousley and Santana \cite{EMMS} showed that a stronger form of Kaneko's conjecture holds for all $n$-vertex
connected graphs with $\alpha(G)\leq5$, and proved Kaneko's conjecture for $d\geq\frac{n}{3}$. Zhou, Sun and Liu \cite{ZSL3} provided two spectral
conditions for a connected graph to contain a spanning tree with leaf distance at least $d=3$. Chen, Lv, Li and Xu \cite{CLLX} investigated the
existence of spanning trees with leaf distance at least $d=4$ in connected graphs and obtained three new results. More results on spanning trees
can be found in \cite{GS,Kyaw,MM,ZW,ZZL,Wc}.

A set $M\subseteq E(G)$ is a matching if no two edges share a vertex. A matching of size $k$ is called a $k$-matching. A matching $M$ is called a
perfect matching (or 1-factor) if it covers all the vertices of $G$. Let $k\geq0$ be an integer. Then a graph $G$ with at least $2k+2$ vertices is
said to be $k$-extendable if every $k$-matching in $G$ can be extended to a perfect matching in $G$. A fractional matching of a graph $G$ is a
function $h$ assigning every edge a real number in $[0,1]$ so that $\sum\limits_{e\in E_G(v)}{h(e)}\leq1$ for any $v\in V(G)$, where $E_G(v)$
denotes the set of edges incident with $v$ in $G$. A fractional matching of $G$ is called a fractional perfect matching if
$\sum\limits_{e\in E_G(v)}{h(e)}=1$ for any $v\in V(G)$. Then a graph $G$ with at least $2k+2$ vertices is said to be fractional $k$-extendable if
every $k$-matching $M$ in $G$ is included in a fractional perfect matching $h$ of $G$ such that $h(e)=1$ for any $e\in M$.

The perfect matching and matching extendability attracted much attention. Tutte \cite{T} provided a characterization for a graph to contain a perfect
matching. Enomoto \cite{E} established a connection between toughness and a perfect matching in a graph. Niessen \cite{N} presented a neighborhood
union condition for a graph to have a perfect matching. O \cite{Os} obtained a spectral radius condition to guarantee that a connected graph has a
perfect matching. Zhang and Lin \cite{ZLp} got a distance spectral condition to guarantee the existence of a perfect matching in a connected graph.
Plummer \cite{P1} first introduced the concept of $k$-extendable graph and obtained some results on $k$-extendable graphs. Ananchuen and Caccetta
\cite{AC}, Lou and Yu \cite{LY}, Cioaba, Koolen and Li \cite{CKL}, Robertshaw and Woodall \cite{RW} investigated the existence of $k$-extendable
graphs. The fractional perfect matching and fractional matching extendability also attracted much attention. Lov\'asz and Plummer \cite{LP} showed
a characterization for the existence of fractional perfect matchings in graphs. Liu and Zhang \cite{LZ} claimed a toughness condition for a graph
to contain a fractional perfect matching in a graph. Ma and Liu \cite{ML} provided a characterization of fractional $k$-extendable graphs. Zhu and
Liu \cite{ZLs} established a relationship between binding numbers and fractional $k$-extendable graphs. Much effort has been devoted to finding
sufficient conditions for the existence of spanning subgraphs (see \cite{Zs,ZSL1,Zr,ZZL2,WZhi,Wp,M,ZWa,ZZS,GWC,Zt,ZZL1}).

Motivated by \cite{Os,EMMS,ML} directly, we are to establish a spectral radius condition for the existence of a spanning tree with leaf distance at
least $d$ in a connected graph, and propose a lower bound on the spectral radius of a connected graph $G$ to guarantee that $G$ is a fractional
$k$-extendable graph. Our main results are shown in the following.

\medskip

\noindent{\textbf{Theorem 1.1.}} Let $G$ be a connected graph of order $n$ with $\alpha(G)\leq5$, and let $d$ be an integer with $16\leq d^{2}\leq n$.
If
$$
\rho(G)\geq\rho(K_{\lceil\frac{d}{2}\rceil-1}\vee(K_{n-\lceil\frac{d}{2}\rceil}\cup K_1)),
$$
then $G$ has a spanning tree with leaf distance at least $d$, unless $G=K_{\lceil\frac{d}{2}\rceil-1}\vee(K_{n-\lceil\frac{d}{2}\rceil}\cup K_1)$.

\medskip

\noindent{\textbf{Theorem 1.2.}} Let $k\geq1$ be an integer, and let $G$ be a connected graph of order $n$ with minimum degree $\delta$ and
$n\geq\max\{2k+9,5\delta+1\}$. If
$$
\rho(G)\geq\max\{\rho(K_{2k}\vee(K_{n-2k-1}\cup K_1)), \rho(K_{\delta}\vee(K_{n-2\delta+2k-1}\cup(\delta-2k+1)K_1))\},
$$
then $G$ is fractional $k$-extendable, unless $G\in\{K_{2k}\vee(K_{n-2k-1}\cup K_1),K_{\delta}\vee(K_{n-2\delta+2k-1}\cup(\delta-2k+1)K_1)\}$.

\medskip

The proofs of Theorems 1.1 and 1.2 will be provided in Sections 3 and 4, respectively.


\section{Preliminary lemmas}

In this section, we put forward some necessary preliminary lemmas, which are very important to the proofs of our main results.

For $t\leq\alpha(G)$, let $\delta_t(G)$ be the minimum order of the neighborhood of an independent set of order $t$ in a graph $G$. Namely,
$\delta_t(G)=\min\{|N_G(I)|: I \ \mbox{is an independent set of order} \ t\}$. Erbes, Molla, Mousley and Santana\cite{EMMS} proved the following
two results.

\medskip

\noindent{\textbf{Lemma 2.1}} (Erbes, Molla, Mousley and Santana \cite{EMMS}). Let $d\geq3$ be an integer, and let $G$ be a connected graph. Then $i(G-S)<\frac{2}{d-2}|S|$ for all nonempty $S\subseteq V(G)$ if and only if $\delta_t(G)>\frac{t(d-2)}{2}$ for all $t$ satisfying $1\leq t\leq\alpha(G)$.

\medskip

\noindent{\textbf{Lemma 2.2}} (Erbes, Molla, Mousley and Santana \cite{EMMS}). Let $d$ be an integer with $d\geq4$, and let $G$ be a connected graph
of order $n$ with $n>d$ and $\alpha(G)\leq5$. If
$$
\delta_{2t}(G)>t(d-2)
$$
for all $t$ satisfying $1\leq t\leq\frac{\alpha(G)}{2}$, then $G$ has a spanning tree with leaf distance at least $d$.

\medskip

Ma and Liu \cite{ML} showed a characterization for a graph to be fractional $k$-extendable.

\medskip

\noindent{\textbf{Lemma 2.3}} (\cite{ML}). Let $k\geq1$ be an integer, and let $G$ be a graph with a $k$-matching. Then $G$ is fractional
$k$-extendable if and only if
$$
i(G-S)\leq|S|-2k
$$
holds for any $S\subseteq V(G)$ such that $G[S]$ contains a $k$-matching.

\medskip

\noindent{\textbf{Lemma 2.4}} (\cite{B}). Let $G$ be a connected graph, and let $H$ be a proper subgraph of $G$. Then $\rho(G)>\rho(H)$.

\medskip

\noindent{\textbf{Lemma 2.5}} (Hong \cite{Ha}). Let $G$ be a graph with $n$ vertices. Then
$$
\rho(G)\leq\sqrt{2e(G)-n+1},
$$
where the equality holds if and only if $G$ is a star or a complete graph.

\medskip

Let $M$ be a real symmetric matrix whose rows and columns are indexed by $V=\{1,2,\cdots,n\}$.
Suppose that $M$ can be written as
\begin{align*}
M=\left(
  \begin{array}{ccc}
    M_{11} & \cdots & M_{1s}\\
    \vdots & \ddots & \vdots\\
    M_{s1} & \cdots & M_{ss}\\
  \end{array}
\right)
\end{align*}
in terms of partition $\pi: V=V_1\cup V_2\cup\cdots\cup V_s$ , wherein $M_{ij}$ is the submatrix (block) of $M$ obtained by rows in $V_i$ and
columns in $V_j$. The average row sum of $M_{ij}$ is denoted by $q_{ij}$. Then matrix $M_{\pi}=(q_{ij})$ is said to be the quotient matrix of $M$.
If the row sum of every block $M_{ij}$ is a constant, then the partition is equitable.

\medskip

\noindent{\textbf{Lemma 2.6}} (\cite{YYSX}). Let $M$ be a real matrix with an equitable partition $\pi$, and let $M_{\pi}$ be the corresponding
quotient matrix. Then every eigenvalue of $M_{\pi}$ is an eigenvalue of $M$. Furthermore, if $M$ is nonnegative, then the largest eigenvalues of
$M$ and $M_{\pi}$ are equal.

\medskip

The subsequent lemma is the well-known Cauchy Interlacing Theorem.

\medskip

\noindent{\textbf{Lemma 2.7}} (Haemers \cite{Hi}). Let $M$ be a Hermitian matrix of order $s$, and let $N$ be a principal submatrix of $M$ with
order $t$. If $\lambda_1\geq\lambda_2\geq\cdots\geq\lambda_s$ are the eigenvalues of $M$ and $\mu_1\geq\mu_2\geq\cdots\geq\mu_t$ are the eigenvalues
of $N$, then $\lambda_i\geq\mu_i\geq\lambda_{s-t+i}$ for $1\leq i\leq t$.


\section{The proof of Theorem 1.1}

In order to verify Theorem 1.1, we first prove the following lemma.

\medskip

\noindent{\textbf{Lemma 3.1.}} Let $d$ is an integer with $d\geq3$, and let $G$ be a connected graph of order $n$ with $n\geq d^{2}$. If
$$
\rho(G)\geq\rho(K_{\lceil\frac{d}{2}\rceil-1}\vee(K_{n-\lceil\frac{d}{2}\rceil}\cup K_1)),
$$
then $\delta_t(G)>\frac{t(d-2)}{2}$ for all $t$ satisfying $1\leq t\leq\alpha(G)$, unless $G=K_{\lceil\frac{d}{2}\rceil-1}\vee(K_{n-\lceil\frac{d}{2}\rceil}\cup K_1)$.

\medskip

\medskip

\noindent{\it Proof.} Suppose that $\delta_t(G)\leq\frac{t(d-2)}{2}$ for some $t$ satisfying $1\leq t\leq\alpha(G)$. According to Lemma 2.1,
we conclude
$$
\frac{(d-2)\cdot i(G-S)}{2}\geq|S|
$$
for some nonempty $S\subseteq V(G)$. By the integrity of $|S|$, we see
$$
\left\lceil\frac{(d-2)\cdot i(G-S)}{2}\right\rceil\geq|S|
$$
for some nonempty $S\subseteq V(G)$. Let $|S|=s$ and $i(G-S)=q$. Then $G$ is a spanning subgraph of
$G_1=K_{\lceil\frac{q(d-2)}{2}\rceil}\vee(K_{n_1}\cup qK_1)$, where $n_1=n-\lceil\frac{q(d-2)}{2}\rceil-q$. In view of Lemma 2.4, we obtain
\begin{align}\label{eq:3.1}
\rho(G)\leq\rho(K_{\lceil\frac{q(d-2)}{2}\rceil}\vee(K_{n-\lceil\frac{q(d-2)}{2}\rceil-q}\cup qK_1)),
\end{align}
where the equality holds if and only if $G=K_{\lceil\frac{q(d-2)}{2}\rceil}\vee(K_{n-\lceil\frac{q(d-2)}{2}\rceil-q}\cup qK_1)$.

If $q=1$, then $G_1=K_{\lceil\frac{d}{2}\rceil-1}\vee(K_{n-\lceil\frac{d}{2}\rceil}\cup K_1)$. Using \eqref{eq:3.1}, we get
$$
\rho(G)\leq\rho(K_{\lceil\frac{d}{2}\rceil-1}\vee(K_{n-\lceil\frac{d}{2}\rceil}\cup K_1)),
$$
with equality if and only if $G=K_{\lceil\frac{d}{2}\rceil-1}\vee(K_{n-\lceil\frac{d}{2}\rceil}\cup K_1)$, a contradiction. In what follows,
we consider $q\geq2$.

Recall that $G_1=K_{\lceil\frac{q(d-2)}{2}\rceil}\vee(K_{n-\lceil\frac{q(d-2)}{2}\rceil-q}\cup qK_1)$. By a direct computation, we have
\begin{align}\label{eq:3.2}
e(G_1)=&\binom{n-q}{2}+q\left\lceil\frac{q(d-2)}{2}\right\rceil\nonumber\\
=&\frac{(n-q)(n-q-1)}{2}+q\left\lceil\frac{q(d-2)}{2}\right\rceil.
\end{align}

It follows from \eqref{eq:3.2} and Lemma 2.5 that
\begin{align}\label{eq:3.3}
\rho(G_1)\leq&\sqrt{2e(G_1)-n+1}\nonumber\\
=&\sqrt{(n-q)(n-q-1)+2q\left\lceil\frac{q(d-2)}{2}\right\rceil-n+1}\nonumber\\
\leq&\sqrt{(n-q)(n-q-1)+2q\cdot\frac{q(d-2)+1}{2}-n+1}\nonumber\\
=&\sqrt{(d-1)q^{2}-(2n-2)q+n^{2}-2n+1}.
\end{align}
Let $\psi_1(q)=(d-1)q^{2}-(2n-2)q+n^{2}-2n+1$. Notice that $n\geq\lceil\frac{q(d-2)}{2}\rceil+q\geq\frac{q(d-2)}{2}+q=\frac{qd}{2}$. Then we
obtain $2\leq q\leq\frac{2n}{d}$. By a direct calculation, we have
$$
\psi_1\left(\frac{2n}{d}\right)-\psi_1(2)=-\frac{4}{d^{2}}(n-d)(n-d^{2})\leq0
$$
by $n\geq d^{2}$. Thus, we see that $\psi_1(q)$ attains its maximum value at $q=2$ for $2\leq q\leq\frac{2n}{d}$. Together with \eqref{eq:3.3},
$n\geq d^{2}$ and $d\geq3$, we get
\begin{align}\label{eq:3.4}
\rho(G_1)\leq&\sqrt{\psi_1(2)}\nonumber\\
=&\sqrt{4(d-1)-2(2n-2)+n^{2}-2n+1}\nonumber\\
=&\sqrt{(n-2)^{2}-2n+4d-3}\nonumber\\
\leq&\sqrt{(n-2)^{2}-2d^{2}+4d-3}\nonumber\\
=&\sqrt{(n-2)^{2}-2(d-1)^{2}-1}\nonumber\\
<&n-2.
\end{align}

Since $K_{n-1}$ is a proper subgraph of $K_{\lceil\frac{d}{2}\rceil-1}\vee(K_{n-\lceil\frac{d}{2}\rceil}\cup K_1)$, it follows from Lemma 2.4
that
\begin{align}\label{eq:3.5}
\rho(K_{\lceil\frac{d}{2}\rceil-1}\vee(K_{n-\lceil\frac{d}{2}\rceil}\cup K_1))>\rho(K_{n-1})=n-2.
\end{align}

Using \eqref{eq:3.1}, \eqref{eq:3.4} and \eqref{eq:3.5}, we conclude
$$
\rho(G)\leq\rho(G_1)<n-2<\rho(K_{\lceil\frac{d}{2}\rceil-1}\vee(K_{n-\lceil\frac{d}{2}\rceil}\cup K_1)),
$$
which contradicts $\rho(G)\geq\rho(K_{\lceil\frac{d}{2}\rceil-1}\vee(K_{n-\lceil\frac{d}{2}\rceil}\cup K_1))$. This completes the proof of
Lemma 3.1. \hfill $\Box$

\medskip

Next, we prove Theorem 1.1.

\medskip

\noindent{\it Proof of Theorem 1.1.} According to Lemma 3.1, we see
$$
\delta_k(G)>\frac{k(d-2)}{2}
$$
for all even $k$ satisfying $2\leq k\leq\alpha(G)$, unless $G=K_{\lceil\frac{d}{2}\rceil-1}\vee(K_{n-\lceil\frac{d}{2}\rceil}\cup K_1)$. Let
$k=2t$. Then we have
$$
\delta_{2t}(G)>t(d-2)
$$
for all $t$ satisfying $1\leq t\leq\frac{\alpha(G)}{2}$, unless $G=K_{\lceil\frac{d}{2}\rceil-1}\vee(K_{n-\lceil\frac{d}{2}\rceil}\cup K_1)$.
Combining this with $n\geq d^{2}\geq16$, $\alpha(G)\leq5$ and Lemma 2.2, we see that $G$ has a spanning tree with leaf distance at least $d$,
unless $G=K_{\lceil\frac{d}{2}\rceil-1}\vee(K_{n-\lceil\frac{d}{2}\rceil}\cup K_1)$. Theorem 1.1 is proved. \hfill $\Box$


\section{The proof of Theorem 1.2}

In this section, we prove Theorem 1.2.

\medskip

\noindent{\it Proof of Theorem 1.2.} Suppose, to the contrary, that $G$ is not fractional $k$-extendable. According to Lemma 2.3, there exists
some nonempty subset $S$ of $V(G)$ such that $|S|\geq2k$ and $i(G-S)\geq|S|-2k+1$. Then $G$ is a spanning subgraph of
$G_1=K_s\vee(K_{n_1}\cup(s-2k+1)K_1)$, where $|S|=s\geq2k$ and $n_1=n-2s+2k-1\geq0$. Using Lemma 2.4, we conclude
\begin{align}\label{eq:4.1}
\rho(G)\leq\rho(G_1),
\end{align}
with equality if and only if $G=G_1$. Notice that $G$ has the minimum degree $\delta$. Thus, we have $\delta(G_1)=s\geq\delta(G)=\delta$. Then
we proceed by the following two cases.

\medskip

\noindent{\bf Case 1.} $\delta\leq2k$.

Obviously, $s\geq2k\geq\delta$. Let $G_2=K_{2k}\vee(K_{n-2k-1}\cup K_1)$. We are to prove that $\rho(G_1)\leq\rho(G_2)$ with equality if and only
if $G_1=G_2$.

It is obvious that $G_1=G_2$ if $s=2k$, and so $\rho(G_1)=\rho(G_2)$. Next, we are to consider $s\geq2k+1$.

In terms of the partition $V(G_1)=V(K_s)\cup V(K_{n-2s+2k-1})\cup V((s-2k+1)K_1)$, the quotient matrix of $A(G_1)$ is equal to
\begin{align*}
B_1=\left(
  \begin{array}{ccc}
    s-1 & n-2s+2k-1 & s-2k+1\\
    s & n-2s+2k-2 & 0\\
    s & 0 & 0\\
  \end{array}
\right).
\end{align*}
Then the characteristic polynomial of the matrix $B_1$ is
\begin{align*}
\varphi_{B_1}(x)=x^{3}+(s-2k+3-n)x^{2}+(2ks-s^{2}-2k+2-n)x+s(s-2k+1)(n-2s+2k-2).
\end{align*}
Since the partition $V(G_1)=V(K_s)\cup V(K_{n-2s+2k-1})\cup V((s-2k+1)K_1)$ is equitable, it follows from Lemma 2.6 that $\rho(G_1)$ is the largest
root of $\varphi_{B_1}(x)=0$. Namely, $\varphi_{B_1}(\rho(G_1))=0$. Let $\gamma_1=\rho(G_1)\geq\gamma_2\geq\gamma_3$ be the three roots of $\varphi_{B_1}(x)=0$
and $Q_1=\mbox{diag}(s,n-2s+2k-1,s-2k+1)$. One checks that
\begin{align*}
Q_1^{\frac{1}{2}}B_1Q_1^{-\frac{1}{2}}=\left(
  \begin{array}{ccc}
    s-1 & s^{\frac{1}{2}}(n-2s+2k-1)^{\frac{1}{2}} & s^{\frac{1}{2}}(s-2k+1)^{\frac{1}{2}}\\
    s^{\frac{1}{2}}(n-2s+2k-1)^{\frac{1}{2}} & n-2s+2k-2 & 0\\
    s^{\frac{1}{2}}(s-2k+1)^{\frac{1}{2}} & 0 & 0\\
  \end{array}
\right)
\end{align*}
is symmetric, and also contains
\begin{align*}
\left(
  \begin{array}{ccc}
    n-2s+2k-2 & 0\\
    0 & 0\\
  \end{array}
\right)
\end{align*}
as its submatrix. Since $Q_1^{\frac{1}{2}}B_1Q_1^{-\frac{1}{2}}$ and $B_1$ have the same eigenvalues, by Lemma 2.7, we conclude
\begin{align}\label{eq:4.2}
\gamma_2\leq n-2s+2k-2<n-2.
\end{align}

Recall that $G_2=K_{2k}\vee(K_{n-2k-1}\cup K_1)$. Then the quotient matrix of $A(G_2)$ by the partition $V(G_2)=V(K_{2k})\cup V(K_{n-2k-1})\cup V(K_1)$
is equal to
\begin{align*}
B_2=\left(
  \begin{array}{ccc}
    2k-1 & n-2k-1 & 1\\
    2k & n-2k-2 & 0\\
    2k & 0 & 0\\
  \end{array}
\right),
\end{align*}
whose characteristic polynomial is
$$
\varphi_{B_2}(x)=x^{3}+(3-n)x^{2}+(2-2k-n)x+2k(n-2k-2).
$$
By virtue of Lemma 2.6, the largest root, say $\rho_2$, of $\varphi_{B_2}(x)=0$ is equal to $\rho(G_2)$.

Note that $K_{n-1}$ is a proper subgraph of $G_2=K_{2k}\vee(K_{n-2k-1}\cup K_1)$, it follows from \eqref{eq:4.2} and Lemma 2.4 that
\begin{align}\label{eq:4.3}
\rho_2=\rho(G_2)>\rho(K_{n-1})=n-2\geq\gamma_2.
\end{align}
Next, we prove $\varphi_{B_1}(\rho_2)=\varphi_{B_1}(\rho_2)-\varphi_{B_2}(\rho_2)>0$. By a direct calculation, we get
\begin{align}\label{eq:4.4}
\varphi_{B_1}(\rho_2)=\varphi_{B_1}(\rho_2)-\varphi_{B_2}(\rho_2)=(s-2k)f(\rho_2),
\end{align}
where $f(\rho_2)=\rho_2^{2}-s\rho_2+(s+1)n-2s^{2}+2ks-4s-2k-2$.

\noindent{\bf Claim 1.} $f(\rho_2)>0$ for $\rho_2>n-2$.

\noindent{\it Proof.} Firstly, we consider $n=2s-2k+1$. Together with $n\geq2k+9$, we deduce $s\geq2k+4$. Combining this with \eqref{eq:4.3}, we get
$$
\frac{s}{2}<2s-2k-1=n-2<\rho_2,
$$
and so
\begin{align*}
f(\rho_2)>&f(n-2)\\
=&n^{2}-3n-2s^{2}+2ks-2s-2k+2\\
=&(2s-2k+1)^{2}-3(2s-2k+1)-2s^{2}+2ks-2s-2k+2\\
=&2s^{2}-(6k+4)s+4k^{2}\\
\geq&2(2k+4)^{2}-(6k+4)(2k+4)+4k^{2}\\
=&16>0.
\end{align*}

Now we consider $n\geq2s-2k+2$. If $s\geq2k+2$, then it follows from \eqref{eq:4.3} that
$$
\frac{s}{2}<2s-2k\leq n-2<\rho_2,
$$
and so
\begin{align*}
f(\rho_2)>&f(n-2)\\
=&n^{2}-3n-2s^{2}+2ks-2s-2k+2\\
\geq&(2s-2k+2)^{2}-3(2s-2k+2)-2s^{2}+2ks-2s-2k+2\\
=&2s^{2}-6ks+4k^{2}-4k\\
\geq&2(2k+2)^{2}-6k(2k+2)+4k^{2}-4k\\
=&8>0.
\end{align*}
If $s=2k+1$, then we deduce
$$
\frac{s}{2}<2s-2k\leq n-2<\rho_2
$$
by \eqref{eq:4.3} and $n\geq2s-2k+2$. Combining this with $n\geq2k+9$, we conclude
\begin{align*}
f(\rho_2)>&f(n-2)\\
=&n^{2}-3n-2s^{2}+2ks-2s-2k+2\\
\geq&(2k+9)^{2}-3(2k+9)-2(2k+1)^{2}+2k(2k+1)-2(2k+1)-2k+2\\
=&8k+52>0.
\end{align*}
Claim 1 is proved. \hfill $\Box$

According to \eqref{eq:4.3}, \eqref{eq:4.4}, $s\geq2k+1$ and Claim 1, we obtain
$$
\varphi_{B_1}(\rho_2)=(s-2k)f(\rho_2)>0.
$$
As $\gamma_2\leq n-2<\rho(G_2)=\rho_2$ (see \eqref{eq:4.3}), we deduce
$$
\rho(G_1)=\gamma_1<\rho_2=\rho(G_2).
$$

From the above discussion, we have
\begin{align}\label{eq:4.5}
\rho(G_1)\leq\rho(G_2),
\end{align}
with equality if and only if $G_1=G_2$. Recall that $G_2=K_{2k}\vee(K_{n-2k-1}\cup K_1)$. It follows from \eqref{eq:4.1} and \eqref{eq:4.5} that
$$
\rho(G)\leq\rho(K_{2k}\vee(K_{n-2k-1}\cup K_1)),
$$
with equality if and only if $G=K_{2k}\vee(K_{n-2k-1}\cup K_1)$, a contradiction.

\medskip

\noindent{\bf Case 2.} $\delta\geq2k+1$.

Clearly, $s\geq\delta\geq2k+1$. Recall that $G_1=K_s\vee(K_{n-2s+2k-1}\cup(s-2k+1)K_1)$, the adjacency matrix $A(G_1)$ of $G_1$ has the quotient
matrix $B_1$, and $B_1$ has the characteristic polynomial $\varphi_{B_1}(x)$. Let $G_3=K_{\delta}\vee(K_{n-2\delta+2k-1}\cup(\delta-2k+1)K_1)$,
where $n\geq2\delta-2k+1$. We are to verify $\rho(G_1)\leq\rho(G_3)$ with equality if and only if $G_1=G_3$.

It is clear that $G_1=G_3$ if $s=\delta$, and so $\rho(G_1)=\rho(G_3)$. In what follows, we are to consider $s\geq\delta+1$.

For the graph $G_3$, its adjacency matrix $A(G_3)$ has the quotient matrix $B_3$ which is formed by replacing $s$ with $\delta$ in $B_1$, and $B_3$
has the characteristic polynomial $\varphi_{B_3}(x)$ which is derived by replacing $s$ with $\delta$ in $\varphi_{B_1}(x)$. Thus, we obtain
\begin{align*}
\varphi_{B_3}(x)=x^{3}+(\delta-2k+3-n)x^{2}+(2k\delta-\delta^{2}-2k+2-n)x+\delta(\delta-2k+1)(n-2\delta+2k-2).
\end{align*}
In terms of Lemma 2.6, the largest root, say $\rho_3$, of $\varphi_{B_3}(x)=0$ equals the spectral radius of $G_3$. That is, $\rho(G_3)=\rho_3$.

Notice that $\varphi_{B_3}(\rho_3)=0$. By plugging the value $\rho_3$ into $x$ of $\varphi_{B_1}(x)-\varphi_{B_3}(x)$, we have
\begin{align}\label{eq:4.6}
\varphi_{B_1}(\rho_3)=\varphi_{B_1}(\rho_3)-\varphi_{B_3}(\rho_3)=(s-\delta)g(\rho_3),
\end{align}
where $g(\rho_3)=\rho_3^{2}-(s+\delta-2k)\rho_3-2s^{2}+ns+6ks-2\delta s-4s-2kn+\delta n+n-2\delta^{2}+6k \delta-4\delta-4k^{2}+6k-2$. Since
$K_{n-\delta+k-1}$ is a proper subgraph of $G_3$, it follows from Lemma 2.4 that
\begin{align}\label{eq:4.7}
\rho_3=\rho(G_3)>\rho(K_{n-\delta+k-1})=n-\delta+k-2.
\end{align}
From \eqref{eq:4.7}, $s\geq\delta+1$ and $n\geq2s-2k+1$, we deduce
\begin{align}\label{eq:4.8}
\frac{s+\delta-2k}{2}<n-\delta+k-2<\rho_3,
\end{align}
which leads to
\begin{align}\label{eq:4.9}
g(\rho_3)>&g(n-\delta+k-2)\nonumber\\
=&n^{2}-(2\delta-2k+3)n-2s^{2}+(5k-\delta-2)s+k\delta+2\delta-k^{2}-2k+2.
\end{align}

We first consider $s\geq\frac{5}{2}\delta+1$. In view of \eqref{eq:4.9} and $n\geq2s-2k+1$, we conclude
\begin{align}\label{eq:4.10}
g(\rho_3)>&n^{2}-(2\delta-2k+3)n-2s^{2}+(5k-\delta-2)s+k\delta+2\delta-k^{2}-2k+2\nonumber\\
\geq&(2s-2k+1)^{2}-(2\delta-2k+3)(2s-2k+1)\nonumber\\
&-2s^{2}+(5k-\delta-2)s+k\delta+2\delta-k^{2}-2k+2\nonumber\\
=&2s^{2}-(5\delta-k+4)s+5k\delta-k^{2}+2k.
\end{align}
Let $h(s)=2s^{2}-(5\delta-k+4)s+5k\delta-k^{2}+2k$. Note that $\delta\geq2k+1$ and
$$
\frac{5\delta-k+4}{4}<\frac{5}{2}\delta+1\leq s,
$$
which implies that
\begin{align*}
h(s)\geq&h\left(\frac{5}{2}\delta+1\right)\\
=&\left(\frac{15}{2}k-5\right)\delta-k^{2}+3k-2\\
\geq&\frac{5}{2}k\delta-k^{2}+3k-2\\
\geq&\frac{5}{2}k(2k+1)-k^{2}+3k-2\\
=&4k^{2}+\frac{11}{2}k-2\\
>&0.
\end{align*}
Combining this with \eqref{eq:4.6}, \eqref{eq:4.10} and $s\geq\frac{5}{2}\delta+1$, we infer
\begin{align}\label{eq:4.11}
\varphi_{B_1}(\rho_3)=(s-\delta)g(\rho_3)>(s-\delta)h(s)>0.
\end{align}

According to \eqref{eq:4.7}, $s\geq\frac{5}{2}\delta+1$ and $n\geq2s-2k+1$, we have
\begin{align}\label{eq:4.12}
\varphi_{B_1}'(\rho_3)=&(s-\delta)g'(\rho_3)\nonumber\\
=&(s-\delta)(2\rho_3-s-\delta+2k)\nonumber\\
>&(s-\delta)(2(n-\delta+k-2)-s-\delta+2k)\nonumber\\
=&(s-\delta)(2n-3\delta-s+4k-4)\nonumber\\
\geq&(s-\delta)(2(2s-2k+1)-3\delta-s+4k-4)\nonumber\\
=&(s-\delta)(3s-3\delta-2)\nonumber\\
>&0.
\end{align}
The inequalities \eqref{eq:4.11} and \eqref{eq:4.12} imply
$$
\rho(G_1)=\gamma_1<\rho_3=\rho(G_3).
$$

From the above discussion, we get
\begin{align}\label{eq:4.13}
\rho(G_1)\leq\rho(G_3),
\end{align}
with equality if and only if $G_1=G_3$. Recall that $G_3=K_{\delta}\vee(K_{n-2\delta+2k-1}\cup(\delta-2k+1)K_1)$. By virtue of \eqref{eq:4.1}
and \eqref{eq:4.13}, we deduce
$$
\rho(G)\leq\rho(K_{\delta}\vee(K_{n-2\delta+2k-1}\cup(\delta-2k+1)K_1)),
$$
with equality if and only if $G=K_{\delta}\vee(K_{n-2\delta+2k-1}\cup(\delta-2k+1)K_1)$, a contradiction.

In what follows, we consider $\delta+1\leq s<\frac{5}{2}\delta+1$. Notice that $\frac{5k-\delta-2}{4}<\delta+1\leq s<\frac{5}{2}\delta+1$.
According to \eqref{eq:4.9}, $\delta\geq2k+1$ and $n\geq5\delta+1$, we obtain
\begin{align}\label{eq:4.14}
g(\rho_3)>&n^{2}-(2\delta-2k+3)n-2s^{2}+(5k-\delta-2)s+k\delta+2\delta-k^{2}-2k+2\nonumber\\
>&n^{2}-(2\delta-2k+3)n-2\left(\frac{5}{2}\delta+1\right)^{2}+(5k-\delta-2)\left(\frac{5}{2}\delta+1\right)\nonumber\\
&+k\delta+2\delta-k^{2}-2k+2\nonumber\\
=&n^{2}-(2\delta-2k+3)n-15\delta^{2}+\frac{27}{2}k\delta-14\delta-k^{2}+3k-2\nonumber\\
\geq&(5\delta+1)^{2}-(2\delta-2k+3)(5\delta+1)-15\delta^{2}+\frac{27}{2}k\delta-14\delta-k^{2}+3k-2\nonumber\\
=&\left(\frac{47}{2}k-21\right)\delta-k^{2}+5k-4\nonumber\\
\geq&\left(\frac{47}{2}k-21\right)(2k+1)-k^{2}+5k-4\nonumber\\
=&46k^{2}-\frac{27}{2}k-25\nonumber\\
>&0.
\end{align}
According to \eqref{eq:4.6}, \eqref{eq:4.14} and $\delta+1\leq s<\frac{5}{2}\delta+1$, we have
\begin{align}\label{eq:4.15}
\varphi_{B_1}(\rho_3)=(s-\delta)g(\rho_3)>0.
\end{align}

Using \eqref{eq:4.7}, $\delta+1\leq s<\frac{5}{2}\delta+1$ and $n\geq5\delta+1$, we obtain
\begin{align}\label{eq:4.16}
\varphi_{B_1}'(\rho_3)=&(s-\delta)g'(\rho_3)\nonumber\\
=&(s-\delta)(2\rho_3-s-\delta+2k)\nonumber\\
>&(s-\delta)(2(n-\delta+k-2)-s-\delta+2k)\nonumber\\
=&(s-\delta)(2n-3\delta-s+4k-4)\nonumber\\
\geq&(s-\delta)(2(5\delta+1)-3\delta-s+4k-4)\nonumber\\
=&(s-\delta)(7\delta-s+4k-2)\nonumber\\
>&0.
\end{align}
The inequalities \eqref{eq:4.15} and \eqref{eq:4.16} yield
$$
\rho(G_1)=\gamma_1<\rho_3=\rho(G_3).
$$

From the above discussion, we conclude
\begin{align}\label{eq:4.17}
\rho(G_1)\leq\rho(G_3),
\end{align}
with equality if and only if $G_1=G_3$. Recall that $G_3=K_{\delta}\vee(K_{n-2\delta+2k-1}\cup(\delta-2k+1)K_1)$. It follows from \eqref{eq:4.1}
and \eqref{eq:4.17} that
$$
\rho(G)\leq\rho(K_{\delta}\vee(K_{n-2\delta+2k-1}\cup(\delta-2k+1)K_1)),
$$
with equality if and only if $G=K_{\delta}\vee(K_{n-2\delta+2k-1}\cup(\delta-2k+1)K_1)$, a contradiction. This completes the proof of Theorem 1.2. \hfill $\Box$


\medskip

\section*{Data availability statement}

My manuscript has no associated data.


\section*{Declaration of competing interest}

The authors declare that they have no conflicts of interest to this work.


\section*{Acknowledgments}

%The authors would like to express their sincere gratitude to the referee for his/her very careful reading of the paper and
%for insightful comments and valuable suggestions, which improved the quality of this paper.

This work was supported by the Natural Science Foundation of Jiangsu Province (Grant No. BK20241949). Project ZR2023MA078 supported by Shandong
Provincial Natural Science Foundation.

\begin{thebibliography}{9999}

\bibitem {Ks} A. Kaneko, Spanning trees with constraints on the leaf degree, Discrete Applied Mathematics 115(2001)73--76.

\bibitem {KKS} A. Kaneko, M. Kano, K. Suzuki, Spanning trees with leaf distance at least four, Journal of Graph Theory 55(2007)83--90.

\bibitem {EMMS} C. Erbes, T. Molla, S. Mousley, M. Santana, Spanning trees with leaf distance at least $d$, Discrete Mathematics 340(2017)1412--1418.

\bibitem {ZSL3} S. Zhou, Z. Sun, H. Liu, $\mathcal{D}$-index and $\mathcal{Q}$-index for spanning trees with leaf degree at most $k$ in graphs, Discrete
Mathematics 347(5)(2024)113927.

\bibitem {CLLX} H. Chen, X. Lv, J. Li, S. Xu, Sufficient conditions for spanning trees with constrained leaf distance in a graph, Discussiones
Mathematicae Graph Theory, https://doi.org/10.7151/dmgt.2530

\bibitem {GS} R. Gould, W. Shull, On spanning trees with few branch vertices, Discrete Mathematics 343(2020)111581.

\bibitem {Kyaw} A. Kyaw, Spanning trees with at most $k$ leaves in $K_{1,4}$-free graphs, Discrete Mathematics 311(2011)2135--2142.

\bibitem {MM} H. Matsuda, H. Matsumura, Degree conditions and degree bounded trees, Discrete Mathematics 309(2009)3653--3658.

\bibitem {ZW} S. Zhou, J. Wu, Spanning $k$-trees and distance spectral radius in graphs, Journal of Supercomputing 80(16)(2024)23357--23366.

\bibitem {ZZL} S. Zhou, Y. Zhang, H. Liu, Spanning $k$-trees and distance signless Laplacian spectral radius of graphs, Discrete Applied Mathematics 358(2024)358--365.

\bibitem {Wc} J. Wu, Characterizing spanning trees via the size or the spectral radius of graphs, Aequationes Mathematicae 98(6)(2024)1441--1455.

\bibitem {T} W. Tutte, The factorization of linear graphs, Journal of the London Mathematical Society 22(1947)107--111.

\bibitem {E} H. Enomoto, Toughness and the existence of $k$-factors III, Discrete Mathematics 189(1998)277--282.

\bibitem {N} T. Niessen, Neighborhood unions and regular factors, Journal of Graph Theory (19)(1)(1995)45--64.

\bibitem {Os} S. O, Spectral radius and matchings in graphs, Linear Algebra and its Applications 614(2021)316--324.

\bibitem {ZLp} Y. Zhang, H. Lin, Perfect matching and distance spectral radius in graphs and bipartite graphs, Discrete Applied Mathematics 304(2021)315--322.

\bibitem {P1} M. Plummer, On $n$-extendable graphs, Discrete Mathematics 31(1980)201--210.

\bibitem {AC} N. Ananchuen, L. Caccetta, Matching extension and minimum degree, Discrete Mathematics 170(1997)1--13.

\bibitem {LY} D. Lou, Q. Yu, Connectivity of $k$-extendable graphs with large $k$, Discrete Applied Mathematics 136(2004)55--61.

\bibitem {CKL} S. Cioaba, J. Koolen, W. Li, Max-cut and extendability of matchings in distance-regular graphs, European Journal of Combinatorics 62(2017)232--244.

\bibitem {RW} A. Robertshaw, D. Woodall, Binding number conditions for matching extension, Discrete Mathematics 248(2002)169--179.

\bibitem {LP} L. Lov\'asz, M. Plummer, Matching Theory, North-Holland, New York (1986).

\bibitem {LZ} G. Liu, L. Zhang, Toughness and the existence of fractional $k$-factors of graphs, Discrete Mathematics 308(2008)1741--1748.

\bibitem {ML} Y. Ma, G. Liu, Some results on fractional $k$-extendable graphs, Chinese Journal of Engineering Mathematics 21(4)(2004)567--573.

\bibitem {ZLs} Y. Zhu, G. Liu, Some results on binding number and fractional perfect matching, Journal of Applied Mathematics and Computing 25(1-2)(2007)339--344.

\bibitem {Zs} S. Zhou, Some spectral conditions for star-factors in bipartite graphs, Discrete Applied Mathematics 369(2025)124--130.

\bibitem {ZSL1} S. Zhou, Z. Sun, H. Liu, Distance signless Laplacian spectral radius for the existence of path-factors in graphs, Aequationes
Mathematicae 98(3)(2024)727--737.

\bibitem {Zr} S. Zhou, Regarding $r$-orthogonal factorizations in bipartite graphs, Rocky Mountain Journal of Mathematics, in press.

\bibitem {ZZL2} S. Zhou, Y. Zhang, H. Liu, Some properties of $(a,b,k)$-critical graphs, Filomat 38(16)(2024)5885--5894.

\bibitem {WZhi} S. Wang, W. Zhang, Independence number, minimum degree and path-factors in graphs, Proceedings of the Romanian Academy, Series A:
Mathematics, Physics, Technical Sciences, Information Science 23(3)(2022)229--234.

\bibitem {Wp} J. Wu, Path-factor critical covered graphs and path-factor uniform graphs, RAIRO-Operations Research 56(6)(2022)4317--4325.

\bibitem {M} H. Matsuda, Fan-type results for the existence of $[a,b]$-factors, Discrete Mathematics 306(2006)688--693.

\bibitem {ZWa} S. Zhou, J. Wu, A spectral condition for the existence of component factors in graphs, Discrete Applied Mathematics 376(2025)141--150.

\bibitem {ZZS} S. Zhou, Y. Zhang, Z. Sun, The $A_{\alpha}$-spectral radius for path-factors in graphs, Discrete Mathematics 347(5)(2024)113940.

\bibitem {GWC} W. Gao, W. Wang, Y. Chen, Tight isolated toughness bound for fractional $(k,n)$-critical graphs, Discrete Applied Mathematics 322(2022)194--202.

\bibitem{Zt} S. Zhou, Toughness, fractional extendability and distance spectral radius in graphs, Journal of the Korean Mathematical Society 62(3)(2025)601--617.

\bibitem{ZZL1} S. Zhou, T. Zhang, H. Liu, Sufficient conditions for fractional $k$-extendable graphs, Filomat 39(8)(2025)2711--2724.

\bibitem {B} R. Bapat, Graphs and Matrices, 2nd edition, Hindustan Book Agency, New Delhi, 2018.

\bibitem {Ha} Y. Hong, A bound on the spectral radius of graphs, Linear Algebra and its Applications 108(1988)135--139.

\bibitem {YYSX} L. You, M. Yang, W. So, W. Xi, On the spectrum of an equitable quotient matrix and its application, Linear Algebra and its Applications
577(2019)21--40.

\bibitem {Hi} W. Haemers, Interlacing eigenvalues and graphs, Linear Algebra and its Applications 227 (1995) 593--616.



\end{thebibliography}
\end{document}
