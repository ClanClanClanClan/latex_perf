\documentclass[12pt,a4wide, reqno]{amsart}
\usepackage[utf8]{inputenc}
\usepackage{amsmath}
\usepackage{amsfonts}
\usepackage{amssymb}
\usepackage{listings}
\usepackage{enumitem}
\usepackage{amsthm}
\usepackage{hyperref}
\usepackage{cleveref}
\usepackage{empheq, bm}
\usepackage[left=2cm,right=2cm,top=1.6cm,bottom=1.5cm]{geometry}
\usepackage{graphicx}
\newtheorem{theorem}{Theorem}[section]
\newtheorem{lemma}{Lemma}
\newtheorem{proposition}{Proposition}
\theoremstyle{definition}
\newtheorem{definition}{Definition}[section]
\newtheorem{corollary}[theorem]{Corollary}
\newtheorem{Con}[theorem]{Conjecture}
\newtheorem{Prob}[theorem]{Problem}
\newtheorem{Fact}[theorem]{Fact}
\theoremstyle{remark}
\newtheorem{remark}{Remark}
\newtheorem{Example}[theorem]{Example}
\numberwithin{equation}{section}
\usepackage{fullpage}
\usepackage{enumitem}
\usepackage{xcolor}
\lstdefinelanguage{Sage}{
  morekeywords={for, print, in, and, if, else, elif, def, return, lambda, import, from, as, while, break, continue},
  morekeywords={[2]GF, PolynomialRing, modulus, append},
  morecomment=[l]\#,
  morestring=[b]',
  morestring=[b]",
  sensitive=true
}
 \lstset
 {
  language=Sage,
  basicstyle=\ttfamily\small,
  keywordstyle=\color{blue},
  stringstyle=\color{red},
  commentstyle=\color{gray},
  numbers=none,
  numberstyle=\tiny\color{gray},
  stepnumber=1,
  numbersep=5pt,
  tabsize=4,
  showstringspaces=false,
  breaklines=true,
  frame=single,
  captionpos=b
}

\begin{document}

\title{Orthomorphism Polynomials of degree $7$ over finite fields}
\author{Bhitali Kousik And Dhiren Kumar Basnet }
\address{department of mathematical sciences, tezpur univeristy, tezpur 784028, Assam, India}
\email{\textnormal{kousikbhitali2016@gmail.in, dbasnet@tezu.ernet.in}}
\keywords{Permutation Polynomial, Orthomorphism Polynomial, SageMath}
\subjclass[2020]{11T06, 11T55, 12Y05}

\begin{abstract}
    In 2019, Xiang Fan \cite{xfan} classified all permutation polynomials of degree $7$ over finite fields of odd characteristics. In this paper, we use this classification to determine the complete list of degree $7$ orthomorphism polynomials over finite fields of order $q\in\{11,~13,~17,~19,~25,~49\}.$ In addition, the non-existence of these polynomials is established for certain fields.
\end{abstract}
 
\maketitle
\section{Introduction} 
Let $\mathbb{F}_q$ be the finite field of order $q,$ where $q=p^r,~p$ is a prime and $r\in\mathbb{N}$. We denote the multiplicative group $\mathbb{F}_q\setminus\{0\}$ by $\mathbb{F}_q^*.$ A polynomial $f\in \mathbb{F}_q[x]$ is called a \textit{permutation polynomial} (PP) over $\mathbb{F}_q$ if the associated map $f:c \mapsto f(c)$ permutes $\mathbb{F}_q.$ We call a polynomial $f\in \mathbb{F}_q[x]$ an \textit{orthomorphism polynomial} (OP) if both $f(x)$ and $f(x)-x$ are PP over $\mathbb{F}_q$ and a \textit{complete permutation polynomial} (CPP) if both $f(x)$ and $f(x)+x$ are PP over $\mathbb{F}_q.$ Orthomorphism polynomials are widely used in design theory, particularly in relation to latin squares \cite{evans, wanless}. For more background and applications related to the equivalent concepts of OP and CPP we refer to \cite{evans2, winterhof, wanless2, wanless3}. 
\begin{definition}
    A PP $f\in \mathbb{F}_q[x]$ of degree $d$ is said to be \textit{normalised} if $f$ is monic, $f(0)=0$ and the coefficient of $x^{d-1}$ is zero if the characteristic $p$ of $\mathbb{F}_q$ does not divide $d.$ 
\end{definition}

In 1982, Neiderreiter and Robinson \cite{CPP1} gave a complete classification of all normalised OPs of degree $<6,$ as well as resolved the degree $6$ case for finite fields of order coprime to $6.$~ Infact they classified  CPPs, but the difference is of little significance. To carry this out the authors used Dickson's classification of the corresponding set of PPs \cite{dickson1, dickson2}. In 2013, Shallue and Wanless \cite{shallue} completed this case by classifying all degree $6$ OPs over finite fields of characteristic $2$ and $3.$ Additionally, they provided the total number of all \textit{canonical orthomorphism polynomials} (OP with zero constant term \cite{wanless2}) over fields of order $q\leq 17,$ classified according to their degree. For example, they found that there are a total of $660\times 11=7260,~494 \times 13=6422$ and $272 \times 17=4624$ OPs of degree $7$ over $\mathbb{F}_{11},~\mathbb{F}_{13}$ and $\mathbb{F}_{17}$ respectively. This observation motivated us to investigate their explicit forms. The present paper furthers the research along this line by characterising all OPs of degree $7$ over $\mathbb{F}_q$ for $q\in\{11,~17,~19,~23,~25,~31,~49\}$ and $q\equiv 6~(\textup{mod}~7).$

In our study, mainly the classification of PPs of degree $7$ over $\mathbb{F}_q$ with any odd $q,$ up to linear transformations proposed by Xiang Fan \cite{xfan} will be used along with some computer programs. Two polynomials $f,~g\in \mathbb{F}_q[x]$ are said to be \textit{related by linear transformation} or \textit{linearly related} if there exist $a,~b\in \mathbb{F}_q^*$ and $c,~d\in \mathbb{F}_q$ such that $$g(x)=af(bx+c)+d.$$
Clearly, if $f$ and $g$ are linearly related, then $f$ is a PP over $\mathbb{F}_q$ if and only if so is $g.$ 
\section{Preliminaries}
We begin this section by outlining some results from Xiang Fan's paper \cite{xfan} which will be instrumental in our subsequent work.
\begin{lemma}\cite[Proposition~7]{xfan}\label{l2}
    Every polynomial in $\mathbb{F}_q[x]$ with $\textup{gcd}(q,~7)=1)$ is linearly related to some $G\in\mathbb{F}_q[x]$ in normalised form, where $G(x)=x^7+\sum_{i=1}^{5}g_ix^i$ with all $g_i\in \mathbb{F}_q.$ 
\end{lemma}
\begin{lemma}\cite[Proposition~11]{xfan}\label{l3}
    Any degree $7$ PP over $\mathbb{F}_q$ is linearly related to $x^7$ or to some $G(x)=x^7+\sum_{i=1}^{t}g_ix^i$ where $1\leq t \leq 5$ and all $g_i\in \mathbb{F}_q$ satisfy the following criteria:
    \begin{enumerate}[label=\textup{(\arabic*)}]
   %\setlist[enumerate]{label=\textbf{\arabic*}.}
        \item 
        $0\neq g_t\in \textup{CK}_q(7-t)$ and $g_{t-1}\in \textup{CI}_q(7-t).$
        \item 
        If $7\mid q,$ then $g_{t-1}=0.$
        \item 
        If $t=5$ and $g_4=0,$ then $g_2\in \{0\}\cup \textup{CI}_q(2).$ 
        \item 
        If $t=4$ and $g_3=0,$ then $g_2\in \{0\}\cup \textup{CI}_q(3).$
        \item 
        If $t=3,~g_2=0$ and $q\equiv 1 (\textup{mod}~4),$ then $g_1\in \{0\}\cup \textup{CI}_q(2),$
    \end{enumerate}
    where $\textup{CK}_q(m)$ and $\textup{CI}_q(m),~m\in\mathbb{Z}$ was defined in \cite{xfan} as\\
    \hspace*{3 cm} $\textup{CK}_q(m)=\{\theta^i: 0\leq i < \text{gcd}(m,~q-1)\},$~and\\
 \hspace*{3 cm} $\textup{CI}_q(m)=\{\theta^j: 0\leq j < (q-1)/\text{gcd}(m,~q-1)\},$ $\theta$ is a generator of $\mathbb{F}_q^*.$

 
    Also, whenever there exists another polynomial $F(x)=x^7+\sum_{i=1}^{s}f_i x^i$ with $1\leq s \leq 5$ and all $s_i\in \mathbb{F}_q$ satisfying the same requirements \emph{(1) \textendash\ (5)}, $G$ and $F$ are linearly related if and only if $G=F.$
\end{lemma}
An \textit{exceptional polynomial} over $\mathbb{F}_q$ refers to a polynomial in $\mathbb{F}_q[x]$ which is a PP over infinitely many extensions of $\mathbb{F}_q$ and a \textit{non-exceptional polynomial} is one that does not satisfy this condition.
Now we present one of the key results of \cite{xfan}, an essential basis of our work.
\begin{lemma}\cite[Theorem~13]{xfan}\label{l4}
    Let $A=\{11,~13,~17,~19,~23,~25,~31\}.$ A non-exceptional PP over $\mathbb{F}_q$ of characteristic $p\notin\{2,~3,~7\}$ and $q>7$ exists if and only if $q\in A.$ Furthermore, a polynomial $F(x)\in\mathbb{F}_q[x]$ is a non-exceptional PP of degree $7$ over $\mathbb{F}_q$ if and only if it is linearly related to some $x^7+\sum_{i=1}^{5}f_ix^i$ with $(f_5,~f_4,~f_3,~f_2,~f_1)\in \mathbb{F}_q^5$ listed as follows. This list is exhaustive and non-redundant, meaning it provides a one-to-one correspondence between all the tuples $(f_5,~f_4,~f_3,~f_2,~f_1)$ and all the linearly related classes of non-exceptional PPs of degree $7$ over $\mathbb{F}_q.$
   \begin{displaymath}
 \begin{matrix*}[l]
 q=11:&(0,~0,~0,~5,~0),&(0,~0,~0,~8,~0),& (0,~1~,0,~0,~4),&(0,~1,~0,~8,~5), \\
 &(0,~1,~0,~9,~5),& (1,~0,~5,~0,~2),&(1,~0,~5,~0,~7),&(1,~0,~5,~8,~6)\\
 &(1,~1,~5,~2,~5),&(1,~2,~5,~9,~8),& (1,~4,~5,~8,~0),&(1,~8,~5,~8,~0),\\
 &(1,~5,~5,~1,~5),&(2,~0,~9,~0,~8),& (2,~0,~9,~0,~9),& (2,~0,~9,~4,~4),\\
 &(2,~1,~9,~5,~3),& (2,~2,~9,~5,~8),&(2,~4,~9,~8,~3),&(2,~8,~9,~5,~2), \\
&(2,~8,~9,~7,~8),&(2,~8,~9,~8,~1), &(2,~5,~9,~5,~2),& (2,~5,~9,~6,~1),\\
&(2,~5,~9,~8,~3).& & & \\
\end{matrix*}
\end{displaymath}
\begin{displaymath}
\begin{matrix*}[l]
q=13:& (0,~0,~0,~0,~2), &(0,~0,~0,~0,~6), & (0,~0~,2,~0,~8),&(0,~0,~4,~0,~4),\\ & (0,~0,~8,~0,~3),& (0,~1,~0,~0,~2) &(0,~1,~1,~10,~5),&(0,~1,~2,~1,~0),\\
& (0,~1,~2,~3,~9), &(0,~1,~8,~7,~11),& (0,~2,~1,~0,~8), &(0,~4,~0,~0,~6),\\
& (0,~4,~1,~7,~1), &(0,~4,~4,~3,~3). & &  \\
q=17:& (0,~1,~10,~0,~16), &(1,~0,~6,~0,~11), & (1,~0,~7,~0,~0),&(1,~0,~13,~0,~7),\\
& (1,~0,~13,~0,~14),& (1,~0,~14,~0,~3) &(1,~3,~13,~11,~10),&(1,~10,~3,~14,~11)\\
& (3,~0,~7,~0,~4), &(3,~0,~10,~0,~14),& (3,~0,~12,~0,~0), &(3,~0,~14,~0,~8),\\
& (3,~0,~15,~0,~2), &(3,~9,~11,~14,~10), & (3,~9,~12,~14,~5),& (3,~9,~15,~14,~12),\\
&(3,~15,~10,~12,~4).&  &  & \\
q=19:& (0,~0,~0,~0,~16), &(1,~0,~3,~14,~11), & (1,~0~,5,~0,~4),&(1,~0,~7,~0,~11),\\
& (1,~0,~11,~0,~16),& (1,~0,~18,~9,~4) &(2,~0,~14,~0,~5),  &(2,~0,~16,~0,~9)\\
& (2,~0,~17,~0,~5). & & & \\
q=23:& (1,~1,~0,~4,~9), &(1,~5,~11,~5,~9), & (1,~2~,6,~19,~21).&  \\
q=25:& (0,~0,~0,~0,~\theta), &(0,~0,~0,~0,~\theta^5), & (\theta,~0~,\theta^2,~0,~0),& with~\theta=\theta^2+2~in~ \mathbb{F}_{25}.  \\
q=31:& (1,~0,~16,~0,~2), &(1,~17,~25,~25,~29), & (3,~1~,14,~19,~10).&   \\ 
\end{matrix*}
\end{displaymath}
\end{lemma}
\begin{remark} \label{r1}
    The author also listed  all the exceptional PPs of degree $7$ over $\mathbb{F}_q$ for the above $q,$ up to linear transformations, (previously classified by M\"{u}ller \cite[Theorem~4]{muller}) as $x^7+\sum_{i=1}^{5}g_ix^i$ with $(g_5,~g_4,~g_3,~g_2,~g_1)\in \mathbb{F}_q^5$ as follows.
    \begin{displaymath}
        \begin{matrix*}[l]
        q=11:&(0,~0,~0,~0,~0),& (1,~0,~5,~0,~9), & (2,~0,~9,~0,~6).\\
        q=13:&(0,~0,~0,~0,~0).&  & \\
        q=17:&(0,~0,~0,~0,~0),& (1,~0,~10,~0,~8), & (3,~0,~5,~0,~11).\\
        q=19:&(0,~0,~0,~0,~0),& (1,~0,~3,~0,~7), & (2,~0,~12,~0,~18).\\
        q=23:&(0,~0,~0,~0,~0),& (1,~0,~20,~0,~8), & (5,~0,~17,~0,~11).\\
         q=25:&(0,~0,~0,~0,~0),& (1,~0,~1,~0,~4), & (\theta,~0,~\theta^2,~0,~4\theta^3).\\
        q=31:&(0,~0,~0,~0,~0),& (1,~0,~18,~0,~19), & (3,~0,~7,~0,~17).\\
       
        \end{matrix*}
    \end{displaymath}
\end{remark}
\begin{remark}\label{r2}
From \Cref{l2} and \Cref{l3}, we know that any polynomial $H(x)=\sum_{i=0}^{7}h_ix^i$ over $\mathbb{F}_q$ with $c_7\neq 0$ and $\textup{gcd}~(q,~7)=1$ can be normalised into $F(x)=x^7+\sum_{i=1}^{5}f_ix^i$ that satisfies all the criteria of \Cref{l3}. For $q\in \{11,~13,~17,~19,~23,~25,~31\},~H$ is a PP over $\mathbb{F}_q$ if and only if $F$ is listed in \Cref{l4} or \Cref{r1} and such a $F$ is unique.
\end{remark}

As stated earlier, if $f$ and $g$ in $\mathbb{F}_q[x]$ are two linearly related polynomials, then $f$ is a PP over $\mathbb{F}_q$ implies so is $g$ and vice versa. However, the same does not hold for OPs. For example, $f(x)=4x+1\in \mathbb{F}_5[x]$ is an OP over $\mathbb{F}_5,$ but $f(3x)$ or $3f(x)$ is not. This was the primary difficulty we faced during our classification process. The following trivial lemma offers some help with this problem.
\begin{lemma} \cite{CPP1} \label{l5}
    If $f\in\mathbb{F}_q[x]$ is an orthomorphism polynomial over $\mathbb{F}_q,$ then for all $c,~d\in\mathbb{F}_q$ the same holds for $f(x+ c)+ d$ also.
\end{lemma}

Finally, we state one more lemma that will be used to enumerate the total number of OPs of degree $7$ over $\mathbb{F}_q$ in the next section.
\begin{lemma}\label{l6}
    Let $F(x)= x^7+\sum_{i=1}^{5}f_ix^i$ be a polynomial over $\mathbb{F}_q.$ Define $g(x)=\alpha F(\beta x)$ and $h(x)=\alpha_1 F(\beta_1 x),$ where $\alpha,~\beta,~\alpha_1,~\beta_1 \in \mathbb{F}_q^*.$~If $g( x+\gamma)+\delta= h(x+\gamma_1)+\delta_1$ for all $x\in \mathbb{F}_q$ with $\gamma,~\delta,~\gamma_1,~\delta_1 \in \mathbb{F}_q,$ then we must have $\gamma=\gamma_1$ and $\delta=\delta_1.$ 
\end{lemma}
\begin{proof}
Expanding both $g( x+\gamma)+\delta$ and $h(x+\gamma_1)+\delta_1$ in the given equality yields the following expression.
\begin{equation}\label{eq}
    \begin{aligned}
        \alpha \beta ^7(x+\gamma)^7+\alpha \sum_{i=1}^{5}f_i\beta ^i(x+\gamma)^i+\delta=\alpha_1 \beta_1 ^7(x+\gamma_1)^7+\alpha_1 \sum_{i=1}^{5}f_i\beta_1 ^i(x+\gamma_1)^i+\delta_1.
    \end{aligned}
\end{equation}
We equate the coefficients of $x^7$ on both sides of \Cref{eq} to get
\begin{equation}\label{eq2}
\begin{aligned}
\alpha \beta ^7&=\alpha_1 \beta_1 ^7.
\end{aligned}
    \end{equation}
   Next, by comparing the coefficients of $x^6,$ we obtain
   \begin{equation}\label{eq3}
\begin{aligned}
\binom{7}{1}\alpha \beta ^7 \gamma&=\binom{7}{1}\alpha_1 \beta_1 ^7 \gamma_1.
\end{aligned}
    \end{equation}
    Applying \Cref{eq2} to \Cref{eq3}, we have $\gamma=\gamma_1.$ In a similar way, we observe all the terms involving $x^5$ in \Cref{eq} and use both \Cref{eq2} and \Cref{eq3} to get $\alpha \beta ^5 f_5=\alpha_1 \beta_1 ^5 f_5.$ The corresponding process applied to $x^4,~x^3,~x^2$ and $x$ yields similar kinds of equations that we omit for brevity. Using all these equations in the final step, where we equate the constant terms on both sides of \Cref{eq} gives us the required result, which is $\delta=\delta_1.$
\end{proof}
\section{Main Results}
For ease of calculation, we present our discussions step by step, according to $q.$ The main methodology is detailed for the case $q=13.$ Similar approach applies to other values of $q$, but to avoid repetition, we do not elaborate on each case here.
\vspace{0.5 cm}
\\
\underline{\Large \textbf{$q=13$~:}}
\vspace{0.2cm}
\\
We combine \Cref{l4} and \Cref{r1} to relist all PPs of the prescribed form $f(x)=x^7+\sum_{i=1}^{5}a_ix^i$ over $\mathbb{F}_{13}$ up to linear transformations in terms of the families of tuples $(a_5,~a_4,~a_3,~a_2,~a_1)\in\mathbb{F}_{13}^5$ as follows:
\begin{equation}
\begin{matrix*}[l]
     \textup{Family 1}:(0,~0,~0,~0,~2), & \textup{Family 2}:(0,~0,~0,~0,~6),\\
     \textup{Family 3}: (0,~0,~2,~0,~8),&\textup{Family 4}:(0,~0,~4,~0,~4), \\
     \textup{Family 5}:(0,~0,~8,~0,~3),& \textup{Family 6}: (0,~1,~0,~0,~2),\\
     \textup{Family 7}:(0,~1,~1,~10,~5),& \textup{Family 8}:(0,~1,~2,~1,~0),\\
 \textup{Family 9}:(0,~1,~2,~3,~9),& \textup{Family 10}:(0,~1,~8,~7,~11),\\
 \textup{Family 11}:(0,~2,~1,~0,~8), & \textup{Family 12}:(0,~4,~0,~0,~6),\\
 \textup{Family 13}:(0,~4,~1,~7,~1),& \textup{Family 14}:(0,~4,~4,~3,~3), \\
 \textup{Family 15}:(0,~0,~0,~0,~0). &  \\
\end{matrix*}
\label{eq1}
\end{equation}

For example, Family 1 represents the polynomial $f_1(x)=x^7+2x.$ We first check whether $f_1$ is an OP or not. For this, it will be sufficient to check whether $h_1(x)=f_1(x)-x=x^7+x$ is a PP or not. According to \Cref{r2}, $h_1$ is a PP if and only if it is linearly related to one of the families listed above. As it satisfies all requirements (1) \textendash (5) of \Cref{l3}, it follows that $h_1,$ represented by the tuple $(0,~0,~0,~0,~1)$ must be equal to one of these families. Clearly this is not possible and therefore, $f_1$ can not be an OP.

Next, we consider Family 2 which represents the polynomial $f_2(x)=x^7+6x.$ Define $h(x)$ as $f_2(x)-x=x^7+5x.$ Does it satisfy the requirements of \Cref{l3}? As we know that $2$ is a generator of $\mathbb{F}_{13}^*,$ so by definition of $\textup{CK}_q(m)$, we have
\begin{equation*}
    \begin{aligned}
        \text{CK}_{13}(7-1)=\text{CK}_{13}(6)&=\{2^i: 0\leq i < \text{gcd}(6,~12)\},\\
        &=\{1,~2,~2^2,~2^3,~2^4,~2^5\},\\
        &=\{1,~2,~3,~4,~6,~8\}.
    \end{aligned}
\end{equation*}
But $a_1=5\notin\text{CK}_{13}(7-1),$ where $a_1$ denotes the coefficient of the least degree term of $h.$ It follows that $h$ does not satisfy condition (1) of \Cref{l3} and hence it need not be equal to any families of those tuples even if $h$ is a PP. \\The immediate next step is to determine whether $h$ is linearly related to any of these families. Suppose that $h(x)=af(bx+c)+d,$ where $~a,b\in\mathbb{F}_{13}^*;~c,~d\in\mathbb{F}_{13}$ and $f(x)=x^7+\sum_{i=1}^{5}a_ix^i$ is a polynomial over $\mathbb{F}_{13}$ represented by one of those families listed in \ref{eq1}. We expand it to obtain
$$x^7+5x=a(bx+c)^7+a\displaystyle\sum_{i=1}^{5}a_i(bx+c)^i+d.$$
Comparing the coefficients of all powers of $x$ on its both sides, we get
$$1=ab^7,~0=c=aa_5b^5=aa_4b^4=aa_3b^3=aa_2b^2,~5=aa_1b~\text{and}~0=d.$$
As $a~\text{and}~b$ are non-zero, we must have $a_5=a_4=a_3=a_2=0$ and $a_1\neq 0.$ Therefore, if $h$ is a PP, then it must be linearly related to one of the families of the form $(0,~0,~0,~0,~a_1)$ with $a_1\in\mathbb{F}_{13}^*.$ By observing the list \ref{eq1}, it is evident that $a_1$ can be either $2$ or $6.$ Finally it requires to solve the following system of equations over $\mathbb{F}_{13}^*:$
\begin{equation}
    \begin{matrix*}[l]
         ab^7=1, & 5=2ab,
    \end{matrix*}
\end{equation}
and
\begin{equation}
    \begin{matrix*}[l]
         ab^7=1, & 5=6ab.
    \end{matrix*}
\end{equation}
These equations can be solved either manually or with the aid of a computer program (the code is provided in the next page). After solving these, we found that none of them is solvable over $\mathbb{F}_{13}^*.$ This means that $h$ is not a PP, and consequently $f_2$ cannot be an OP. 

Using similar processes, we went through all the remaining families but could not find any OPs. This observation led us to investigate their linearly related classes. For each family $f$ we only need to determine the values of $\alpha$ and $\beta$ belonging to $\mathbb{F}_{13}^*$ for which $\alpha f(\beta x)$ is an OP. By \Cref{l5}, each such pair $(\alpha,~\beta)$ yields an OP $\alpha f(\beta (x+\gamma))+\delta$, for all $\gamma,~\beta\in\mathbb{F}_{13}.$ We begin this process with Family 1.\\
\textbf{Family 1:}~$(0,~0,~0,~0,~2)$\\ 
 We define 
$$f_1(x)=x^7+2x,~\textup{the polynomial represented by Family 1},$$
$$g_1(x)=\alpha f_1(\beta x)=\alpha \beta^7x^7+2\alpha \beta x,~~~\text{and}$$
$$h_1(x)=\alpha f_1(\beta x)-x=\alpha \beta^7x^7+(2\alpha \beta-1) x.$$
Suppose $h_1$ is a PP. Then it must be linearly related to one of the families of  (\ref{eq1}), say $f'$. It means that there must exist $a,~b\in\mathbb{F}_{13}^*$ and $c,~d\in\mathbb{F}_{13}$ such that
\begin{equation*}
    \begin{aligned}
        h_1(x)&=af'(bx+c)+d
        \end{aligned}
\end{equation*}
which implies that,
\begin{equation*}
    \begin{aligned}
        \alpha \beta^7x^7+(2\alpha \beta-1) x&=a(bx+c)^7+a\sum_{i=1}^{5}a_i(bx+c)^i+d.
        \end{aligned}
\end{equation*}
Comparing the coefficients of all powers of $x,$ we get
$$\alpha \beta^7=ab^7,~0=c=aa_5b^5=aa_4b^4=aa_3b^3=aa_2b^2,~2\alpha \beta-1=aa_1b~\text{and}~0=d.$$
It follows from these conditions that $f'$ must come from families of the form $(0,~0,~0,~0,~a_1)$ with $a_1\in\mathbb{F}_{13},$ namely Family 1 $(a_1=0),$ Family 2 $(a_1=6)$ and Family 15 $(a_1=0).$ Ultimately, our problem reduces to solving the following system of equations:
\begin{equation}\label{eq4}
    \begin{matrix*}[l]
         \alpha \beta ^7=ab^7, & 2\alpha \beta -1=2ab,
    \end{matrix*}
\end{equation}
\begin{equation}\label{eq5}
    \begin{matrix*}[l]
         \alpha \beta ^7=ab^7, & 2\alpha \beta -1=6ab,
    \end{matrix*}
\end{equation}
and
\begin{equation}\label{eq6}
    \begin{matrix*}[l]
         \alpha \beta ^7=ab^7, & 2\alpha \beta -1=0.
    \end{matrix*}
\end{equation}
 To exhaust all the OPs from this family, we only need to determine the values of $\alpha$ and $\beta.$ We run the following code \textbf{OP7(13)} in SageMath to find all the possible values of $a,~b,~\alpha$ and $\beta.$~Let us start by working on (\ref{eq4}).
 \vspace{0.5 cm}
 \hrule
 \vspace{0.2 cm}
 \textbf{OP7(13)}:
 \vspace{0.2 cm}
 \vspace{0.1 cm}
 \begin{lstlisting}
 # Define base field and extension
F13 = GF(13)

# Find all nonzero elements of F13
nonzero = [z for z in F13 if z != 0]

# Store all solutions
solutions = []

# Step 1: Brute-force all (a, b, A, B) in (F13^*)^4. We use A and B in place of alpha and beta respectively
for a in nonzero:
    for b in nonzero:
        ab7 = a * b^7
        ab = a * b
        for A in nonzero:
            for B in nonzero:
                AB7 = A * B^7
                AB = A * B
                # Check both conditions of equation (3.4)
                if AB7 == ab7 and 2 * AB - 1  == 2 * ab:
                    solutions.append((a, b, A, B))

# Print all solutions
print(f"\nTotal tuples of the form (a, b, A, B) found: {len(solutions)} and they are \n")

for i, (a, b, A, B) in enumerate(solutions, 1):
    print(f"{i}: ({a}, {b}, {A}, {B})")

# Step 2: From all those solutions avoid repetitive polynomials
seen_AB = set()
final_unique_solutions = []

for a, b, A, B in solutions:
    AB_pair = (A*B^7, A*B)
    if AB_pair not in seen_AB:
        seen_AB.add(AB_pair)
        final_unique_solutions.append((a, b, A, B))

# Print the final solutions
print(f"\nFinal values of (A, B):\n")
for i, (a, b, A, B) in enumerate(final_unique_solutions, 1):
    print(f"{i}: (A = {A}, B = {B})")

print(f"\nTotal unique solutions: {len(final_unique_solutions)}")
\end{lstlisting}
The output of this program from Step 1 consists of $72$ tuples of the form $(a,~b,~\alpha,~\beta).$ Among them are $(1,~3,~2,~5),~(1,~3,~5,~2),~(1,~3,~6,~2),~(3,~1,~2,~5)$ and many more. Clearly, the first and third tuples yield the same $(\alpha,~\beta)$ pair and thus the same polynomial. Although the first and second tuples give us distinct pairs of $(\alpha,~\beta),$ the corresponding values of $a$ and $b$  indicate that they represent the same polynomial $g_1(x)=\alpha \beta^7x^7+2\alpha \beta x=ab^7x^7+(2ab+1)x;~a=1,~b=3.$ ~Step 2 is utilized at this point to discard all tuples producing the same values for both $\alpha\beta^7$ and $\alpha\beta,$ retaining only one representative from each such tuple. The final values of $(\alpha,~beta)$ obtained from Step 2 are $(2,~5)$ and $(1,~10)$ and both these pairs give rise to two distinct orthomorphism polynomials which are $2f_1(5x)=3x^7+7x$ and $f_1(10x)=10x^7+10x.$ 

We apply similar processes to both systems of equations (\ref{eq5}) and (\ref{eq6}). In each case, orthomorphism polynomials are found to exist. Specifically, (\ref{eq5}) yields four $(\alpha,~\beta)$ pairs, whereas from (\ref{eq6}) we obtain two of such pairs. They are $(1,~3),~(1,~5),~(2,~9),~(2,~8),~(2,~10)$ and $(1,~7).$ According to \Cref{l5}, each such pair $(\alpha,~\beta)$ generates OP of the form $\alpha f_1(\beta(x+\gamma))+\delta,~\text{for all choices of}~\gamma,~\delta \in \mathbb{F}_{13}.$ Furthermore, \Cref{l6} ensures that all these polynomials are distinct. Corresponding to each such $(\alpha,~\beta),$ there are $13$ possible choices for both $\gamma$ and $\delta.$ Since there are eight pairs in total, this yields $8\times 13\times 13=1352$ distinct OPs from Family 1. 

This forms our main methodology. For each of the remaining families over $\mathbb{F}_{13}$ we apply the same technique with minor changes to the code to obtain our results. For brevity, we only list the values of $(\alpha,~\beta),$ if they exist, for each family together with the corresponding systems of equations.\\
\textbf{Family 2:}~$(0,~0,~0,~0,~6)$ 
\begin{equation}\label{eq8}
    \begin{matrix*}[l]
         \alpha \beta ^7=ab^7, & 6\alpha \beta -1=2ab,
    \end{matrix*}
\end{equation}
\begin{equation}\label{eq9}
    \begin{matrix*}[l]
         \alpha \beta ^7=ab^7, & 6\alpha \beta -1=6ab,
    \end{matrix*}
\end{equation}
\begin{equation}\label{eq10}
    \begin{matrix*}[l]
         \alpha \beta ^7=ab^7, & 6\alpha \beta -1=0.
    \end{matrix*}
\end{equation}
\begin{itemize}
    \item 
    (\ref{eq8}) : $(5,~1),~(1,~10),~(1,~5),~(2,~5).$
    \item 
    (\ref{eq9}) : $(2,~6),~(1,~12).$
    \item 
    (\ref{eq10}) : $(2,~12),~(1,~11).$ 
\end{itemize}
\textbf{Family 3:}~$(0,~0,~2,~0,~8)$ 
\begin{equation}\label{eq11}
    \begin{matrix*}[l]
         \alpha \beta ^7=ab^7, & 2\alpha \beta^3=2ab^3, & 8\alpha\beta -1=8ab,
    \end{matrix*}
\end{equation}
\begin{equation}\label{eq12}
    \begin{matrix*}[l]
        \alpha \beta ^7=ab^7, & 2\alpha \beta^3=4ab^3, & 8\alpha\beta -1=4ab,
    \end{matrix*}
\end{equation}
\begin{equation}\label{eq13}
    \begin{matrix*}[l]
         \alpha \beta ^7=ab^7, & 2\alpha \beta^3=8ab^3, & 8\alpha\beta -1=3ab.
    \end{matrix*}
\end{equation}
\begin{itemize}
    \item 
    (\ref{eq11}) : $(5,~7),~(3,~3),~(2,~11),~(6,~8),~(1,~9),~(4,~12).$
    \item 
    (\ref{eq12}) : No pair.
    \item 
    (\ref{eq13}) : No pair. 
\end{itemize}
\textbf{Family 4:}~$(0,~0,~4,~0,~4)$ 
\begin{equation}\label{eq14}
    \begin{matrix*}[l]
         \alpha \beta ^7=ab^7, & 4\alpha \beta^3=2ab^3, & 4\alpha\beta -1=8ab,
    \end{matrix*}
\end{equation}
\begin{equation}\label{eq15}
    \begin{matrix*}[l]
        \alpha \beta ^7=ab^7, & 4\alpha \beta^3=4ab^3, & 4\alpha\beta -1=4ab,
    \end{matrix*}
\end{equation}
\begin{equation}\label{eq16}
    \begin{matrix*}[l]
         \alpha \beta ^7=ab^7, & 4\alpha \beta^3=8ab^3, & 4\alpha\beta -1=3ab.
    \end{matrix*}
\end{equation}
\begin{itemize}
    \item 
    (\ref{eq14}) : No pair.
    \item 
    (\ref{eq15}) : $(5,~1),~(3,~6),~(2,~9),~(6,~3),~(1,~5),~(4,~11).$
    \item 
    (\ref{eq16}) : No pair. 
\end{itemize}
\textbf{Family 5:}~$(0,~0,~8,~0,~3)$ 
\begin{equation}\label{eq17}
    \begin{matrix*}[l]
         \alpha \beta ^7=ab^7, & 8\alpha \beta^3=2ab^3, & 3\alpha\beta -1=8ab,
    \end{matrix*}
\end{equation}
\begin{equation}\label{eq18}
    \begin{matrix*}[l]
        \alpha \beta ^7=ab^7, & 8\alpha \beta^3=4ab^3, & 3\alpha\beta -1=4ab,
    \end{matrix*}
\end{equation}
\begin{equation}\label{eq19}
    \begin{matrix*}[l]
         \alpha \beta ^7=ab^7, & 8\alpha \beta^3=8ab^3, & 3\alpha\beta -1=3ab.
    \end{matrix*}
\end{equation}
\begin{itemize}
    \item 
    (\ref{eq17}) : No pair.
    \item 
    (\ref{eq18}) : No pair.
    \item 
    (\ref{eq19}) : $(5,~10),~(3,~8),~(2,~12),~(6,~4),~(1,~11),~(4,~6).$ 
\end{itemize}
\textbf{Family 15:}~$(0,~0,~0,~0,~0)$ 
\begin{equation}\label{eq20}
    \begin{matrix*}[l]
         \alpha \beta ^7=ab^7, & 12=2ab,
    \end{matrix*}
\end{equation}
\begin{equation}\label{eq21}
    \begin{matrix*}[l]
        \alpha \beta ^7=ab^7, &  12=6ab.
    \end{matrix*}
\end{equation}
\begin{itemize}
    \item 
    (\ref{eq20}) : $(1,~6),~(1,~7).$
    \item 
    (\ref{eq21}) : $(1,~2),~(1,~11).$
\end{itemize}
No additional pairs were found in the remaining families. Thus, we have found a total of $8+8+6+6+6+4=38$ pairs of $(\alpha,~\beta)$ from all the families resulting in total $38\times 13 \times 13=6422$ orthomorphism polynomials over $\mathbb{F}_{13},$ of which four are exceptional polynomials belonging to Family 15. These data match exactly with \cite[Table~3.2]{shallue}.

Subsequently, the same methodology is applied to $q=11,~17,~19,~23,~25~\text{and}~31,$ with minor modifications to the code \textbf{OP7(13)}. We list the corresponding values of $(\alpha,~\beta)$ wherever they exist for each family with the corresponding systems of equations.
\vspace{0.5 cm}
\\
\underline{\Large \textbf{$q=11$~:}}
\vspace{0.2cm}
\\
Analogously to $q=13,$ we combine \Cref{l4} and \Cref{r1} to relist all PPs of the prescribed form $f(x)=x^7+\sum_{i=1}^{5}a_ix^i$ over $\mathbb{F}_{11}$ up to linear transformations in terms of the families of tuples $(a_5,~a_4,~a_3,~a_2,~a_1)\in\mathbb{F}_{11}^5$ as follows:
\begin{equation*}
\begin{matrix*}[l]
    \textup{Family 1}:(0,~0,~0,~5,~0), & \textup{Family 2}:(0,~0,~0,~8,~0),\\
    \textup{Family 3}: (0,~1~,0,~0,~4), &\textup{Family 4}:(0,~1,~0,~8,~5), \\
     \textup{Family 5}:(0,~1,~0,~9,~5),& \textup{Family 6}: (1,~0,~5,~0,~2),\\
    \textup{Family 7}:(1,~0,~5,~0,~7),& \textup{Family 8}:(1,~0,~5,~8,~6),\\
 \textup{Family 9}:(1,~1,~5,~2,~5),& \textup{Family 10}:(1,~2,~5,~9,~8),\\
 \textup{Family 11}:(1,~4,~5,~8,~0), & \textup{Family 12}:(1,~8,~5,~8,~0),\\
 \textup{Family 13}:(1,~5,~5,~1,~5),& \textup{Family 14}:(2,~0,~9,~0,~8), \\
 \textup{Family 15}:(2,~0,~9,~0,~9), & \textup{Family 16}:(2,~0,~9,~4,~4),\\
 \textup{Family 17}:(2,~1,~9,~5,~3), & \textup{Family 18}:(2,~2,~9,~5,~8),\\
 \textup{Family 19}:(2,~4,~9,~8,~3), & \textup{Family 20}:(2,~8,~9,~5,~2),\\
 \textup{Family 21}:(2,~8,~9,~7,~8), & \textup{Family 22}:(2,~8,~9,~8,~1),\\
 \textup{Family 23}:(2,~5,~9,~5,~2), & \textup{Family 24}:(2,~5,~9,~6,~1),\\
 \textup{Family 25}:(2,~5,~9,~8,~3), & \textup{Family 26}:(0,~0,~0,~0,~0),\\
 \textup{Family 27}:(1,~0,~5,~0,~9), & \textup{Family 28}:(2,~0,~9,~0,~6).\\
\end{matrix*}
\end{equation*}
\textbf{Family 6:}~$(1,~0,~5,~0,~2)$ 
\begin{equation}\label{eq22}
    \begin{matrix*}[l]
         \alpha \beta ^7=ab^7, & \alpha \beta ^5=ab^5, & 5\alpha \beta ^3=5ab ^3, & 2\alpha\beta-1=2ab,
    \end{matrix*}
\end{equation}
\begin{equation}\label{eq23}
    \begin{matrix*}[l]
          \alpha \beta ^7=ab^7, & \alpha \beta ^5=ab^5, & 5\alpha \beta ^3=5ab ^3, & 2\alpha\beta-1=7ab,
    \end{matrix*}
\end{equation}
\begin{equation}\label{eq24}
    \begin{matrix*}[l]
          \alpha \beta ^7=ab^7, & \alpha \beta ^5=2ab^5, & 5\alpha \beta ^3=9ab ^3, & 2\alpha\beta-1=8ab,
    \end{matrix*}
\end{equation}
\begin{equation}\label{eq25}
    \begin{matrix*}[l]
          \alpha \beta ^7=ab^7, & \alpha \beta ^5=2ab^5, & 5\alpha \beta ^3=9ab^3, & 2\alpha\beta-1=9ab,
    \end{matrix*}
\end{equation}
\begin{equation}\label{eq26}
    \begin{matrix*}[l]
          \alpha \beta ^7=ab^7, & \alpha \beta ^5=ab^5 & 5\alpha \beta ^3=5ab ^3, & 2\alpha\beta-1=9ab,
    \end{matrix*}
\end{equation}
\begin{equation}\label{eq27}
    \begin{matrix*}[l]
          \alpha \beta ^7=ab^7, & \alpha \beta ^5=2ab^5, & 5\alpha \beta ^3=9ab ^3, & 2\alpha\beta-1=6ab.
    \end{matrix*}
    \end{equation}
\begin{itemize}
    \item 
    (\ref{eq22}) : No pair.
    \item 
    (\ref{eq23}) : $(1,~2),~(2,~1),~(3,~8),~(4,6),~(5,~7).$
    \item 
    (\ref{eq24}) : No pair.
    \item 
    (\ref{eq25}) : No pair.
    \item 
    (\ref{eq26}) : $(1,~3),~(2,~7),~(3,~1),~(4,~9),~(5,~5).$
    \item
    (\ref{eq27}) : No pair.
    \end{itemize}
    
\textbf{Family 7:}~$(1,~0,~5,~0,~7)$ 
\begin{equation}\label{eq28}
    \begin{matrix*}[l]
         \alpha \beta ^7=ab^7, & \alpha \beta ^5=ab^5, & 5\alpha \beta ^3=5ab ^3, & 7\alpha\beta-1=2ab,
    \end{matrix*}
\end{equation}
\begin{equation}\label{eq29}
    \begin{matrix*}[l]
          \alpha \beta ^7=ab^7, & \alpha \beta ^5=ab^5, & 5\alpha \beta ^3=5ab ^3, & 7\alpha\beta-1=7ab,
    \end{matrix*}
\end{equation}
\begin{equation}\label{eq30}
    \begin{matrix*}[l]
          \alpha \beta ^7=ab^7, & \alpha \beta ^5=2ab^5, & 5\alpha \beta ^3=9ab ^3, & 7\alpha\beta-1=8ab,
    \end{matrix*}
\end{equation}
\begin{equation}\label{eq31}
    \begin{matrix*}[l]
          \alpha \beta ^7=ab^7, & \alpha \beta ^5=2ab^5, & 5\alpha \beta ^3=9ab^3, & 7\alpha\beta-1=9ab,
    \end{matrix*}
\end{equation}
\begin{equation}\label{eq32}
    \begin{matrix*}[l]
          \alpha \beta ^7=ab^7, & \alpha \beta ^5=ab^5 & 5\alpha \beta ^3=5ab ^3, & 7\alpha\beta-1=9ab,
    \end{matrix*}
\end{equation}
\begin{equation}\label{eq33}
    \begin{matrix*}[l]
          \alpha \beta ^7=ab^7, & \alpha \beta ^5=2ab^5, & 5\alpha \beta ^3=9ab ^3, & 7\alpha\beta-1=6ab.
    \end{matrix*}
\end{equation}

\begin{itemize}
    \item 
    (\ref{eq28}) : $(1,~9),~(2,~10),~(3,~3),~(4,~5),~(5,~4).$
    \item 
    (\ref{eq29}) : No pair.
    \item 
    (\ref{eq30}) : No pair.
    \item 
    (\ref{eq31}) : No pair.
    \item 
    (\ref{eq32}) : $(1,~5),~(2,~8),~(3,~9),~(4,~4),~(5,~1).$
    \item
    (\ref{eq33}) : No pair.
    \end{itemize}

    \textbf{Family 14:}~$(2,~0,~9,~0,~8)$ 
\begin{equation}\label{eq34}
    \begin{matrix*}[l]
         \alpha \beta ^7=ab^7, & 2\alpha \beta ^5=ab^5, & 9\alpha \beta ^3=5ab ^3, & 8\alpha\beta-1=2ab,
    \end{matrix*}
\end{equation}
\begin{equation}\label{eq35}
    \begin{matrix*}[l]
          \alpha \beta ^7=ab^7, & 2\alpha \beta ^5=ab^5, & 9\alpha \beta ^3=5ab ^3, & 8\alpha\beta-1=7ab,
    \end{matrix*}
\end{equation}
\begin{equation}\label{eq36}
    \begin{matrix*}[l]
          \alpha \beta ^7=ab^7, & 2\alpha \beta ^5=2ab^5, & 9\alpha \beta ^3=9ab ^3, & 8\alpha\beta-1=8ab.
    \end{matrix*}
\end{equation}
\begin{equation}\label{eq37}
    \begin{matrix*}[l]
          \alpha \beta ^7=ab^7, & 2\alpha \beta ^5=2ab^5, & 9\alpha \beta ^3=9ab^3, & 8\alpha\beta-1=9ab,
    \end{matrix*}
\end{equation}
\begin{equation}\label{eq38}
    \begin{matrix*}[l]
          \alpha \beta ^7=ab^7, & 2\alpha \beta ^5=ab^5 & 9\alpha \beta ^3=5ab ^3, & 8\alpha\beta-1=9ab,
    \end{matrix*}
\end{equation}
\begin{equation}\label{eq39}
    \begin{matrix*}[l]
          \alpha \beta ^7=ab^7, & 2\alpha \beta ^5=2ab^5, & 9\alpha \beta ^3=9ab ^3, & 8\alpha\beta-1=6ab.
    \end{matrix*}
\end{equation}

\begin{itemize}
    \item 
    (\ref{eq34}) : No pair.
    \item 
    (\ref{eq35}) : No pair.
    \item 
    (\ref{eq36}) : No pair.
    \item 
    (\ref{eq37}) : $(1,~10),~(2,~5),~(3,~7),~(4,~8),~(5,~2).$
    \item 
    (\ref{eq38}) : No pair.
    \item
    (\ref{eq39}) : $(1,~6),~(2,~3),~(3,~2),~(4,~7),~(5,~10).$
    \end{itemize}

    \textbf{Family 15:}~$(2,~0,~9,~0,~9)$ 
\begin{equation}\label{eq40}
    \begin{matrix*}[l]
         \alpha \beta ^7=ab^7, & 2\alpha \beta ^5=ab^5, & 9\alpha \beta ^3=5ab ^3, & 9\alpha\beta-1=2ab,
    \end{matrix*}
\end{equation}
\begin{equation}\label{eq41}
    \begin{matrix*}[l]
          \alpha \beta ^7=ab^7, & 2\alpha \beta ^5=ab^5, & 9\alpha \beta ^3=5ab ^3, & 9\alpha\beta-1=7ab,
    \end{matrix*}
\end{equation}
\begin{equation}\label{eq42}
    \begin{matrix*}[l]
          \alpha \beta ^7=ab^7, & 2\alpha \beta ^5=2ab^5, & 9\alpha \beta ^3=9ab ^3, & 9\alpha\beta-1=8ab,
    \end{matrix*}
\end{equation}
\begin{equation}\label{eq43}
    \begin{matrix*}[l]
          \alpha \beta ^7=ab^7, & 2\alpha \beta ^5=2ab^5, & 9\alpha \beta ^3=9ab^3, & 9\alpha\beta-1=9ab,
    \end{matrix*}
\end{equation}
\begin{equation}\label{eq44}
    \begin{matrix*}[l]
          \alpha \beta ^7=ab^7, & 2\alpha \beta ^5=ab^5 & 9\alpha \beta ^3=5ab ^3, & 9\alpha\beta-1=9ab,
    \end{matrix*}
\end{equation}
\begin{equation}\label{eq45}
    \begin{matrix*}[l]
          \alpha \beta ^7=ab^7, & 2\alpha \beta ^5=2ab^5, & 9\alpha \beta ^3=9ab ^3, & 9\alpha\beta-1=6ab.
    \end{matrix*}
\end{equation}

\begin{itemize}
    \item 
    (\ref{eq40}) : No pair.
    \item 
    (\ref{eq41}) : No pair.
    \item 
    (\ref{eq42}) : $(1,~1),~(2,~6),~(3,~4),~(4,~3),~(5,~9).$
    \item 
    (\ref{eq43}) : No pair.
    \item 
    (\ref{eq44}) : No pair.
    \item
    (\ref{eq45}) : $(1,~4),~(2,~2),~(3,~5),~(4,~1),~(5,~3).$
    \end{itemize}
    
    \textbf{Family 27:}~$(1,~0,~5,~0,~9)$ 
\begin{equation}\label{eq46}
    \begin{matrix*}[l]
         \alpha \beta ^7=ab^7, & \alpha \beta ^5=ab^5, & 5\alpha \beta ^3=5ab ^3, & 9\alpha\beta-1=2ab,
    \end{matrix*}
\end{equation}
\begin{equation}\label{eq47}
    \begin{matrix*}[l]
          \alpha \beta ^7=ab^7, & \alpha \beta ^5=ab^5, & 5\alpha \beta ^3=5ab ^3, & 9\alpha\beta-1=7ab,
    \end{matrix*}
\end{equation}
\begin{equation}\label{eq48}
    \begin{matrix*}[l]
          \alpha \beta ^7=ab^7, & \alpha \beta ^5=2ab^5, & 5\alpha \beta ^3=9ab ^3, & 9\alpha\beta-1=8ab,
    \end{matrix*}
\end{equation}
\begin{equation}\label{eq49}
    \begin{matrix*}[l]
          \alpha \beta ^7=ab^7, & \alpha \beta ^5=2ab^5, & 5\alpha \beta ^3=9ab^3, & 9\alpha\beta-1=9ab,
    \end{matrix*}
\end{equation}
\begin{equation}\label{eq50}
    \begin{matrix*}[l]
          \alpha \beta ^7=ab^7, & \alpha \beta ^5=ab^5 & 5\alpha \beta ^3=5ab ^3, & 9\alpha\beta-1=9ab,
    \end{matrix*}
\end{equation}
\begin{equation}\label{eq51}
    \begin{matrix*}[l]
          \alpha \beta ^7=ab^7, & \alpha \beta ^5=2ab^5, & 5\alpha \beta ^3=9ab ^3, & 9\alpha\beta-1=6ab.
    \end{matrix*}
\end{equation}

\begin{itemize}
    \item 
    (\ref{eq46}) : $(1,~8),~(2,~4),~(3,~10),~(4,~2),~(5,~6).$
    \item 
    (\ref{eq47}) : $(1,~6),~(2,~3),~(3,~2),~(4,~7),~(5,~10).$
    \item 
    (\ref{eq48}) : No pair.
    \item 
    (\ref{eq49}) : No pair.
    \item 
    (\ref{eq50}) : No pair.
    \item
    (\ref{eq51}) : No pair.
    \end{itemize}

     \textbf{Family 28:}~$(2,~0,~9,~0,~6)$ 
\begin{equation}\label{eq52}
    \begin{matrix*}[l]
         \alpha \beta ^7=ab^7, & 2\alpha \beta ^5=ab^5, & 9\alpha \beta ^3=5ab ^3, & 6\alpha\beta-1=2ab,
    \end{matrix*}
\end{equation}
\begin{equation}\label{eq53}
    \begin{matrix*}[l]
          \alpha \beta ^7=ab^7, & 2\alpha \beta ^5=ab^5, & 9\alpha \beta ^3=5ab ^3, & 6\alpha\beta-1=7ab,
    \end{matrix*}
\end{equation}
\begin{equation}\label{eq54}
    \begin{matrix*}[l]
          \alpha \beta ^7=ab^7, & 2\alpha \beta ^5=2ab^5, & 9\alpha \beta ^3=9ab ^3, & 6\alpha\beta-1=8ab,
    \end{matrix*}
\end{equation}
\begin{equation}\label{eq55}
    \begin{matrix*}[l]
          \alpha \beta ^7=ab^7, & 2\alpha \beta ^5=2ab^5, & 9\alpha \beta ^3=9ab^3, & 6\alpha\beta-1=9ab,
    \end{matrix*}
\end{equation}
\begin{equation}\label{eq56}
    \begin{matrix*}[l]
          \alpha \beta ^7=ab^7, & 2\alpha \beta ^5=ab^5 & 9\alpha \beta ^3=5ab ^3, & 6\alpha\beta-1=9ab,
    \end{matrix*}
\end{equation}
\begin{equation}\label{eq57}
    \begin{matrix*}[l]
          \alpha \beta ^7=ab^7, & 2\alpha \beta ^5=2ab^5, & 9\alpha \beta ^3=9ab ^3, & 6\alpha\beta-1=6ab.
    \end{matrix*}
\end{equation}

\begin{itemize}
    \item 
    (\ref{eq52}) : No pair.
    \item 
    (\ref{eq53}) : No pair.
    \item 
    (\ref{eq54}) : $(1,~5),~(2,~8),~(3,~9),~(4,~4),~(5,~1).$
    \item 
    (\ref{eq55}) : $1,~7),~(2,~9),~(3,~6),~(4,~10),~(5,~8).$
    \item 
    (\ref{eq56}) : No pair.
    \item
    (\ref{eq57}) : No pair.
    \end{itemize}
    
    This leads us to the conclusion that there are total $60\times 11 \times 11 =7260$ orthomorphism polynomials over $\mathbb{F}_{11}.$ Our findings for $\mathbb{F}_{11}$ also match perfectly with \cite[Table~3.2]{shallue}. Among the $7260$ polynomials obtained, $10+10=20$ are exceptional coming from Family 27 and Family 28.
    \vspace{0.5 cm}\\
\underline{\Large \textbf{$q=17$~:}}
\vspace{0.2cm}\\
The following is the list of all PPs of the form $f(x)=x^7+\sum_{i=1}^{5}a_ix^i$ over $\mathbb{F}_{17}$ up to linear transformations in terms of the families of tuples $(a_5,~a_4,~a_3,~a_2,~a_1)\in\mathbb{F}_{17}^5$ given by \Cref{l4} and \Cref{r1}.
\begin{equation*}
\begin{matrix*}[l]
     \textup{Family 1}:(0,~1,~10,~0,~6), & \textup{Family 2}:(1,~0,~6,~0,~11),\\
     \textup{Family 3}: (1,~0,~7,~0,~0),&\textup{Family 4}:(1,~0,~13,~0,~7), \\
     \textup{Family 5}:(1,~0,~13,~0,~14),& \textup{Family 6}: (1,~0,~14,~0,~3),\\
     \textup{Family 7}:(1,~3,~13,~11,~10),& \textup{Family 8}:(1,~10,~3,~14,~11),\\
 \textup{Family 9}:(3,~0,~7,~0,~4),&\textup{Family 10}:(3,~0,~10,~0,~14),\\
 \textup{Family 11}:(3,~0,~12,~0,~0), & \textup{Family 12}:(3,~0,~14,~0,~8),\\
 \textup{Family 13}:(3,~0,~15,~0,~2),& \textup{Family 14}:(3,~9,~11,~14,~10), \\
 \textup{Family 15}:(3,~9,~12,~14,~5), & \textup{Family 16}:(3,~9,~15,~14,~12),\\
 \textup{Family 17}:(3,~15,~10,~12,~4),& \textup{Family 18}:(0,~0,~0,~0,~0), \\
 \textup{Family 19}:(1,~0,~10,~0,~8),& \textup{Family 20}:(3,~0,~5,~0,~12). 
 
\end{matrix*}
\end{equation*}
\textbf{Family 4:}~$(1,~0,~13,~0,~7)$ 
\begin{equation}\label{eq58}
    \begin{matrix*}[l]
         \alpha \beta ^7=ab^7, & \alpha \beta ^5=ab^5, & 13\alpha \beta ^3=6ab ^3, & 7\alpha\beta-1=11ab,
    \end{matrix*}
\end{equation}
\begin{equation}\label{eq59}
    \begin{matrix*}[l]
          \alpha \beta ^7=ab^7, & \alpha \beta ^5=ab^5, & 13\alpha \beta ^3=7ab ^3, & 7\alpha\beta-1=0,
    \end{matrix*}
\end{equation}
\begin{equation}\label{eq60}
    \begin{matrix*}[l]
          \alpha \beta ^7=ab^7, & \alpha \beta ^5=ab^5, & 13\alpha \beta ^3=13ab ^3, & 7\alpha\beta-1=7ab,
    \end{matrix*}
\end{equation}
\begin{equation}\label{eq61}
    \begin{matrix*}[l]
          \alpha \beta ^7=ab^7, & \alpha \beta ^5=ab^5, & 13\alpha \beta ^3=13ab^3, & 7\alpha\beta-1=14ab,
    \end{matrix*}
\end{equation}
\begin{equation}\label{eq62}
    \begin{matrix*}[l]
          \alpha \beta ^7=ab^7, & \alpha \beta ^5=ab^5 & 13\alpha \beta ^3=14ab ^3, & 7\alpha\beta-1=3ab,
    \end{matrix*}
\end{equation}
\begin{equation}\label{eq63}
    \begin{matrix*}[l]
          \alpha \beta ^7=ab^7, & \alpha \beta ^5=3ab^5, & 13\alpha \beta ^3=7ab^3, & 7\alpha\beta-1=4ab,
    \end{matrix*}
\end{equation}
\begin{equation}\label{eq64}
    \begin{matrix*}[l]
          \alpha \beta ^7=ab^7, & \alpha \beta ^5=3ab^5, & 13\alpha \beta ^3=10ab ^3, & 7\alpha\beta-1=14ab,
    \end{matrix*}
\end{equation}
\begin{equation}\label{eq65}
    \begin{matrix*}[l]
          \alpha \beta ^7=ab^7, & \alpha \beta ^5=3ab^5, & 13\alpha \beta ^3=12ab ^3, & 7\alpha\beta-1=0.
    \end{matrix*}
\end{equation}
\begin{equation}\label{eq66}
    \begin{matrix*}[l]
          \alpha \beta ^7=ab^7, & \alpha \beta ^5=3ab^5, & 13\alpha \beta ^3=14ab ^3, & 7\alpha\beta-1=8ab,
    \end{matrix*}
\end{equation}
\begin{equation}\label{eq67}
    \begin{matrix*}[l]
          \alpha \beta ^7=ab^7, & \alpha \beta ^5=3ab^5, & 13\alpha \beta ^3=15ab ^3, & 7\alpha\beta-1=2ab,
    \end{matrix*}
\end{equation}
\begin{equation}\label{eq68}
    \begin{matrix*}[l]
          \alpha \beta ^7=ab^7, & \alpha \beta ^5=ab^5, & 13\alpha \beta ^3=10ab ^3, & 7\alpha\beta-1=8ab,
    \end{matrix*}
\end{equation}
\begin{equation}\label{eq69}
    \begin{matrix*}[l]
          \alpha \beta ^7=ab^7, & \alpha \beta ^5=3ab^5, & 13\alpha \beta ^3=5ab ^3, & 7\alpha\beta-1=12ab.
    \end{matrix*}
\end{equation}
\begin{itemize}
    \item 
    (\ref{eq58}) : No pair.
    \item 
    (\ref{eq59}) : No pair.
    \item 
    (\ref{eq60}) : No pair.
    \item 
    (\ref{eq61}) : $(1,~12),~(2,~6),~(3,~4),~(4,~3),~(5,~16),~(6,~2),~(7,~9),~(8,~10).$
    \item 
    (\ref{eq62}) : No pair.
    \item
    (\ref{eq63}) : No pair.
    \item 
    (\ref{eq64}) : No pair.
    \item 
    (\ref{eq65}) : No pair.
    \item 
    (\ref{eq66}) : No pair.
    \item 
    (\ref{eq67}) : No pair.
    \item 
    (\ref{eq68}) : No pair.
    \item 
    (\ref{eq69}) : No pair.
    \end{itemize}

    \textbf{Family 5:}~$(1,~0,~13,~0,~14)$ 
\begin{equation}\label{eq70}
    \begin{matrix*}[l]
         \alpha \beta ^7=ab^7, & \alpha \beta ^5=ab^5, & 13\alpha \beta ^3=6ab ^3, & 7\alpha\beta-1=11ab,
    \end{matrix*}
\end{equation}
\begin{equation}\label{eq71}
    \begin{matrix*}[l]
          \alpha \beta ^7=ab^7, & \alpha \beta ^5=ab^5, & 13\alpha \beta ^3=7ab ^3, & 7\alpha\beta-1=0,
    \end{matrix*}
\end{equation}
\begin{equation}\label{eq72}
    \begin{matrix*}[l]
          \alpha \beta ^7=ab^7, & \alpha \beta ^5=ab^5, & 13\alpha \beta ^3=13ab ^3, & 7\alpha\beta-1=7ab,
    \end{matrix*}
\end{equation}
\begin{equation}\label{eq73}
    \begin{matrix*}[l]
          \alpha \beta ^7=ab^7, & \alpha \beta ^5=ab^5, & 13\alpha \beta ^3=13ab^3, & 7\alpha\beta-1=14ab,
    \end{matrix*}
\end{equation}
\begin{equation}\label{eq74}
    \begin{matrix*}[l]
          \alpha \beta ^7=ab^7, & \alpha \beta ^5=ab^5 & 13\alpha \beta ^3=14ab ^3, & 7\alpha\beta-1=3ab,
    \end{matrix*}
\end{equation}
\begin{equation}\label{eq75}
    \begin{matrix*}[l]
          \alpha \beta ^7=ab^7, & \alpha \beta ^5=3ab^5, & 13\alpha \beta ^3=7ab^3, & 7\alpha\beta-1=4ab,
    \end{matrix*}
\end{equation}
\begin{equation}\label{eq76}
    \begin{matrix*}[l]
          \alpha \beta ^7=ab^7, & \alpha \beta ^5=3ab^5, & 13\alpha \beta ^3=10ab ^3, & 7\alpha\beta-1=14ab,
    \end{matrix*}
\end{equation}
\begin{equation}\label{eq77}
    \begin{matrix*}[l]
          \alpha \beta ^7=ab^7, & \alpha \beta ^5=3ab^5, & 13\alpha \beta ^3=12ab ^3, & 7\alpha\beta-1=0,
    \end{matrix*}
\end{equation}
\begin{equation}\label{eq78}
    \begin{matrix*}[l]
          \alpha \beta ^7=ab^7, & \alpha \beta ^5=3ab^5, & 13\alpha \beta ^3=14ab ^3, & 7\alpha\beta-1=8ab,
    \end{matrix*}
\end{equation}
\begin{equation}\label{eq79}
    \begin{matrix*}[l]
          \alpha \beta ^7=ab^7, & \alpha \beta ^5=3ab^5, & 13\alpha \beta ^3=15ab ^3, & 7\alpha\beta-1=2ab,
    \end{matrix*}
\end{equation}
\begin{equation}\label{eq80}
    \begin{matrix*}[l]
          \alpha \beta ^7=ab^7, & \alpha \beta ^5=ab^5, & 13\alpha \beta ^3=10ab ^3, & 7\alpha\beta-1=8ab,
    \end{matrix*}
\end{equation}
\begin{equation}\label{eq81}
    \begin{matrix*}[l]
          \alpha \beta ^7=ab^7, & \alpha \beta ^5=3ab^5, & 13\alpha \beta ^3=5ab ^3, & 7\alpha\beta-1=12ab.
    \end{matrix*}
\end{equation}
\begin{itemize}
    \item 
    (\ref{eq70}) : No pair.
    \item 
    (\ref{eq71}) : No pair.
    \item 
    (\ref{eq72}) : $(1,~5),~(2,~11),~(3,~13),~(4,~14),~(5,~1),~(6,~15),~(7,~8),~(8,~7).$
    \item 
    (\ref{eq73}) : No pair.
    \item 
    (\ref{eq74}) : No pair.
    \item
    (\ref{eq75}) : No pair.
    \item 
    (\ref{eq76}) : No pair.
    \item 
    (\ref{eq77}) : No pair.
    \item 
    (\ref{eq78}) : No pair.
    \item 
    (\ref{eq79}) : No pair.
    \item 
    (\ref{eq80}) : No pair.
    \item 
    (\ref{eq81}) : No pair.
    \end{itemize}
    
We find exactly $8\times 17 \times 17=4624$ orthomoprphism polynomials over $\mathbb{F}_{17},$ consistent with Shallue and Wanless's tabulated data \cite[Table~3.2]{shallue}.
 \vspace{0.5 cm}\\
\underline{\Large \textbf{$q=19$~:}}
\vspace{0.2cm}\\
Using \Cref{l4} and \Cref{r1}, the PPs of the specific form $f(x)=x^7+\sum_{i=1}^{5}a_ix^i$ over $\mathbb{F}_{19}$ up to linear transformations are presented below via the families of tuples $(a_5,~a_4,~a_3,~a_2,~a_1)\in\mathbb{F}_{19}^5.$
\begin{equation*}
\begin{matrix*}[l]
     \textup{Family 1}:(0,~0,~0,~0,~16), & \textup{Family 2}:(1,~0,~3,~14,~11),\\
     \textup{Family 3}: (1,~0,~5,~0,~4),&\textup{Family 4}:(1,~0,~7,~0,~11), \\
     \textup{Family 5}:(1,~0,~11,~0,~16),& \textup{Family 6}: (1,~0,~18,~9,~4),\\
     \textup{Family 7}:(2,~0,~14,~0,~5),& \textup{Family 8}:(2,~0,~16,~0,~9),\\
 \textup{Family 9}:(2,~0,~17,~0,~5),&\textup{Family 10}:(0,~0,~0,~0,~0),\\
 \textup{Family 11}:(1,~0,~3,~0,~7), & \textup{Family 12}:(2,~0,~12,~0,~18).\\ 
 
\end{matrix*}
\end{equation*}
\textbf{Family 1:}~$(0,~0,~0,~0,~16)$ 
\begin{equation}\label{eq82}
    \begin{matrix*}[l]
         \alpha \beta ^7=ab^7, & 16\alpha\beta-1=16ab,
    \end{matrix*}
\end{equation}
\begin{equation}\label{eq83}
    \begin{matrix*}[l]
         \alpha \beta ^7=ab^7, & 16\alpha\beta-1=0.
    \end{matrix*}
\end{equation}
\begin{itemize}
    \item 
    (\ref{eq82}) : $(2,~13),~(4,~14),~(1,~18),~(4,~16),~(2,~9),~(1,~7).$
    \item 
    (\ref{eq83}) : $(4,~11),~(1,~6),~(2,~3).$
    \end{itemize}
    \textbf{Family 10:}~$(0,~0,~0,~0,~0)$ 
\begin{equation}\label{eq84}
    \begin{matrix*}[l]
         \alpha \beta ^7=ab^7, & 18=16ab.
    \end{matrix*}
\end{equation}
\begin{itemize}
    \item 
    (\ref{eq84}) : $(1,~13),~(1,~10),~(1,~15).$
    \end{itemize}
We have found a total of $12\times 19\times 19=4332$ orthomorphism polynomials over $\mathbb{F}_{19}$ including $4$ exceptional polynomials. While Shallue et al. \cite{shallue} provided data only up to $\mathbb{F}_{17},$ our results for $\mathbb{F}_{19}$ serve as a natural extension of that tabulated data.
\vspace{0.5 cm}
\\
\underline{\Large \textbf{$q=25$~:}}
\vspace{0.2cm}
\\
As discussed in \Cref{l4} and \Cref{r1}, any polynomial over $\mathbb{F}_{25}$ is a PP over $\mathbb{F}_{25}$ if and only if it is linearly related to exactly one of the polynomials of specific form $f(x)=x^7+\sum_{i=1}^{5}a_ix^i$ over $\mathbb{F}_{25}$ with coefficients presented below via the families of tuples $(a_5,~a_4,~a_3,~a_2,~a_1)\in\mathbb{F}_{25}^5.$
\begin{equation*}
\begin{matrix*}[l]
     \textup{Family 1}:(0,~0,~0,~0,~\theta), & \textup{Family 2}:(0,~0,~0,~0,~\theta^5),\\
     \textup{Family 3}: (\theta,~0,~\theta^2,~0,~0),&\textup{Family 4}:(0,~0,~0,~0,~0), \\
     \textup{Family 5}:(1,~0,~1,~0,~4),& \textup{Family 6}: (\theta,~0,~\theta^2,~0,~4\theta^3),~~\textup{with}~\theta=\theta^2+2~\textup{in}~\mathbb{F}_{25}.\\
     \end{matrix*}
     \end{equation*}
As arithmetic in $\mathbb{F}_{25}$ differs significantly from that in prime fields such as $\mathbb{F}_{13}$, we provide the corresponding SageMath code \textbf{OP7(25)} for readers's convenience.
\newpage
 \vspace{0.5 cm}
 \hrule
 \vspace{0.2 cm}
 \textbf{OP7(25)}:
 \begin{lstlisting}
 # Define base field and extension
F5 = GF(5)
R.<x> = PolynomialRing(F5)

# Define F25 as F5[x]/(x^2 - x + 2)
K.<e> = GF(5^2, name='theta', modulus=x^2 - x + 2)

# Find all nonzero elements of F25
nonzero = [z for z in K if z != 0]

# Find theta such that theta^2 - theta+ 2 = 0
theta_candidates = [z for z in K if z^2 - z + 2 == 0]
theta = theta_candidates[0]

# Store valid solutions
solutions = []
# Brute-force all (a, b, A, B) in (F25^*)^4. We use A and B in place of alpha and beta respectively
for a in nonzero:
    for b in nonzero:
        ab7 = a * b^7
        ab = a * b
        for A in nonzero:
            for B in nonzero:
                AB7 = A * B^7
                AB = A * B
                # Check both conditions (equation 3.84)
                if AB7 == ab7 and theta* AB - 1  == theta* ab:
                    solutions.append((a, b, A, B))
                    
# Avoid repetitive polynomials                  
seen_AB = set()
final_solutions = []

for a, b, A, B in solutions:
    AB_pair = (A*B^7, A*B)
    if AB_pair not in seen_AB:
        seen_AB.add(AB_pair)
        final_solutions.append((a, b, A, B))
        
print(f"\nFinal values of (A, B):\n")
for i, (a, b, A, B) in enumerate(solutions, 1):
    print(f"{i}: (A = {A}, B = {B})")

print(f"\nTotal number of unique solutions: {len(final_solutions)}")
 \end{lstlisting}


     In what follows, we specify only the number of $(\alpha,~\beta)$ pairs associated with each family. The explicit forms of these pairs will be presented in the subsequent theorem.\\
\textbf{Family 1:}~$(0,~0,~0,~0,~\theta)$ 
\begin{equation}\label{eq85}
    \begin{matrix*}[l]
         \alpha \beta ^7=ab^7, & \theta\alpha\beta-1=\theta ab,
    \end{matrix*}
\end{equation}
\begin{equation}\label{eq86}
    \begin{matrix*}[l]
         \alpha \beta ^7=ab^7, & \theta\alpha\beta-1=\theta^5ab.
    \end{matrix*}
\end{equation}
\begin{equation}\label{eq87}
    \begin{matrix*}[l]
         \alpha \beta ^7=ab^7, & \theta\alpha\beta-1=0.
    \end{matrix*}
\end{equation}
\begin{itemize}
    \item 
    (\ref{eq85}) : $12$ pairs.
    \item 
    (\ref{eq86}) : $16$ pairs.
    \item 
    (\ref{eq87}) : $4$ pairs.
    \end{itemize}
    
   \textbf{Family 2:}~$(0,~0,~0,~0,~\theta^5)$ 
\begin{equation}\label{eq88}
    \begin{matrix*}[l]
         \alpha \beta ^7=ab^7, & \theta^5\alpha\beta-1=\theta ab,
    \end{matrix*}
\end{equation}
\begin{equation}\label{eq89}
    \begin{matrix*}[l]
         \alpha \beta ^7=ab^7, & \theta^5\alpha\beta-1=\theta^5ab.
    \end{matrix*}
\end{equation}
\begin{equation}\label{eq90}
    \begin{matrix*}[l]
         \alpha \beta ^7=ab^7, & \theta^5\alpha\beta-1=0.
    \end{matrix*}
\end{equation}
\begin{itemize}
    \item 
    (\ref{eq88}) : $16$ pairs.
    \item 
    (\ref{eq89}) : $12$ pairs.
    \item 
    (\ref{eq90}) : $4$ pairs.
    \end{itemize}
\textbf{Family 3:}~$(\theta,~0,~\theta^2,~0,~0)$ 
\begin{equation}\label{eq91}
    \begin{matrix*}[l]
         \alpha \beta ^7=ab^7, & \theta\alpha\beta^5=\theta ab^5, & \theta^2\alpha\beta^3=\theta^2ab^3, & 4=4\theta^3ab,
    \end{matrix*}
\end{equation}
\begin{equation}\label{eq92}
    \begin{matrix*}[l]
         \alpha \beta ^7=ab^7, & \theta \alpha\beta^5=ab^5, & \theta^2\alpha\beta^3=ab^3, & 4=ab,
    \end{matrix*}
\end{equation}
\begin{itemize}
    \item 
    (\ref{eq91}) : $12$ pairs.
    \item 
    (\ref{eq92}) : No pair.
    \end{itemize}
    \textbf{Family 4:}~$(0,~0,~0,~0,~0)$ 
\begin{equation}\label{eq93}
    \begin{matrix*}[l]
         \alpha \beta ^7=ab^7,  & 4=\theta ab,
    \end{matrix*}
\end{equation}
\begin{equation}\label{eq94}
    \begin{matrix*}[l]
         \alpha \beta ^7=ab^7,  & 4=\theta^5ab,
    \end{matrix*}
\end{equation}
\begin{itemize}
    \item 
    (\ref{eq93}) : $4$ pairs.
    \item 
    (\ref{eq94}) : $4$ pairs.
    \end{itemize}
\textbf{Family 5:}~$(1,~0,~1,~0,~4)$\\
$$\textup{No pairs of}~ (\alpha,~\beta)~\textup{are found for this family.}$$\\
\textbf{Family 6:}~$(\theta,~0,~\theta^2,~0,~4\theta^3)$ 
\begin{equation}\label{eq95}
    \begin{matrix*}[l]
         \alpha \beta ^7=ab^7, & \theta \alpha\beta^5=\theta ab^5, & \theta^2\alpha\beta^3=\theta^2ab^3, & 4\theta^3\alpha\beta-1=0,
    \end{matrix*}
\end{equation}
\begin{equation}\label{eq96}
    \begin{matrix*}[l]
         \alpha \beta ^7=ab^7, & \theta\alpha\beta^5=ab^5, & \theta^2\alpha\beta^3=ab^3, & 4\theta^3\alpha\beta-1=4ab,
    \end{matrix*}
\end{equation}
\begin{equation}\label{eq97}
    \begin{matrix*}[l]
         \alpha \beta ^7=ab^7, & \theta\alpha\beta^5=ab^5, & \theta^2\alpha\beta^3=ab^3, & 4\theta^3\alpha\beta-1=4\theta^3ab,
    \end{matrix*}
\end{equation}
\begin{itemize}
    \item 
    (\ref{eq95}) : $12$ pairs.
    \item 
    (\ref{eq96}) : No pair.
    \item 
    (\ref{eq97}) : No pair.
    \end{itemize}
Therefore, the total number of orthomorphism polynomials of degree $7$ over $\mathbb{F}_{25}$ is $96\times 25\times 25=60,000,$ among which $20$ are exceptional polynomials.
 \vspace{0.5 cm}\\
\underline{\Large \textbf{$q=23$}~and \textbf{$q=31$:}}
\vspace{0.2cm}\\
For both $\mathbb{F}_{23}$ and $\mathbb{F}_{31},$ no pairs of $(\alpha,~\beta)$ exist for any of their respected families, implying the non-existence of orthomorphism polynomials of degree $7$ over these fields.
We now summarize all of the preceding findings in the form of the following theorem.
\begin{theorem}\label{thm3.1}
    Let $\mathbb{F}_q$ be a finite field of characteristic $p\notin\{2,~3,~7\}$ order $q> 7.$ A non-exceptional orthomorphism polynomial $f$ of degree $7$ exists over $\mathbb{F}_q$ if and only if $q\in\{11,~13,~17,~19,~25\}.$ These polynomials, along with all the exceptional polynomials of degree $7$ over these fields are listed below. The complete collection can be obtained by taking $f(x+\gamma)+\delta,$ where $\gamma,~\delta\in\mathbb{F}_q.$
    \begin{itemize}
        \item 
        $q=11$
        \begin{enumerate}[label=\textup{(\arabic*)}]
        \item 
        $\alpha\beta^7x^7+\alpha\beta^5x^5+5\alpha\beta^3x^3+2\alpha\beta x,$ \\
        where $(\alpha,~\beta)\in\{(1,~2),~(2,~1),~(3,~8),~(4,~6),~(5,~7),~(1,~3),~(2,~7),~(3,~1),~(4,~9),\\~(5,~5)\}.$
        \item 
        $\alpha\beta^7x^7+\alpha\beta^5x^5+5\alpha\beta^3x^3+7\alpha\beta x,$\\ where $(\alpha,~\beta)\in\{(1,~9),~(2,~10),~(3,~3),~(4,~5),~(5,~4),~(1,~5),~(2,~8),~(3,~9),~(4,~4),\\~(5,~1)\}.$
        \item 
        $\alpha\beta^7x^7+2\alpha\beta^5x^5+9\alpha\beta^3x^3+8\alpha\beta x,$\\
        where $(\alpha,~\beta)\in\{(1,~10),~(2,~5),~(3,~7),~(4,~8),~(5,~2),~(1,~6),~(2,~3),~(3,~2),~(4,~7),\\~(5,~10)\}.$
        \item 
        $\alpha\beta^7x^7+2\alpha\beta^5x^5+9\alpha\beta^3x^3+9\alpha\beta x,$\\
        where $(\alpha,~\beta)\in\{(1,~1),~(2,~6),~(3,~4),~(4,~3),~(5,~9),~(1,~4),~(2,~2),~(3,~5),~(4,~1),\\~(5,~3)\}.$
        \item 
        $\alpha\beta^7x^7+\alpha\beta^5x^5+5\alpha\beta^3x^3+9\alpha\beta x,$\\
        where $(\alpha,~\beta)\in\{(1,~8),~(2,~4),~(3,~10),~(4,~2),~(5,~6),~(1,~6),~(2,~3),~(3,~2),~(4,~7),\\~(5,~10)\}.$
        \item 
        $\alpha\beta^7x^7+2\alpha\beta^5x^5+9\alpha\beta^3x^3+6\alpha\beta x,$\\
        where $(\alpha,~\beta)\in\{(1,~5),~(2,~8),~(3,~9),~(4,~4),~(5,~1),~(1,~7),~(2,~9),~(3,~6),~(4,~10),\\~(5,~8)\}.$
    \end{enumerate}
    \item 
    $q=13$
    \begin{enumerate}[label=\textup{(\arabic*)}]
     \item 
        $\alpha\beta^7x^7+2\alpha\beta x,$\\
        where $(\alpha,~\beta)\in\{(2,~5),~(1,~10),~(1,~3),~(1,~5),~(2,~9),~(2,~8),~(2,~10),~(1,~7)\}.$
         \item 
        $\alpha\beta^7x^7+6\alpha\beta x,$\\
        where $(\alpha,~\beta)\in\{(5,~1),~(1,~10),~(1,~5),~(2,~5),~(2,~6),~(1,~12),~(2,~12),~(1,~11)\}.$
         \item 
        $\alpha\beta^7x^7+2\alpha\beta^3 x^3+8\alpha\beta x,$\\
        where $(\alpha,~\beta)\in\{(5,~7),~(3,~3),~(2,~11),~(6,~8),~(1,~9),~(4,~12)\}.$
        \item 
        $\alpha\beta^7x^7+4\alpha\beta^3 x^3+4\alpha\beta x,$\\
        where $(\alpha,~\beta)\in\{(5,~1),~(3,~6),~(2,~9),~(6,~3),~(1,~5),~(4,~11)\}.$
        \item 
        $\alpha\beta^7x^7+8\alpha\beta^3 x^3+3\alpha\beta x,$\\
        where $(\alpha,~\beta)\in\{(5,~10),~(3,~8),~(2,~12),~(6,~4),~(1,~11),~(4,~6)\}.$
        \item
        $\alpha\beta^7x^7,$\\
        where $(\alpha,~\beta)\in\{(1,~6),~(1,~7),~(1,~2),~(1,~11)\}.$
    \end{enumerate}
    \item
    $q=17$
    \begin{enumerate}[label=\textup{(\arabic*)}]
         \item 
        $\alpha\beta^7x^7+\alpha\beta^5 x^5+13\alpha\beta^3x^3+7\alpha\beta x,$\\
        where $(\alpha,~\beta)\in\{(1,~12),~(2,~6),~(3,~4),~(4,~3),~(5,~16),~(6,~2),~(7,~9),~(8,~10)\}.$
        \item 
        $\alpha\beta^7x^7+\alpha\beta^5 x^5+13\alpha\beta^3x^3+14\alpha\beta x,$\\
        where $(\alpha,~\beta)\in\{(1,~5),~(2,~11),~(3,~13),~(4,~14),~(5,~1),~(6,~15),~(7,~8),~(8,~7)\}.$
        \end{enumerate}
         \item
    $q=19$
    \begin{enumerate}[label=\textup{(\arabic*)}]
         \item 
        $\alpha\beta^7x^7+16\alpha\beta x,$\\
        where $(\alpha,~\beta)\in\{(2,~13),~(4,~14),~(1,~18),~(4,~16),~(2,~9),~(1,~7),~(4,~11),~(1,~6),\\~(2,~3)\}.$
        \item 
        $\alpha\beta^7x^7,$\\
        where $(\alpha,~\beta)\in\{(1,~13),~(1,~10),~(1,~15)\}.$
        \end{enumerate}
        \item 
        $q=25$
        \begin{enumerate}[label=\textup{(\arabic*)}]
        \item 
        $\alpha\beta^7x^7+\theta\alpha\beta x,$\\
        where $(\alpha,~\beta)\in\{(4\theta+3,~4\theta+2),~(\theta+3,,~3\theta+1),~(2\theta+2,~\theta),~(2\theta+2,~4\theta),\\~(4\theta+3,~2\theta+1),~(\theta,~2\theta+2),~(\theta,~3\theta+3),~(2\theta+2,~2\theta),~(\theta+3,~4\theta+3),\\~(\theta+3,~\theta+2),~(\theta,~4\theta+4),~(4\theta+3,~\theta+3),~(\theta,~\theta+3),~(2\theta+2,~\theta),\\~(4\theta+3,~2\theta+4),~(\theta+3,~2),~(\theta+3,~\theta),~(\theta,~3\theta+1),~(2\theta+2,~3\theta+4),\\~(4\theta+3,~4\theta+1),~(\theta+3,~2\theta+1),~(\theta,~2\theta+3),~(2\theta+2,~2\theta+2),~(4\theta+3,~1),\\~(\theta+3,~\theta+1),~(\theta,~2\theta),~(2\theta+2,~2\theta+3),~(4\theta+3,~4\theta),~(\theta+3,~2\theta+4),\\~(4\theta+3,~3\theta+4),~(2\theta+2,~3\theta),~(\theta,~\theta+1)\}.$
            \item 
            $\alpha\beta^7x^7+\theta^5\alpha\beta x,$\\
            where $(\alpha,~\beta)\in\{(4\theta+3,~2\theta+4),~(\theta,,~4\theta+2),~(2\theta+2,~\theta+1),\\~(\theta+3,~4\theta+2),~(2\theta+2,~3\theta+4),~(\theta+3,~4\theta),~(\theta,~\theta),~(4\theta+3,~4\theta),~(\theta,~2\theta+3),\\~(4\theta+3,~4),~(\theta+3,~1),~(2\theta+2,~4),~(\theta+3,~\theta+1),~(2\theta+2,~3\theta+2),\\(4\theta+3,~2\theta+3),~(\theta,~\theta+2),~(\theta+3,~4\theta+1),~(2\theta+2,~2\theta+4),\\(4\theta+3,~4\theta+4),~(4\theta+3,~2\theta+2),~(\theta,~1),~(2\theta+2,~3\theta+1),~(2\theta+2,~4\theta+3),\\(\theta+3,~2\theta+3),~(\theta,~4),~(\theta,~2),~(4\theta+3,~\theta+1),~(\theta+3,~3\theta+2),\\(4\theta+3,~3\theta+3),~(2\theta+2,~\theta+2),~(\theta,~3),~(\theta+3,~\theta+4)\}.$
            \item 
            $\alpha\beta^7x^7+\theta\alpha\beta^5x^5+\theta^2\alpha\beta^3 x^3,$\\
            where $(\alpha,~\beta)\in\{(\theta,~3\theta+4),~(\theta+3,~3\theta),~(4\theta+3,~3),~(2\theta+2,~\theta+4),\\(4\theta+1,~3\theta+3),~(2,~\theta+2),~(2\theta,~4\theta+2),~(2\theta+1,~4\theta),~(3\theta+1,~4),\\(4\theta+4,~3\theta+2),~(3\theta+2,~4\theta+4),~(4,~3\theta+1)\}.$
            \item 
            $\alpha\beta^7x^7,$\\
            where $(\alpha,~\beta)\in\{(\theta,~4\theta+4),~(\theta,~3\theta+3),~(\theta,~\theta+1),~(\theta,~2\theta+2),~(\theta,~2),~(\theta,~4),~(\theta,~3),\\(\theta,~1)\}.$
            \item $\alpha\beta^7x^7+\theta\alpha\beta^5x^5+\theta^2\alpha\beta^3x^3w+4\theta^3\alpha\beta x,$\\
            where $(\alpha,~\beta)\in\{(\theta,~2\theta+1),~(\theta+3,~2\theta),~(4\theta+3,~2),~(2\theta+2,~4\theta+1),\\(4\theta+1,~2\theta+2),~(2,~4\theta+3),~(2\theta,~\theta+3),~(2\theta+1,~\theta),~(3\theta+1,~1),\\(4\theta+4,~2\theta+3),~(3\theta+2,~\theta+1),~(4,~2\theta+4)\},$
            \end{enumerate}
            with $\theta=\theta^2+2$ in $\mathbb{F}_{25}.$
    \end{itemize}
    
\end{theorem}

\begin{corollary}
    Any orthomorphism polynomial over $\mathbb{F}_{17}$ is non-exceptional.
\end{corollary}
\begin{proof}
    It follows from the fact that a linear transformation of a non-exceptional polynomial remains non-exceptional.
\end{proof}
\begin{theorem}
    There does not exist any orthomorphism polynomial over $\mathbb{F}_{23}$ and $\mathbb{F}_{31}.$
\end{theorem}
\begin{remark}\label{r3}
Shallue et al. \cite{shallue}  established that no orthomorphism polynomials of degree $7$ exist over fields of even characteristics. The same can be proven by employing our technique applied for earlier values of $q$ together with the known results of Li et al. \cite[Theorem~4.4]{li} on degree $7$ PPs over fields of characteristic $2.$
\end{remark}

To classify orthomorphism polynomials of degree $7$ over fields of characteristic $3$ and $7$ it suffices to consider fields of order greater than or equal to $27$ and $49$ respectively. The following theorem from \cite{xfan} provides a classification of PPs over these fields.
\begin{theorem}\cite[Theorem~14,  Theorem~16]{xfan} \label{thmm}(1) Each PP over degree $7$ over $\mathbb{F}_{27}$ is linearly related to either exceptional $x^7$ or non-exceptional $x^7-x^3+x.$ If $k\geq 4,$ all PPs of degree $7$ over $\mathbb{F}_{3^k}$ are exceptional.\\
(2) Let $\theta$ be a root of $x^2+6x+3$ in $\mathbb{F}_{49}.$ All non-exceptional PPs of degree $7$  over $\mathbb{F}_{49}$ are linearly related to $$x^7+\theta x^5+2\theta^2x^3+4\theta^3x$$ and each exceptional polynomial of degree $7$ over $\mathbb{F}_{49}$ is linearly related to exactly one of the following polynomials:
\begin{itemize}
    \item 
    $x^7+ax$ with $a\in\{0,~\theta,~\theta^2,~\theta^3,~\theta^4,~\theta^5\},$
    \item 
    $x^7+x^4+2x,$
    \item 
    $x^7+\theta^2x^4+2\theta^4x,$ and
    \item 
    $x^7+\theta x^5+\theta+5\theta^2x^3+6\theta^3x.$
\end{itemize}
(3) Each PP over degree $7$ over $\mathbb{F}_{7^k},~k\geq 3$ is exceptional.
\end{theorem}
Using \Cref{thmm} we follow a similar approach as used earlier to determine the existence of orthomorphism polynomials of degree $7$ over $\mathbb{F}_{27}$ and $\mathbb{F}_{49}$ to obtain the following results.
\begin{corollary}\label{cor3.5}
    (1) There does not exist any orthomorphism polynomial over $\mathbb{F}_{27}.$\\
    (2) Any orthomorphism polynomial over $\mathbb{F}_{49}$ and hence over $\mathbb{F}_{7^t},~t\geq 2$ is exceptional and there are a total of $3,937,640$ orthomorphism polynomials of degree $7$ over $\mathbb{F}_{49}.$
    \end{corollary}
    All such orthomorphism polynomials over $\mathbb{F}_{49}$ are of the form $f(x+\gamma)+\delta$ with $\gamma,~\delta \in \mathbb{F}_{49}$;~$f(x)=\alpha\beta^7x^7+a\theta \alpha\beta x,$ where $\theta$ is a root of $x^2+6x+3$ in $\mathbb{F}_{49},~a\in\{0,~\theta,~\theta^2,~\theta^3,~\theta^4,~\theta^5\}$ and $(\alpha,~\beta)$ takes specific values in $\mathbb{F}_{49}^2$ that can be obtained by the method we previously used for $\mathbb{F}_{25}.$ The total number of pairs of $(\alpha,~\beta)$ is $40$ when $a=0$ and $320$ for each of the values of $a=\theta,~\theta^2,~\theta^3,~\theta^4$ and $\theta^5.$ Owing to the large number of $(\alpha,~\beta)$ pairs ($1640$ in total), we list only those corresponding to $a=0$ for illustration and they are;\\
     $(\theta,~4\theta+1),~(\theta,~5\theta,+1),~(\theta,~4\theta+2),~(\theta,~5),~(\theta,~3\theta+6),~(\theta,~2\theta+6),~(\theta,~3\theta+5),~(\theta,~2),\\(\theta,~2\theta+2),~(\theta,~2\theta+3),~(\theta,~4\theta),~(\theta,~3\theta+4),~(\theta,~5\theta+5),~(\theta,~5\theta+4),~(\theta,~3\theta),\\(\theta,~4\theta+3),~(\theta, ~4\theta+ 5),~(\theta, ~ 6\theta + 3),~ (\theta,~ 4),~ (\theta,~ \theta+ 2),~(\theta, ~3\theta+ 2),~(\theta, ~\theta+ 4),~(\theta,~ 3),\\(\theta, ~ 6\theta + 5),~(\theta,~ 3\theta+ 1),~(\theta,~ 6\theta),~(\theta, ~\theta+ 6),~(\theta,~ 4\theta + 4),~(\theta, ~4\theta+ 6),~(\theta,~ \theta),~(\theta, ~ 6\theta + 1),\\(\theta,~ 3\theta+ 3),~(\theta,~ 2\theta+ 1),~(\theta,~ 6),~(\theta, ~5\theta+ 3),~(\theta,~ \theta+3),~(\theta, ~5\theta+ 6),~(\theta,~ 1),~(\theta,~ 2\theta+ 4),~(\theta,~6\theta + 4),$ where $\theta^2+6\theta+3=0$ in $\mathbb{F}_{49}.$


\Cref{r3} and \Cref{cor3.5} together yield the following broader version of \Cref{thm3.1}.
\begin{theorem}
    For any finite field $\mathbb{F}_q$ of order greater than $7,$ a non-exceptional orthomorphism polynomial of degree $7$ exists if and only if $q\in\{11,~13,~17,~19,~25\}.$
\end{theorem}
\noindent
The following theorem also appears in \cite{xfan}.
\begin{theorem}
    Let $\mathbb{F}_q$ be a finite field, where $q\equiv 6 ~(\textup{mod}~7)$ and $q\notin\{13,~27\}.$ Then any polynomial $f$ of degree $7$ is a PP over $\mathbb{F}_q$ if and only if it is linearly related to $x^7.$
\end{theorem}
\begin{corollary}
    No orthomorphism polynomials exist over $\mathbb{F}_q;$ where $q\equiv 6 ~(\textup{mod}~7),~q\neq 13.$
\end{corollary}
\section{Conclusion}
We have determined all orthomorphism polynomials of degree $7$ over finite fields of order $ q\in\{11,~17,~19,~23,~25,~27,~31,~~49\}$ and $q\equiv 6~(\textup{mod}~7),$ thereby extending the existing classification of such polynomials \cite{CPP1, shallue}. As a direction for future work, the remaining orders can be explored. According to \cite{xfan}, to complete the classification of all orthomorphism polynomials of degree $7$, it suffices to examine the finite fields $\mathbb{F}_{7^r},~r\geq 3$ and $\mathbb{F}_q$ with $q> 31,~q\not\equiv 0,~1,~6 ~(\textup{mod}~7).$ For all these cases, only exceptional polynomials will be obtained.

\begin{thebibliography}{999}
\bibitem{dickson1}
Dickson, L. E. The analytic representation of substitutions on a power of a prime number of letters with a discussion of the linear group. \emph{Annals of Mathematics}, 11(1):65-120, 1896/97.
\bibitem{dickson2}
Dickson, L. E. \emph{Linear Groups}. Dover, New york, 1958.
\bibitem{evans}
Evans, A. B. \emph{Orthogonal Latin Squares Based on Groups}. Developments in Mathematics, vol. 57, Springer, Cham, Switzerland, 2018.
\bibitem{evans2}
Evans, A. B. \emph{Orthomorphism graphs of groups}. Springer, 1992.
\bibitem{xfan}
 Fan, X. A classification of permutation polynomials of degree $7$ over finite fields. \emph{Finite Fields and Their Applications}, 59:1-21, 2019.
 \bibitem{li}
 Li, J., Chandler, D. B., and Xiang Q. Permutation polynomials of degree 6 or 7 over finite fields of characteristic 2. \emph{Finite Fields and Their Application}, 16:406-419, 2010.
 \bibitem{muller}
M\"{u}ller, P. A Weil-bound free proof of Schur's conjecture. \emph{Finite Fields and Their Applications}, 3:25-32, 1997.
\bibitem{CPP1}
Neiderreiter, H., and Robinson, K. H. Complete Mappings of Finite Fields. \emph{Journal of the Australian Mathematical Society} 33(2):197-212, 1982.
\bibitem{winterhof}
Shaheen, R., and Winterhof, A. Permutations of finite fields for check digit systems. \emph{Designs, Codes and Cryptograohy}, 57:361-371, 2010.
\bibitem{shallue}
Shallue, C. J., and  Wanless, I. M. Permutation polynomials and orthomorphism polynomials of degree six. \emph{Finite Fields and Their Applications}, 20:84-92, 2013.
\bibitem{wanless2}
Stones, D. S.,and  Wanless, I. M. Compound orthomorphism of the cyclic group. \emph{Finite Fields and Their Applications}, 16:277-289, 2010.
\bibitem{wanless3}
Wanless, I. M. Atomic Latin Squares based on Cyclotomic Orthomorphisms. \emph{The Electronic Journal of Combinatorics}, 12:R22, 2005.
\bibitem{wanless}
Wanless, I. A. Diagonally cyclic latin squares. \emph{European Journal of Combinatorics}, 25:393-413, 2004.
\end{thebibliography}






\end{document}
