\noindent We continue in the setting of Section \ref{trace formula} with $G=\GL(n)$ and $K_f=K(N)$. It is clear from the definition that to understand analytic torsion, we must analyze the terms
\begin{align}\label{mellin}
    \FP_{s=0}\left(\frac{1}{s\Gamma(s)}\int_0^\infty \Trreg(e^{-t\Delta_p(\tau)})t^{s-1}dt\right),
\end{align}
where we by the integral in fact means its meromorphic continuation to all of $\C$ as a function of $s$. As $\Gamma(s)$ has a simple pole at $s=0$ with residue $1$, we see that $\frac{1}{s\Gamma(s)}$ is holomorphic at $s=0$ with value $1$. Thus, we are reduced to examining the meromorphic continuation of the Mellin transform of $\Trreg(e^{-t\Delta_p(\tau)})$.

From the standard theory of Mellin transforms (see e.g. \cite{Zagier}), we can understand the meromorphic continuation of the Mellin transform of a function $f$ by giving asymptotics of $f(t)$ for $t\to 0$ and $t\to\infty$. More precisely, it is sufficient to establish the following:
\begin{align}\label{largef}
    f(t) &= O(e^{-ct}) \qquad\qquad\qquad\qquad\qquad\qquad \text{as }t\to\infty \\\label{smalltf}
    f(t) &= \sum_{i=0}^B\sum_{j=0}^{r_i}c_{ij}t^{\alpha_i}(\log t)^j +O(t^{\alpha_{B+1}}) \qquad \text{as }t\to 0
\end{align}
for $(\alpha_i)_{i\in\N_0}$ a sequence of real, possibly negative numbers with $\alpha_i<\alpha_{i+1}$ and tending to $+\infty$, and $c>0$. Then we know that the Mellin transform of $f$ converges in some half plane and has a meromorphic continuation to all of $\C$. Furthermore, its residues at its poles can be described in terms of the coefficients $c_{i,j}$. In the following, we present such asymptotics in the different settings we will need for our proof.


\subsection{Large $t$ asymptotic of the spectral side}

During the proof of the main theorem, we will need to control large $t$ behaviour of the entire trace formula at once to control the error term incurred from truncating the the Mellin transform. This was done in (\cite{MzM2}, Corollary $6.7$), using the fine spectral expansion of the spectral side of the Arthur trace formula and results on logarithmic derivatives of intertwining operators. The result is
\begin{align}\label{larget}
    \left\vert J_{\text{spec}}(h_t^{\tau,p}\otimes \chi_{K(N)})\right\vert \leq Ce^{-c t}\vol(Y(N))
\end{align}
for some $C,c>0$, for all $t\geq 1$, $p=0,\dots,n$ and $N\in\N$.
\begin{rmk}
    Going carefully through the proof of the above result in (\cite{MzM2}), one sees that we may pick $c=\lambda(1-\epsilon)$ for any $0<\epsilon<1$, where $\lambda$ is the spectral gap guaranteed by assuming $\tau$ is $\lambda$-strongly acyclic. Indeed, in  (\cite{MzM2}, $(6.20)$) they pick $c=\frac{\lambda}{2}$, but the proof works for any multiple of $\lambda$ with a factor less than $1$.  This result is needed in order to allow the swap from our standard test function to its compactification.
\end{rmk} 

\subsection{Small $t$ asymptotics of orbital integrals}

By the definition of the regularized trace of the heat operator, along with Proposition \ref{compactunip} and the fine geometric expansion (\ref{finegeom}), once we have switched to a compactified test function it is sufficient to establish the above asymptotics for the weighted orbital integrals. By the decomposition into archimedean and non-archimedean part (\ref{localparts}), as the non-archimedean part does not depend on the variable $t$ in our case, we are reduced to analyzing the archimedean parts.

The desired small-$t$ asymptotics (\ref{smalltf}) for our archimedean weighted orbital integrals given in (\ref{inforbit}) were shown in (\cite{MzM1}). Combining (\cite{MzM1}, Proposition $12.3$) and (\cite{MzM1}, $(13.14)$) we get
\begin{prop}\label{orbitsmallt}
Let $M\in \mathcal{L}$ and $\mf{u}\in \mathcal{U}_M(\Q)$ with $(M,\mf{u})\neq (G,\lbrace 1\rbrace)$. For every $N\geq 3$, there is an expansion
    \begin{align*}
        J^G_M((h_{t,R}^{\tau,p})_Q,\mf{u}) = t^{-(d-k)/2}\sum_{j=0}^N\sum^{r_j}_{i=0}c_{ij}(\tau,p)t^{j/2}(\log t)^i+O(t^{(N-d+k+1)/2})
    \end{align*}
    as $t\to 0^+$.
\end{prop}
\noindent Here $k$ is the dimension of the Lie algebra of $V(\R)$, and $c_{ij}(\tau,p)$ are certain coefficients depending only on $i,j,\tau,p$. Note that the results (\cite{MzM1}, Proposition $12.3$) is stated for $M\neq G$, but their proof holds in the more general case above without modification. Furthermore, it is sufficient for our purposes to state the result for $L=G$ as above, since every Levi subgroup $L$ of $\GL(n)$ is canonically isomorphic to a finite direct product of $\GL(m)$'s, $m\leq n$, and the orbital integral splits accordingly.

\begin{rmk}\label{independentR}
    The proof of (\cite{MzM1}, Proposition $12.3$) does not utilize the fact that the support of the test function is compactified, i.e. their proof holds when replacing $h_{t,R}^{\tau,p}$ by $h_t^{\tau,p}$ - this can be seen in the first inequality of Section $12.2$. In particular, neither the coefficients nor the implied constants in the error term depend on the radius of compactification $R$. 
\end{rmk}


\subsection{Large $t$ asymptotics of orbital integrals}


Using Proposition \ref{traceheatdecay}, we now show exponential decay as $t\to\infty$ of the orbital integrals appearing in (\ref{inforbit}). Simply inserting the bound of the proposition into (\ref{inforbit}), we get
\begin{align*}
    \left\vert\int_{V(\R)}h^{\tau,p}_{t,R}(v)\omega(v)dv\right\vert \leq C\,e^{-\lambda t}\left\vert\int_{V(\R)\cap B_R}\omega(v)dv\right\vert.
\end{align*}
As a consequence of log-homogeneity (\cite{MzM1}, Proposition $7.1$), the weight function $\omega(v)$ is bounded by a polynomial of uniformly bounded degree in powers of $\log||v||$, and thus by applying Lemma \ref{distancebound} we have a bound on the above latter integral of the form
\begin{align*}
    \left\vert\int_{V(\R)\cap B_R}\omega(v)dv\right\vert \leq C \vol(B(R)) (\log R)^b \leq C' R^a(\log R)^b.
\end{align*}
This proves the following result.
\begin{prop}\label{orbitasymp}
    Assume $t\geq 1$. There exists constants $a,b$ only depending on the group $G$ and a constant $D_{\tau,p}>0$ only depending on $\tau$ and $p$ such that
    \begin{align*}
        \left\vert\int_{V(\R)}h^{\tau,p}_{t,R}(v)\omega(v)dv\right\vert \leq D_{\tau,p} \,e^{-\lambda t}R^a(\log R)^b.
    \end{align*}
\end{prop}
\noindent This gives us the desired large $t$ asymptotics (\ref{largef}). We will put this to work in the following section.

