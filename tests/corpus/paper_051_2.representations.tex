\noindent We continue with the setup from Section \ref{heatker}. In particular, $G$ is a connected semisimple algebraic group. In the setting that $\Gamma\backslash\Tilde{X}$ is compact, (\cite{BV}, §$4$) defines an irreducible finite-dimensional representation $(\tau,V_\tau)$ of $G(\R)$ to be \textit{strongly acyclic} if there exists some positive constant $\eta>0$ such that every eigenvalue of $\Delta_p(\tau)$ is $\geq \eta$ for every choice of $\Gamma$ and every $p$. Furthermore, they show that for any such $\tau$ not fixed by the Cartan involution, then $\tau$ must be strongly acyclic. This latter condition is shown to give the necessary bounds in the noncompact setting in (\cite{MzM2}, Lemma $6.1$), and is used throughout that paper. 

For our purposes, we need a slightly stronger condition that we define now. This will be the central object of this section.
\begin{defn}
    Let $(\tau,V_\tau)$ be a finite-dimensional representation of $G(\R)$, and let $\lambda>0$. We say that $\tau$ is $\lambda$\textit{-strongly acyclic} if
    \begin{align*}
        \tau(\Omega)-\pi(\Omega) \geq \lambda
    \end{align*}
    for every irreducible unitary representation $\pi$ satisfying $\Hom_K(\Lambda^p\mf{p}\otimes V_\tau^*,\pi)\neq 0$, for some $p$.
\end{defn}
\noindent We see the analogy to the definition of strongly acyclic by comparing to (\ref{casimirglobal}). For $\tau$ a $\lambda$-strongly acyclic representation, we will sometimes say that $\lambda$ is the \textit{spectral gap} of $\tau$. We now prove that there are plenty of such representations to choose from.

\begin{prop}\label{kstrong}
    For any $\lambda>0$, there exists infinitely many $\lambda$-strongly acyclic representations of $G(\R)$ if $\delta(G(\R))\geq 1$.
\end{prop}

\noindent The proof takes up the remainder of the section. Let $F=V_\tau^*$ be the dual of the representation space. Consider $\mathfrak{g}\coloneqq \text{Lie}(G(\R))$ with $\mf{g}_\C$ its complexification. Recall that $\theta$ is a Cartan involution on $\mf{g}$ with Cartan decomposition $\mf{g}=\mf{k}\oplus\mf{p}$. This also gives the decomposition $\mf{g}_\C=\mf{k}_\C\oplus\mf{p}_\C$. Let $\mf{h}^+$ be a Cartan subalgebra in $\mf{k}$. Denote by $\mf{h}$ the centralizer of $\mf{h}^+$ in $\mf{g}$, which is then a Cartan subalgebra in $\mf{g}$.

The following two lemmas are elementary and certainly well known, but we were not able to find precise references. Hence, we will give the proofs.

\begin{lem}\label{decomp}
    There exists an abelian subalgebra $\mf{h}^-\subset \mf{p}$ such that $\mf{h}=\mf{h}^+\oplus \mf{h}^-$.
\end{lem}

\begin{proof}
    Take any basis $(K_1,\dots,K_d)$ of $\mf{h}^+$, and extend it to a basis 
    $$(K_1,\dots, K_d,X_1,\dots,X_s)$$
    of $\mf{h}$. Using the Cartan decomposition, write $X_i = M_i+P_i$ for $M_i\in \mf{k}$ and $P_i\in \mf{p}$. Since $[K,K']\in \mf{k}$ and $[P,K]\in \mf{p}$ for any $\lambda,K'\in\mf{k}$ and $P\in\mf{p}$, and $\mf{k}\cap \mf{p}=0$, we see that $[M_i,K_j]+[P_i,K_j] = [X_i,K_j]=0$ for all $i,j$ implies that $[M_i,K_j]=[P_i,K_j]=0$. In particular, $M_i\in \mf{k}$ commutes with all of $\mf{h}^+$, but as this is a \textit{maximal} abelian subalgebra of $\mf{k}$, we get $M_i\in \mf{h}^+$. Thus, it becomes clear that 
    $$\mf{h} = \text{span}\lbrace K_1,\dots, K_d,P_1,\dots,P_s\rbrace.$$
\end{proof}


\noindent Let $\Phi_k$ be the roots of $\mf{h}^+_\C$ in $\mf{k}_\C$, and let $\Phi$ be the roots of $\mf{h}_\C$ in $\mf{g}_\C$. Pick $\Phi^+_k$ a set of positive roots of $\Phi_k$, and let $\Phi^+$ be a \textit{compatible} choice of positive roots of $\Phi$ (see \cite{BW}, II, §$6.6$). A compatible root system $\Phi^+_k$ is defined as a root system which is closed under the Cartan involution (acting by precomposition), and every element in $\Phi^+_k$ is the restriction of some element in $\Phi^+$ to $\mf{h}^+_\C$. Let $\rho$ and $\rho_k$ be the respective half-sum of positive roots. Set $W$ to be the Weyl group of $\mf{g}_\C$, and let $W^1$ be the subset of elements $w$ such that $w\Phi^+$ is again a compatible system of positive roots.

We will be concerned with the weight lattice of $\mf{g}_\C$, which lives inside the real span of the roots, $\sp_\R \Phi$. To utilize our decomposition above, we need the following elementary lemma. We extend $(\mf{h}^+)^*$ to a subspace of $\mf{h}^*$ be setting elements to be identically zero on $\mf{h}^-$, and extend $(\mf{h}^-)^*$ by setting elements identically zero on $\mf{h}^+$. We further identify these as (real) subspaces of $\mf{h}^*_\C$ using that $\mf{h}^*_\C= \mf{h}^*\oplus i\mf{h}^*$. 

\begin{lem}\label{dualdecomp}
    We have a decomposition $\sp_\R \Phi = i(\mf{h}^+)^*\oplus (\mf{h}^-)^*$. The Killing form induces a $W$- and $\theta$-invariant inner product on this space such that the decomposition is orthogonal.
\end{lem}

\begin{proof}
    Define $\mf{h}_0^* \coloneqq \sp_\R \Phi$. Given $H\in \mf{h}$, note that the values $\phi(H)$ for $\phi\in\Phi\sqcup \lbrace 0 \rbrace$ are exactly the eigenvalues of $\ad_H$, by definition. Also, it is an elementary fact that for any element $X$ in a compact subalgebra, all eigenvalues of $\ad_X$ are imaginary -- one could argue as follows: Its Lie group is compact, thus its action on the Lie algebra by conjugation has eigenvalues of absolute value $1$. Now take logarithms. 

As $\mf{k}$ is a compact Lie subalgebra of $\mf{g}_\C$, the above implies that for $K\in\mf{h}^+$ and $\phi\in\Phi_k$ we have $\phi(K)\in i\R$. Thus, the roots must take values in $\R$ on $i\mf{h}^+$. Similarly, as $i\mf{p}$ is a subspace of the compact real form $\mf{u}=\mf{k}\oplus i\mf{p}$ of $\mf{g}_\C$, the roots must take values in $i\R$ on $i\mf{p}$, and thus values in $\R$ on $\mf{p}$. This proves the inclusion $\mf{h}_0^* \subset i(\mf{h}^+)^*\oplus (\mf{h}^-)^*$, and equality follows from comparing dimensions.

To prove the second statement, recall that the Killing form $B$ on $\mf{g}_\C$ is negative definite on $\mf{k}$, thus positive definite on $i\mf{k}$, and positive definite on $\mf{p}$. Hence, it is positive definite on $i\mf{k}\oplus \mf{p}$, giving an inner product on the space, and in particular also on the subspace $i\mf{h}^+\oplus \mf{h}^-$. Using this to identify $i\mf{h}^+\oplus \mf{h}^-$ with its dual, we get an inner product on $\mf{h}_0^*$. It is well known that the Killing form is invariant under any $\mf{g}_\C$-automorphism, and in particular it is invariant under the Weyl group $W$ and the Cartan involution $\theta$. Finally, the fact that $\mf{k}\oplus \mf{p}$ is an orthogonal decomposition under the Killing form immediately implies the same for $i\mf{k}\oplus \mf{p}$, which also translates to its dual.
\end{proof}

\noindent As this may be a nonstandard choice of positive root systems to the reader, let us consider an example.

\begin{exmp}
    Consider $G(\R) = \SL(3,\R)$ with the usual Cartan involution on its Lie algebra $\theta(X) =-X^t$ and Cartan decomposition $\mf{g}=\mf{k}\oplus\mf{p}$ into skew-symmetric and symmetric traceless real $3\times 3$ matrices. Let
    \begin{align*}
        K=\begin{pmatrix}
            &1& \\
            -1&& \\
            &&
        \end{pmatrix}
        \:,\quad P = \begin{pmatrix}
            1 & & \\
            & 1 & \\
            & & -2
        \end{pmatrix}.
    \end{align*}
    \noindent Then $\mf{h}\coloneqq \sp_\R\lbrace K \rbrace$ is a Cartan subalgebra of $\mf{k}$, and its centralizer is $\mf{h} = \sp_\R\lbrace K,P\rbrace$. Complexifying, we have $\mf{h}_\C = \sp_\C\lbrace K,P\rbrace$. One can check that the adjoint action of $K$ on $\mf{k}_\C$ has eigenvalues $0,i,-i$ with respective eigenspaces
\begin{align*}
    \mf{h}^+_\C, \quad \sp_\C  \begin{pmatrix}
        & & 1\\
        & & i\\
        -1&-i&
    \end{pmatrix}, \quad 
    \sp_\C  \begin{pmatrix}
        & & 1\\
        & & -i\\
        -1&i&
    \end{pmatrix} .
\end{align*}
In particular, we may choose a positive root system $\Phi^+_k\coloneqq \lbrace \phi\rbrace$ for $\mf{h}^+_\C$ with $\phi:\mf{h}^+_\C\to \C$ given by $\phi(K) = i$. 

Consider now the dual space $\mf{h}^*_\C$ and the dual basis $\lbrace K^*,P^*\rbrace$ defined by $K^*(K) = 1$, $K^*(P) = 0$ and analogously for $P^*$. One can then check that the six roots of $\mf{h}_\C$, in terms of this basis, are $\pm( 2i,0), \pm( i,3)$ and $\pm( i,-3)$. As the Cartan involution acts by precomposition, we see that the dual basis inherits the action of the Cartan involution from the original basis, i.e. $\theta(K^*) = K^*$ and $\theta(P^*)=-P^*$.

In particular, the action of the Cartan involution on a vector written in the dual basis is $(a,b)\mapsto (a,-b)$. We then have exactly one choice of a compatible positive root system for $\mf{h}$: As it must restrict to $\Phi_k^+$, we must pick $(i,3)$ or $(i,-3)$, and as we have to be closed under the Cartan involution we have to pick both. Thus, by the axioms of positive root systems, we see that $\Phi^+ = \lbrace (i,3),(i,-3),(2i,0)\rbrace$.     
\end{exmp}


\noindent We return to the general setting. As the adjoint action of $\mf{k}$ preserves the Killing form on $\mf{p}$, which defines $\mf{so}(\mf{p})$, we get a natural map $\mf{k}\to\mf{so}(\mf{p})$, and thus, we get an induced representation of $\mf{k}$ on $S\coloneqq\text{Spin}(\mf{p})$. In the proof of (\cite{BV}, Lemma $4.1$), it is noted that every highest weight of $F\otimes S$ is of the form $\frac12(\mu+\theta\mu)+w\rho-\rho_k$ for some weight $\mu$ of $F$ and some $w\in W^1$. We use this to give a more precise statement.

\begin{lem}\label{weight}
    Any highest weight of $F\otimes S$ is always of the form $\frac12(\nu+\theta\nu)+w\rho-\rho_k$ for $\nu$ the \textit{highest} weight of $F$, and some $w\in W^1$.
\end{lem}

\begin{proof}
    Let $\frac12(\mu_0+\theta\mu_0)+w_0\rho-\rho_k$ be a highest weight of $F\otimes S$, for some $\mu_0$ and $w_0\in W^1$. We show that this $\mu_0$ is $\nu$. Let $M(\mu)\coloneqq \frac12(\mu+\theta\mu)+w_0\rho-\rho_k$. We claim that if $\mu\succ \mu'$, then $M(\mu)\succ M(\mu')$. Indeed, as the Cartan involution fixes the set of positive roots, it fixes the fundamental Weyl chamber, such that $\mu$ is a positive combination of positive roots if and only if the same is true for $\theta(\mu)$. Now we see that
    $$M(\mu)-M(\mu') = \frac12((\mu-\mu')+\theta(\mu-\mu')),$$
    and so, if we assume $(\mu-\mu')$ is a positive combination, so is $\theta(\mu-\mu')$, and thus so is their half-sum, proving the claim. 
    
    As every irreducible representation has a unique highest weight, we have a unique highest weight $\nu$ for $F$. By the above, we immediately get $M(\nu)\succeq M(\mu_0)$, with equality iff $\nu=\mu_0$.
\end{proof}

\noindent It is shown in (\cite{BV}, Lemma $4.1$) that 
\begin{align}\label{BVeigenval}
    \tau(\Omega)-\pi(\Omega)\geq \eta(\tau)
\end{align}
for each irreducible unitary representation $\pi$ of $G(\R)$ satisfying $\Hom_K(\Lambda^p\mf{p}\otimes V_\tau^*,\pi)\neq 0$, where $\eta(\tau)$ is given by
\begin{align}\label{delta}
    \eta(\tau) = |\nu+\rho|^2-\left|\frac12(\mu+\theta\mu)+w\rho\right|^2
\end{align}
for the choice of $\mu$ and $w$ given above (\cite{BV}, ($4.1.2$)), and the norm associated to the inner product induced by the Killing form. By Lemma \ref{delta}, this $\mu$ must be $\nu$, so we have
\begin{align*}
    \eta(\tau) = |\nu+\rho|^2-\left|\frac12(\nu+\theta\nu)+w\rho\right|^2
\end{align*}
Consider $(K_1^*,\dots,K_d^*,P_1^*,\dots,P_s^*)$, an orthonormal basis respecting the decomposition in Lemma \ref{dualdecomp}. With respect to this basis, we will write any weight $\mu$ as $\mu=\mu^++ \mu^-$, where $\mu^+=(\mu^+_1,\dots,\mu^+_d,0,\dots,0)$ and $\mu^-=(0,\dots,0,\mu^-_1,\dots,\mu^-_s)$. Recall the fact that $\mf{k}_\C$, $\mf{p}_\C$ are the eigenspaces of $\theta$ with eigenvalues $1$ and $-1$, respectively. As the dual basis inherits the action from the Cartan involution, we get that $\theta(\mu^++ \mu^-)=\mu^+- \mu^-$. In particular, $\frac12(\mu+\theta\mu) = \mu^+$. Thus, for $\nu=(\nu_1^+,\dots,\nu_d^+,\nu_1^-,\dots,\nu_s^-)$ the highest weight of $F$, expressed in the basis above, we may write
\begin{align}\label{eta}
    \eta(\tau) = |\nu+\rho|^2-\left\vert\frac12(\nu+\theta\nu)+w\rho\right\vert^2 = \sum_{i=1}^s |\nu_i^-|^2+\text{linear terms in }\nu.
\end{align}

\noindent Note that the second equality above follows from the orthogonality of the basis. By (\ref{BVeigenval}), we need only argue that there exists infinitely many $\tau$ such that $\eta(\tau)\geq \lambda$. By the expression above, it is sufficient to be able to find infinitely many representations with highest weight having large $\mf{h}^-$-part. This is possible by the theorem of the highest weight, stating that every dominant integral element is the unique highest weight of an irreducible representation. The dominant integral elements constitute a lattice in the fundamental Weyl chamber associated to $\Phi^+$, which is some cone in the weight space. By the assumption $\delta(G(\R))\geq 1$, we know that $s\geq 1$. 

To be precise, we do the following: Fix any half-line in the fundamental Weyl chamber starting at $0$ and passing through a lattice point, and not lying in the hyperplane given by $x_{d+1}=x_{d+2} = \dots = x_{d+s}=0$. This line is guaranteed to pass through infinitely many lattice points. For any $\lambda>0$ and any linear polynomial $p$ in $d+s$ variables there exists some $r>0$ such that at distance at least $r$ from $0$, every point on this line will have $\sum^s_{i=1}|x_{d+i}|^2-|p(x_1,\dots,x_{d+s})|$ larger than $\lambda$. Considering the expression (\ref{eta}), we see that any lattice point lying on the line with distance at least $r$ to $0$ must have $\eta(\tau)\geq \lambda$. Thus, the line contains infinitely many lattice points associated to $\lambda$-strongly acyclic representations. This finishes the proof of Proposition \ref{kstrong}.

\medskip

\noindent As $\delta(\SL(n,\R))\geq 1$ for $n\geq 3$, we have an immediate corollary.

\begin{cor}
    For any $\lambda>0$ and $n\geq 3$, There exists infinitely many $\lambda$-strongly acyclic representations of $\SL(n,\R)$.
\end{cor}

