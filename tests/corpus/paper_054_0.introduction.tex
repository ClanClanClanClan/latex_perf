\noindent Analytic torsion is an invariant of Riemannian manifolds, defined by Ray and Singer in the $70$'s (\cite{RS}). More recently, it was used by Bergeron and Venkatesh to study the torsion in the homology of cocompact arithmetic groups (\cite{BV}). This made it desirable to extend the definition of analytic torsion to manifolds associated to non-cocompact arithmetic groups. This was first accompished by Raimbault (\cite{Raimbault}) and Müller--Pfaff (\cite{MP}) for finite volume hyperbolic manifolds. Later, it was extended further by Matz and Müller, first for congruence subgroups of $\SL(n)$ in \cite{MzM1}, and later for more general arithmetic groups in \cite{MzM3}. 

One of the reasons for studying torsion in the homology of arithmetic groups is the Ash conjecture (\cite{Ash}), partially proven by Scholze (\cite{Scholze}). In rough terms, this conjecture links systems of Hecke eigenvalues occurring in the mod $p$ cohomology of arithmetic groups to Galois representations with matching Frobenius eigenvalues. See (\cite{BV}, Section $6$) for a discussion on the connection between analytic torsion and the Ash conjecture. As explained in Scholze's paper, by applying the work of Bergeron--Venkatesh and others, in certain situations the Ash conjecture predicts the existence of many Galois representations, more than is predicted by the global Langlands correspondence. 

Further control of the behaviour of analytic torsion should translate to better understanding of torsion homology, and conjecturally to existence of Galois representations. With this motivation we give improved asymptotics of analytic torsion in terms of the volume of the associated manifolds. To state our theorem and the connection to homology, we recall the setups of \cite{BV} and \cite{MzM2}.

\subsection{Analytic torsion in the compact setting}

Let $X$ be a compact oriented Riemannian manifold of dimension $d$, and let $\rho$ be a finite-dimensional representation of $\pi_1(X)$. To such a representation, one may associate a flat vector bundle $E_\rho\to X$ over $X$. Fixing a Hermitian fiber metric on $E_\rho$, we let $\Delta_p(\rho)$ be the Laplace operator on $E_\rho$-valued $p$-forms. For $t>0$, we define $e^{-t\Delta_p(\rho)}$ to be the heat operator, and we let $k_p(\rho) =\dim\ker \Delta_p(\rho)$. One can associate a zeta function $\zeta_p(s,\rho)$ to $\Delta_p(\rho)$, a meromorphic function on $\C$ defined by
\begin{align}\label{zeta}
    \zeta_p(s,\rho) = \frac{1}{\Gamma(s)}\int_0^\infty (\Tr e^{-t\Delta_p(\rho)}-k_p(\rho))t^{s-1} dt
\end{align}

\noindent for $\Re(s)>\frac{d}{2}$. We then define analytic torsion $T_X(\rho)$ of $M$ by
\begin{align}
    \log T_X(\rho) =\frac12 \sum_{p=1}^d (-1)^p p\frac{d}{ds}\zeta_p(s;\rho)\vert_{s=0}.
\end{align}

\noindent There is a corresponding $L^2$-invariant, called $L^2$-torsion denoted $T^{(2)}_X(\rho)$, defined by Lott and Mathai (resp. \cite{Lott}, \cite{Mathai}). For $\Tilde{X}$ the universal covering of $X$, it may be expressed as
\begin{align}\label{l2torsion}
    \log T^{(2)}_X(\rho) = t^{(2)}_{\Tilde{X}}(\rho)\vol(X),
\end{align}
where $t^{(2)}_{\Tilde{X}}(\tau)$ is an invariant associated to $\Tilde{X}$ and $\rho$.

To relate this to arithmetic groups, let $G$ be a semisimple algebraic group with $G(\R)$ of non-compact type, let $K$ be a maximal compact subgroup of $G(\R)$, and suppose $\Gamma\subset G(\Q)$ is a torsion-free congruence lattice. Assume further that $\Gamma$ is cocompact in $G(\R)$. Then $\Tilde{X}\coloneqq G(\R)/K$ is a Riemannian symmetric space, and $X\coloneqq \Gamma\backslash \Tilde{X}$ is a compact locally symmetric space, and as $\Tilde{X}$ is contractible, we have $\pi_1(X)=\Gamma$. Thus, given $\Gamma$ and a finite-dimensional representation $\rho$ of $\Gamma$, we associate to it the analytic torsion $T_X(\rho)$ of the space $X$ as defined above. 

By a theorem of Cheeger and Müller (\cite{Cheeger},\cite{Müller1},\cite{Müller2}), there is an equality of analytic torsion and \textit{Reidemeister torsion} on $X$, the latter being a topological invariant of $X$ computed in terms of homology (see \cite{Reidemeister}, \cite{Franz}). Since $X$ is a classifying space for $\Gamma$, the homology of $X$ computes the homology of $\Gamma$, and thus one may study the homology of $\Gamma$ through its associated analytic torsion.

Now, consider a descending chain of finite index subgroups $\Gamma=\Gamma_0\supseteq\Gamma_1\supseteq\dots$ in $G(\Q)$ satisfying $\bigcap_{i=1}^\infty\Gamma_i = \lbrace1\rbrace$, and let $X_i\coloneqq \Gamma_i\backslash \Tilde{X}$. Let $\tau$ be an irreducible finite-dimensional representation of $G(\R)$. By abuse of notation, we set $T_{X_i}(\tau)\coloneqq T_{X_i}(\tau|_{\Gamma_i})$. Given a non-degeneracy condition on $\tau$, namely requiring that it be \textit{strongly acyclic}, Bergeron and Venkatesh prove (\cite{BV}, Theorem $4.5$) that
\begin{align}\label{BVapprox}
    \lim_{i\to\infty} \frac{\log T_{X_i}(\tau)}{\vol(X_i)} = t^{(2)}_{\Tilde{X}}(\tau).
\end{align}

\noindent Bergeron and Venkatesh show that $t^{(2)}_{\Tilde{X}}(\tau)$ is non-zero if and only if the \textit{deficiency} of $G$, defined $\delta(G)\coloneqq\rank G(\R)-\rank K$, is $1$ (\cite{BV}, Proposition $5.2$). The assumption that $\tau$ is strongly acyclic also guarantees that the free homology is trivial, so Reidemeister torsion is expressed solely in torsion homology. Using the Cheeger-Müller theorem, this is interpreted as there being a lot of torsion in the homology when $\delta(G)=1$, and little torsion in all other cases. "Little torsion" should here be understood in a weak sense, as we only know that there is not full exponential growth of analytic torsion in terms of the volume. A natural question is then, also presented in \cite{AGMY}: what growth rate should we expect when the deficiency is not $1$, and can we more precisely describe the behaviour of analytic torsion in terms of the volume? The results of this paper is a first step in answering these questions.

The result (\ref{BVapprox}) can be thought of as an approximation theorem. In particular, noting that every $\Gamma_i$ is a finite index subgroup of $\Gamma$, we see that $\vol(X_i)=\vol(X)[\Gamma:\Gamma_i]$, and thus the result is equivalent to 
\begin{align}
    \lim_{i\to\infty} \frac{\log T_{X_i}(\tau)}{[\Gamma:\Gamma_i]} = \log T^{(2)}_{X}(\tau).
\end{align}

\noindent In words, analytic torsion associated to a tower of subgroups can be used to approximate the $L^2$-torsion associated to the group. Keeping in mind the equality of analytic torsion and Reidemeister torsion, this statement should be thought of as a torsion version of the analogous result on Betti numbers $b_p(X)$ and $L^2$-Betti numbers $b_p^{(2)}(X)$, proven by W. Lück (\cite{Lück1}), which we state here in our setting.
\begin{align}
    \lim_{i\to\infty} \frac{b_p(X_i)}{[\Gamma:\Gamma_i]} = b_p^{(2)}(X).
\end{align}

\noindent To further the analogy, here there is a criterion based on the deficiency as well: We have that $b_p^{(2)}(X)=0$ unless $\delta(G) = 0$. For a comprehensive survey on approximation of $L^2$-invariants, see \cite{Lück2}.

For certain approximation theorems, their rate of convergence has been explored (see e.g. \cite{BLLS}, and \cite{Lück2} Chapter 5), but is not very developed for analytic and $L^2$-torsion. This is mostly due to the fact that this approximation result is still conjectural in general, and only proven in very particular cases, such as the setting of this paper. Viewed in this light, Theorem \ref{mytheorem} presented below can be seen as a first example of a rate of convergence result in this setting.

Many of the most important arithmetic groups in number theory are not cocompact. In the general setting of non-cocompact arithmetic groups in semisimple Lie groups, we do not have a replacement for the Cheeger-Müller theorem, though certain special cases have recently been proven (see \cite{MR1}, \cite{MR3}). The non-cocompact setting do, however, have one  advantage, as it allows certain terms to contribute (namely the non-identity unipotent part, see section \ref{applyingfine}) which vanish in the cocompact setting, and these terms are very probable candidates for where second order terms should show up. For this reason, the present paper will focus on the non-cocompact setting. 

\subsection{Analytic torsion in the non-compact setting} 

To state our main theorem, let us recall the setup of Matz and Müller (\cite{MzM1}, \cite{MzM2}). We switch to an adelic framework and focus our attention on $\SL(n)$, $n\geq 2$. Let $\A$ be the ring of adeles of $\Q$, with $\A_f$ the finite adeles. Let $X=\SL(n,\R)/\SO(n)$. For $N\geq 3$, let $K(N)\subset \SL(n,\A_f)$ be the open compact subgroup given by $K(N)=\prod_p K_p(p^{\nu_p(N)})$, with $K_p(p^e)\coloneqq \ker(\SL(n,\Z_p)\to \SL(n,\Z_p/p^e\Z_p))$. Define
\begin{align*}
    X(N) \coloneqq \SL(n,\Q)\backslash (\Tilde{X}\times \SL(n,\A_f))/K(N)).
\end{align*}

\noindent By strong approximation, we get that $X(N) = \Gamma(N)\backslash \Tilde{X}$, with $\Gamma(N)\subset \SL(n,\Z)$ the standard principal congruence subgroup of level $N$. Given a finite-dimensional representation $\tau$ of $\SL(n,\R)$, one can now follow the construction of the Laplace operator $\Delta_p(\tau)$ of the Laplace operator on $p$-forms with values in $E_\tau$, as above. As $X(N)$ is not compact, however, $\Delta_p(\tau)$ has continuous spectrum, and hence we need a refined definition of the analytic torsion. This is constructed by Matz and Müller in \cite{MzM1}. Most strikingly, they define the regularized trace of the heat kernel $\Trreg(e^{-t\Delta_p(\tau)})$ as the geometric side of the \textit{Arthur trace formula} for $\SL(n,\R)$ applied to the test function $h_t^{\tau,p}\otimes \chi_{K(N)}$, where $h_t^{\tau,p}$ is the trace of the heat kernel and $\chi_{K(N)}$ the normalized characteristic function of $K(N)$. We give the full definition in 
Section \ref{torsion}. This reduces to the standard definition in the cocompact setting. In \cite{MzM2}, they prove the same limit behaviour as Bergeron-Venkatesh for this setup, namely the approximation formula
\begin{align*}
    \lim_{N\to\infty}\frac{\log T_{X(N)}(\tau)}{\vol(X(N))} = t^{(2)}_{\Tilde{X}}(\tau).
\end{align*}

\noindent Formulated in terms of asymptotics of analytic torsion, the above is equivalent to
\begin{align}\label{MzMasymp}
    \log T_{X(N)}(\tau)=\log T^{(2)}_{X(N)}(\tau)+o(\vol(X(N))) \quad\text{as}\quad N\to\infty.
\end{align}

\subsection{Results} 

As a step towards better understanding the behavior of analytic torsion in terms of the volume, in this paper we prove a stronger version of the asymptotic (\ref{MzMasymp}). In particular, the new result provides an improved upper bound on analytic torsion when the deficiency is not $1$, as well as a bound on second order terms when the deficiency is $1$. We need a slightly stronger non-degeneracy assumption on $\tau$ which we call $\lambda$-strongly acyclic. Our result is the following.
\begin{thm}\label{mytheorem}
    Assume $\tau$ is a $\lambda$-strongly acyclic representation of $\SL(n,\R)$, for a certain $\lambda$ depending only on $n$. Then there exists some $a>0$ such that
    $$\log T_{X(N)}(\tau) = \log T_{X(N)}^{(2)}(\tau)+O(\vol(X(N))N^{-(n-1)}\log(N)^a)$$
    as $N$ tends to infinity.
\end{thm}

\noindent Computing the size of $\vol(X(N))$ in terms of $N$ (see the appendix), the theorem implies a more intrinsic version:
\begin{align}\label{mytheoreminvar}
    \log T_{X(N)}(\tau) = \log T_{X(N)}^{(2)}(\tau)+O(\vol(X(N))^{1-\frac{1}{n+1}}\log(\vol(X(N))^a)
\end{align}
\noindent as $N\to\infty$.

\noindent To ensure that the technical constraint is justified, we prove the existence of infinitely many $\lambda$-strongly acyclic representations for any connected semisimple algebraic group $G$ with $\delta(G)\geq 1$. This result is an extension of the result (\cite{BV}, Section $8.1$), stating that strongly acyclic representations always exist for semisimple $G$ with $\delta(G)=1$.

It seems likely that up to log-terms, the bound in the theorem is strict in the case of deficiency $1$, though this is not proven here. When the deficiency is not $1$, a better upper bound is expected. A lower bound would be extremely interesting as well, but this is not in the scope of this paper.

Because of technical reasons, we will work with $\GL(n)$ instead of $\SL(n)$. For $Y(N)$ the analogous adelic locally symmetric space of $\GL(n)$, one can define analytic torsion $T_{Y(N)}(\tau)$ in a similar manner (see (\ref{torsion}) for the general definition). Our proof is then that of Theorem \ref{glmytheorem}, which is the analogous statement of Theorem \ref{mytheorem} in the setting of $Y(N)$, and we show the two theorems are equivalent.

The proof is based on the framework of \cite{MzM2}. The main contribution of this paper lies in the change of perspective to allow certain parameters, namely the compactification parameter (see Section \ref{compactsub}) and the truncation parameter $T$ (see (\ref{trunc})) to vary with the level $N$, and in the work needed to allow for this variation. In particular, we show the existence of infinitely many $\lambda$-strongly acyclic representations of any connected semisimple algebraic group (Proposition \ref{kstrong}), and we prove a new bound on the trace of the heat kernel in the large time aspect (Proposition \ref{traceheatdecay}) which may be of independent interest. The bound is used to gain control over large time behaviour of the archimedean orbital integrals showing up in the analysis of the Arthur trace formula.

\subsection{Organization of the paper} 

In Section \ref{definitions}, we present the setup for the heat kernel on symmetric spaces. In Section \ref{representations}, we prove the existence of infinitely many $\lambda$-strongly acyclic representations for any semisimple algebraic group. Section \ref{plancherel} gives a proof of large $t$ asymptotics of the trace of the heat kernel. We give a brief introduction to the Arthur trace formula and its geometric expansions in Section \ref{trace formula} and define analytic torsion, as well as express the trace formula applied to a compactification of our test function. In Section \ref{asymptotics}, we handle the asymptotics of our local orbital integrals. Finally, we combine all the ingredients in Section \ref{conclusion} to prove our main theorem.

\subsection{Aknowledgement}

The author would like to thank Jasmin Matz, for suggesting this topic and for her invaluable advice throughout the project. I thank Bonn University for their hospitality while working on this project. I would also like to thank Werner Müller, for our helpful discussions while in Bonn and for his comments to a draft of this paper. Finally, I wish to express my gratitude to Oscar Harr, Desirée Gijón Gómez, Fadi Mezher and Huaitao Gui for useful conversations at UCPH. 

This work was supported by the Carlsberg Foundation.