\section{Künneth morphisms for analytic stacks}\label{$6$-functor formalism for analytic stacks}
\subsection{The $6$-functor formalism for analytic stacks}\label{6functorsforanalyticstacks}
In this subsection we recall the construction of the $6$-functor formalism of analytic stacks, following Clausen-Scholze \Cite[Lecture 16-20]{AnSt}. Since we are mainly interested in formal questions of $6$-functor formalisms, we will not recall all the relevant definitions and refer to \Cite{AnSt}, \Cite{clausen2022condensed}.   We denote by  $D(\mathbb{Z})\coloneqq D(\text{CondAb})$ be the derived category of light condensed abelian groups. 
\begin{definition}\label{defanring}\Cite[Lecture 8]{AnSt}
    An analytic ring $A=(A^{\tri},D(A))$ is the datum of a condensed animated ring $A^{\tri}$ and a fully faithful subcategory $D(A) \overset{j_A}{\subset} \Mod_{A^{\tri}}D(\mathbb{Z})$ such that
    \begin{enumerate}
       \item The category $D(A)$ is stable under all limits and colimits in $\Mod_{A^{\tri}}D(\mathbb{Z})$ and there is a left adjoint of the inclusion $j_A: D(A) \subset \Mod_{A^{\tri}}D(\mathbb{Z})$ which we denote by $(-)_A^\wedge$.
       \item For all $M \in D(\mathbb{Z})$ and $N \in D(A)$, we have that $R\underline{\text{Hom}}_{\mathbb{Z}}(M,N) \in D(A)$.
       \item The left adjoint $(-)_A^\wedge$ sends connective to connective objects.
       \item The underlying condensed animated ring $A^{\tri}$ lies in  $D(A)$.
    \end{enumerate}
\end{definition}

We denote by  $\AnR$ be the category of analytic rings, by $\Aff\coloneqq \AnR^{op} $ its opposite category and by $\SpecAn(-)\coloneqq (-)^{op}: \AnR \to\Aff$ the opposite functor. Condition (1) in \Cref{defanring} implies that $D(A)$ is an accessible localisation of $\Mod_{A^{\tri}}D(\mathbb{Z})$ and thus a presentable category. For any morphism $f: \SpecAn(B) \to \SpecAn(A)$ in $\Aff$, there is an associated  functor $f^* : D(A) \to D(B)$ in $\PrL$ which is defined by the diagram 
\[\begin{tikzcd}
	{\Mod_{A^{\tri}}D(\mathbb{Z})} & {\Mod_{B^{\tri}}D(\mathbb{Z})} \\
	{D(A)} & {D(B)}
	\arrow["{f^{\tri *}}", from=1-1, to=1-2]
	\arrow["{(-)_B^{\wedge}}", from=1-2, to=2-2]
	\arrow["{j_A}", hook, from=2-1, to=1-1]
	\arrow["{f^*}"', from=2-1, to=2-2]
\end{tikzcd}\]
Here we denote by $f^{\tri,*}\coloneqq B^{\tri} \otimes_{A^{\tri}}(-)$ the base change functor. Denoting the right adjoint of $f^*$  by $f_*$ we thus obtain a functor 
\begin{align*}
  D : \Aff^{op} \to \CAlg(\PrL), \ \SpecAn(A) &\mapsto D(A)   \\
   (f: \SpecAn(B) \to \SpecAn(A)) &\mapsto f^* : D(A) \to D(B)
\end{align*}
\begin{remark}
In order to apply \Cref{tannakian lift}, it will be convenient to work with  (condensed) $\mathbb{E}_{\infty}$-rings as opposed to (condensed) animated rings and to consider analytic $\mathbb{E}_{\infty}$-rings. We denote by $\text{AnCR}$ the category of analytic complete $\mathbb{E}_{\infty}$-rings as in \Cite[Definition 2.1.1]{camargo2024analytic}. Indeed, by \Cite[Lemma 2.1.3]{camargo2024analytic} the resulting $6$-functor formalism on $\text{AnCR}$ is Tannakian. Everything we say in the following will be correct as well for  $\text{AnCR}$ replacing $\AnR$.
\end{remark}
Our aim is now to apply the construction result \Cite[Proposition 3.3.3]{heyer20246}. As a first step we thus need an appropriate class of morphisms $I,P$ and $E$ in $\Aff$.  
\begin{definition}\Cite[lecture 17]{AnSt}
Let $\pi: X\to Y$ be a map in $\Aff$.
\begin{enumerate}
    \item We say $\pi$ is proper if $\pi_*$ commutes with small colimits and satisfies the projection formula.
    \item We say $\pi$ is an open immersion if it is a $D$-open immersion in the sense of \Cref{def open immersion for D Cop}.
    \item We say $\pi$ is $!$-able if it can by written as a composition $\pi=p \circ j$ of an open immersion $j$ and a proper map $p$.
\end{enumerate}
We denote by $E$ the class of $!$-able maps and by $I,P \subset E$ the class of open immersions and of proper maps, respectively.
\end{definition}
 We will need the following alternative description of proper maps. 
\begin{lemma}\label{alternative char. of open and proper in aff}
Let $\pi: \SpecAn(A) \to \SpecAn(B)$ be a map in $\Aff$. 
\begin{enumerate}
    \item $\pi$ is proper if and only if $A$ carries the induced analytic ring structure of $B$, i.e.\  $A=A^{\tri}_{/B}\coloneqq (A^{\tri},\Mod_{A^{\tri}}D(B))$. 
    \item If $\pi : X \to Y$ is proper then it satisfies proper base change, i.e. for any  Cartesian square
    \[\begin{tikzcd}
	{X'} & Z \\
	X & Y
	\arrow["{\pi'}", from=1-1, to=1-2]
	\arrow["{f'}", from=1-1, to=2-1]
	\arrow["f", from=1-2, to=2-2]
	\arrow["\pi", from=2-1, to=2-2]
\end{tikzcd}\]
the natural morphism of functors $f^*\pi_* \to f'^*\pi'_*$ is an equivalence. The base change of a proper map is again proper. Any composition of proper maps is proper.
   
\end{enumerate}
\begin{proof}
1) Assume $A$ carries the induced analytic ring structure. Then $\pi_* : \Mod_{A^{\tri}}D(B) \to D(B)$ is the forgetful functor, and hence commutes with colimits and is $D(B)$-linear and thus satisfies the projection formula.  Now assume that $\pi$ is proper. By the projection formula we have for any $N \in D(B)$ equivalences 
\begin{align*}
\pi_*(A^{\tri}\otimes_A \pi^*(N)) &\cong \pi_*(A^{\tri})\otimes_B N \\
& \cong  (\pi_*(A^{\tri}) \otimes_{B^{\tri}}N)^{\wedge}_B.
\end{align*}
However, the left hand side identifies to $\pi_*( (A^{\tri} \otimes_B N)^{\wedge}_A)$. We thus obtain that the element $A^{\tri}\otimes_B N \in \Mod_{A^{\tri}}D(B) $ lies automatically in $D(A)$, since the former is generated by colimits of the elements $A^{\tri}\otimes_B N$ for $N\in D(B)$  we obtain the desired equivalence $D(A) \cong \Mod_{A^{\tri}}D(B)$. \\
2)  Assume  that $\pi$ is proper. Let $f: \SpecAn(C^{\tri},D(C))\to \SpecAn(B^{\tri},D(B))$ any morphism of affine analytic stacks, then we obtain by the description of pushouts in $\AnR$  !!!REF!!! that the fibre product is given by $\SpecAn(A)\times_{\SpecAn(C)}\SpecAn(B)\cong \SpecAn((A^{\tri}\otimes_{B^{\tri}}C^{\tri},\Mod_{A^{\tri}\otimes_{B^{\tri}}C^{\tri}}D(C))$. Let $\tilde{\pi}$ be the base change of $\pi$ along $f$. The base change diagram reduces to the diagram 

\[\begin{tikzcd}
	{\Mod_{A^{\tri}\otimes_{B^{\tri}}C^{\tri}}D(C)} & {D(C)} \\
	{\Mod_{A^{\tri}}D(B)} & {D(B)}
	\arrow["{\tilde{\pi}_*}", from=1-1, to=1-2]
	\arrow["{\tilde{f}^*}"', from=2-1, to=1-1]
	\arrow["{\pi_*}", from=2-1, to=2-2]
	\arrow["{f^*}"', from=2-2, to=1-2]
\end{tikzcd}\]
which indeed commutes. In particular we see that the fibre product carries the induced analytic ring structure from $(C^{\tri},D(C))$, which shows that $\tilde{\pi}$ is proper by part 1).  The claim that a composition of proper maps is again proper can be checked easily from the definition of proper maps. 
\end{proof}

\end{lemma}

\begin{lemma}\label{pbc and stability under pullback of open imersion in aff}
Let $\pi: \SpecAn(A) \to \SpecAn(B)$ be a map in $\Aff$.  If $\pi$ is an open immersion then it satisfies proper base change. The base change of an open immersion is again an open immersion. Any composition of open immersions is an open immersion.
\end{lemma}
\begin{proof}
  Let $I\in D(B)$ be the idempotent algebra corresponding to the open immersion $\pi$, so $D(A) \subset D(B)$ identifies with the full subcategory of elements $M \in D(B)$ for which $I \otimes M\cong 0$. Let $f: \SpecAn(C) \to \SpecAn(B)$ be any map in $\Aff$. By the definition of pushouts in $\AnR$, we know that $C\otimes_B A$ is given by the completion of the analytic ring structure on $C^{\tri}$ such that  module $M \in \Mod_{C^{\tri}\otimes_{B^{\tri}} A^{\tri}}D(\mathbb{Z})$ lies in $D(C\otimes_B A)$ if and only its restriction to $\Mod_{C^{\tri}}D(\mathbb{Z})$ respectively $\Mod_{A^{\tri}}D(\mathbb{Z})$ lies in $D(C)$ respectively $D(A)$. Thus $M \in D(C)$ lies in $D(C \otimes_B A)$ if and only $M\otimes I\cong 0$. This shows that the base change is again an open immersion. The claim that proper base change holds follows by the construction the equivalence $D(C \otimes_B A)\cong \{M\in D(C)\mid f^*(I)\otimes M\cong 0 \}$. The claim that a composition of open immersions is again an open immersion can be checked easily from the definition of open immersions. 
\end{proof}

\begin{lemma}\label{AffE is geometric setup}
The pair $(\Aff, E)$ is a geometric setup
\end{lemma}
\begin{proof}
 Since open immersions and proper maps are stable under arbitrary base change in $\Aff$ (cf. \Cref{alternative char. of open and proper in aff}, \Cref{pbc and stability under pullback of open imersion in aff}), the same is true for any $!$-able map. We now check stability under composition. Let $f : \SpecAn(A) \to \SpecAn(B)$ and $g: \SpecAn(B) \to \SpecAn(C)$ be $!$-able maps and write $f=p_1 \circ j_1$  and $g= p_2 \circ j_2$ for $j_i \in I$ and $p_i\in P$. We consider the factorisation 
\[\begin{tikzcd}
	& {\SpecAn(A^{\tri}_{/C})} \\
	{\SpecAn(A)} && {\SpecAn(C)}
	\arrow["p", from=1-2, to=2-3]
	\arrow["\iota", from=2-1, to=1-2]
	\arrow["{g\circ f}"', from=2-1, to=2-3]
\end{tikzcd}\]
Here the map $p$ is clearly proper, since it is given by the induced analytic ring structure from $C$. It thus suffices to check that $\iota $ is an open immersion. Note that we have a cartesian diagram:
\[\begin{tikzcd}
	{\SpecAn(A^{\tri}_{/B})} & {\SpecAn(A^{\tri}_{/C})} \\
	{\SpecAn(B)} & {\SpecAn(B^{\tri}_{/C})}
	\arrow["{\tilde{j_2}}", hook, from=1-1, to=1-2]
	\arrow["{\tilde{p_1}}", from=1-1, to=2-1]
	\arrow["p_1", from=1-2, to=2-2]
	\arrow["j_2", hook, from=2-1, to=2-2]
\end{tikzcd}\]
 It thus follows that $\iota \cong {\tilde{j_2}} \circ j_1$ is an open immersion since it is a composition of open immersions (cf. \Cref{pbc and stability under pullback of open imersion in aff}).
\end{proof}



\begin{lemma}\label{suitable decomposition I,P}
The classes $I,P \subset E $ satisfy the following properties: 
\begin{enumerate}
    \item $I$ and $P$ contain all identity morphisms and are stable under compositions and arbitrary base change in $\Aff$.
    \item Every $f\in E$ can be written as a composition $f= p \circ j$ with $p\in P$, $j\in I$.
    \item  $I$ and $P$ are right cancellative. i.e. if $p, p\circ q \in I$ (respectively in $P$) then $q \in I$ (respectively in $P$)
    \item Every morphism $f \in I\cap P$ is $n$-truncated for some $n \geq -2$ (possibly depending on $f$).
\end{enumerate}
\end{lemma}
\begin{proof}
1) An isomorphism $X\cong Y$ in $\Aff$ is clearly an open immersion and proper.  The claim that $I$ and $P$ are stable under base change and composition is \Cref{alternative char. of open and proper in aff} and \Cref{pbc and stability under pullback of open imersion in aff}. The condition 2) is satisfied by definition of the class $E$. \\
3) Consider the diagram 
\[\begin{tikzcd}
	{\SpecAn(A)} & {\SpecAn(B)} \\
	& {\SpecAn(C)}
	\arrow["q", from=1-1, to=1-2]
	\arrow["{p\circ q}"', from=1-1, to=2-2]
	\arrow["p", from=1-2, to=2-2]
\end{tikzcd}\]


and consider the proper case. Note that for any proper map $p: \SpecAn(A^{\tri},\Mod_{A^{\tri}}D(B))\to \SpecAn((B^{\tri},D(B)))$ the functor $p_* : \Mod_{A^{\tri}}D(B) \to D(B)$ is the forgetful functor and hence conservative (cf. \Cref{alternative char. of open and proper in aff}). Assume $p, p\circ q$ are proper. Then $q_*$ commutes with all colimits since $p_*$ commutes with colimits and $p_*\circ q_*$ is conservative. The fact that $q_*$ satisfies the projection formula follows from \Cref{examples of internal leftadjoint}.\\
We now consider the case of open immersions. Let $j\coloneqq p\circ q$ and consider $D\coloneqq \mathrm{cofib}(j_!(1)\to 1)\in D(C)$ the idempotent algebra corresponding to the open immersion $j$. Then $p^*(D)\in D(B)$ is again idempotent and we claim that it defines an open immersion via \Cref{alternative char. of open and proper in aff}. We thus need to show 
\[
\mathrm{ker}(q^*)\cong \Mod_{p^*(D)} D(B)
\]
as full subcategories of $D(B)$. If $M\in \Mod_{p^*(D)} D(B)$, then $p^*(D)\otimes_B M\cong M$ so applying $q^*(-)$ we obtain $q^*p^*(D)\otimes_A q^*(M)\cong q^*(M)\cong 0$ since $q^*p^*(D)\cong j^*(D)\cong 0$, so $M \in \mathrm{ker}(q^*)$. Conversely, let $M\in \mathrm{ker}(q^*)$, and let $X\coloneqq \SpecAn(A), Y\coloneqq \SpecAn(B), Z\coloneqq \SpecAn(C) $. Consider the following cartesian diagram:

\[\begin{tikzcd}
	{X\times_ZY} & {Y\times_ZY} & Y \\
	X & Y & Z
	\arrow["{\tilde{q}}", from=1-1, to=1-2]
	\arrow["{\tilde{j}}", curve={height=-18pt}, from=1-1, to=1-3]
	\arrow["h", from=1-1, to=2-1]
	\arrow["\lrcorner"{anchor=center, pos=0.125}, draw=none, from=1-1, to=2-2]
	\arrow["{\tilde{p}}", from=1-2, to=1-3]
	\arrow["{\tilde{p}}", from=1-2, to=2-2]
	\arrow["\lrcorner"{anchor=center, pos=0.125}, draw=none, from=1-2, to=2-3]
	\arrow["p", from=1-3, to=2-3]
	\arrow["q", from=2-1, to=2-2]
	\arrow["j", curve={height=18pt}, from=2-1, to=2-3]
	\arrow["p", from=2-2, to=2-3]
\end{tikzcd}\]
We want to verify that $p^*(D)\otimes_B M\cong M$. Note that 
\begin{align*}
 p^*(D)\otimes_B M & \cong \mathrm{cofib}(p^*(j_!(1))\otimes_BM\to M) \\
 & \cong \mathrm{cofib}(\tilde{j}_!\tilde{j}^*(M)\to M).
\end{align*}
Here we used the symmetric monoidality of $p^*(-)$ in the first line and the fact that $\tilde{j}$ an open immersion, as it is a base change of the open immersion $j$ and hence satisfies the projection formula for $\tilde{j}_!$. It thus suffices to show that $\tilde{j}_!(\tilde{j}^*(M))\cong 0$, which follows by the following diagram chase: 
\begin{align*}
 \tilde{j}_!\tilde{j}^*(M)&\cong q_!h_!\tilde{q}^*\tilde{p}^*(M) \\
 & \cong q_! q^*\tilde{p}_!\tilde{p}^*(M)\\
 & \cong q_!q^*p^*p_!(M) \\
 & \cong q_!q^*(M) \cong 0.
\end{align*}
Here we used the definition of $j$ and the commutativity of the left square in the first line, proper base change for the left square in the second line, proper base change for the right square in the third line and  fully faithfulness of $p_!$ and that $M\in \mathrm{ker}(q^*)$ in the last line. \\
4)
For any open immersion $j: U \to X$ we have (for example by the description in terms of idempotent algebras in \Cref{alternative defclosed open inSym}) that the diagonal $\Delta_j : U \to U \times_X U$ is an isomorphism, thus $j$ is a monomorphism in $\Aff$ and hence $(-1)$-truncated.
\end{proof}
\begin{definition}\Cite[Definition 3.3.2]{heyer20246} 
Let $(C,E)$ be a geometric setup such that $C$ admits finite products. A suitable decomposition of $E$ is a pair $I,P \subset E$ such that the conditions 1)-4) of the previous lemma are satisfied.
\end{definition}
Thus the pair $I,P \subset E$ of open immersions and proper maps define a suitable decomposition of $E$ in $\Aff$. 

\begin{lemma}\label{proper maps arre Künneth}
    Any proper map $p: \SpecAn(B) \to \SpecAn(A)$ is Künneth. 
\end{lemma}
\begin{proof}
 Let $g: \SpecAn(C)\to \SpecAn(A)$ be any morphism in $\Aff$. By \Cref{alternative char. of open and proper in aff}, we have an equivalence $D(B)\cong \Mod_{B^{\tri}}D(A)$. Since any base change of a proper map is proper by \Cref{alternative char. of open and proper in aff}, we obtain that $D(C\otimes_AB)\cong \Mod_{g^*(B^{\tri})}D(C)$. On the other hand we have  equivalences 
 \[
 D(C)\otimes_{D(A)}D(B)\cong D(C)\otimes_{D(A)}\Mod_{B^{\tri}}D(A)\cong \Mod_{g^*(B^{\tri})}D(C).
 \]
\end{proof}
\begin{lemma}\label{I and P interact condition c}
Consider a cartesian diagram in $\Aff$
\[\begin{tikzcd}
	{\SpecAn(D)} & {\SpecAn(C)} \\
	\SpecAn(A) & \SpecAn(B)
	\arrow["{j'}", from=1-1, to=1-2]
	\arrow["{p'}", from=1-1, to=2-1]
	\arrow["\ulcorner"{anchor=center, pos=0.125}, draw=none, from=1-1, to=2-2]
	\arrow["p", from=1-2, to=2-2]
	\arrow["j", from=2-1, to=2-2]
\end{tikzcd}\]
with $j\in I$ and $p\in P$. Then the natural map  $j_!p'_*\to p_*j'_! $ is an isomorphism of functors.
\end{lemma}
\begin{proof}
Since $j'$ is an open immersion the functor $j'^*$ is essentially surjective. It thus suffices to show the isomorphism for $M\coloneqq j'^*(N)$, $N\in D(X')$. Let $I \in D(B)$ be the idempotent algebra corresponding to $j$, then since $p$ (and hence $p'$) is proper we have that $C\cong C^{\tri}_{/B}$ the idempotent algebra corresponding to $j'$ is given by by $p^*(I)\cong I \otimes_B C^{\tri}$. We claim that there are the following equivalences:
\begin{align*}
  p_*j'_!j'^*(N)& \cong p_*(\text{fib}(N\to N\otimes_{C}p^*(I)) \\
   &\cong \text{fib}(p_*(N)\to p_*(N)\otimes_{B}I )
\end{align*}
Here we used the definition of $j'_!$ in the first line and that $p_*$ commutes with limits and satisfies proper base change in the second line. On the other hand, we obtain $j_!p'_*j'^*(N)\cong j_!j^*p_*(N)$ by proper base change. Using the definition of $j_!j^*$ we thus obtain 
\begin{align*}
    j_!p'_*j'^*(N)& \cong j_!j^*p_*(N) \\
    & \cong \text{fib}(p_*(N) \to p_*(N)\otimes_B I)
\end{align*}
which shows that the natural map $j_!p'^*\to p_*j_!'$ is an equivalence.
\end{proof}

\begin{proposition}\Cite[lecture 17]{AnSt}\label{affinekünneth}
The pair $(\Aff,E)$ defines a geometric setup and the functor $D: \Aff^{op} \to \PrL$ extends to a $6$-functor formalism  $D: Corr(\Aff,E) \to \PrL$ which is Künneth (and Tannakian if we work with $\text{AnCR}$).
\end{proposition}
\begin{proof}
 To prove that the functor $D$ extends to a $6$-functor formalism we need to show that the assumptions a), b) and c) in \Cite[Proposition 3.3.3]{heyer20246} are verified. The conditions a) and b) are satisfied by definition of the classes $I$ and $P$ and by the base change results for categorical open immersions and proper maps \Cref{alternative char. of open and proper in aff}, \Cref{pbc and stability under pullback of open imersion in aff}. The last condition c) is satisfied by \Cref{I and P interact condition c}. By \Cite[Proposition 3.3.3]{heyer20246} we thus obtain a $6$-functor formalism 
 \[
 D: Corr(\Aff,E) \to \PrL
 \]
 such that for all $p\in P$ the functor $p_!$ is right adjoint to $p^*$ and for all $j\in I$ the functor $j_!$ is left adjoint to $j^*$. \\
 We will now prove that any $!$-able map $f : \SpecAn(A)\to \SpecAn(R)$ is Künneth. By definition $f\cong p \circ  j$ for $p$ a proper map and $j$ an open immersion. By \Cref{proper maps arre Künneth} and \Cref{open immersion and closed immersions are Künneth}, open immersion and proper maps are Künneth, so we conclude by \Cref{dualisbe stable under compposition}. The claim that $D$ is Tannakian is \Cite[Lemma 2.1.3]{camargo2024analytic}.
\end{proof}

We work with the following definition of analytic stacks which is slightly simplified compared to the correct definition in \Cite[Lecture 19]{AnSt}.

\begin{definition}\Cite[Lecture 19]{AnSt}
 An analytic stack is a sheaf $X \in \Shv(\Aff, \Ani)$ for the $!$-topology on $\Aff$.
 
\end{definition}
\begin{remark}
Since the category $\AnR$ is not small one needs to be careful when considering sheaves on it. One way to proceed is to pick a large enough infinite regular cardinal $\kappa$ and to consider $\kappa$-small objects in $\AnR$. 
\end{remark}
By \Cref{affinekünneth} we have a sheafy $6$-functor formalism $D : Corr(\Aff, E) \to \PrL$ on the geometric setup $(\Aff, E)$, with $E$ the class of $!$-able maps. Using the extension result \Cite[Theorem 3.4.11]{heyer20246} we obtain a sheafy $6$-functor formalism
\[
\mathit{D}_{qc} : Corr(\AnSt, \tilde{E}) \to \CAlg(\PrL)
\]
on the category of analytic stacks with $\tilde{E}$ the class of morphisms as in \Cref{E' properties}.

\begin{corollary}
 Let $f: X \to S$ be a morphism of analytic stacks in $\tilde{E}$ and assume that there is a $!$-cover $g: S'\to S$ with $S'\in C$ and $g\in E_0$. Then $f$ is Künneth (and Tannakian if we work with analytic $\mathbb{E}_\infty$-rings).
\end{corollary}
\begin{proof}
    This  follows from \Cref{affinekünneth} together with \Cref{Künnethextension-main corollary}.
\end{proof}



\subsection{Examples of analytic stacks and $!$-covers}
We will give some examples of analytic stacks and $!$-covers.
\begin{example}\Cite[Lecture 19]{AnSt}
For any $A\in \AnR$, the functor 
\[
\SpecAn(A) : F: \AnR \to \Ani , \ B \mapsto \text{Hom}(A,B)
\]
is an analytic stack.
\end{example}
\begin{example}\Cite[Lecture 19]{AnSt}
 Let $S\in \text{Prof}^{\text{light}}$ be a light profinite set, then we associate the analytic stack $\SpecAn(C(S,\mathbb{Z}))$, where $C(S,\mathbb{Z})$ is the ring of $\mathbb{Z}$-valued continuous functions on $S$, with its discrete analytic ring structure. This gives a functor 
 \[
 (-)_{\text{Betti}}: \text{Prof}^{\text{light}} \to \AnSt.
 \]
 One can check that this sends surjections of light profinite sets to $!$-covers and thus gives a functor from the category of compact Hausdorff spaces $\text{LCHaus}$ to analytic stacks
 \[
 (-)_{\text{Betti}}: \text{CHaus} \to \AnSt.
 \]
\end{example}

\begin{example}
    Let us sketch how to realise algebraic stacks in the category $\AnSt$. Let $\text{Ring}$ be the category of animated commutative rings. For $R \in \text{Ring} $ we can associate to it the analytic ring $R^{\tri}\coloneqq (\underline{R},\Mod_{\underline{R}}D(\mathbb{Z}))$), giving a functor 
    \[
    (-)^{\tri}: \text{Ring} \to \AnR .   \]
    This functor sends fpqc covers to $!$-covers, we thus obtain a functor
     \[
    (-)^{\tri}: \Shv_{fpqc}(\text{Ring},\Ani) \to \AnSt.  \]
\end{example}


\begin{example}
 Let $(A,A^+)$ be a sheafy analytic Huber pair, then by \Cite[Proposition 3.34]{andreychev2021pseudocoherent} we can associate to it an analytic ring $(A,A^+)_{\square}$. Denoting by $\text{AnH}$ the category of complete sheafy analytic Huber pairs we thus obtain a functor 
 \[
 (-)_{\square}: \text{AnH} \to \AnSt,  \ (A,A^+) \mapsto \SpecAn((A,A^+)_{\square}).
 \]
 This functor sends open covers to $!$-covers, inducting a functor from the category $\text{AdicSp}$ of analytic adic spaces to analytic stacks
\[
 (-)_{\square}: \text{AdicSp} \to \AnSt.
\]
\end{example}



\subsection{A $p$-adic version of Drinfeld's lemma}\label{Examples of Künneth morphism for analytic stacks}\label{Drinfelds section}
In this subsection we want to give an application of our results which we hope to be of interest for a $p$-adic version of the work of Fargues and Scholze (cf. \Cite{fargues2021geometrization}). A central object in their geometrisation of the $\ell$-adic local Langlands correspondence is the following analytic adic space
\[
Y_{(0,\infty),S}\coloneqq \Spa(W(R^+))\backslash V([\pi]p),
\]
where $W(-)$ denotes the $p$-typical Witt vectors, $S=\Spa(R,R^+)$ an affinoid perfectoid space in characteristic $p$ and $\pi\in R^+$ a pseudo-uniformiser. Let us fix an algebraic closure $k\coloneqq \bar{\mathbb{F}}_p$. One main insight of the work of Anschütz, Le Bras and Mann \cite{anschütz20246functorformalismsolidquasicoherent}, is that the $p$-adic pro-étale cohomology of $S$ can be understood by quasi-coherent cohomology of the quotient of the analytic adic space $Y_{(0,\infty),S}$ by the Frobenius $\varphi$ induced by $R^+$
\[
Y_{(0,\infty),S}/\varphi^{\mathbb{Z}}.
\]
The Drinfeld lemma in the $\ell$-adic setting of Fargues Scholze deals with the $v$-stack \newline $\Div_k^1\coloneqq (\Spa(\mathbb{Q}_p^{\cyc})/\mathbb{Z}_p^\times)/\varphi^{\mathbb{Z}}$ and the full subcategory $D_{dlb}(Y,\Lambda)\subset D_{et}(Y,\Lambda)$ of dualisable objects in the category of étale sheaves with coefficients in $\Lambda\coloneqq \Bar{\Q}_{\ell}$ on a small $v$-stack $Y$. It can be phrased as an equivalence of idempotent complete stable categories 
\[
D_{dlb}(\Div_k^1,\Lambda)\otimes_{D_{dlb}(\Spd(k),\Lambda)}D_{dlb}(W,\Lambda) \cong D_{dlb}(\Div_k^{1}\times_{\Spd(k)}W,\Lambda) 
\]
for $W$ any small $v$-stack over $\Spd(k)$ (cf. \Cite[Chapter IV.7.3]{fargues2021geometrization}). 
Guided by the work of Anschütz, Le Bras and Mann and motivated to obtain a $p$-adic version of Drinfeld's lemma we consider the following analytic stacks

\begin{align*}
    \Div_{\mathbb{Q}_p}^{1,\ct}\coloneqq & ((Y_{(0,\infty), \Q_p^{\cyc,b}})_{\square}/(\Z_p^{\times})_{\text{Betti}})/\varphi^{\Z} \\
\Div_{\mathbb{Q}_p}^{1,\la}\coloneqq & (\lim_{T\mapsto (T+1)^p-1}\mathbb{D}_{\square} \backslash \{0\} )/\Q_p^{\times, \la}.
\end{align*}

By  \Cref{Künnethextension-main theorem}, any morphism $X\to S$ of analytic stacks in $\tilde{E}$ with $S\in \AnSt$ such that $\Delta_S$ is representable and admitting a $!$-cover $U\to S$ with $U \in C$ give examples of Künneth morphisms for the $6$-functor formalism $\mathit{D}_{qc} : Corr(\AnSt, \tilde{E}) \to \PrL$. In particular we obtain the following example.
\begin{corollary}\label{Drinfeld cont und la}
The structural morphism $\Div_{\mathbb{Q}_p}^{1,?} \to \SpecAn(\Q_{p,\square})$  for $?\in \{ \ct, \la \}$ lies in $\tilde{E}$ and is thus Künneth.
\end{corollary}


We end with an example for a morphism which is not Künneth. One way to construct such a morphism is to consider stacks $S$ with non-affine diagonal:

\begin{example}
Let $k$ be an algebraically closed field and consider $E \to \Spec(k)$ an elliptic curve. We consider $E^{\tri}\to \SpecAn(k^{\tri})\coloneqq \ast$ with its discrete induced analytic structure. Then the morphism $f : \ast  \to \ast/E^{\tri}$ is not Künneth. Indeed, note that it lies in $\tilde{E}$ as it has a section. One can show that $f$ is a descendable cover and that $f_*(1)\cong 1$. 
We thus obtain an isomorphism 
\[
D_{qc}(\ast/E^{\tri})\cong \coMod_{f^*f_*(1)}D_{qc}(\ast)\cong D_{qc}(\ast).
\]

In particular, we observe that the morphism
\[
D_{qc}(\ast) \otimes_{D_{qc}(\ast/E^{\tri}) }D_{qc}(\ast) \cong D(\ast) \to D_{qc}(\ast \times_{\ast/E^{\tri} }\ast) \cong D_{qc}(E^{\tri})
\]
is not an isomorphism. Note that $f$ has non-representable diagonal $\Delta_f$ as the pullback along itself gives $\ast \times_{\ast/E^{\tri}} \ast \cong E^{\tri}$ which does not lie in $\Aff$.
\end{example}