\section{Künneth formulas for stacks}\label{Künneth formulas for stacks}
We introduce the notion of Künneth morphisms for general $3$-functor formalisms and discuss its relation to the dualisable presentable modules discussed in the previous chapter. For a $6$-functor formalism $D: Corr(C,E)\to \PrL$, a morphism $f: X \to Z$ in $E$ will be called Künneth if a general categorical Künneth formula holds, by which we mean that for all $W\to Z$ in $C$ the natural morphism
\[
m_{X,W}: D(X) \otimes_{D(Z)} D(W) \to D(X\times_Z W)
\]
is an equivalence. We show that Künneth morphisms are stable under composition and base change (\Cref{dualisbe stable under compposition}, \Cref{dualisable stable under bc}) and in certain cases,  $!$-local on the source and target (\Cref{dualisability is !-local on source}, \Cref{dualisability is !-local on target}). Finally, we recall the $6$-functor formalisms of analytic stacks and give examples of Künneth morphisms in this context, giving a generalisation of the discussion in \Cite[Chapter 3-4]{ben2010integral} in the context of quasi-coherent sheaves on perfect stacks. We refer to \Cite[Chapter 1-4]{heyer20246} for a reference on general $6$-functor formalisms.
\subsection{Künneth morphisms}\label{Künneth modules in $6$-functor formalisms}
A very powerful tool to construct $3$-functor formalisms is the construction result of Heyer-Mann \Cite[Proposition 3.3.3]{heyer20246}. We start by recalling the following definition.
\begin{definition}\Cite[Definition 2.1.1, Remark 2.1.2]{heyer20246}
A geometric setup is a tuple $(C,E)$ of a category $C$ together with a homotopy class of edges $E$ in $C$ such that the following conditions are satisfied.
\begin{enumerate}
    \item $E$ contains all isomorphisms. 
    \item $E$ is stable under compositions and under pullback along edges in $C$. 
    \item If $f :X \to Y$ lies in $E$, then so does its diagonal $\Delta_f: X \to X \times_Y X$.
\end{enumerate}
  
\end{definition}
For the construction result of Heyer-Mann, assume we are given a geometric setup $(C,E)$, where $C$ is a category with finite limits, and a lax symmetric monoidal functor 
\begin{align*}
  D : C^{op} \to \CAlg(\PrL), \ X &\mapsto D(X)   \\
   (f: Y \to X) &\mapsto  D(f)\coloneqq f^* : D(X) \to D(Y)
\end{align*}
where we consider $C$ with its cartesian monoidal structure. We will furthermore assume that the class $E$ decomposes into a suitable class $P\subset E$ of "proper maps" and a class $I\subset E$ of "local isomorphisms, such that any $f\in E$ can be written as a composition of a "local isomorphism" and a "proper map". We now want to extend the functor $D$ to a $3$-functor formalism 
\[
D: Corr(C,E) \to \CAlg(\PrL)
\]
such that for any $f\in P$ the functor $f_!$ is a right adjoint of $f^*$ and for any $f\in I$ the functor $f_!$ is a left adjoint to $f^*$. The construction theorem of Mann states that such an extension to a $3$-functor formalism exists given a certain amount of data (cf. \Cite[Proposition 3.3.3]{heyer20246} for the conditions).\\



In view of this construction theorem, we will begin by discussing categorical Künneth formulas quite abstractly. Let $C$ be a category with finite limits and consider  $C$ with its cartesian monoidal structure. Let  $D: C^{op}\to \CAlg(\PrL)$ be a lax symmetric monoidal functor. Let us recall how to define the morphisms $m_{X,W}$ in this context.
Let $X,W\in C$ be two objects and $p_X : X\times_Z W \to X$ and $p_W : X \times_Z W\to W$ the canonical projections. Then the functors  $p_X^*$ and $p_W^*$ are $D(Z)$-linear and combine to a $D(Z)$-bilinear functor 
\[
D(X)\times D(W) \to D(X \times_Z W), \quad (A, B) \mapsto p_X^*(A) \otimes p_W^*(B).
\]
By the universal property of the Lurie-tensor product, this functor extends to a $D(Z)$-linear functor which we denote by
\[
m_{X,W}: D(X)\otimes_{D(Z)} D(W) \to D(X \times_Z W).
\]

\begin{lemma}\label{naturality of m1}
Let $(C,E)$ be a geometric setup, and consider $C^{op}$ as a symmetric monoidal category with its coCartesian symmetric monoidal structure and $D: C^{op} \to \CAlg(\PrL)$ be a lax symmetric monoidal functor. Let $X\to Z, W \to Z$ and $f: Y\to W$ be any maps in $C$. Then the following diagram commutes
\[\begin{tikzcd}
	{D(X)\otimes_{D(Z)}D(W)} & {D(X)\otimes_{D(Z)}D(Y)} \\
	{D(X\times_Z W)} & {D(X\times_Z Y)}
	\arrow["{id_{D(X)}\otimes f^*}", from=1-1, to=1-2]
	\arrow["{(id_X\times f)^*}", from=2-1, to=2-2]
	\arrow["{m_{X,W}}"', from=1-1, to=2-1]
	\arrow["{m_{X,Y}}", from=1-2, to=2-2]
\end{tikzcd}
\]
\end{lemma}
\begin{proof}
This follows from the lax monoidality of $D: C^{op} \to \CAlg(\PrL)$.
\end{proof}


We introduce the following definition.
\begin{definition}\label{def.künneth6f}
Let $(C,E)$ be a geometric setup, and consider $C^{op}$ as a symmetric monoidal category with its coCartesian symmetric monoidal structure. 
\begin{enumerate}
    \item Let $D: C^{op} \to \CAlg(\PrL)$ be a lax symmetric monoidal functor. Let $f: X\to Z$ be a map in $E$. We call $f$ Künneth if for all objects $W \in C_{/Z}$ the natural morphism 
\[
m_{X,Y} : D(X)\otimes_{D(Z)} D(W) \to  D(X\times_Z W)
\]
is an isomorphism. We say that $D$ satisfies Künneth for a class $P \subset E$ if any morphism $f \in P$ is Künneth. We say $D$ satisfies Künneth if it satisfies Künneth for $P=E$.
\item If $D: Corr(C,E) \to \CAlg(\PrL)$ is a $3$-functor formalism and $f: X \to Y$ in $E$, we call $f$ Künneth if it is Künneth for the functor $C^{op}\to Corr(C,E)\to \CAlg(\PrL)$. 
\end{enumerate}
\end{definition}


\subsection{Monoidal properties of Künneth morphisms}\label{monoidal properties of Künneth}
In this subsection we consider Künneth morphisms for $D: C^{op} \to \CAlg(\PrL)$  a lax symmetric monoidal functor for $(C,E)$ a geometric setup. We show certain closure properties of Künneth morphisms such as stability under composition and base change and give first examples of Künneth morphisms. 

\begin{lemma}\label{dualisbe stable under compposition}
Let $D: C^{op} \to \CAlg(\PrL)$ be a lax symmetric monoidal functor, $f: X \to Y$, $g: Y \to Z$ two morphisms. If $f$ and $g$ are Künneth, then $g\circ f$ is Künneth. 
\end{lemma}
\begin{proof}
 Let $W \to Z$ be any morphism in $C$. Since $f$ is Künneth and using the isomorphism $X \times_Z W \cong X \times_Y (Y\times_Z W)$ we obtain an equivalence 
\[
m_{X,Y\times_ZW}: D(X) \otimes_{D(Y)}D(Y\times_Z W)\cong D(X\times_ZW ).
\]
Using that $g$ is Künneth, we obtain equivalences 
\[
D(X) \otimes_{D(Y)}D(Y\times_Z W)\cong D(X)\otimes_{D(Z)}D(W).
\]
By functoriality of pullbacks and \Cref{naturality of m1} we see that the resulting equivalence $D(X)\otimes_{D(Z)}D(W) \to D(X\times_Z W)$ is given by $m_{X,W}$ which proves that $g\circ f$ is Künneth. 
\end{proof}
\begin{lemma}\label{dualisable stable under bc}
Let $D: C^{op} \to \CAlg(\PrL)$ be a lax symmetric monoidal functor, $f: X \to Z$, $g: Y \to Z$ two morphisms. If $f$ is Künneth, then the base change $f' : X' \coloneq X\times_Z Y \to Y$ of $f$ along $g$ is Künneth.
\end{lemma}
\begin{proof}
 Let $W \to Y$ be an arbitrary morphism. We have a chain of equivalences 
 \begin{align*}
     D(X') \otimes_{D(Y)}D(W) & \cong (D(X)\otimes_{D(Z)}D(Y) )\otimes_{D(Y)}D(W) \\
     & \cong D(X) \otimes_{D(Z)}D(W) \\
     & \cong D(X' \times_Y W)
 \end{align*}
 Here we used that $f$ is Künneth in the first line, the natural isomorphism in the second line and the natural isomorphism $X' \times_Y W \cong X\times_Z W$ together with the fact that $f$ is Künneth in the last line.
\end{proof}

\begin{lemma}\label{dualisbe and cancelation}
Let $D: C^{op} \to \CAlg(\PrL)$ be a lax symmetric monoidal functor, $f: X \to Y$, $g: Y \to Z$ two morphisms. Assume that  $g$ and the diagonal $\Delta_g: Y \to Y\times_Z Y$ are Künneth. If  $g\circ f$ is Künneth, then $f$ is Künneth.
\end{lemma}
\begin{proof}
We write $f$ as the composition $ X \overset{(id,f)}{\to} X \times_Z Y \overset{pr_2}{\to} Y$. By \Cref{dualisbe stable under compposition}, it suffices to show that $(id,f)$ and $pr_2$ are Künneth. The morphism $(id,f)$ is given by the base change of $(f,id): X\times_Z Y \to Y \times_Z Y$  along the diagonal $\Delta_g : Y \to Y\times_Z Y$ an is thus Künneth by \Cref{dualisable stable under bc}. Similarly, the morphism $pr_2$ is the base change of $g \circ f$ along $g$ and is thus Künneth by \Cref{dualisable stable under bc}.
\end{proof}

\begin{lemma}\label{dualisability is stable under product}
Let  $D: C^{op} \to \CAlg(\PrL)$ be a lax symmetric monoidal functor. If  $f: X_1 \to Z$, $g: X_2 \to Z$ are Künneth, then $X_1\times_Z X_2 \to Z$ is Künneth.
\end{lemma}
\begin{proof}
This follows from \Cref{dualisable stable under bc} and \Cref{dualisbe stable under compposition}.
\end{proof}




We will now give first examples of Künneth morphisms. We recall the following definition.
\begin{definition}\Cite[Lemma 6.4, Proposition 6.5]{clausen2022condensed}\label{abstractdef of open and closed in Sym}
Let $F : \mathcal{D} \to \mathcal{C}$ be a map in $\CAlg(\Pr_{st}^L)$.  \\
We call $F$ a categorical open immersion, if $F$ has a fully faithful left adjoint $F_L: \mathcal{C} \to \mathcal{D}$ which satisfies the projection formula, i.e.  for any $M\in \mathcal{C}$ and any $N\in \mathcal{D}$ the natural map 
    \[
   F_L(M\otimes F(N)) \to  F_L(M) \otimes N
    \]
    is an equivalence. 

\end{definition}
\begin{proposition}\Cite[Proposition 6.5]{clausen2022condensed}\label{alternative defclosed open inSym}
Let $F : \mathcal{D} \to \mathcal{C}$ be a map in $\CAlg(\Pr_{st}^L)$. Then $F$ is a categorical open immersion if and only if there is a (necessarily unique) idempotent algebra $A\in \mathcal{D}$ such that $F(A)\cong 0$ and the induced map $\mathcal{D}/\Mod_A\mathcal{D} \to \mathcal{C}$ is an equivalence.

\end{proposition}
\begin{proof}
  Assume that $F$ is a categorical open immersion. We claim that $A\coloneqq \mathrm{cofib}(F_L(1) \to 1)\in \mathcal{D}$ is an idempotent algebra. Indeed, note that $F(A)\cong \mathrm{cofib}(F (F_L(1))\to 1)\cong 0$ since $F_L$ is fully faithful.  We thus have equivalences
\begin{align*}\label{idempotetn algebra etc}
A\otimes A & \cong \mathrm{cofib}(F_L(1)\otimes A \to A)\\
& \cong \mathrm{cofib}(F_L(F(A)) \to A) \\
& \cong \mathrm{cofib}(0 \to A)\cong A
\end{align*}
where the first equivalence follows since cofibers commute with tensor products and the second line follows by the projection formula and the last line follows since $F(A)\cong 0$. Thus $A\in \mathcal{D}$ is an idempotent algebra and it suffices to show that
\[
\mathrm{ker}(F)\cong \Mod_A\mathcal{D}
\]
as full subcategories of $\mathcal{D}$. If $M\in \mathrm{ker}(F)$, then $F_L(1)\otimes M \cong F_L (F(M)) \cong 0$ by the projection formula, so $A \otimes M\cong M $ and thus $M$ lies in $\Mod_A\mathcal{D}$. If $M\in \Mod_A\mathcal{D}$ we have $A \otimes M \cong M$ and thus $F(A \otimes M) \cong F(A) \otimes F(M) \cong 0$ since $F(A)=0$. Conversely, assume that there exists an idempotent algebra $A\in \mathcal{D}$ such that $F :\mathcal{D} \to \mathcal{D}/\Mod_A\mathcal{D} \cong \mathcal{C}$. Then $F(-)\cong \mathrm{fib}(1\to A)\otimes (-) $ and  $F_L$ is the inclusion, so the projection formula is trivially satisfied.
\end{proof}
\begin{lemma}\Cite[Lemma 6.4, Corollary 6.6]{clausen2022condensed}\label{compos bc and cancel for openclosedimmsersion in sym}
\begin{enumerate}
    \item Let $F : \mathcal{D} \to \mathcal{C}$, $G:\mathcal{C}\to \mathcal{B} $ be categorical open immersions, then $G\circ F$ is a categorical open immersion.
\item Let $F : \mathcal{D} \to \mathcal{C}$ be a categorical open immersion, $G: \mathcal{D} \to \mathcal{B}$ be a functor in $\CAlg(\Pr_{st}^L)$. Then the pushout $\tilde{F}: \mathcal{B} \to \mathcal{C}\otimes_{\mathcal{D}}\mathcal{B}$ is again a categorical open immersion. 
\item Let $F : \mathcal{D} \to \mathcal{C}$, $G:\mathcal{C}\to \mathcal{B} $ be maps in $\CAlg(\Pr_{st}^L)$. If $G\circ F$ is a categorical open immersion, then $G$ is a categorical open immersion.
    \item Let $G : \mathcal{D} \to \mathcal{B}$ be a map in $\CAlg(\Pr_{st}^L)$ and consider the pushout diagram in $\CAlg(\Pr_{st}^L)$.
    \[\begin{tikzcd}
	{\mathcal{B}} & {\mathcal{B}\otimes_{\mathcal{D}}\mathcal{C}} \\
	{\mathcal{D}} & {\mathcal{C}}
	\arrow["{\tilde{F}}", from=1-1, to=1-2]
	\arrow["{G}"', from=2-1, to=1-1]
	\arrow["F", from=2-1, to=2-2]
	\arrow["\tilde{G}"', from=2-2, to=1-2]
\end{tikzcd}\]
If  $F$ is a categorical open immersion, the diagram above is left adjointable.
\end{enumerate}
\end{lemma}


\begin{definition}\label{def open immersion for D Cop}
$D: C^{op} \to \CAlg(\Pr_{st}^L)$ be a lax symmetric monoidal functor and $f: X \to Y$ any morphism in $E$. We call $f$ a $D$-closed  (respectively $D$-open) immersion if $D(f)\coloneqq f^* : D(Y) \to D(X)$ is a categorical closed (respectively open) immersion in the sense of \Cref{abstractdef of open and closed in Sym}.
\end{definition}
We will now see that in the context of a $3$-functor formalisms, these notions yield our first examples of Künneth morphisms.
\begin{proposition}\label{open immersion and closed immersions are Künneth}
$D: Corr(C,E) \to \CAlg(\Pr_{st}^L)$ be a $3$-functor formalism and $f: X \to Y$ any morphism in $E$. 
Let $f$ be a $D$-open immersion, then $f$ is Künneth.

\end{proposition}
\begin{proof}
 Let $f: X \to Y$ be a $D$-open immersion and $g: W \to Y$ any morphism in $C$. Let $I \in D(Y)$ be the idempotent algebra associated with $f$ via \Cref{alternative char. of open and proper in aff}. We have a localisation sequence
 \[
 \Mod_ID(Y) \hookrightarrow D(Y) \overset{f^*}{\to} D(X).
 \]
 Now the Lurie-tensor product in $\Mod_{D(Y)}(\PrL)$ preserves localizations (cf.\Cite[Corollary 1.46]{ramzi2024dualizable}), thus after applying $(-)\otimes_{D(Y)}D(W)$ we obtain a localization sequence
\[\begin{tikzcd}
	{\Mod_ID(Y)\otimes_{D(Y)}D(W)} & {D(Y)\otimes_{D(Y)}D(W)} & {D(X)\otimes_{D(Y)}D(W)} \\
	{\Mod_{g^*(I)}D(W)} & {D(W)} & {D(X\times_YW)} \\
	{}
	\arrow[hook, from=1-1, to=1-2]
	\arrow["\wr"', from=1-1, to=2-1]
	\arrow["{f^*\otimes id_{D(W)}}", shift left, from=1-2, to=1-3]
	\arrow["{\wr \ m_{Y,W}}", from=1-2, to=2-2]
	\arrow["{m_{X,W}}", from=1-3, to=2-3]
	\arrow[hook, from=2-1, to=2-2]
	\arrow["{\tilde{f}^*}", from=2-2, to=2-3]
\end{tikzcd}\]
Note that $g^*(I)$ is idempotent so the lower left functor is again fully faithful. The right square commutes by \Cref{naturality of m1}. We need to show $\text{ker}(\tilde{f}^*)\cong \Mod_{g^*(I)}D(W)$ as full subcategories of $D(W)$. The inclusion $\Mod_{g^*(I)}D(W)\subset \text{ker}(\tilde{f}^*)$ is obvious by the commutativity of the diagram. Now let $M \in  \text{ker}(\tilde{f}^*)$, we want to show that $g^*(I) \otimes M \cong M$. This follows by the following calculation
\begin{align*}
    g^*(I) \otimes M & \cong \text{cofib}(g^*f_!(1)\otimes M \to M) \\
    & \cong \text{cofib}(\tilde{f}_!(1)\otimes M \to M) \\
    & \cong \text{cofib}(\tilde{f}_!(\tilde{f}^*(M)) \to M) \\
    & \cong  \text{cofib}(0\to M)\cong M.
\end{align*}
Thus we conclude that $\tilde{f}$ is a $D$-open immersion and since $m_{Y,W}$ and the left vertical morphism are isomorphisms also $m_{X,W}$ is an isomorphism.
\end{proof}

\begin{corollary}\label{bc for open immersions}
 $D: Corr(C,E) \to \CAlg(\PrL)$ be a $3$-functor formalism and  $f: X \to Y$ a $D$-open immersion, $g: Z \to Y$ any morphism in $C$. Then the base change $\tilde{f}: X \times_Y Z \to Z$ is again a $D$-open immersion.
\end{corollary}
\begin{proof}
    This follows by \Cref{open immersion and closed immersions are Künneth} together with point $2$ in \Cref{compos bc and cancel for openclosedimmsersion in sym}.
\end{proof}

\subsection{Dualisable modules in $6$-functor formalisms}\label{Dualisable modules in $6$-functor formalisms}
Now let $D : Corr(C,E) \to \PrL$ be a $6$-functor formalism satisfying Künneth and $f: X \to Z$ a map in $E$. We want to see that $D(X)$ is then automatically dualisable in $\Mod_{D(Z)}\PrL$ via the map $f^*$.

\begin{lemma}\label{funnydiagrams}
Let $(C,E)$ be a geometric setup,  $\pi : X \to Z$ a map in $E$ and $\Delta : X \to X\times_Z X$ the diagonal map. Then the following two compositions  in $\mathrm{Corr}(C,E)$ represent the identity. In particular, $X$ is self-dual in $\mathrm{Corr}(C,E)$.
\[\begin{tikzcd}
	& {X\times_Z X} & {} & {X\times_Z X} \\
	X && {X\times_Z X \times_Z X} & {} & X
	\arrow["{\Delta\times id_X}", from=1-2, to=2-3]
	\arrow["{\pi\times id_X}"', from=1-2, to=2-1]
	\arrow["{id_X \times \Delta}"', from=1-4, to=2-3]
	\arrow["{id_X \times \pi}", from=1-4, to=2-5]
\end{tikzcd}\]
\[\begin{tikzcd}
	& {X\times_Z X} & {} & {X\times_Z X} \\
	X && {X \times_Z X \times_Z X} & {} & X.
	\arrow["{id_X \times \Delta}", from=1-2, to=2-3]
	\arrow["{\Delta \times id_X }"', from=1-4, to=2-3]
	\arrow["{\pi \times id_X}", from=1-4, to=2-5]
	\arrow["{id_X \times \pi}"', from=1-2, to=2-1]
\end{tikzcd}\]

\end{lemma}
\begin{proof}
The first composition is given by 
\[\begin{tikzcd}
	&& X \\
	& {X\times_Z X} & {} & {X\times_Z X} \\
	X && {X\times_Z X \times_Z X} & {} & X.
	\arrow["{\Delta\times id_X}", from=2-2, to=3-3]
	\arrow["{id_X \times \Delta}"', from=2-4, to=3-3]
	\arrow["{id_X \times \pi}", from=2-4, to=3-5]
	\arrow["{\pi\times id_X}"', from=2-2, to=3-1]
	\arrow["\Delta"', from=1-3, to=2-2]
	\arrow["\Delta", from=1-3, to=2-4]
\end{tikzcd}\]
As $(\pi \times id_X)\circ \Delta=id_X= (id_X \times \pi)\circ \Delta$ this proves the claim. The claim for the second composition follows by the same argument.
\end{proof}

\begin{remark}\label{* and ! linearity}
    Let  $D : Corr(C,E) \to \PrL$ be a $6$-functor formalism, $Y,W, Z\in C$ with morphisms $\pi_Y: Y \to Z$ and $\pi_W: W \to Z$ and  $f: Y \to W$ a map in $E$ such that $\pi_Y =\pi_W \circ f$. Then the functors $f_!$ and $f^*$ are $D(Z)$-linear with respect to the $D(Z)$-module structures on $D(Y)$ and $D(W)$ induced by the functors $\pi_Y^*$ and $\pi_W^*$, respectively (cf. \Cite[Lecture 3, Remark 3.13]{6functors}).
\end{remark}

\begin{corollary}\label{D(complicated) equals id}
Let  $D : Corr(C,E) \to \PrL$ be a $6$-functor formalism, $\pi : X \to Z$ a map in $E$ and $\Delta : X \to X\times_Z X$ the diagonal map. Then we have isomorphisms of functors
\[
(id_X \times \pi)_!(id_X \times \Delta)^*(\Delta \times id_X)_!(\pi \times id_X)^* \cong id_{D(X)} \hspace{3mm}, \
(\pi \times id_X)_!(\Delta \times id_X)^*(id_X \times \Delta)_!(id_X \times \pi)^* \cong id_{D(X)}.
\]

\end{corollary}
\begin{proof}
Apply the functor $D(-)$ to the diagrams of \Cref{funnydiagrams} and use that they represent the identity morphisms in $\mathrm{Corr}(C,E)$.
\end{proof}

For  $\pi : X \to Z$ our fixed morphism, we introduce the notation $m \coloneq m_{X,X}$, $m_{12} \coloneq m_{X,X\times_Z X}$, $m_{21}\coloneq m_{X\times_Z X,X}$. As the next two propositions show, in order to deduce that $D(X)$ is dualisable as a $D(Z)$-module, it is sufficient for the map $m$ to be an equivalence.   
\begin{proposition}\label{magicmonsterdiagram proposition}
Let $D : Corr(C,E) \to \PrL$ be a $6$-functor formalism  and let $\pi: X \to Z$ be a morphism in $E$. If the functor 
 \[
 m: D(X)\otimes_{D(Z)}D(X) \to D(X\times_Z X)
 \]
 is an equivalence, then $D(X)$ is self-dual as a $D(Z)$-module.
\end{proposition}
 \begin{proof}
 We introduce the functors
 \begin{align*}
 t \coloneq & \pi_! \Delta^* : D(X\times_Z X) \to D(Z) \\
 u \coloneq & \Delta_! \pi^* : D(Z) \to D(X \times_Z X)
 \end{align*}
 which are $D(Z)$-linear by \Cref{* and ! linearity}. Using the $D(Z)$-linear equivalence  $m$, we can define the evaluation $ev$ and coevaluation $co$ by the diagrams 
\[\begin{tikzcd}
	{D(X\times_Z X)} & {D(Z)} & {D(X\times_Z X)} & {D(X)\otimes_{D(Z)}D(X)} \\
	{D(X)\otimes_{D(Z)}D(X)} && {D(Z)}
	\arrow["t", from=1-1, to=1-2]
	\arrow["m", from=2-1, to=1-1]
	\arrow["ev"', from=2-1, to=1-2]
	\arrow["u", from=2-3, to=1-3]
	\arrow["\sim"', from=1-4, to=1-3]
	\arrow["m", from=1-4, to=1-3]
	\arrow["co"', from=2-3, to=1-4]
\end{tikzcd}\]

\noindent  In order to see that $D(X)$ is dualizable as a $D(Z)$-module, we need to verify that 
 \[
 (ev\otimes id_{D(X)})\circ (id_{D(X)} \otimes co) \cong id_{D(X)}.
 \]
 To see this, consider the following diagram: 
\begin{figure*}[h]
 \tikzcdset{scale cd/.style={every label/.append style={scale=#1},
    cells={nodes={scale=#1}}}}

\begin{flushleft}
\[\begin{tikzcd}[scale cd=0.75]\label{bigdiagram}
	&& {D(X)\otimes_{D(Z)} D(X) \otimes_{D(Z)} D(X)} \\
	\\
	& {D(X)\otimes_{D(Z)}D(X\times_Z X)} && {D(X\times_Z X)\otimes_{D(Z)}D(X)} & {} \\
	{D(X)} && {D(X\times_Z X \times_Z X)} && {D(X)}
	\arrow["{(id \times \Delta)_!(id \times \pi)^*}"', from=4-1, to=4-3]
	\arrow["{(\pi \times id)_!(\Delta \times id)^*}"', from=4-3, to=4-5]
	\arrow["{m_{21}}"'{pos=0.6}, from=3-4, to=4-3]
	\arrow["{t\otimes id_{D(X)}}"'{pos=0.4}, from=3-4, to=4-5]
	\arrow["{m_{12}}"{pos=0.6}, from=3-2, to=4-3]
	\arrow["{id_{D(X)}\otimes u}"'{pos=0.6}, from=4-1, to=3-2]
	\arrow["{id_{D(X)}\otimes co}", curve={height=-30pt}, from=4-1, to=1-3]
	\arrow["{id_{D(X)} \otimes m}"{pos=0.3}, from=1-3, to=3-2]
	\arrow["{ev \otimes id_{D(X)}}", curve={height=-30pt}, from=1-3, to=4-5]
	\arrow["{m\otimes id_{D(X)}}"'{pos=0.3}, from=1-3, to=3-4]
\end{tikzcd}\]
\end{flushleft}
\caption{}
\end{figure*}

Note that we suppressed the equivalence $m_{X,Z}: D(X)\otimes_{D(Z)}D(Z) \cong D(X)$ in the lower left corner and the equivalence $m_{Z,X}: D(Z)\otimes_{D(Z)}D(X)\cong D(X)$ in the lower right corner of Figure 1 above.
First observe that the outer triangles involving $ev$ and $co$ commute by definition of $ev$ and $co$. Next we show the commutativity of the lower triangles, which follows by the naturality of $m$: Indeed, let $f: Y \to W$ be any morphism, then it follows from \Cref{naturality of m1} that the following diagram commutes


\begin{equation}\label{naturality of m 2}
\begin{tikzcd}
	{D(Y)\otimes_{D(Z)}D(X)} & {D(W)\otimes_{D(Z)}D(X)} \\
	{D(Y\times_Z X)} & {D(W\times_Z X).}
	\arrow["{f_! \otimes id_{D(X)}}", from=1-1, to=1-2]
	\arrow["{(f\times id_X)_!}", from=2-1, to=2-2]
	\arrow["{m_{Y,X}}"', from=1-1, to=2-1]
	\arrow["{m_{W,X}}", from=1-2, to=2-2]
\end{tikzcd}
\end{equation}

Combining \Cref{naturality of m1} and \Cref{naturality of m 2}, we see that each of the two lower triangles in Figure 1 commute. Finally, one easily verifies that the central square commutes by definition of the morphisms $m_{i,j}$. Thus we infer that 
\[
(ev\otimes id_{D(X)}) \circ (id_{D(X)} \otimes co) \cong (\pi \times id)_!(\Delta \times id)^*(id \times \Delta)_!(id \times \pi)^*.
\]
But this is isomorphic to the identity by \Cref{D(complicated) equals id}. The other identity 
\[
(id_{D(X)} \otimes ev)\circ (co \otimes id_{D(X)})\cong id_{D(X)}
\]
follows by an analogous argument. This shows that $D(X)\in \Mod_{D(Z)}(\PrL)$ is self-dual.
\end{proof}

\begin{corollary}\label{Künnethmorphismimplies dualisable}
    Let $f: X\to Z$ be Künneth. Then $D(X)\in \Mod_{D(Z)}(\PrL)$ is dualisable.
\end{corollary}
\begin{proof}
    Since $f$ is Künneth we have in particular that the morphism $m_{X,X}: D(X)\otimes_{D(Z)}D(X) \to D(X\times_Z X)$ is an equivalence. The result then follows  by \Cref{magicmonsterdiagram proposition}.
\end{proof}

\begin{corollary}\label{generalequivalence is given by a kernel}
Let $D : Corr(C,E) \to \PrL$ be a $6$-functor formalism  and let $X \to Z$ be a morphism in $C$. Assume that the functor $m: D(X)\otimes_{D(Z)}D(X) \to D(X\times_Z X)$ is an equivalence. Then we have an equivalence 
\[
\mathrm{Fun}_{D(Z)}^{L}(D(X),D(X))\cong D(X \times_Z X).
\]
 \end{corollary}
 \begin{proof}
We claim that there are equivalences
 \[
  \mathrm{Fun}_{D(Z)}^{L}(D(X),D(X))\cong  \mathrm{Fun}_{D(Z)}^{L}(D(Z), D(X)\otimes_{D(Z)} D(X))\cong D(X)\otimes_{D(Z)} D(X).
 \]
Indeed, by \Cref{magicmonsterdiagram proposition}, we know that $D(X)$ is self dual as a $D(Z)$ module which shows the first equivalence. The second equivalence is natural.
 \end{proof}

\subsection{Descent properties of Künneth morphisms}\label{Properties of Künneth morphisms}
 

In this subsection we will be interested in the interaction of the notion of Künneth morphisms with descent of the $6$-functor formalism $D: Corr(C,E) \to \PrL$. We denote by 
\[
D^*: C^{op} \to \PrL
\]
the functor given by $D^*(X)=D(X)$ for $X\in C$ and $D^*(f)=f^*$ for morphisms $f$ in $C$ and
\[
D^!: C_E^{op} \to \Cat
\]
the functor given by $D^!(X)=D(X)$ for $X\in C$ and $D^!(f)=f^!$ for morphisms $f$ in $E$ (c.f \Cite[Definition 3.1.4 and 3.2.3]{heyer20246}). In the following we use the notion of sieves as defined in \Cite[Definition A.4.1]{heyer20246}. 
\begin{definition}
 Let $C,D$ be categories and assume that all limits exist in $D$, let $\mathcal{U} \subset C_{/U}$ a sieve on an object $U\in C$ and $F: C^{op}\to D$ a functor. 
We say $F$ descends along $\mathcal{U}$ if the natural map 
     \[
     F(U) \to \lim_{V \in \mathcal{U}^{op}}F(V)
     \]
     is an isomorphism. We say $F$ descends universally along  $\mathcal{U}$ if it descends along every pullback of $\mathcal{U}$.
\end{definition}
Let us recall the definition of a universal $!$- and $*$-covers. 
\begin{definition}\Cite[Definition 3.4.6]{heyer20246}
 Let $D : Corr(C,E) \to \PrL$ be a  $6$-functor formalism. 
 \begin{enumerate}
     \item We say that a sieve $\mathcal{U} \subset (C_E)_{/U}$ is a small $!$-cover if it is generated by a small family of maps $(U_i \to U)_i$ and the functor $D^!$ descends along $\mathcal{U}$. We say $\mathcal{U}$ is a universal $!$-cover if for every map $V \to U$ in $C$ the family $(U_i\times_U V \to V)_i$ generates a small $!$-cover.
     \item We say that a sieve $\mathcal{U}$ in $C$ is a (universal) $*$-cover if $D^*$ descends (universally) along $\mathcal{U}$.
     \item The $D$-topology is the site on $C$ where covers are canonical covers which have universal $!$-and $*$-descent.
 \end{enumerate}
\end{definition}
For our purposes the following two classes of examples of $!$-covers will be particularly important.

\begin{proposition}\label{smooth and descendable D-covers}
Let $D : Corr(C,E) \to \Pr_{st}^L$ be a stable $6$-functor formalism, $f:X\to Y $ a morphism in $E$. Assume either of the following conditions are satisfied.
\begin{enumerate}
    \item (smooth cover) The object $1_X$ is $f$-smooth and the functor $f^*$ is conservative.
    \item (descendable cover) The object $1_X$ is $f$-proper and $f_*(1_X)\in \CAlg(D(Y))$ is descendable.
\end{enumerate}
Then the arrow $f : X \to Y$ is a universal $!$- and a universal $*$-cover.
\end{proposition}
\begin{proof}
This is \Cite[Proposition 6.18, Proposition 6.19] {6functors}.
\end{proof}
We will call a morphism $f:X\to Y $ in $E$ a smooth (respectively descendable) $!$-cover if it satisfies  condition $1$ (respectively condition $2$) of \Cref{smooth and descendable D-covers}.
As the following proposition shows, for $f: X \to Y$ a descendable or smooth $!$-cover, the category $D(Y)$ can be made more explicit.

\begin{proposition}\Cite[Proposition 3.1.27]{camargo2024analytic}\label{modulcomodul}
    Let $D : Corr(C,E) \to \Pr_{st}^L$ be a stable $6$-functor formalism and  $f: X \to Y$ be a morphism in $E$. 
 \begin{enumerate}   
  \item   If $f$ is a descendable $!$-cover we have 
    \[
    D(Y)\cong \coMod_{f^*f_*}D(X).
    \]
 \item    If $f$ is a smooth $!$-cover we have 
    \[
    D(Y)\cong \Mod_{f^!f_!}D(X).
    \]
\end{enumerate}
\end{proposition}

\begin{definition}\Cite[Definition A.4.5]{heyer20246}
$D : Corr(C,E) \to \PrL$ be a $6$-functor formalism and  $(f_i: {X}_i\to X)_{i\in I}$ be a universal $!$-cover of $X\in C$. We denote by $\Delta_I \to \Delta$ the right fibration associated with the functor $\Delta^{op}\to \Ani ,\  [n]\mapsto I^{n+1}$ by unstraightening. An object in $\Delta_I$ is a pair $([n]\in \Delta,i_\bullet\in I^{n+1})$ and a morphism $([n],i_\bullet)\to ([m],j_\bullet) $ in $\Delta_I$ is a morphism $\alpha : [n] \to [m]$ in $\Delta$ such that $i_k=j_{\alpha(k)}$ for all $k\in [n]$. We denote by $f_{\bullet} : X_{\bullet}\to X$ the Cech-nerve given by morphisms $f_{[n],i_\bullet} : X_{[n],i_\bullet}\coloneqq X_{i_0}\times_X ... \times_X X_{i_n} \to X $.
\end{definition}


We will now show one of our main results, which is that the property of a morphism being Künneth can by checked $!$-locally on the source. 
\begin{proposition}\label{dualisability is !-local on source}
Let $D : Corr(C,E) \to \PrL$ be a $6$-functor formalism and $g: X \to Z$ be any morphism in $E$. Let $(f_i: {X}_i\to X)_{i\in I}$ be a universal $!$-cover. If $g\circ f_{[n],i_\bullet} : X_{i_0}\times_X ... \times_X X_{i_n} \to Z$ is Künneth for all $([n],i_\bullet)\in \Delta_I$, then $g$ is Künneth.
\end{proposition}
\begin{proof}
By $!$-codescend, we have an equivalence $D(X)\cong \colim_{!,([n],i_\bullet)\in \Delta_I} D(X_{[n],i_\bullet})$ in $\PrL$. Using that the forgetful functor $\Mod_{D(Z)}(\Pr^L)\to \PrL$ is conservative and \Cite[Lemma 3.2.5]{heyer20246} (by passing to the induced geometric setup $(C_{/Z},E)$ of the slice category $C_{/Z}$) we see that this is also an equivalence in $\Mod_{D(Z)}(\PrL)$. Let $W \to Z$ be an arbitrary map. We claim that the following equivalences hold
\begin{align*}
    D(X)\otimes_{D(Z)} D(W)& \cong \colim_{!,([n],i_\bullet)\in \Delta_I} D(X_{[n],i_\bullet})\otimes_{D(Z)}D(W) \\
    & \cong \colim_{!,([n],i_\bullet)\in \Delta_I}  D(X_{[n],i_\bullet} \times_Z W) \\
    & \cong  D(X\times_Z W).
\end{align*}
In the first line we used that the functor $D(W) \otimes_{D(Z)} (-)$ commutes with colimits (see \Cref{tensor in Prl commutes with colimits}). The second line follows since the morphisms $X_i \to Z$ are Künneth. The last line follows by $!$-descent by noting that $(g_i :X_i\times_Z W \to X\times_Z W)_{i\in I}$ is a $!$-cover, as it is the base change of the universal $!$-cover $(X_i \to X)_{i\in I}$ and by observing that the morphisms $g_{[n],i_\bullet}: X_{[n],i_\bullet} \times_Z W \to X\times_Z W $ are given by the base change of $f_{[n],i_\bullet} : X_{[n],i_\bullet} \to X$ by $X\times_ZW \to X$. We now show that the equivalences 
\[
m_{{[n],i_\bullet}}: D(X_{[n],i_\bullet})\otimes_{D(Z)}D(W)\cong D(X_{[n],i_\bullet}\times_{Z}W)
\] 

are compatible with the colimit systems: By considering \Cref{naturality of m 2} for the morphims $h : X_{[n],i_\bullet} \to X_{[m],j_\bullet}$ in the colimit system we observe that the following diagram commutes.

\[\begin{tikzcd}
	{D(X_{[n],i_\bullet})\otimes_{D(Z)}D(W)} & {D(X_{[m],j_\bullet})\otimes_{D(Z)}D(W)} \\
	{D(X_{[n],i_\bullet}\times_Z W)} & {D(X_{[m],j_\bullet}\times_Z W).}
	\arrow["{h_{!}\otimes id_{D(W)}}", from=1-1, to=1-2]
	\arrow["{(h \times id_{W})_!}", from=2-1, to=2-2]
	\arrow["{m_{{[n],i_\bullet}}}"', from=1-1, to=2-1]
	\arrow["{m_{[m],j_\bullet}}", from=1-2, to=2-2]
\end{tikzcd}\]

This shows the compatibility of the equivalences $m_{{[n],i_\bullet}}$ with the the colimit systems. 
\end{proof}


Similarly, the notion of a morphism being Künneth can be checked $!$-locally on the target. 
\begin{proposition}
    
\label{dualisability is !-local on target}
 Let $D : Corr(C,E) \to \PrL$ be a $6$-functor formalism and $f: X \to Z$ be any morphism in $E$ and $(g_i: Z_i\to Z)_{i\in I}$ be a universal $!$-cover. Denote by  $f_{([n],i_\bullet)}: X_{([n],i_\bullet)}\coloneqq X \times_Z Z_{[n],i_\bullet} \to Z_{[n],i_\bullet}$ the base change of $f$ by $g_{[n],i_\bullet}: Z_{[n],i_\bullet} \to Z$. If $f_{[n],i_\bullet}$ is Künneth for all $([n],i_\bullet)\in \Delta_I$, then $f$ is Künneth.
\end{proposition}
\begin{proof}
Let $W \to Z$ be an arbitrary map and $W_{([n],i_\bullet)}\coloneqq W \times_Z Z_{[n],i_\bullet} \to Z_{[n],i_\bullet}$ the base change. Using $!$-descent for  the $!$-covers $X_{([n],i_\bullet)}\to X$,$Z_{([n],i_\bullet)}\to Z$, $W_{([n],i_\bullet)}\to W$ we obtain equivalences 
\begin{align*}
    D(X)\otimes_{D(Z)}D(W) \cong &\colim_{!,([n],i_\bullet)\in \Delta_I^{op}}D(X_{([n],i_\bullet)}) \otimes_{\colim_{!,([l],j_\bullet)\in \Delta_I^{op}}D(Z_{([l],j_\bullet)})}\colim_{!,([m],k_\bullet)\in \Delta_I^{op}}D(W_{([m],k_\bullet)}) \\
    \cong & \colim_{!,([n],i_\bullet)\in \Delta_I^{op}}D(X_{([n],i_\bullet)}\times_{Z_{([n],i_\bullet)}} W_{([n],i_\bullet)}) \\
    \cong & \colim_{!,([n],i_\bullet)\in \Delta_I^{op}}D(X\times_Z W \times_Z Z_{([n],i_\bullet)}) \\
    \cong & D(X\times_Z W)
\end{align*}
Here we used $!$-descent for the $!$-covers $X_{([n],i_\bullet)}\to X$,$Z_{([n],i_\bullet)}\to Z$, $W_{([n],i_\bullet)}\to W$ in the first line, the fact that  the category $\Delta_I^{op}$ is sifted in the second line, the fact that $f_{[n],i_\bullet}$ is Küneth for all $([n],i_\bullet)\in \Delta_I$ in the third line and  $!$-descent for $(X\times_Z W \times_Z Z_{([n],i_\bullet)} \to X\times_Z W$ in the last line.
\end{proof}
 In view of our applications to analytic stacks we investigate the following situation of an extension of a $6$-functor formalism.
\begin{proposition}\Cite[Theorem 3.4.11]{heyer20246}\label{E' properties}
    Let $D: Corr(C,E) \to \PrL$ be a sheafy $6$-functor formalism, where $C$ is a subcanoncial site. Then there is a collection of edges $\tilde{E}$ in $\mathcal{X}\coloneqq \Sh(C)$ with the following properties:
\begin{enumerate}[(i)]
\item The inclusion $C \hookrightarrow \mathcal{X}$ defines a morphism of geometric setups $(C,E) \to (\mathcal{X},\tilde{E})$ and $D$ extends uniquely to a $6$-functor formalism $\tilde{D}$ on $ (\mathcal{X},\tilde{E})$.
    \item $\tilde{E}$ is $*$-local on the target: Let $f: X \to Y$ in $C$ be map for which the pullback to any object in $C$ lies in $E$, then $f$ lies in $\tilde{E}$. 
    \item $\tilde{E}$ is $!$-local: Let $f : X \to Y$ be a map which is $!$-local on the source or target in $\tilde{E}$, then $f$ lies in $\tilde{E}$.
    \item $\tilde{E}$ is tame: Every map $f : X \to Y$ in $\tilde{E}$ with $Y\in C$ is $!$-locally on the source in $E$.
\end{enumerate}
Moreover, there is a minimal choice of $\tilde{E}$.
\end{proposition}

\begin{theorem}\label{Künnethextension-main theorem}
    Let $D: Corr(C,E) \to \PrL$ be a sheafy $6$-functor formalism on a subcanonical site $C$. Let $S\in C$ and assume that $D_S: ((C_E){{_{/S}}})^{op} \to \Mod_{D(S)}(\PrL)$ is symmetric monoidal. Then the extension $\tilde{D}_S: ((\Shv(C)_{\tilde{E}}){{_{/S}}})^{op}\to \Mod_{D(S)}(\PrL) $ is symmetric monoidal.
\end{theorem}
\begin{proof}
We review the inductive construction of $\tilde{E}$ in the proof of \Cite[Theorem 3.4.11.]{heyer20246} and show that the property of morphisms being Künneth is preserved under each extension step. Let $E_0$ be the class of morphisms in $\mathcal{X}$ which are representable in $E$ and by $A$ the collections of edges $E'$ such that $E'$ satisfies $(i)$ and $(iv)$ in \Cref{E' properties}. For $E' \in A$, let $(E')'\coloneqq E''$ be the class of morphisms in $\mathcal{X}$ which are $D^{!'}$-locally in $E'$. For $E'\in A$, we denote by $E_S'$ the class of morphisms $X \to S$ in $E'$ and by $Ku(E_S')\subset E_S'$ the class of morphisms $f: X \to S$ in $E'_S$ which are Künneth. Note that $E_0 \in A$. We first observe that $Ku(E_{0,S})=E_{0,S}$. Indeed, by representability we have that any $f : X \to S$ in $E_0$ lies in $E$ since $S\in C$, thus $f$ is Künneth since $D_S: ((C_E){{_{/S}}})^{op} \to \Mod_{D(S)}(\PrL)$ is symmetric monoidal. We show the following claim.
\[
(\ast) \ \text{If $E'\in A$ and $Ku(E_S')= E_S'$, then $Ku(E_S'')= E_S''$}
\]

Indeed, let $f: X \to S$ be an element in $E_S''$. By construction, there exists a small $!-$cover $(X_i\to X)_{i\in I}$ with $g_i:X_i\to X$ in $E'$ and the composition $f\circ g_i \in E'$. By stability under base change and composition of $E'$ we have that all the morphisms $f\circ g_{[n],i_\bullet}: X_{[n],i_\bullet} \to S$ are in $E'$. It thus follows that $f$ is Künneth by \Cref{dualisability is !-local on source} and thus $Ku(E_S'')=E_S''$. \\ For $E'\in A$, we define $E'_!\coloneqq \bigcup_{n\geq2}E^n$ where $E^2\coloneqq E''$ and $E^{n+1}\coloneqq (E^n)'$ for $n \geq 2$. By \Cite[Theorem 3.4.11]{heyer20246} we have that $E'_! \in A$ and satisfies $(iii)$. We have the following claim:

\[
(\ast') \ \text{If $E'\in A$ and $Ku(E_S')= E_S'$, then $Ku({E_{!S}'})= E'_{!S}$}
\]
Indeed, let $f: X \to S$ be in $E'_{!S}$. By definition, we know that $f\in E^n_S$ for some $n\ge2$, thus $f$ is Künneth by applying $(\ast)$ repeatedly. \\


For $E' \in A$ satisfying $(iii)$, we follow Heyer-Mann and define $E'_*$ to be the class of morphisms $X \to Y$ in $\mathcal{X}$ which after pullback under any morphism $U \to Y$ for $U\in C$ lie in $E'$. By \Cite[Theorem 3.4.11]{heyer20246} this defines again an element $E'_* \in A$. Since $E_{*S}'=E_S'$, we obtain the following claim:
\[
(\ast \ast) \ \text{If $E'\in A$  and $Ku(E_S')= E_S'$, then $Ku(E_{*S}')= E_{*S}'$}
\]
By construction we have $\tilde{E}\coloneqq \bigcup_{n \geq 1} \tilde{E}^n$, where $\tilde{E}^1\coloneqq E_{0!*}$ and $\tilde{E}^{n+1}\coloneqq (\tilde{E}^{n})_{!*}$. It follows from $(\ast),(\ast')$ and $(\ast \ast)$ that $Ku(\tilde{E}_S)=\tilde{E}_S$.
\end{proof}
\begin{remark}
\Cref{dualisability is !-local on source} together with \Cref{modulcomodul} can be used to construct dualisable comodule categories: For example, consider a $6$-functor formalism $D$ on $(\Sh(C),\tilde{E})$ as above and assume that the restriction of $D$ to $(C,E)$ satisfies Künneth. Let $G \in \Sh(C)$ be a commutative group object such that $h: G\to \ast$ is $D$-cohomologically proper and the natural map $f: \ast \to \ast/G$ is a descendable $D$-cover. We note that in this case $h_*\cong h_!$ satisfies the projection formula. Then using \Cref{modulcomodul} and the proper base change along the cartesian square 
\[\begin{tikzcd}
	G & \ast \\
	\ast & {\ast/G}
	\arrow["h", from=1-1, to=1-2]
	\arrow[from=1-1, to=2-1]
	\arrow["f", from=1-2, to=2-2]
	\arrow["f"', from=2-1, to=2-2]
\end{tikzcd}\]
we obtain an equivalence $D(\ast/G)\cong \coMod_{h_*(1)}D(\ast)$. In fact, this equivalence is stable under arbitry base change, that is for any $\pi: Y \to \ast$ we obtain an equivalence 
\[
D(\ast/G \times_\ast Y)\cong \coMod_{\pi^*h_*(1)}D(Y).
\]
Using  \Cref{dualisability is !-local on source} we note that we obtain a Künneth formula for comodules: 
\[
\coMod_{h_*(1)}D(\ast) \otimes_{D(\ast)} D(Y)\cong \coMod_{\pi^*h_*(1)}D(Y). 
\]
Although the corresponding statement for smooth $D$-covers reduces to a well-known statement about categories of modules (cf. \Cite[Proposition 4.1]{ben2010integral}) a corresponding abstract statement for comodules is unknown to the author.
\end{remark}

\begin{theorem}\label{Künnethextension-main corollary}
    Let $D: Corr(C,E) \to \PrL$ be a $6$-functor formalism which is Künneth and consider the extension $\tilde{D}: Corr(\Shv(C),\tilde{E})\to \PrL$. Assume that $S\in \Shv(C)$ admits a $!$-cover $(S_i\to S)_{i\in I}$ with $S_i \to S$ in $E_0$ and $S_i \in C$ for all $i\in I$. Then any morphism $f : X \to S$ in $\tilde{E}$ is Künneth.
\end{theorem}
\begin{proof}
Since $S_i \to S$ in $E_0$ we have that $S_{[n],i_\bullet}\in C$ for all $([n],i_\bullet) \in \Delta_I$. The result follows now by \Cref{Künnethextension-main theorem} and \Cref{dualisability is !-local on target}. 
\end{proof}
\begin{remark}
   The previous corollary applies for example if $S \in \Shv(C)_{\tilde{E}/\ast}$ has representable diagonal, $\Delta_S : S \to S \times S \in E_0$. In the context of derived algebraic geometry and with $D(-)$ given by $\QCoh(-)$ this  corresponds to the affineness condition on the diagonal of the base in the definition of perfect stacks (c.f \Cite[Definition. 3.2]{ben2010integral}). One could slightly generalise this to the affineness condition to allow "quasi-affine" diagonals (to reproduce the results of \Cite[Theorem 1.0,3]{stefanich2023tannaka}) using the fact that open immersion are Künneth \Cref{open immersion and closed immersions are Künneth}.
\end{remark}
We include the following lemma
\begin{lemma}\label{fibreproductcommutes with colimits}
    Let $\Sh(C)$ be the category of sheaves of anima on a small site $C$, $Z$ an object in $\Sh(C)$ and $X,Y\in \Sh(C)_{/Z}$ . Then the functor given by taking fibre products
\[
X\times_Z (-) : \Sh(C)_{/Z} \to \Sh(C)_{/Z} ,  \ Y \mapsto X\times_Z Y
\]
commutes with all colimits. In particular, it has a left adjoint which is given by the Hom-stack  
\[
\Hom_Z(X,Y) : C_{/Z} \to \Ani ,  \ U \mapsto \text{Hom}_Z(X \times_Z U,Y).
\]
Here, the category $C_{/Z} \hookrightarrow \Sh(C)_{/Z}$ is given by pullback of the forgetful functor $\Sh(C)_{/Z}\to \Sh(C)$ along the fully faithful inclusion $C \hookrightarrow \Sh(C)$ via Yoneda.
\end{lemma}
\begin{proof}
The fact that the functor $X\times_Z (-)$ commutes with all colimits is immediate as $\Sh(C)$ and hence $\Sh(C)_{/Z}$ are $\infty-$topoi. Since $\Sh(C)_{/Z}$ is presentable, we have a right adjoint $L_X$ which is given on  $U\in C_{/Z}$ by
\begin{align*}
L_X(U)\cong & \text{Hom}_Z(U, L_X) \\
\cong & \text{Hom}_Z(X \times_Z U,Y) \\
\cong & \Hom_Z(X,Y)(U).
\end{align*}
\end{proof}

\begin{lemma}\label{Künneth remains künneth under extension}
  Let $D_0 : Corr(C_0,E_0) \to \PrL$ be a $6$-functor formalism  on a subcanonical site $C_0$ and consider an extension $D : Corr(\Sh(C_0), \tilde{E}_0) \to \PrL$ as in \Cref{E' properties}. Then $f : X \to Y$ in $E_0$ is Künneth with respect to $D_0$ if and only if $f$ is Künneth with respect to $D$.
\end{lemma}
\begin{proof}
If $f$ is Künneth with respect to $D$ then it is also Künneth with respect to $D_0$ since $C_0 \subset \Sh(C_0)$. Conversely, since $f\in E_0$ we obtain $f\in \tilde{E}_0$ by the definition of the extension $\tilde{E}_0$. Let $W \to Z$ be a morphism in $\Sh(C_0)$ and write $D(W) \cong \lim_{n\in J} D_0(W_n)$ with $W_n\in C_0$. By \Cref{Künnethmorphismimplies dualisable} we obtain that $D_0(X) \in \Mod_{D_0(Z)}(\PrL)$ is dualisable.  We claim that there is a chain of equivalences
\begin{align*}
D_0(X) \otimes_{D_0(Y)}D(W) & \cong \lim_{n\in J}D_0(X) \otimes_{D_0(Y)}D_0(W_n) \\
& \cong \lim_{n\in J} D_0(X\times_Y W_n) \\
& \cong D(X \times_Y W).
\end{align*}
Indeed, the first line follows by \Cref{tensor with dualisable object commmutes lim}, the second equivalence follows from the fact that $f$ is Künneth with respect to $D_0$ and the last one follows from \Cref{fibreproductcommutes with colimits}.
\end{proof}

\begin{remark}\label{remark diagobal and rightcancellativenenss}
Let $D : Corr(C,E) \to \PrL$ be a $6$-functor formalism. By \Cref{dualisability is stable under product},\Cref{dualisable stable under bc} and \Cref{dualisability is !-local on source} we see that the class $K$ of Künneth morphisms in $C$ has good properties such as closure under composition, arbitrary base change and $!$-locality on the source. Note however, that $(C,K)$ does not define a geometric setup in the sense of \Cite[Definition 2.1.1]{heyer20246}. The problem is that the class $K$ is not right cancellative in general, or in view of \Cite[Remark 2.1.2]{heyer20246} that for $f: X \to Y$ Künneth the diagonal $\Delta_f: X \to X\times_Y X$ is not necessarily Künneth. 
\end{remark}

\subsection{Consequences: Tannakian reconstruction}\label{Consequences: Tannakian lifting}
Let $D: Corr(C,E)\to \CAlg(\PrL)$ be a $6$ functor formalism and let $S \in C$. It is an interesting question whether one can recover maps $\text{Hom}_{C_{/S}}(X,Y)$ in $C_{/S}$ from the category of morphisms $\Fun_{D(S)}^{L,\otimes}(D(Y),D(X))$ in $\CAlg(\Mod_{D(S)}(\PrL))$. We introduce the following terminology.
\begin{definition}\label{tannakian morphism}
  Let $D: Corr(C,E)\to \CAlg(\PrL)$ be a $6$-functor formalism and $f: X \to S$ in $C$. We call $f$ Tannakian if the functor 
  \[
 \text{Hom}_{C_{/S}}(Y,X) \to \Fun_{D(S)}^{L,\otimes}(D(X),D(Y))
  \]
  is an equivalence for all $Y \in C_{/S}$. We call $D$ Tannakian if all morphisms in $C$ are Tannakian.
\end{definition}
Given a $6$-functor formalism $D: Corr(C,E)\to \CAlg(\PrL)$ on a subcanonical site $C$ which is Tannakian and Künneth, we will investigate which morphisms $f :X \to S$ in the extension $\tilde{D}: Corr(\Shv(C),\tilde{E})\to \CAlg(\PrL)$ given by \Cref{E' properties} are Tannakian.  Our arguments basically follow \Cite[Section 4]{stefanich2023tannaka} by showing that the class of $!$-covers in \Cref{Künnethextension-main corollary} satisfy $2$-descent. More generally we show the following lemma.
\begin{lemma}\label{shrik descent implies 2 descent}
Let $D: Corr(C,E)\to \CAlg(\PrL)$ be a $6$-functor formalism and $f: X' \to X$ in $E$ satisfying Künneth. Then 1) implies 2) implies 3):
\begin{enumerate}
    \item $f$ is a universal $!$-cover
    \item For all $M\in \Mod_{D(X)}\PrL$ we have that the natural map $M \cong \underset{[n]\in \Delta}{\lim}^* M \otimes_{D(X)}D(X'_{[n]})$ is an isomorphism. In particular $f$ satisfies $^*$-descent.
    \item $f$ satisfies $2$-descent, i.e. we have an isomorphism $\Mod_{D(X)}\PrL \cong \lim_{[n]\in \Delta} \Mod_{D(X'_{[n])})}\PrL $.
\end{enumerate}
\end{lemma}
\begin{proof}
We start by showing that 1) implies 2). Let $M\in \Mod_{D(X)}\PrL$, then we have equivalences 
\begin{align*}
M\cong & \Fun_{D(X)}^L(D(X),M) \\
\cong & \Fun_{D(X)}^L(\colim_{[n]\in \Delta^{op}}D(X'_{[n])}),M)\\
\cong & \lim_{[n]\in \Delta}\Fun_{D(X)}^L(D(X'_{[n])}),M)\\ 
\cong & \lim_{[n]\in \Delta}\Fun_{D(X)}^L(D(X), D(X'_{[n])})\otimes_{D(X)}M)\cong \lim_{[n]\in \Delta} M \otimes_{D(X)}D(X'_{[n])})
\end{align*}
Here we used that $f: X' \to X$ and hence also the basechanges $f: X'_{[n]} \to X$ are Künneth, which implies that $D(X'_{[n]})\in \Mod_{D(X)}\PrL$ are self-dual by  \Cref{Künnethmorphismimplies dualisable} in the fourth line. Taking $M=D(X)$ it follows that $f$ satisfies $*$-descent.\\
To see that 2) implies 3), note that the natural functor 
\[
B: \Mod_{D(X)}\PrL \to \lim_{[n]\in \Delta} \Mod_{D(X'_{[n]})}\PrL  \ \ , M \mapsto M_n\coloneqq M \otimes_{D(X)}D(X'_{[n]})
\]
admits a right adjoint given by $\lim_{[n]\in \Delta} \Mod_{D(X'_{[n]})}\PrL \to \Mod_{D(X)}\PrL$, $M_n \mapsto \lim_{[n]\in \Delta} f_{n,*}M_n $ where $f_{n,*}: \Mod_{D(X'_{[n]})}\PrL \to \Mod_{D(X)}\PrL $ is the forgetful functor. By 2), the unit of this adjunction is an equivalence. Thus, it suffices to verify that the counit is an equivalence. Since  $D(X'_{[n]})\in \Mod_{D(X)}\PrL$ are dualisable, it suffices to show that the canonical morphism 
\[
(\lim_{[n]\in \Delta} f_{n,*}M_n) \otimes_{D(X)}D(X'_{[m]}) \to  M_m
\]
is an equivalence. We calculate 
\begin{align*}
  (\lim_{[n]\in \Delta} f_{n,*}M_n) \otimes_{D(X)}D(X'_{[m]}) \cong  \lim_{[n]\in \Delta} (f_{n,*}M_n \otimes_{D(X)}D(X'_{[m]})) \cong M_m. 
\end{align*}
\end{proof}

We can now formulate an extension lemma for Tannkian morphisms. The idea of the proof is adapted from \Cite[Theorem 4.2.1]{stefanich2023tannaka}.
\begin{theorem}\label{tannakian lift}
    Let $C$ be a subcanonical site and $D: Corr(C,E)\to \CAlg(\PrL)$ be a Tannakian sheafy $6$-functor formalism satisfying Künneth. Consider its extension $D: Corr(\Shv(C),\tilde{E})\to \CAlg(\PrL)$. Let $f: X \to S$ be a morphism in $\tilde{E}$ and assume that there is a $!$-cover $g: S'\to S$ with $S'\in C$ and $g\in E_0$. Then $f$ is Tannakian.
\end{theorem}
\begin{proof}
  By iterating the following argument we can assume $S= \ast \in C$ the final object.  Consider the following commutative diagram 
\[\begin{tikzcd}
	{(C_{/X})^{op}} & {\CAlg(\Mod_{D(\ast)}(\PrL))_{D(X)/}} \\
	{C^{op}} & {\CAlg(\Mod_{D(\ast)}(\PrL))}
	\arrow["D", from=1-1, to=1-2]
	\arrow[from=1-1, to=2-1]
	\arrow[from=1-2, to=2-2]
	\arrow["D", hook, from=2-1, to=2-2]
\end{tikzcd}\]
Note that the lower horizontal functor is fully faithful since  $D: Corr(C,E)\to \CAlg(\Mod_{D(\ast)}(\PrL))$ is Tannakian by assumption. It suffices to show that this is a pullback diagram, the morphism 
\[
\text{Hom}_{C/\ast}(Y,X) \to \Fun_{D(\ast)}^{L,\otimes}(D(X),D(Y))
\]
arises as the fibre over $D(X) \in \CAlg(\Mod_{D(\ast)}(\PrL))$ of the left fibration $\CAlg(\Mod_{D(\ast)}(\PrL))_{D(X)/}\to \CAlg(\Mod_{D(\ast)}(\PrL))$.
Let $P$ be the pullback of the diagram, which we can identify as as full subcategory $\Mod_{D(X)}^{\text{aff}}\PrL$ of $\Mod_{D(X)}\PrL$ consisting of those $M\in \Mod_{D(X)}\PrL $ for which $M\cong D(Z)$ for some $Z\in C$. Let 
\[
\Theta : (C_{/X})^{op} \to  \Mod_{D(X)}^{\text{aff}}\PrL
\]
be the induced functor. We want to show it is an equivalence. Note that since $S' \to S'$ satisfies $!$-descent and is Künneth by \Cref{Künnethextension-main corollary}, this is also true for $X'\coloneqq X\times_S S' \to X$ by \Cref{dualisable stable under bc}. By \Cref{shrik descent implies 2 descent} we have $2$-descent for $X' \to X$ and since $g\in E_0$ we see that the equivalence from 3) in \Cref{shrik descent implies 2 descent} descents to an equivalence 
\[
\Mod_{D(X)}^{\text{aff}}\PrL \cong \lim_{[n]\in \Delta} \Mod_{D(X'_{[n]})}^{\text{aff}}\PrL.
\].
We thus obtain a diagram 
\[\begin{tikzcd}
	{(C_{/X})^{op}} && {\CAlg(\Mod_{D(X)}^{\text{aff}}\PrL)} \\
	{\underset{[n]\in \Delta}{\lim}(C_{/X'_{[n]}})^{op}} && {\underset{[n]\in \Delta}{\lim}\CAlg(\Mod_{D(X'_{[n]})}^{\text{aff}}\PrL)}
	\arrow["\Theta", from=1-1, to=1-3]
	\arrow[hook', from=1-1, to=2-1]
	\arrow[hook', from=1-3, to=2-3]
	\arrow["{\underset{[n]\in \Delta}{\lim}\Theta_n}"', from=2-1, to=2-3]
\end{tikzcd}\]
Note that for any $n$, the functor $\Theta_n$ is an equivalence, hence the limit $\lim_n\Theta_n$ is an equivalence. It follows that $\Theta$ is fully faithful. We now show essential surjectivity of $\Theta$. Let $D(X) \to D(Z)\in \CAlg(\Mod_{D(X)}^{\text{aff}}\PrL)$, then by fully faithfullness of the right vertical arrow and the equivalence $\underset{[n]\in \Delta}{\lim}\Theta_n$ we obtain compatible arrows $Z\times_X X'_{[n]} \to  X'_{[n]}$. But this gives an arrow $Z \to X$ by $!$-descent. so $\Theta$ is an equivalence.

\end{proof}

