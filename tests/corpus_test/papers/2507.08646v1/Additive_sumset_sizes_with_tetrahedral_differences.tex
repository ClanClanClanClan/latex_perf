%Additive sumset sizes with tetrahedral differences
%July 10, 2025 


\documentclass{amsart}

 

\usepackage{amsmath,amssymb,amsthm,latexsym}




 
\newtheorem{theorem}{Theorem}
\newcommand{\bt}{\begin{theorem}}
\newcommand{\et}{\end{theorem}}
\newtheorem{lemma}{Lemma}
\newcommand{\bl}{\begin{lemma}}
\newcommand{\el}{\end{lemma}}
\newtheorem{corollary}{Corollary}
\newcommand{\bc}{\begin{corollary}}
\newcommand{\ec}{\end{corollary}}
\newcommand{\bconj}{\begin{conjecture}}
\newcommand{\econj}{\end{conjecture}}
\newtheorem{problem}{Problem}
\newcommand{\bprob}{\begin{problem}}
\newcommand{\eprob}{\end{problem}}
\newcommand{\beq}{\begin{equation}}
\newcommand{\eeq}{\end{equation}}
\newcommand{\benum}{\begin{enumerate}}
\newcommand{\eenum}{\end{enumerate}}
\newcommand{\N}{\ensuremath{ \mathbf N }}
\newcommand{\Z}{\ensuremath{\mathbf Z}}
\newcommand{\Q}{\ensuremath{\mathbf Q}}
\newcommand{\R}{\ensuremath{\mathbf R}}
\newcommand{\Rn}{\ensuremath{\mathbf R^n}}
\newcommand{\C}{\ensuremath{\mathbf C}}
\newcommand{\mbA}{\ensuremath{ \mathbf A}}
\newcommand{\mbB}{\ensuremath{ \mathbf B}}
\newcommand{\mbC}{\ensuremath{ \mathbf C}}
\newcommand{\mbF}{\ensuremath{ \mathbf F}}
\newcommand{\PP}{\ensuremath{ \mathbf P}}


\newcommand{\mca}{\ensuremath{ \mathcal A}}
\newcommand{\mcb}{\ensuremath{ \mathcal B}}
\newcommand{\mcc}{\ensuremath{ \mathcal C}}
\newcommand{\mcd}{\ensuremath{ \mathcal D}}
\newcommand{\mce}{\ensuremath{ \mathcal E}}
\newcommand{\mcf}{\ensuremath{ \mathcal F}}
\newcommand{\mcg}{\ensuremath{ \mathcal G}}
\newcommand{\mch}{\ensuremath{ \mathcal H}}
\newcommand{\mci}{\ensuremath{ \mathcal I}}
\newcommand{\mcj}{\ensuremath{ \mathcal J}}
\newcommand{\mck}{\ensuremath{ \mathcal K}}
\newcommand{\mcl}{\ensuremath{ \mathcal L}}
\newcommand{\mcm}{\ensuremath{ \mathcal M}}
\newcommand{\mcn}{\ensuremath{ \mathcal N}}
\newcommand{\mco}{\ensuremath{ \mathcal O}}
\newcommand{\mcp}{\ensuremath{ \mathcal P}}
\newcommand{\mcr}{\ensuremath{ \mathcal R}}
\newcommand{\mcs}{\ensuremath{ \mathcal S}}
\newcommand{\mct}{\ensuremath{ \mathcal T}}
\newcommand{\mcu}{\ensuremath{ \mathcal U}}
\newcommand{\mcv}{\ensuremath{ \mathcal V}}
\newcommand{\mcw}{\ensuremath{ \mathcal W}}
\newcommand{\mcx}{\ensuremath{ \mathcal X}}

\newcommand{\mba}{\ensuremath{ \mathbf a}}
\newcommand{\mbb}{\ensuremath{ \mathbf b}}
\newcommand{\mbc}{\ensuremath{ \mathbf c}}
\newcommand{\mbe}{\ensuremath{ \mathbf e}}
\newcommand{\mbm}{\ensuremath{ \mathbf m}}
\newcommand{\mbo}{\ensuremath{ \mathbf 0}}
\newcommand{\mbu}{\ensuremath{ \mathbf u}}
\newcommand{\mbv}{\ensuremath{ \mathbf v}}
\newcommand{\mbw}{\ensuremath{ \mathbf w}}
\newcommand{\mbx}{\ensuremath{ \mathbf x}}
\newcommand{\mby}{\ensuremath{ \mathbf y}}
\newcommand{\mbz}{\ensuremath{ \mathbf z}}


\newcommand{\lex}{\ensuremath{\preceq_{\text{lex}}}}
\newcommand{\slex}{\ensuremath{\prec_{\text{lex}}}}

\newcommand{\glex}{\ensuremath{\succeq_{\text{lex}}}}
\newcommand{\sglex}{\ensuremath{\succ_{\text{lex}}}}

\DeclareMathOperator{\supp}{\text{supp}}
\DeclareMathOperator{\card}{\text{card}}
\DeclareMathOperator{\conv}{\text{conv}}
\DeclareMathOperator{\diam}{\text{diam}}
\DeclareMathOperator{\height}{\text{ht}}
\DeclareMathOperator{\id}{id}
\DeclareMathOperator{\kernel}{\text{kernel}}
\DeclareMathOperator{\length}{length}
\DeclareMathOperator{\orbit}{\text{orbit}}
\DeclareMathOperator{\perm}{\text{perm}}
\DeclareMathOperator{\rank}{\text{rank}}
\DeclareMathOperator{\vol}{\text{vol}}

\newcommand{\bmat}{\left(\begin{matrix}}
\newcommand{\emat}{\end{matrix}\right)}
\newcommand{\bsmallmat}{\left(\begin{smallmatrix}}
\newcommand{\esmallmat}{\end{smallmatrix}\right)}
\DeclareMathOperator{\qand}{\quad\text{and}\quad}
\DeclareMathOperator{\qqand}{\qquad\text{and}\qquad}

%\DeclareMathOperator{\supp}{\text{supp}}
\DeclareMathOperator{\support}{\text{support}}
\DeclareMathOperator{\image}{\text{image}}
\DeclareMathOperator{\Perm}{\text{Perm}}
\DeclareMathOperator{\Tet}{\text{Tet}}

\DeclareMathOperator{\vectoran}{\left( \begin{matrix} a_1 \\ \vdots \\ a_n \end{matrix}\right)}
\DeclareMathOperator{\vectorbn}{\left( \begin{matrix} b_1 \\ \vdots \\ b_n \end{matrix}\right)}
\DeclareMathOperator{\vectorcn}{\left( \begin{matrix} c_1 \\ \vdots \\ c_n \end{matrix}\right)}
\DeclareMathOperator{\vectorrn}{\left( \begin{matrix} r_1 \\ \vdots \\ r_n \end{matrix}\right)}
\DeclareMathOperator{\vectorxn}{\left( \begin{matrix} x_1 \\ \vdots \\ x_n \end{matrix}\right)}
\DeclareMathOperator{\vectoryn}{\left( \begin{matrix} y_1 \\ \vdots \\ y_n \end{matrix}\right)}
\DeclareMathOperator{\vectorzn}{\left( \begin{matrix} z_1 \\ \vdots \\ z_n \end{matrix}\right)}
\DeclareMathOperator{\vectorxy}{\left( \begin{matrix} x \\ y \end{matrix} \right)}
\DeclareMathOperator{\vectorxyz}{\left( \begin{matrix} x \\ y \\z \end{matrix} \right)}
\DeclareMathOperator{\vectoram}{\left( \begin{matrix} a_1 \\ \vdots \\ a_m \end{matrix}\right)}
\DeclareMathOperator{\vectorbm}{\left( \begin{matrix} b_1 \\ \vdots \\ b_m \end{matrix}\right)}
\DeclareMathOperator{\vectorcm}{\left( \begin{matrix} c_1 \\ \vdots \\ c_m \end{matrix}\right)}
\DeclareMathOperator{\vectorrm}{\left( \begin{matrix} r_1 \\ \vdots \\ r_m \end{matrix}\right)}
\DeclareMathOperator{\vectorxm}{\left( \begin{matrix} x_1 \\ \vdots \\ x_m \end{matrix}\right)}
\DeclareMathOperator{\vectorym}{\left( \begin{matrix} y_1 \\ \vdots \\ y_m \end{matrix}\right)}
\DeclareMathOperator{\vectorzm}{\left( \begin{matrix} z_1 \\ \vdots \\ z_m \end{matrix}\right)}

\DeclareMathOperator{\vectorad}{\left( \begin{matrix} a_1 \\ \vdots \\ a_d \end{matrix}\right)}
\DeclareMathOperator{\vectorbd}{\left( \begin{matrix} b_1 \\ \vdots \\ b_d \end{matrix}\right)}
\DeclareMathOperator{\vectorcd}{\left( \begin{matrix} c_1 \\ \vdots \\ c_d \end{matrix}\right)}
\DeclareMathOperator{\vectorrd}{\left( \begin{matrix} r_1 \\ \vdots \\ r_d \end{matrix}\right)}
\DeclareMathOperator{\vectorxd}{\left( \begin{matrix} x_1 \\ \vdots \\ x_d \end{matrix}\right)}
\DeclareMathOperator{\vectoryd}{\left( \begin{matrix} y_1 \\ \vdots \\ y_d \end{matrix}\right)}
\DeclareMathOperator{\vectorzd}{\left( \begin{matrix} z_1 \\ \vdots \\ z_d \end{matrix}\right)}


\DeclareMathOperator{\vectorch}{\left( \begin{matrix} c_1 \\ \vdots \\ c_h \end{matrix}\right)}
\DeclareMathOperator{\vectorsmallch}{\left( \begin{smallmatrix} c_1 \\ \vdots \\ c_h\end{smallmatrix}\right)}


\DeclareMathOperator{\vectorzero}{\left( \begin{matrix} 0 \\ \vdots \\ 0 \end{matrix} \right)}
\DeclareMathOperator{\vectorone}{\left( \begin{matrix} 1 \\ \vdots \\ 1 \end{matrix} \right)}

\DeclareMathOperator{\vectorsmallan}{\left( \begin{smallmatrix} a_1 \\ \vdots \\ a_n \end{smallmatrix}\right)}
\DeclareMathOperator{\vectorsmallbn}{\left( \begin{smallmatrix} b_1 \\ \vdots \\ b_n \end{smallmatrix}\right)}
\DeclareMathOperator{\vectorsmallcn}{\left( \begin{smallmatrix} c_1 \\ \vdots \\ c_n \end{smallmatrix}\right)}
\DeclareMathOperator{\vectorsmallrn}{\left( \begin{smallmatrix} r_1 \\ \vdots \\ r_n \end{smallmatrix}\right)}
\DeclareMathOperator{\vectorsmallun}{\left( \begin{smallmatrix} u_1 \\ \vdots \\ u_n \end{smallmatrix}\right)}
\DeclareMathOperator{\vectorsmallvn}{\left( \begin{smallmatrix} v_1 \\ \vdots \\ v_n \end{smallmatrix}\right)}
\DeclareMathOperator{\vectorsmallwn}{\left( \begin{smallmatrix} w_1 \\ \vdots \\ w_n \end{smallmatrix}\right)}
\DeclareMathOperator{\vectorsmallxn}{\left( \begin{smallmatrix} x_1 \\ \vdots \\ x_n \end{smallmatrix}\right)}
\DeclareMathOperator{\vectorsmallyn}{\left( \begin{smallmatrix} y_1 \\ \vdots \\ y_n \end{smallmatrix}\right)}
\DeclareMathOperator{\vectorsmallzn}{\left( \begin{smallmatrix} z_1 \\ \vdots \\ z_n \end{smallmatrix}\right)}
\DeclareMathOperator{\vectorsmallxy}{\left( \begin{smallmatrix} x \\ y \end{smallmatrix} \right)}
\DeclareMathOperator{\vectorsmallxyz}{\left( \begin{smallmatrix} x \\ y \\z \end{smallmatrix} \right)}

\DeclareMathOperator{\vectorsmallam}{\left( \begin{smallmatrix} a_1 \\ \vdots \\ a_m \end{smallmatrix}\right)}
\DeclareMathOperator{\vectorsmallbm}{\left( \begin{smallmatrix} b_1 \\ \vdots \\ b_m \end{smallmatrix}\right)}
\DeclareMathOperator{\vectorsmallcm}{\left( \begin{smallmatrix} c_1 \\ \vdots \\ c_m \end{smallmatrix}\right)}
\DeclareMathOperator{\vectorsmallrm}{\left( \begin{smallmatrix} r_1 \\ \vdots \\ r_m \end{smallmatrix}\right)}
\DeclareMathOperator{\vectorsmallxm}{\left( \begin{smallmatrix} x_1 \\ \vdots \\ x_m \end{smallmatrix}\right)}
\DeclareMathOperator{\vectorsmallym}{\left( \begin{smallmatrix} y_1 \\ \vdots \\ y_m \end{smallmatrix}\right)}

\DeclareMathOperator{\vectorsmallzero}{\left( \begin{smallmatrix} 0 \\ \vdots \\ 0 \end{smallmatrix} \right)}
\DeclareMathOperator{\vectorsmallone}{\left( \begin{smallmatrix} 1 \\ \vdots \\ 1 \end{smallmatrix} \right)}

\DeclareMathOperator{\vector02}{\left( \begin{smallmatrix} 0 \\ 0\end{smallmatrix} \right)}
\DeclareMathOperator{\vector03}{\left( \begin{smallmatrix} 0 \\ 0 \\ 0 \end{smallmatrix} \right)}




 \title[Tetrahedral differences]{Additive sumset sizes with tetrahedral differences} 
\author{Melvyn B.  Nathanson}
\address{Department of Mathematics\\Lehman College (CUNY)\\Bronx, NY 10468}
\email{melvyn.nathanson@lehman.cuny.edu}

\date{\today}


\subjclass[2000]{11B13, 11B05, 11B75,  11P70, 11Y16, 11Y55}
\keywords{Sumset, sumset size, popular sumset size, tetrahedral number, 
additive number theory, combinatorial number theory}
\thanks{Supported in part by  PSC-CUNY Research Award Program grant 66197-00 54.}

\begin{document}
 

\begin{abstract}
Experimental calculations suggest that the $h$-fold sumset 
sizes of 4-element sets of integers are concentrated at $h$ numbers 
that are differences of tetrahedral numbers.  In this paper it is proved 
that these ``popular'' sumset sizes exist and explicit $h$-adically defined 
sets are constructed for each of these numbers. 
\end{abstract}



 \maketitle


\section{The sumset size problem}
The \emph{$h$-fold sum} of a set $A$ of integers is the set of all sums of $h$ 
not necessarily distinct elements of $A$. 
The core problem of additive number theory is to understand $h$-fold sumsets.  

If $A$ is a finite set of $k$ integers, then $hA$ is a finite set and 
\beq          \label{h-adic:basicIneq}
h(k-1)+1 \leq |hA| \leq \binom{h+k-1}{k-1}.
\eeq
We have $|hA| = h(k-1)+1$ if and only if $A$ is an arithmetic progression of length $k$, 
and $|hA| = \binom{h+k-1}{k-1}$ if and only if $A$ is a $B_h$-set, that is, a set 
such that every integer in the sumset $hA$ has a unique representation 
(up to permutation of the summands) as a sum of $h$ 
not necessarily distinct elements of $A$. 

The \emph{integer interval} defined by real numbers $u$ and $v$ is the set 
\[
[u,v] = \{n \in \Z: u \leq n \leq v\}.
\]
The \emph{integer part} of the real number $u$ is denoted $[u]$.

Let $\mcr_{\Z}(h,k)$ be the set of $h$-fold sumset sizes of sets of size $k$, that is, 
\[
\mcr_{\Z}(h,k) = \left\{ |hA|:A\subseteq \Z \text{ and } |A|=k \right\}.
\]
Inequality~\eqref{h-adic:basicIneq} implies 
\beq          \label{h-adic:basicSubset}
\mcr_{\Z}(h,k)  \subseteq \left[ h(k-1)+1, \binom{h+k-1}{k-1} \right].
\eeq

Not every possible sumset size is actually the size of a sumset.  
For example, relation~\eqref{h-adic:basicSubset} gives  
\[
\mcr_{\Z}(3,3)  \subseteq \left[ 7, 10 \right].
\]
We have 
\[
3\{0,1,2\} = \{0,1,2,3,4,5,6\} \qqand |3\{0,1,2\}| = 7
\]
\[
3\{0,1,3\} = \{0,1,2,3,4,5,6,7,9\} \qqand |3\{0,1,3\}| = 9
\]
\[
3\{0,1,4\} = \{0,1,2,3,4,5,6,8,9,12\} \qqand |3\{0,1,4\}| = 10 
\]
and so 
\[
 \{7,9,10\} \subseteq  \mcr_{\Z}(3,3).  
\]
However, there exists no set $A$ of integers 
with $|A| = 3$ and $|3A|=8$ (Nathanson~\cite{nath25bb}).  
Thus,  
\[
\mcr_{\Z}(3,3) = \{7,9,10\}.  
\]

This example motivates the following  problem: 
For all positive integers $h$ and $k$, determine the full range 
of sumset sizes of $h$-fold sums of sets of 
$k$ integers, that is, compute the set $\mcr_{\Z}(h,k)$.\footnote{There is the 
analogous problem in every additive abelian group or semigroup $G$:  Determine 
the set $\mcr_{G}(h,k)$ of the sizes of $h$-fold sums of $k$-element subsets of $G$.}
For all $h$ and $k$, we have 
\[
\mcr_{\Z}(h,1) =  \{1\} \qqand \mcr_{\Z}(1,k) =  \{k\}.
\]
Sets $A$ and $ B$ are \emph{affinely equivalent} if there exist numbers 
$\lambda \neq 0$ and $\mu$ 
such that $\lambda \ast A + B = \{\lambda a + \mu: a \in A\}$.  If $A$ and $B$ are affinely equivalent, 
then $|hA| = |hB|$ for all positive integers $h$.  
Every finite set $A$ of integers is affinely equivalent to a set $B$ with $\min B = 0$ 
and $\gcd(B) = 1$.  In particular, every set of size 2 is affinely equivalent to the set $\{0,1\}$.
It follows that 
\[
\mcr_{\Z}(h,2) =  \{h+1\}. 
\]
Erd\H os and Szemer\' edi~\cite{erdo-szem83} stated that 
\[
\mcr_{\Z}(2,k) = \left[ 2k-1, \binom{k+1}{2} \right].
\]
(A proof is in~\cite{nath25bb}.)  
Thus, the unsolved problem is to determine $\mcr_{\Z}(h,k)$ 
for $h \geq 3$ and $k \geq 3$. 

A first step is to fix a positive integer $k$ and find the possible sizes 
of $h$-fold sums of sets of size $k$.  
Recall that the \emph{$j$th triangular number} 
$f_2^j = \binom{j+1}{2}$ is the sum of the first $j$ positive integers. 
The \emph{$j$th tetrahedral number} 
$f_3^j = \binom{j+2}{3}$ is the sum of the first $j$ triangular numbers (Dickson~\cite{dick20}).  
For $k = 3$, Nathanson~\cite{nath25bb} proved that 
\[
\mcr_{\Z}(h,3) = \left\{\binom{h+2}{2} - \binom{i_0+1}{2} : i_0 \in [0,h-1] \right\}.
\]
Thus, every sumset size of a 3-element set is 
of the form $f_2^{h+1} -  f_2^{i_0}$,  
that is, a difference of triangular numbers.
For $k \geq 4$, the problem is still open.  
Numerical experiments (Nathanson~\cite{nath25q} 
and O'Bryant~\cite{obry25}) suggest that, for $k = 4$, 
the ``most popular'' sumset sizes are the integers 
\[
f_3^{h+1} -  f_3^{i_0} = \binom{h+3}{3} - \binom{i_0+2}{3} 
\]
for $i_0 \in [0,h-1]$.    
These are the differences between the tetrahedral 
number  $f_3^{h+1}  = \binom{h+3}{3}$, 
which is also the size of a 4-element $B_h$-set, 
and the $h$ consecutive tetrahedral numbers 
$f_3^{0} ,  f_3^{1} ,\ldots, f_3^{h-1}$. 
It had been an open problem to decide if the  integers 
$f_3^{h+1}  - f_3^{i_0}  $ are, indeed,    
sumset sizes for all $h \geq 2$ and $i_0 \in [0,h-1]$.  
The goal of this paper is to prove that these sumset sizes do exist 
for all $h$ and $i_0$, that is, 
\[
\left\{\binom{h+3}{3} - \binom{i_0+2}{3} : i_0 \in [0,h-1] \right\} 
\subseteq \mcr_{\Z}(h,4)
\]
and to construct explicit $h$-adically defined sets with 
exactly these sumset sizes. 

 For related work on sumset size problems in additive number theory, 
 see~\cite{fox-krav-zhan25}--\cite{schi25}. 
 


\section{A family of $h$-adic sets} 
 
\bt             \label{h-adic:theorem:main} 
For all $h \geq 2$ and  $i_0 \in [0,h-1]$, let 
\[
c = (h + 1 - i_0)(h+1). 
\]
The set  
\begin{align*}
A & = \{0,1,h+1,c\} 
\end{align*}
satisfies $|A|=4$ and 
\[
|hA| = \binom{h+3}{3} - \binom{i_0+2}{3}.
\]
\et


\begin{proof}  
The set 
\[
B  = \{0,1,h+1\}  
\]
is a $B_{h}$-set  and so $B$ is a $B_{h-i}$-set for all $i \in [0,h-1]$ and 
\[
|(h-i)B| = \binom{h-i+2}{2}.
\]
Let  $0B = \{0\}$. 

We have 
\begin{align*}
A  = B \cup \{c\} & = \{0,1,h+1, (h+1-i_0)(h+1)\}.
\end{align*} 
The inequality $h+1-i_0 \geq 2$ implies $|A| = 4$.  

If $i_0 = 0$, then $c = (h+1)^2$ and 
\[
A = \left\{ 0,1,h+1, (h+1)^2\right\}. 
\]
The uniqueness of $(h+1)$-adic representations implies that 
$A$ is a $B_4$-set and so 
\[
|hA| = \binom{h+3}{3} = \binom{h+3}{3} - \binom{i_0+2}{3}.
\]

Let $i_0 \in [1,h-1]$.  
We decompose the sumset $hA$ as follows:  
\[
hA  = \bigcup_{i=0}^{h} \left( (h-i)B+ic\right)  = \bigcup_{i=0}^{h} L_i 
\]
where 
\begin{align}
L_i   & = (h-i)B+ic        \nonumber  \\ 
& = \bigcup_{j=0}^{h-i}\left(   (h-i-j)(h+1)  + [0,j] \right) + ic     \nonumber \\ 
& = \bigcup_{j=0}^{h-i} \left(  (h+ (h-i_0)i-j) (h+1) + [0,j] \right).            \label{h-adic:Li}
\end{align} 
We have 
\beq            \label{h-adic:Li-size}
|L_i | = |(h-i)B+ic| = |(h-i)B| = \binom{h-i+2}{2}. 
\eeq 

The set $L_i$ is the union of $h-i+1$ pairwise 
disjoint integer intervals whose smallest elements are multiples of $h+1$ 
and whose lengths are at most $h$.  
It follows that if $n \in L_i$ and $n = q(h+1) + r$ with $r \in [0,h]$, 
then $q = h+ (h-i_0)i-j$ for some $j \in [0,h-i]$ and $r \in [0,j]$.
Then $L_i$ contains the integer interval $q(h+1) + [0, j]$.

For all $i \in [0,h-1]$, we have 
\begin{align*} 
\min\left(L_i\right)  & =  ic  <  (i+1)c  = \min\left( L_{i+1} \right)  
\end{align*} 
and 
\begin{align*} 
\max\left(L_i \right) & = (h-i)(h+1) + ic  \\
& < (h-i-1)(h+1) + (i+1)c \\
& = \max\left( L_{i+1} \right)  
\end{align*} 
and so the sets $L_i$ ``move to  the right'' as $i$ increases 
from 0 to $h$. 
Moreover, 
\[
\max\left( L_i \right) <  \min\left(L_{i+1} \right)  
\]
if and only if 
\[
 (h-i)(h+1)  + ic <  (i+1)c
\]
if and only if 
\[
i > h - \frac{c}{h+1} 
\]
if and only if 
\[
i \geq 1 + \left[  h - \frac{c}{h+1} \right] = i_0.  
\]
Thus, the sets $ L_i$ and $L_j$ are disjoint if $i_0 \leq i < j \leq h$ 
and 
\[
\left| \bigcup_{i=i_0+1}^h L_i \right| = \sum_{i=i_0+1}^h  \left|  L_i \right| 
= \sum_{i=i_0+1}^h  \binom{h-i+2}{2}. 
\]
Because the sets $L_i$ move to the right, we have 
\[
\left( \bigcup_{i=0}^{i_0} L_i  \right) \cap \left( \bigcup_{i=i_0+1}^h L_i \right) = \emptyset    
\] 
and 
\begin{align}                   \label{h-adic:PartialSum}
|hA| & = \left| \bigcup_{i=0}^h L_i \right| 
= \left|  \bigcup_{i=0}^{i_0} L_i \right| + \left|  \bigcup_{i=i_0+1}^{h} L_i \right|  
\nonumber  \\
& = \left|  \bigcup_{i=0}^{i_0} L_i \right| +  \sum_{i=i_0+1}^h  \binom{h-i+2}{2}. \end{align}



We shall compute $L_i \cap L_{i+t} $ for all $i \in [1,h-1]$ and $t \in [1,h-i]$.  
Relation~\eqref{h-adic:Li} implies 
\[
L_{i+t} =  \bigcup_{j=0}^{h-i-t} \left(  (h+ (h-i_0)(i+t)-j) (h+1) + [0,j] \right).  
\]
We have $q(h+1) \in L_i \cap L_{i+t} $ if and only if there exist $j_0 \in [0,h-i]$ 
and $j_t \in [0,h-i-t]$ such that 
\[
q = h+ (h-i_0)i-j_0 = h+ (h-i_0)(i+t)-j_t  
\]
if and only if 
\begin{align*}
j_0 & = j_t - (h-i_0)t \\
&  \in [0,h-i] \cap [-(h-i_0)t, h-i - t - (h-i_0)t] \\ 
& = [0,  h-i - t -(h - i_0)t].  
\end{align*}
It follows that 
\beq           \label{h-adic:Lt-intersect}
L_i \cap L_{i+t} = \bigcup_{j=0}^{h-i - t - (h - i_0)t} 
\left(  (h+ (h-i_0)i-j) (h+1) + [0,j] \right) 
\eeq
and so 
\begin{align*}
\left|  L_i \cap L_{i+t}\right| 
& = \left|  \bigcup_{j=0}^{h-i - t -(h -i_0)t} 
\left(  (h+ (h-i_0)i-j) (h+1) + [0,j] \right)\right| \\ 
& =  \sum_{j=0}^{h-i - t -(h -i_0)t} 
\left| (h+ (h-i_0)i-j) (h+1) + [0,j] \right| \\
& = \sum_{j=0}^{h-i - t - (h -i_0)t} (j+1) \\
& = \binom{h-i - t - (h - i_0)t+2}{2}.
\end{align*}
In particular, 
\beq                  \label{h-adic:L1-intersect}
\left|  L_i \cap L_{i+1}\right| = \binom{i_0+1-i}{2}.
\eeq
Relation~\eqref{h-adic:Lt-intersect} also implies that, for $t \in [1,h-i]$,  
\[
L_i \setminus L_{i+t} = \bigcup_{j=h-i - t -(h - i_0)t +1}^{h-i}
\left(  (h+ (h-i_0)i-j) (h+1) + [0,j] \right) 
\]
and so 
\[
L_i \setminus  L_{i+1} \subseteq L_i \setminus  L_{i+2} \subseteq \cdots \subseteq 
L_i \setminus  L_{h}.
 \]
Therefore, 
\begin{align*}
L_i \setminus \left(\bigcup_{t = 1}^{i_0-i} L_{i+t} \right) 
& = L_i \cap \left( \bigcup_{t = 1}^{i_0-i} L_{i+t} \right)^c 
 = L_i \cap \left(\bigcap_{t = 1}^{i_0 -i} L_{i+t}^c \right) \\
& = \bigcap_{t = 1}^{i_0 -i} \left( L_i \cap  L_{i+t}^c \right) 
 = \bigcap_{t = 1}^{i_0 -i} \left( L_i \setminus  L_{i+t}  \right) \\
&  = L_i \setminus  L_{i+1}.
\end{align*}
%%%%%%%%%%%%%%%%%%%%%%%
%%%%%%%%%%%%%%%%%%%%%%%
The sets 
\[
L_i \setminus \left( \bigcup_{t = 1}^{i_0-i} L_{i+t} \right) 
\]
are pairwise disjoint for $i \in [0,i_0]$ and 
\[
 \bigcup_{i=0}^{i_0} L_i 
 = L_{i_0} \cup  \bigcup_{i=0}^{i_0-1} \left( L_i \setminus \bigcup_{t = 1}^{i_0-i} L_{i+t}  \right).
\]
From~\eqref{h-adic:Li-size} and~\eqref{h-adic:L1-intersect}, we obtain  
\begin{align*}        
\left| \bigcup_{i=0}^{i_0} L_i \right| 
& = | L_{i_0}| + \sum_{i=0}^{i_0-1}  \left|   L_i \setminus \bigcup_{t = 1}^{i_0-i} L_{i+t} \right| \\ 
&  = | L_{i_0}| +  \sum_{i=0}^{i_0-1}  \left| L_i \setminus  L_{i+1}\right|  \\ 
& =  | L_{i_0}| + \sum_{i=0}^{i_0-1} \left(\left| L_i \right|  -  \left| L_i \cap L_{i+1}\right| \right) \\
&  = \sum_{i=0}^{i_0} \left| L_i \right|  
 -  \sum_{i=0}^{i_0 -1}  \left| L_i \cap L_{i+1}\right| \\
& = \sum_{i=0}^{i_0} \binom{h-i+2}{2} - \sum_{i=0}^{i_0-1}  \binom{i_0+1-i}{2}.
\end{align*}
Relation~\eqref{h-adic:PartialSum} gives 
\begin{align*} 
|hA| 
& =  \left|  \bigcup_{i=0}^{i_0} L_i \right| +  \sum_{i=i_0+1}^h  \binom{h-i+2}{2} \\ 
& =   \sum_{i=0}^h  \binom{h-i+2}{2} -  \sum_{i=0}^{i_0 -1}  \binom{i_0+1-i}{2} \\
& =   \sum_{i=0}^h  \binom{ i+2}{2} -  \sum_{i=0}^{i_0-1}  \binom{i+2}{2} \\
& =  \binom{ h+3}{3} - \binom{i_0+2}{3}. 
\end{align*}
This completes the proof. 
 \end{proof}
 
 
 
 
 
\section{Open problems} 
\bprob
This paper considers the special class of $h$-adically defined 4-element sets 
\[
A   = \{0,1,h+1,h^2+h+1-p \}
\]
with 
\[
p = 1 + (i_0-1)(h+1) 
\] 
and $i_0 \in [0,h-1]$.  
It is of interest to compute, for all $h \geq 3$ and all $p\in [0,h^2-1\}$, 
the $h$-fold sumset sizes of the sets  
\[
A = \{0,1,h+1,h^2+h+1-p\}. 
\]
\eprob

\bprob
For all $h \geq 3$, compute the set of $h$-fold sumset sizes of the sets 
\[
A = \{0,1,a,b\}
\]
for $2 \leq a \leq h$ and $a+1 \leq b \leq ha+1$.
\eprob


\bprob 
A next step is to determine the popular sumset sizes of 5-element sets of integers.  
The fundamental problem is to obtain a complete description 
of the sumset size set $\mcr_{\Z}(h,k)$ 
for all positive integers $h$ and $k$, 
to explain the distribution of sumset sizes for fixed $h$ and $k$, 
 and to understand why some numbers cannot be sumset sizes.  
 A solution to this problem might be called the 
\emph{Second Fundamental Theorem of Additive Number Theory}. 
\eprob





\begin{thebibliography}{99}

\bibitem{dick20}
Leonard Eugene Dickson, \emph{History of the Theory of Numbers}, 
Volume II, 
Carnegie Institute of Washington, 1920.
Reprinted by Chelsea Publishing Company, New York, 1971.


\bibitem{erdo-szem83}
P. Erd\H os and E. Szemer\' edi, On sums and products of integers, in:
Studies in Pure Mathematics, Birkh\" auser, Basel-Boston, 1983, pp. 213--218.

\bibitem{fox-krav-zhan25}
J. Fox, N. Kravitz, and S. Zhang,
Finer control on relative sizes of iterated susmets,
preprint, 2025. 


\bibitem{hegy96}
N. Hegyv{\' a}ri, 
On representation problems in the additive number theory, 
Acta Math. Hungar. 72 (1996), 35--44. 

\bibitem{krav25}
N. Kravitz, Relative sizes of iterated sumsets, 
J. Number Theory 272 (2025), 113--128, 



\bibitem{nath25aa}
M.  B. Nathanson, Inverse problems for sumset sizes of finite sets of integers, 
Fibonacci Quarterly (2025), to appear. 
arXiv:2411.02365.

\bibitem{nath25bb}
M.  B. Nathanson, Problems in additive number theory, VI:  
Sizes of sumsets of finite sets, 
Acta Math. Hungar., to appear.
arXiv:2411.02365, 2024.

\bibitem{nath25cc}
M.  B. Nathanson, 
Explicit sumset sizes in additive number theory, \\
arXiv: 2505.05329, 2025.

\bibitem{nath25}
M.  B. Nathanson, 
Compression and complexity for sumset sizes in additive number theory,
arXiv:2505.20998. 

\bibitem{nath25q}
M.  B. Nathanson, Triangular and tetrahedral number differences of sumset sizes 
in additive number theory, 
arXiv:2506.15015.


\bibitem{obry25}
K. O'Bryant, 
On Nathanson's triangular number phenomenon, 
arXiv:2506.20836. 

\bibitem{peri-roto25}
P. P\' eringuey and A. de Roton, 
A note on iterated sumsets races,
arXiv:2505.11233.  


\bibitem{schi25}
V. Schinina, On the sumset  of sets of size $k$, 
arXiv:2505.07679.  


\end{thebibliography}



\end{document}
%%%%%%%%%%%%%%%%%%%%%%%%%%%%%%%%%%%%%%%%%%%%%%%%%%%%%%%%%%%%%%%%%%%%%%%%%%%%%%%%%%%%%%%%%%%%%%%%%%%%%%%%%%%%%%%%%%%%%%%%%%%%%%%%%%%%%%%%%%%%%%%%%%%%%%%%%%%%%%%%%%%%%%%%%%%%%%%%%%%%%%%%%%%%%%%%%%%%%%%%%%%%%%%%%%%%%%%%%%%%%%%%%%%%%%%%%%%%%%%%%%%%%%%%%%%%%%%%%%%%%%%%%%%%%%%%%%%%%%%%%%%%%%%%%%%%%%%%%%%%%%%%%%%%%%%%%%%%%%%%%%%%%%%%%%%%%%%%%%%%%%%%%%%%%%%%%%%%%%%%%%%%%%%%%%%%%%%%%%%%%%%%%%%%%%%%%%%%%%%%%%%%%%%%%%%%%%%%%%%%%%%