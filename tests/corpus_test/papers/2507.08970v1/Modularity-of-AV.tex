%%%%%%%%%%%%%%%%%%%%%%%%%%%%%%%%%%%%%%%%%%%%%%%%%%%%%%%%%%%%%%%%%%%%%%%%%%%%%%%%%%%%%%%%%%%%%%%%%%%%%%%%%%%%%%%%%%%%%%%%%%%%%%%%%%%%%%%%%%%%%%%%%%%%%%%%%%%%%%%%%%%%%%%%%%%%%%%%%%%%%%%%%%%%%%
%%%%%%%%%%%%%%%%%%%%%%%%%%%%%%%%%%%%%%%%%%%%%%%%%%%%%%%%%%%%%%%%%%%%%%%%%%%%%%%%%%%%%%%%%%%%%%%%%%%%%%%%%%%%%%%%%%%%%%%%%%%%%%%%%%%%%%%%%%%%%%%%%%%%%%%%%%%%%%%%%%%%%%%%%%%%%%%%%%%%%%%%%%%%%%
%%%%%%%%%%%%%%%%%%%%%%%%%%%%%%%%%%%%%%%%%%%%%%%%%%%%%%%%%%%%%%%%%%%%%%%%%%%%%%%%%%%%%%%%%%%%%%%%%%%%%%%%%%%%%%%%%%%%%%%%%%%%%%%%%%%%%%%%%%%%%%%%%%%%%%%%%%%%%%%%%%%%%%%%%%%%%%%%%%%%%%%%%%%%%%
%%%%%%%%%%%%%%%%%%%%%%%%%%%%%%%%%%%%%%%%%%%%%%%%%%%%%%%%%%%%%%%%%%%%%%%%%%%%%%%%%%%%%%%%%%%%%%%%%%%%%%%%%%%%%%%%%%%%%%%%%%%%%%%%%%%%%%%%%%%%%%%%%%%%%%%%%%%%%%%%%%%%%%%%%%%%%%%%%%%%%%%%%%%%%%
%%%%%%%%%%%%%%%%%%%%%%%%%%%%%%%%%%%%%%%%%%%%%%%%%%%%%%%%%%%%%%%%%%%%%%%%%%%%%%%%%%%%%%%%%%%%%%%%%%%%%%%%%%%%%%%%%%%%%%%%%%%%%%%%%%%%%%%%%%%%%%%%%%%%%%%%%%%%%%%%%%%%%%%%%%%%%%%%%%%%%%%%%%%%%%
%%%%%%%%%%%%%%%%%%%%%%%%%%%%%%%%%%%%%%%%%%%%%%%%%%%%%%%%%%%%%%%%%%%%%%%%%%%%%%%%%%%%%%%%%%%%%%%%%%%%%%%%%%%%%%%%%%%%%%%%%%%%%%%%%%%%%%%%%%%%%%%%%%%%%%%%%%%%%%%%%%%%%%%%%%%%%%%%%%%%%%%%%%%%%%
%%%%%%%%%%%%%%%%%%%%%%%%%%%%%%%%%%%%%%%%%%%%%%%%%%%%%%%%%%%%%%%%%%%%%%%%%%%%%%%%%%%%%%%%%%%%%%%%%%%%%%%%%%%%%%%%%%%%%%%%%%%%%%%%%%%%%%%%%%%%%%%%%%%%%%%%%%%%%%%%%%%%%%%%%%%%%%%%%%%%%%%%%%%%%%
%%%%%%%%%%%%%%%%%%%%%%%%%%%%%%%%%%%%%%%%%%%%%%%%%%%%%%%%%%%%%%%%%%%%%%%%%%%%%%%%%%%%%%%%%%%%%%%%%%%%%%%%%%%%%%%%%%%%%%%%%%%%%%%%%%%%%%%%%%%%%%%%%%%%%%%%%%%%%%%%%%%%%%%%%%%%%%%%%%%%%%%%%%%%%%
%%%%%%%%%%%%%%%%%%%%%%%%%%%%%%%%%%%%%%%%%%%%%%%%%%%%%%%%%%%%%%%%%%%%%%%%%%%%%%%%%%%%%%%%%%%%%%%%%%%%%%%%%%%%%%%%%%%%%%%%%%%%%%%%%%%%%%%%%%%%%%%%%%%%%%%%%%%%%%%%%%%%%%%%%%%%%%%%%%%%%%%%%%%%%%
%%%%%%%%
%%%%%%%%
%%%%%%%%                                The Modularity of an Abelian Variety
%%%%%%%%
%%%%%%%%                                             by Jae-Hyun Yang
%%%%%%%%
%%%%%%%%                                           first draft: June 2024
%%%%%%%%
%%%%%%%%%%%%%%%%%%%%%%%%%%%%%%%%%%%%%%%%%%%%%%%%%%%%%%%%%%%%%%%%%%%%%%%%%%%%%%%%%%%%%%%%%%%%%%%%%%%%%%%%%%%%%%%%%%%%%%%%%%%%%%%%%%%%%%%%%%%%%%%%%%%%%%%%%%%%%%%%%%%%%%%%%%%%%%%%%%%%%%%%%%%%%%
%%%%%%%%%%%%%%%%%%%%%%%%%%%%%%%%%%%%%%%%%%%%%%%%%%%%%%%%%%%%%%%%%%%%%%%%%%%%%%%%%%%%%%%%%%%%%%%%%%%%%%%%%%%%%%%%%%%%%%%%%%%%%%%%%%%%%%%%%%%%%%%%%%%%%%%%%%%%%%%%%%%%%%%%%%%%%%%%%%%%%%%%%%%%%%
%%%%%%%%%%%%%%%%%%%%%%%%%%%%%%%%%%%%%%%%%%%%%%%%%%%%%%%%%%%%%%%%%%%%%%%%%%%%%%%%%%%%%%%%%%%%%%%%%%%%%%%%%%%%%%%%%%%%%%%%%%%%%%%%%%%%%%%%%%%%%%%%%%%%%%%%%%%%%%%%%%%%%%%%%%%%%%%%%%%%%%%%%%%%%%
%%%%%%%%%%%%%%%%%%%%%%%%%%%%%%%%%%%%%%%%%%%%%%%%%%%%%%%%%%%%%%%%%%%%%%%%%%%%%%%%%%%%%%%%%%%%%%%%%%%%%%%%%%%%%%%%%%%%%%%%%%%%%%%%%%%%%%%%%%%%%%%%%%%%%%%%%%%%%%%%%%%%%%%%%%%%%%%%%%%%%%%%%%%%%%
%%%%%%%%%%%%%%%%%%%%%%%%%%%%%%%%%%%%%%%%%%%%%%%%%%%%%%%%%%%%%%%%%%%%%%%%%%%%%%%%%%%%%%%%%%%%%%%%%%%%%%%%%%%%%%%%%%%%%%%%%%%%%%%%%%%%%%%%%%%%%%%%%%%%%%%%%%%%%%%%%%%%%%%%%%%%%%%%%%%%%%%%%%%%%%
%%%%%%%%%%%%%%%%%%%%%%%%%%%%%%%%%%%%%%%%%%%%%%%%%%%%%%%%%%%%%%%%%%%%%%%%%%%%%%%%%%%%%%%%%%%%%%%%%%%%%%%%%%%%%%%%%%%%%%%%%%%%%%%%%%%%%%%%%%%%%%%%%%%%%%%%%%%%%%%%%%%%%%%%%%%%%%%%%%%%%%%%%%%%%%
%%%%%%%%%%%%%%%%%%%%%%%%%%%%%%%%%%%%%%%%%%%%%%%%%%%%%%%%%%%%%%%%%%%%%%%%%%%%%%%%%%%%%%%%%%%%%%%%%%%%%%%%%%%%%%%%%%%%%%%%%%%%%%%%%%%%%%%%%%%%%%%%%%%%%%%%%%%%%%%%%%%%%%%%%%%%%%%%%%%%%%%%%%%%%%
%%%%%%%%%%%%%%%%%%%%%%%%%%%%%%%%%%%%%%%%%%%%%%%%%%%%%%%%%%%%%%%%%%%%%%%%%%%%%%%%%%%%%%%%%%%%%%%%%%%%%%%%%%%%%%%%%%%%%%%%%%%%%%%%%%%%%%%%%%%%%%%%%%%%%%%%%%%%%%%%%%%%%%%%%%%%%%%%%%%%%%%%%%%%%%
%%%%%%%%%%%%%%%%%%%%%%%%%%%%%%%%%%%%%%%%%%%%%%%%%%%%%%%%%%%%%%%%%%%%%%%%%%%%%%%%%%%%%%%%%%%%%%%%%%%%%%%%%%%%%%%%%%%%%%%%%%%%%%%%%%%%%%%%%%%%%%%%%%%%%%%%%%%%%%%%%%%%%%%%%%%%%%%%%%%%%%%%%%%%%%




\documentclass[11pt]{amsart}
%\usepackage{amsmath,amsfonts,amssymb,mathrsfs}
\usepackage{amssymb,mathrsfs}
%\usepackage[draft]{Mytheorems}
%\usepackage{Mytheorems}
\setlength{\unitlength}{1cm} \setlength{\topmargin}{0.1in}
\setlength{\textheight}{8.4in} \setlength{\textwidth}{6in}
\setlength{\oddsidemargin}{0.1in}
\setlength{\evensidemargin}{0.1in}


\newtheorem{theorem}{Theorem}[section]
\newtheorem{lemma}[theorem]{Lemma}
\newtheorem{corollary}[theorem]{Corollary}
\newtheorem{proposition}[theorem]{Proposition}
\newtheorem{remark}[theorem]{Remark}
\newtheorem{definition}[theorem]{Definition}
\newtheorem{conjecture}[theorem]{Conjecture}
\newtheorem{example}[theorem]{Example}
\newtheorem{exercise}[theorem]{Exercise}
\newtheorem{problem}[theorem]{Problem}
%\newtheorem{proof}{Proof}[section]

\renewcommand{\theequation}{\thesection.\arabic{equation}}
\renewcommand{\thetheorem}{\thesection.\arabic{theorem}}






\title{The Modularity of an Abelian Variety}


\begin{document}


\author{Jae-Hyun Yang}
%\date{June 13 (Thursday), 2024}


\address{Yang Institute for Advanced Study
%\newline\indent
%Hyundai 41 Tower, No. 1905
%\newline\indent
%293 Mokdongdong-ro, Yangcheon-gu
\newline\indent
Seoul 07989, Korea
\vskip 2mm
and
\vskip 2mm
%}
%\address{
Department of Mathematics
\newline\indent
Inha University
\newline\indent
Incheon 22212, Korea}

\email{jhyang@inha.ac.kr\ \ or\ \ yangsiegel@naver.com}






%\newtheorem{theorem}{Theorem}[section]
%\newtheorem{lemma}{Lemma}[section]
%\newtheorem{proposition}{Proposition}[section]
%\newtheorem{remark}{Remark}[section]
%\newtheorem{definition}{Definition}[section]

\renewcommand{\theequation}{\thesection.\arabic{equation}}
\renewcommand{\thetheorem}{\thesection.\arabic{theorem}}
\renewcommand{\thelemma}{\thesection.\arabic{lemma}}
\newcommand{\BR}{\mathbb R}
\newcommand{\BQ}{\mathbb Q}
\newcommand{\BF}{\mathbb F}
\newcommand{\BT}{\mathbb T}
\newcommand{\BM}{\mathbb M}
\newcommand{\bn}{\bf n}
\def\charf {\mbox{{\text 1}\kern-.24em {\text l}}}
\newcommand{\BC}{\mathbb C}
\newcommand{\BZ}{\mathbb Z}

\thanks{\noindent{Subject Classification:} Primary 14Kxx, 14G35, 11F46, 11F80\\
\indent Keywords and phrases: modularity, abelian varieties, Siegel modular variety, Siegel modular forms,
Galois
\newline \indent
representation, zeta functions}






\begin{abstract}
We introduce the concept of the modularity of an abelian variety defined over the rational number field
extending the modularity of an elliptic curve. We discuss the modularity of an abelian variety over $\BQ$. We conjecture that a simple abelian variety over $\BQ$ is modular. The detailed article will appear later.
\end{abstract}



\maketitle


%\tableofcontents
\newcommand\tr{\triangleright}
\newcommand\al{\alpha}
\newcommand\be{\beta}
\newcommand\g{\gamma}
\newcommand\gh{\Cal G^J}
\newcommand\G{\Gamma}
\newcommand\de{\delta}
\newcommand\e{\epsilon}
\newcommand\z{\zeta}
%\newcommand\th{\theta}
\newcommand\vth{\vartheta}
\newcommand\vp{\varphi}
%\newcommand\r{\rho}
\newcommand\om{\omega}
\newcommand\p{\pi}
\newcommand\la{\lambda}
\newcommand\lb{\lbrace}
\newcommand\lk{\lbrack}
\newcommand\rb{\rbrace}
\newcommand\rk{\rbrack}
\newcommand\s{\sigma}
\newcommand\w{\wedge}
\newcommand\fgj{{\frak g}^J}
\newcommand\lrt{\longrightarrow}
\newcommand\lmt{\longmapsto}
\newcommand\lmk{(\lambda,\mu,\kappa)}
\newcommand\Om{\Omega}
\newcommand\ka{\kappa}
\newcommand\ba{\backslash}
\newcommand\ph{\phi}
\newcommand\M{{\Cal M}}
\newcommand\bA{\bold A}
\newcommand\bH{\bold H}
\newcommand\D{\Delta}

\newcommand\Hom{\text{Hom}}
\newcommand\cP{\Cal P}

\newcommand\cH{\Cal H}

\newcommand\pa{\partial}

\newcommand\pis{\pi i \sigma}
\newcommand\sd{\,\,{\vartriangleright}\kern -1.0ex{<}\,}
\newcommand\wt{\widetilde}
\newcommand\fg{\frak g}
\newcommand\fk{\frak k}
\newcommand\fp{\frak p}
\newcommand\fs{\frak s}
\newcommand\fh{\frak h}
\newcommand\Cal{\mathcal}

\newcommand\fn{{\frak n}}
\newcommand\fa{{\frak a}}
\newcommand\fm{{\frak m}}
\newcommand\fq{{\frak q}}
\newcommand\CP{{\mathcal P}_n}
\newcommand\Hnm{{\mathbb H}_n \times {\mathbb C}^{(m,n)}}
\newcommand\BD{\mathbb D}
\newcommand\BH{\mathbb H}
\newcommand\CCF{{\mathcal F}_n}
\newcommand\CM{{\mathcal M}}
\newcommand\Gnm{\Gamma_{n,m}}
%\newcommand\Hgh{{\mathbb H}_{g,h}}
\newcommand\Cmn{{\mathbb C}^{(m,n)}}
\newcommand\Yd{{{\partial}\over {\partial Y}}}
\newcommand\Vd{{{\partial}\over {\partial V}}}

\newcommand\Ys{Y^{\ast}}
\newcommand\Vs{V^{\ast}}
\newcommand\LO{L_{\Omega}}
\newcommand\fac{{\frak a}_{\mathbb C}^{\ast}}









%%%%%%%%%%%%%%%%%%%%%%%%%%%%%%%%%%%%%%%%%%%%%%%%%%%%%%%%%%%%%%%%%%%%%%%%%%%%%%%%%%%%%%%%%%%%%%%%%%%%%%%%%%%%%%%%%%%%%%%%%%%%%%%%%%%%%%%%%%%%%%%%%%%%%%%%%%%%%%%%%%%%%%%%%%%%%%%%%%%%%%%%%%%%%%
%%%%%%%%%%%%%%%%%%%%%%%%%%%%%%%%%%%%%%%%%%%%%%%%%%%%%%%%%%%%%%%%%%%%%%%%%%%%%%%%%%%%%%%%%%%%%%%%%%%%%%%%%%%%%%%%%%%%%%%%%%%%%%%%%%%%%%%%%%%%%%%%%%%%%%%%%%%%%%%%%%%%%%%%%%%%%%%%%%%%%%%%%%%%%%
%%%%%%%%%%%%%%%%%%%%%%%%%%%%%%%%%%%%%%%%%%%%%%%%%%%%%%%%%%%%%%%%%%%%%%%%%%%%%%%%%%%%%%%%%%%%%%%%%%%%%%%%%%%%%%%%%%%%%%%%%%%%%%%%%%%%%%%%%%%%%%%%%%%%%%%%%%%%%%%%%%%%%%%%%%%%%%%%%%%%%%%%%%%%%%
%%%%%%%%%%%%%%%%%%%%%%%%%%%%%%%%%%%%%%%%%%%%%%%%%%%%%%%%%%%%%%%%%%%%%%%%%%%%%%%%%%%%%%%%%%%%%%%%%%%%%%%%%%%%%%%%%%%%%%%%%%%%%%%%%%%%%%%%%%%%%%%%%%%%%%%%%%%%%%%%%%%%%%%%%%%%%%%%%%%%%%%%%%%%%%
%%%%%%%%%%%%%%%%%%%%%%%%%%%%%%%%%%%%%%%%%%%%%%%%%%%%%%%%%%%%%%%%%%%%%%%%%%%%%%%%%%%%%%%%%%%%%%%%%%%%%%%%%%%%%%%%%%%%%%%%%%%%%%%%%%%%%%%%%%%%%%%%%%%%%%%%%%%%%%%%%%%%%%%%%%%%%%%%%%%%%%%%%%%%%%
%%%%%%%%%%%%%%%%%%%%%%%%%%%%%%%%%%%%%%%%%%%%%%%%%%%%%%%%%%%%%%%%%%%%%%%%%%%%%%%%%%%%%%%%%%%%%%%%%%%%%%%%%%%%%%%%%%%%%%%%%%%%%%%%%%%%%%%%%%%%%%%%%%%%%%%%%%%%%%%%%%%%%%%%%%%%%%%%%%%%%%%%%%%%%%
%%%%%%%%%%%%
%%%%%%%%%%%%
%%%%%%%%%%%%
%%%%%%%%%%%%                                        Section 1. The Modularity of an Elliptic Curve
%%%%%%%%%%%%
%%%%%%%%%%%%
%%%%%%%%%%%%
%%%%%%%%%%%%%%%%%%%%%%%%%%%%%%%%%%%%%%%%%%%%%%%%%%%%%%%%%%%%%%%%%%%%%%%%%%%%%%%%%%%%%%%%%%%%%%%%%%%%%%%%%%%%%%%%%%%%%%%%%%%%%%%%%%%%%%%%%%%%%%%%%%%%%%%%%%%%%%%%%%%%%%%%%%%%%%%%%%%%%%%%%%%%%%
%%%%%%%%%%%%%%%%%%%%%%%%%%%%%%%%%%%%%%%%%%%%%%%%%%%%%%%%%%%%%%%%%%%%%%%%%%%%%%%%%%%%%%%%%%%%%%%%%%%%%%%%%%%%%%%%%%%%%%%%%%%%%%%%%%%%%%%%%%%%%%%%%%%%%%%%%%%%%%%%%%%%%%%%%%%%%%%%%%%%%%%%%%%%%%
%%%%%%%%%%%%%%%%%%%%%%%%%%%%%%%%%%%%%%%%%%%%%%%%%%%%%%%%%%%%%%%%%%%%%%%%%%%%%%%%%%%%%%%%%%%%%%%%%%%%%%%%%%%%%%%%%%%%%%%%%%%%%%%%%%%%%%%%%%%%%%%%%%%%%%%%%%%%%%%%%%%%%%%%%%%%%%%%%%%%%%%%%%%%%%
%%%%%%%%%%%%%%%%%%%%%%%%%%%%%%%%%%%%%%%%%%%%%%%%%%%%%%%%%%%%%%%%%%%%%%%%%%%%%%%%%%%%%%%%%%%%%%%%%%%%%%%%%%%%%%%%%%%%%%%%%%%%%%%%%%%%%%%%%%%%%%%%%%%%%%%%%%%%%%%%%%%%%%%%%%%%%%%%%%%%%%%%%%%%%%
%%%%%%%%%%%%%%%%%%%%%%%%%%%%%%%%%%%%%%%%%%%%%%%%%%%%%%%%%%%%%%%%%%%%%%%%%%%%%%%%%%%%%%%%%%%%%%%%%%%%%%%%%%%%%%%%%%%%%%%%%%%%%%%%%%%%%%%%%%%%%%%%%%%%%%%%%%%%%%%%%%%%%%%%%%%%%%%%%%%%%%%%%%%%%%
%%%%%%%%%%%%%%%%%%%%%%%%%%%%%%%%%%%%%%%%%%%%%%%%%%%%%%%%%%%%%%%%%%%%%%%%%%%%%%%%%%%%%%%%%%%%%%%%%%%%%%%%%%%%%%%%%%%%%%%%%%%%%%%%%%%%%%%%%%%%%%%%%%%%%%%%%%%%%%%%%%%%%%%%%%%%%%%%%%%%%%%%%%%%%%
%%%%%%%%%%%%%%%%%%%%%%%%%%%%%%%%%%%%%%%%%%%%%%%%%%%%%%%%%%%%%%%%%%%%%%%%%%%%%%%%%%%%%%%%%%%%%%%%%%%%%%%%%%%%%%%%%%%%%%%%%%%%%%%%%%%%%%%%%%%%%%%%%%%%%%%%%%%%%%%%%%%%%%%%%%%%%%%%%%%%%%%%%%%%%%
%%%%%%%%%%%%%%%%%%%%%%%%%%%%%%%%%%%%%%%%%%%%%%%%%%%%%%%%%%%%%%%%%%%%%%%%%%%%%%%%%%%%%%%%%%%%%%%%%%%%%%%%%%%%%%%%%%%%%%%%%%%%%%%%%%%%%%%%%%%%%%%%%%%%%%%%%%%%%%%%%%%%%%%%%%%%%%%%%%%%%%%%%%%%%%
%%%%%%%%%%%%%%%%%%%%%%%%%%%%%%%%%%%%%%%%%%%%%%%%%%%%%%%%%%%%%%%%%%%%%%%%%%%%%%%%%%%%%%%%%%%%%%%%%%%%%%%%%%%%%%%%%%%%%%%%%%%%%%%%%%%%%%%%%%%%%%%%%%%%%%%%%%%%%%%%%%%%%%%%%%%%%%%%%%%%%%%%%%%%%%


\begin{section}{{\bf The Modularity of an Elliptic Curve}}
\setcounter{equation}{0}
\vskip 3mm
We set $\G_1:=SL(2,\BZ)$. For a positive integer $N$, we let $\G (N),\ \G_1 (N)$ and
$\G_0 (N)$ be the congruence subgroups of $\G_1$ such that $\G (N)\subset \G_1(N)
\subset \G_0 (N)\subset \G_1.$ We refer to \cite[pp.\,13--14,\,p.\,21]{Di-S}
for the precise definitions and properties of $\G (N),\ \G_1 (N)$ and $\G_0 (N)$.
Let $\BH_1$ be the Poincar{\'e} upper half plance. The quotient
\begin{equation*}
  Y_1(N):=\G_1(N)\ba \BH_1 \ ({\rm resp}.\  Y_0(N):=\G_0(N)\ba \BH_1)
\end{equation*}
be the complex manifold which has a natural model $Y_1(N)/\BQ$ (resp. $Y_0(N)/\BQ$).
We let $X_1(N)$ (resp. $X_0(N)$) be the smooth projective curve which contains
$Y_1(N)$ (resp. $Y_0(N)$) as a dense Zariski open subset (cf.\,see \cite[pp.\,45--60]{Di-S}).

\vskip 3mm
Let $S_k(N)$ be the space of cusp forms of weight $k\geq 1$ and level $N\geq 1$.
Here $k$ and $N$ be positive integers. We recall that
if $f\in S_k(N)$, it satisfies the following properties\,:
\vskip 2mm
(C1) $f((a\tau+b)(c\tau+d)^{-1})=(c\tau+d)^k f(\tau)$ for all
$\begin{pmatrix}
   a & b \\
   c & d
 \end{pmatrix}\in \G_1(N)$ and $\tau\in \BH_1$;
\vskip 2mm
(C2) $| f(\tau)|^2 ({\rm Im}\,\tau)^k$ is bounded in $\BH_1$ and
\vskip 2mm
(C3) the Fourier expansion of $f(\tau)$ is given by
$$ f(\tau)=\sum_{n=1}^{\infty} a_n (f)\, q^n,\quad {\rm where}\ q=e^{2\pi i\tau}.$$

We define the $L$-series of $f\in S_k(N)$ to be
$$ L(f,s):=\sum_{n=1}^{\infty} a_n (f)\,n^{-s}.$$

\vskip 2mm
For each prime $p\!\not| N$, we recall that the Hecke operator $T_p:S_k(N)\lrt S_k(N)$ is
defined by
\begin{equation*}
  (T_pf)(\tau)=p^{-1}\sum_{i=0}^{p-1} f\left(\frac{\tau+i}{p}\right)
  +p^{k-1} f\left(\frac{ap\tau+b}{cp\tau+d}\right), \quad f\in S_k (N)
\end{equation*}
for any $\begin{pmatrix}
   a & b \\
   c & d
 \end{pmatrix}\in \G_1$ with $c\equiv 0\,({\rm mod}\,N)$ and $d\equiv p\,({\rm mod}\,N)$.
We refer to \cite[p.\,844]{BCDT} or \cite[pp.\,170--171]{Di-S} for more detail. The Hecke
operators $T_p\,(p\!\!\not| N)$ can be simultaneously diagonalized on $S_k (N)$ and
a simultaneous eigenvector a {\it Hecke\ eigenform} or simply an {\sf eigenform}.

\vskip 2mm
Let $\lambda$ be a place of the algebraic closure $\bar \BQ$ of $\BQ$ in $\BC$ lying over
a rational integer $\ell$ and ${\bar \BQ}_\lambda$ denote the algebraic closure of $\BQ_\ell$
via $\lambda$. Let $G_\BQ:={\rm Gal}({\bar \BQ}/\BQ)$ be the absolute Galois group of $\BQ$.
It is well known that if $f\in S_k (N)$ is a normalized eigenform with
$a_1(f)=1$, then there exists a unique continuous irreducible Galois representation
\begin{equation*}
  \rho_{f,\lambda}:G_\BQ\lrt GL(2,{\bar \BQ}_\lambda)
\end{equation*}
such that $\rho_{f,\lambda}$ is unramified at $p$ for all primes $p\!\!\not| \,\ell N$ and
\begin{equation*}
  {\rm Tr}\left(\rho_{f,\lambda}({\rm Frob}_p)\right)= a_p(f)\quad {\rm for\ any\ prime}\
p\!\!\not| \, \ell N.
\end{equation*}
The existence of $\rho_{f,\lambda}$ is due to Shimura if $k=2$ \cite{Sh2}, due to Deligne if
$k>2$ \cite{De} and due to Deligne and Serre if $k=1$ \cite{D-S}. We see that $\rho_{f,\lambda}$ is
odd in the sense that $\det \rho_{f,\lambda}$ of complex conjugation is -1. Moreover
$\rho_{f,\lambda}$ is potentially semi-stable at $\ell$ in the sense of Fontaine \cite{F-M}.

\vskip 3mm
We may choose a conjugate of $\rho_{f,\lambda}$ which is valued in
$GL(2,{\mathcal O}_{\bar{\BQ}_\lambda})$ and reducing modulo the maximal ideal and
semi-simplyfing yields an irreducible continuous representation
\begin{equation*}
{\overline \rho}_{f,\lambda}:G_\BQ\lrt GL(2,\overline{\mathbb F}_\ell)
\end{equation*}
which, up to isomorphism, does not depend on the choice of conjugate of $\rho_{f,\lambda}$.

\begin{definition}\label{def:1.1}
Let
$$\rho: G_\BQ\lrt GL(2,\bar{\BQ}_\ell)$$
be an irreducible continuous Galois representation which is unramified outside finitely
many primes and for which the restriction of $\rho$ to a decomposition group at $\ell$ is
potentially semi-stable at $\ell$ in the sense of Fontaine. Then $\rho$ is called
${\sf modular}$ if $\rho$ is equivalent to $\rho_{f,\lambda}$
(denoted $\rho\sim \rho_{f,\lambda}$) for some normalized eigenform $f$ and some place
$\lambda | \ell.$
\end{definition}

\begin{definition}\label{1.2}
Let
$${\overline\rho}: G_\BQ\lrt GL(2,\overline{\mathbb F}_\ell)$$
be a two-dimensional irreducible continuous representation of $G_\BQ$.
Then ${\overline\rho}$ is called
${\sf modular}$ if ${\overline\rho}\sim {\overline\rho}_{f,\lambda}$
for some normalized eigenform $f$ and some place $\lambda | \ell.$
\end{definition}

\vskip 3mm
Let $E$ be an elliptic curve over $\BQ$. We define
\begin{equation*}
  a_p(E):=p+1-|E({\mathbb F}_p)| \quad {\rm for\ a\ prime}\ p.
\end{equation*}
The $L$-function $L(E,s)$ of $E$ is defined by the product of the local $L$-factors
\begin{equation*}
  L(E,s):=\prod_{p|D}\left( \frac{1}{1-a_p(E)p^{-s}}\right)
  \prod_{p\not| D}\left( \frac{1}{1-a_p(E)\,p^{-s}+p^{1-2s}}\right).
\end{equation*}
Then $L(E,s)$ converges absolutely for ${\rm Re}\,s> \frac{3}{2}$ and extends to
an entire function by \cite{BCDT}.
\begin{definition}\label{1.3}
An elliptic curve $E$ over $\BQ$ is called
${\sf modular}$ if there exists a Hecke eigenform $f\in S_2(N)$ such that
\begin{equation*}
  L(E,s)=L(f,s).
\end{equation*}
\end{definition}
Let $E$ be an elliptic curve over $\BQ$ with its conductor $N(E)$.
Let $$\rho_{E,\ell}:G_\BQ \lrt GL(2,\bar{\BQ}_\ell)$$
be the $\ell$-adic representation of $G_\BQ$ with the Tate module
$T_\ell (E)$ as its representation space. Let
\begin{equation*}
  J_1(N):=\Omega^1(X_1(N))^{\vee}/H_1(X_1(N),\BZ)\cong S_2(\G_1(N))^{\vee}/H_1(X_1(N),\BZ)
\end{equation*}
be the Jacobian variety of the modular curve $X_1(N)$. Here $\Omega^1(X_1(N))$
denotes the complex vector space of holomorphic 1-forms on $X_1(N)$ and $W^{\vee}$
denotes the dual space of a complex vector space $W$.
It is known that the following statements are equivalent\,:
\vskip 2mm
(a) $E$ is modular.
\vskip 2mm
(b) There is a non-constant holomorphic mapping $X_1(N)\lrt E(\BC)$ for some
positive \par
\ \ \ \ \ \ integer $N$.
\vskip 2mm
(c) There is a non-constant holomorphic mapping $J_1(N)\lrt E(\BC)$ for some positive \par
\ \ \ \ \ \,integer $N$.
\vskip 2mm
(d) $\rho_{E,\ell}$ is modular for a prime $\ell$.

\vskip 3mm
The above statements have been called the Taniyama-Shimura conjecture. We refer to
\cite{Sh3} for the historical story of this conjecture.
The implication (a)\,$\Longrightarrow$\,(b) follows from a construction of Shimura\,\cite{Sh2}
and a theorem of Faltings\,\cite{F}. The implication (b)\,$\Longrightarrow$\,(d) is due to
Mazur\,\cite{MA}. The implication (d)\,$\Longrightarrow$\,(a) follows from a theorem of
Carayol\,\cite{C}. The implication (c)\,$\Longrightarrow$\,(b) is obvious. Wiles\,\cite{TW,W}
proved that a semistable eliptic curve over $\BQ$ is modular by proving the statement (d).
Thereafter Breuil, Conrad, Diamond and Taylor\,\cite{BCDT} proved that every eliptic curve
over $\BQ$ is modular.

\vskip 3mm
Serre \cite{S} conjectured the following\,:
\vskip 2mm\noindent
{\bf Serre's\ Modularity\ Conjecture\,:} Let $\overline{\rho}:G_\BQ\lrt GL(2,{\mathbb F})$ be a two-dimensional
absolutely irreducible, continuous, odd representation of $G_\BQ$. Here $\mathbb F$ is a finite
field of characteristic $p$. Then $\overline{\rho}$ is modular, i.e., arises from (with respect to
some fixed embedding $\imath:\overline{\BQ}\hookrightarrow \overline{\BQ_p})$ a newform $f$ of
some weight $k\geq 2$ and level $N$ prime to $p$.

\vskip 3mm
In 2009, Khare and Wintenberger \cite{K-W1, K-W2} proved that Serre's Modularity Conjecture is true.





\end{section}

\vskip 10mm
%%%%%%%%%%%%%%%%%%%%%%%%%%%%%%%%%%%%%%%%%%%%%%%%%%%%%%%%%%%%%%%%%%%%%%%%%%%%%%%%%%%%%%%%%%%%%%%%%%%%%%%%%%%%%%%%%%%%%%%%%%%%%%%%%%%%%%%%%%%%%%%%%%%%%%%%%%%%%%%%%%%%%%%%%%%%%%%%%%%%%%%%%%%%%%
%%%%%%%%%%%%%%%%%%%%%%%%%%%%%%%%%%%%%%%%%%%%%%%%%%%%%%%%%%%%%%%%%%%%%%%%%%%%%%%%%%%%%%%%%%%%%%%%%%%%%%%%%%%%%%%%%%%%%%%%%%%%%%%%%%%%%%%%%%%%%%%%%%%%%%%%%%%%%%%%%%%%%%%%%%%%%%%%%%%%%%%%%%%%%%
%%%%%%%%%%%%%%%%%%%%%%%%%%%%%%%%%%%%%%%%%%%%%%%%%%%%%%%%%%%%%%%%%%%%%%%%%%%%%%%%%%%%%%%%%%%%%%%%%%%%%%%%%%%%%%%%%%%%%%%%%%%%%%%%%%%%%%%%%%%%%%%%%%%%%%%%%%%%%%%%%%%%%%%%%%%%%%%%%%%%%%%%%%%%%%
%%%%%%%%%%%%%%%%%%%%%%%%%%%%%%%%%%%%%%%%%%%%%%%%%%%%%%%%%%%%%%%%%%%%%%%%%%%%%%%%%%%%%%%%%%%%%%%%%%%%%%%%%%%%%%%%%%%%%%%%%%%%%%%%%%%%%%%%%%%%%%%%%%%%%%%%%%%%%%%%%%%%%%%%%%%%%%%%%%%%%%%%%%%%%%
%%%%%%%%%%%%%%%%%%%%%%%%%%%%%%%%%%%%%%%%%%%%%%%%%%%%%%%%%%%%%%%%%%%%%%%%%%%%%%%%%%%%%%%%%%%%%%%%%%%%%%%%%%%%%%%%%%%%%%%%%%%%%%%%%%%%%%%%%%%%%%%%%%%%%%%%%%%%%%%%%%%%%%%%%%%%%%%%%%%%%%%%%%%%%%
%%%%%%%%%%%%%%%%%%%%%%%%%%%%%%%%%%%%%%%%%%%%%%%%%%%%%%%%%%%%%%%%%%%%%%%%%%%%%%%%%%%%%%%%%%%%%%%%%%%%%%%%%%%%%%%%%%%%%%%%%%%%%%%%%%%%%%%%%%%%%%%%%%%%%%%%%%%%%%%%%%%%%%%%%%%%%%%%%%%%%%%%%%%%%%
%%%%%%%%%%%%
%%%%%%%%%%%%
%%%%%%%%%%%%
%%%%%%%%%%%%              Section 2. The Modularity of an Abelian Variety
%%%%%%%%%%%%
%%%%%%%%%%%%
%%%%%%%%%%%%
%%%%%%%%%%%%%%%%%%%%%%%%%%%%%%%%%%%%%%%%%%%%%%%%%%%%%%%%%%%%%%%%%%%%%%%%%%%%%%%%%%%%%%%%%%%%%%%%%%%%%%%%%%%%%%%%%%%%%%%%%%%%%%%%%%%%%%%%%%%%%%%%%%%%%%%%%%%%%%%%%%%%%%%%%%%%%%%%%%%%%%%%%%%%%%
%%%%%%%%%%%%%%%%%%%%%%%%%%%%%%%%%%%%%%%%%%%%%%%%%%%%%%%%%%%%%%%%%%%%%%%%%%%%%%%%%%%%%%%%%%%%%%%%%%%%%%%%%%%%%%%%%%%%%%%%%%%%%%%%%%%%%%%%%%%%%%%%%%%%%%%%%%%%%%%%%%%%%%%%%%%%%%%%%%%%%%%%%%%%%%
%%%%%%%%%%%%%%%%%%%%%%%%%%%%%%%%%%%%%%%%%%%%%%%%%%%%%%%%%%%%%%%%%%%%%%%%%%%%%%%%%%%%%%%%%%%%%%%%%%%%%%%%%%%%%%%%%%%%%%%%%%%%%%%%%%%%%%%%%%%%%%%%%%%%%%%%%%%%%%%%%%%%%%%%%%%%%%%%%%%%%%%%%%%%%%
%%%%%%%%%%%%%%%%%%%%%%%%%%%%%%%%%%%%%%%%%%%%%%%%%%%%%%%%%%%%%%%%%%%%%%%%%%%%%%%%%%%%%%%%%%%%%%%%%%%%%%%%%%%%%%%%%%%%%%%%%%%%%%%%%%%%%%%%%%%%%%%%%%%%%%%%%%%%%%%%%%%%%%%%%%%%%%%%%%%%%%%%%%%%%%
%%%%%%%%%%%%%%%%%%%%%%%%%%%%%%%%%%%%%%%%%%%%%%%%%%%%%%%%%%%%%%%%%%%%%%%%%%%%%%%%%%%%%%%%%%%%%%%%%%%%%%%%%%%%%%%%%%%%%%%%%%%%%%%%%%%%%%%%%%%%%%%%%%%%%%%%%%%%%%%%%%%%%%%%%%%%%%%%%%%%%%%%%%%%%%
%%%%%%%%%%%%%%%%%%%%%%%%%%%%%%%%%%%%%%%%%%%%%%%%%%%%%%%%%%%%%%%%%%%%%%%%%%%%%%%%%%%%%%%%%%%%%%%%%%%%%%%%%%%%%%%%%%%%%%%%%%%%%%%%%%%%%%%%%%%%%%%%%%%%%%%%%%%%%%%%%%%%%%%%%%%%%%%%%%%%%%%%%%%%%%
%%%%%%%%%%%%%%%%%%%%%%%%%%%%%%%%%%%%%%%%%%%%%%%%%%%%%%%%%%%%%%%%%%%%%%%%%%%%%%%%%%%%%%%%%%%%%%%%%%%%%%%%%%%%%%%%%%%%%%%%%%%%%%%%%%%%%%%%%%%%%%%%%%%%%%%%%%%%%%%%%%%%%%%%%%%%%%%%%%%%%%%%%%%%%%
%%%%%%%%%%%%%%%%%%%%%%%%%%%%%%%%%%%%%%%%%%%%%%%%%%%%%%%%%%%%%%%%%%%%%%%%%%%%%%%%%%%%%%%%%%%%%%%%%%%%%%%%%%%%%%%%%%%%%%%%%%%%%%%%%%%%%%%%%%%%%%%%%%%%%%%%%%%%%%%%%%%%%%%%%%%%%%%%%%%%%%%%%%%%%%
%%%%%%%%%%%%%%%%%%%%%%%%%%%%%%%%%%%%%%%%%%%%%%%%%%%%%%%%%%%%%%%%%%%%%%%%%%%%%%%%%%%%%%%%%%%%%%%%%%%%%%%%%%%%%%%%%%%%%%%%%%%%%%%%%%%%%%%%%%%%%%%%%%%%%%%%%%%%%%%%%%%%%%%%%%%%%%%%%%%%%%%%%%%%%%

\begin{section}{{\bf The Modularity of an Abelian Variety}}
\setcounter{equation}{0}
\vskip 2mm
Let $G:=Sp(2g,\BR)$ and $K=U(g).$
Let
$$\BH_g:=\{\,\Omega\in\BC^{(g,g)}\,|\ \Omega=\,{}^t\Omega,\ \,{\rm{Im}}\,\Omega>0\ \}$$
be the Siegel upper half plane of degree $g$.
Then $G$ acts on ${\mathbb H}_g$ transitively by

\begin{equation*}
  \alpha\cdot \Omega=(A\Omega+B)(C\Omega+D)^{-1},
\end{equation*}
where $\alpha=\begin{pmatrix}
           A & B \\
           C & D
         \end{pmatrix}\in G$ and $\Omega\in \BH_g.$
The stabilizer of the action (2.1) at $iI_g$ is

\begin{equation*}
  \left\{ \begin{pmatrix} \,A & B \\ -B & A \end{pmatrix} \Big| \ A+iB\in U(g)\,\right\}
  \cong U(g).
\end{equation*}
Thus we get the biholomorphic map
\begin{equation*}
G/K \lrt \BH_g, \qquad \alpha K \mapsto \alpha\!\cdot\! iI_g,  \quad \alpha\in G.
\end{equation*}
It is known that $\BH_g$ is an Einstein-K{\"a}hler Hermitian symmetric space.

\vskip 3mm
Let $\G_g:=Sp(2g,\BZ)$ be the Siegel modular group of degree $g$.
For a positive integer $N$, we let
\begin{equation*}
  \G_g(N):=\left\{ \g\in \G_g\,|\ \g\equiv I_{2g}\ ({\rm{mod}}\,N)\,\right\}
\end{equation*}
be the the principal congruence subgroup of $\G_g$ of level $N$. Let
\begin{equation*}
  \G_{g,0}(N):=\left\{ \g\in \G_g\,\big|\ \g=\begin{pmatrix}
                                       A & B \\
                                       C & D
                                     \end{pmatrix},\quad C\equiv 0\ ({\rm{mod}}\,N)\,\right\}
\end{equation*}
and
\begin{equation*}
  \G_{g,1}(N):=\left\{ \g\in \G_g\,\big| \
  \g=\begin{pmatrix}
      A & B \\
      C & D
     \end{pmatrix}\equiv \begin{pmatrix}
      I_g & * \\
      0 & I_g
     \end{pmatrix}\ ({\rm{mod}}\,N)\,\right\}
\end{equation*}
be the congruence subgroups of Level $N$. Then we have the relation
\begin{equation*}
  \G_g (N)\subset \G_{g,1} (N)\subset \G_{g,0} (N)\subset \G_g.
\end{equation*}

\begin{definition}\label{def (2.1)}
Let $\G$ be a congruence subgroup of $\G_g$ and let $k$ be a nonnegative integer $k$.
A function $F:\BH_g\lrt \BC$ is called a
$\sf{Siegel\ modular\ form}$ of degree $g$ and weight $k$ with respect to $\G$
if it satisfies the following conditions\,:
\vskip 2mm
{\sf{(S1)}} $F(\Om)$ is holomorphic on $\BH_g$\,;
\vskip 2mm
{\sf{(S2)}} $F(\g\cdot \Om)=(C\Om+D)^k F(\Om)$ \ \ for all
$\g=\begin{pmatrix}
      A & B \\
      C & D
     \end{pmatrix}\in \G$ and $\Om\in \BH_g$\,;
\vskip 2mm
{\sf{(S3)}} $F(\Om)$ is bounded in any domain $Y\geq Y_0 > 0$ in the case $g=1$.
\end{definition}
We denote the space of all Sigel modular forms of degree $g$ and weight $k$
with respect to $\G$ by $[\G,k]$.
\vskip 2mm
We define the so-called $\sf{Siegel\ operator}$
\begin{equation*}
  \Phi_g: [\G_g,k]\lrt [\G_{g-1},k]
\end{equation*}
by
\begin{equation*}
  (\Phi_g(F))(\Om_1):=\lim_{t\lrt\infty}
  F  \begin{pmatrix}
      \Om_1 & 0 \\
      0 & i\,t
     \end{pmatrix},\quad \Om_1\in \BH_{g-1}.
\end{equation*}
Then $\Phi_g$ is a well-defined linear mapping (cf,\,\cite[pp.\,187--189]{M}).
A Siegel modular form $F\in [\G_g,k]$ is called a $\sf{Siegel\ cusp\ form}$ if
$\Phi_g(F)=0$\,(cf.\,\cite[p.\,198]{M}). We denote by $[\G_g,k]_0$
the space of all Siegel cusp forms in $[\G_g,k]$.

\vskip 3mm
Let $\G$ be a congruence subgroup of $\G_g$. If $F\in [\G,k]$, then $F$ has
a Fourier expansion
\begin{equation*}
  F(\Om)=\sum_{T} a(T;F)\,e^{2\pi i\,{\rm Tr}(T\Om)},
\end{equation*}
where $T$ runs through all $g\times g$ half-integral semi-positive symmetric matrices.
Here ${\rm Tr}(M)$ denotes the trace of a square matrix $M$.
Following the Hecke's method, Maass\,\cite[pp.\,202--210]{M} associated with
$F(\Om)$ the Dirichlet series
\begin{equation*}
D(F,s):=\sum_{\{ T \}}\frac{a(T;F)}{\varepsilon (T)}\,(\det T)^{-s},
\end{equation*}
where the summation indicates that $T$ runs through a complete set of representatives
of the sets
$$
\left\{ \, T[U]\,|\ U \ \rm{unimodular}\,\right\}, \ T>0
$$
and $\varepsilon (T)$ denotes the number of unimodular matrices $U$ which satisfy
the equation $T[U]=T.$ We note that the numbers $\varepsilon (T)$ are finite.

\begin{definition}\label{def:2.2}
Let $F$ be a nonzero Siegel Hecke eigenform in $[\G_g,k]_0$. Let
$\alpha_{p,0},\alpha_{p,1},\cdots,\alpha_{p,g}$ be the $p$-Satake parameters of $F$ at
a prime $p$. We define the $\sf{local\ spinor\ zeta\ function}$ $Z_{F,p}(t)$ of $F$ at $p$
by
$$
Z_{F,p}(t):=(1-\alpha_{p,0}\,t)\sum_{r=1}^g\sum_{1\leq i_1<\cdots < i_r\leq g}
(1-\alpha_{p,0}\alpha_{p,i_1}\cdots\alpha_{p,i_r}\,t).
$$
The $\sf{spinor\ zeta\ function}$ $Z_{F}(s)$ of $F$ is defined to be the following function
\begin{equation*}
Z_{F}(s):=\sum_{p:{\rm prime}} Z_{F,p}(p^{-s})^{-1},\quad {\rm Re}\,s \gg 0.
\end{equation*}
Secondly one has the so-called $\sf{standard\ zeta\ function}$ $D_{F}(s)$ of a Siegel Hecke
eigenform $F$ in $[\G_g,k]_0$ defined by
\begin{equation*}
D_{F}(s):=\sum_{p:{\rm prime}} D_{F,p}(p^{-s})^{-1},\quad {\rm Re}\,s \gg 0,
\end{equation*}
where
$$
D_{F,p}(t)=(1-t) \sum_{i=1}^g (1-\alpha_{p,i}\,t)(1-\alpha_{p,i}^{-1}\,t).
$$
We refer to \cite[p.\,249]{Y}.
\end{definition}

\begin{remark}\label{rk:2.3}
(1) If $g=1$, the spinor zeta function $Z_f(s)$ of a Hecke eigenform $f$ in $S_k(\G_1)$
is nothing but the Hecke $L$-function $L(f,s)$ of $f$.
\vskip 2mm\noindent
(2) If $g=1$, the standard zeta function $D_f(s)$ of a Hecke eigenform
$f(\tau)=\sum_{n=1}^{\infty} a(n)\,e^{2\pi i n\tau}$ in $S_k(\G_1)$ has the following equation
$$
D_f(s-k+1)=\sum_{p:{\rm prime}}(1+p^{k-s-1})^{-1}\cdot\sum_{n=1}^{\infty} a(n^2)\,n^{-s}.
$$
\end{remark}

\vskip 3mm
Let $A$ be a $g$-dimensional simple abelian variety defined over $\BQ$. For a prime $\ell$, we set
\begin{equation*}
  A[\ell^n]:=\{ \, x\in A(\overline{\BQ})\,|\ \ell^n\cdot x=0\,\}.
\end{equation*}
Then $A[\ell^n]\cong (\BZ/\ell^n \BZ)^g\times (\BZ/\ell^n \BZ)^g$\,(cf.\,\cite{MU}). Then the
Tate module of $A$ is given by
\begin{equation*}
  T_{\ell}(A):=\lim_{\longleftarrow} A[\ell^n]\cong \BZ_\ell^g\times \BZ_\ell^g
  \cong \BZ_{\ell}^{2g}.
\end{equation*}
Therefore we have the $2g$-dimensional $\ell$-adic Galois representation of $G_\BQ$
\begin{equation*}
\rho_{A,\ell}:G_\BQ\lrt GL(2g,\BZ_\ell)\subset GL(2g,\BQ_\ell).
\end{equation*}

\begin{definition}\label{def:2.4}
A $2g$-dimensional $\ell$-adic Galois representation $\rho$ of $G_\BQ$ given by
$$
\rho:G_\BQ\lrt GL(2g,\BZ_\ell)\subset GL(2g,\BQ_\ell)
$$
is called $\sf{modular}$ if there is a Siegel Hecke eigenform $F(\Om)\in [\G_{g,0}(N), g+1]_0$
of weight $g+1$ with respect to $\G_{g,0}(N)$ such that
\begin{equation*}
  {\rm Tr}\left( \rho({\rm Frob}_p)\right)=a(pI_g;F)\quad {\rm and}\quad
  \det \left( \rho({\rm Frob}_p)\right)=p^g\quad {\rm for\ any\ prime}\ p\not|\, \ell N,
\end{equation*}
where
\begin{equation*}
  F(\Om)=\sum_{T} a(T;F)\,e^{2\pi i\,{\rm Tr}(T\Om)}
\end{equation*}
is a Fourier expansion of $F(\Om)$.
\end{definition}

\begin{definition}\label{def:2.5}
Let $A$ be a $g$-dimensional simple abelian variety defined over $\BQ$ and let $\ell$ be a prime.
 For a prime $p$, we let
\begin{equation*}
  L_p(A,s):=\left\{ \det\left( I_{2g}-p^{-s}\cdot
  \rho_{A,\ell}({\rm Frob}_p)\big|_{T_{\ell}(A)}\right)\right\}^{-1}
\end{equation*}
be the local $L$-function of $A$ at $p$. We define the $L$-function $L(A,s)$ of $A$ by
\begin{equation*}
  L(A,s)=\prod_{p:\,{\rm prime}}L_p(A,s).
\end{equation*}
\end{definition}

\begin{definition}\label{def:2.6}
Let $A$ be a $g$-dimensional simple abelian variety defined over $\BQ$.
$A$ is called $\sf{modular}$ if there exists a Siegel Hecke eigenform
$F(\Om)\in [\G_{g,0}(N), g+1]_0$
of weight $g+1$ with respect to $\G_{g,0}(N)$ such that
\begin{equation*}
  L(A,s)=D(F,s),\ Z_F(s)\ {\rm or}\ D_F(s).
\end{equation*}
\end{definition}
\vskip 3mm
For two positive integers $g$ and $N$, we let
\begin{equation*}
 {\mathcal A}_{g,0}(N):=\G_{g,0}(N)\ba \BH_g
\end{equation*}
be the Siegel modular variety of level structure $N$ and let
${\mathcal A}_{g,0}^{\rm tor}(N)$ be a smooth toroidal compactification of
${\mathcal A}_{g,0}(N)$\,(cf.\,\cite{AMRT, F-C}). We denote by
\begin{equation*}
 \Omega^i \left( {\mathcal A}_{g,0}^{\rm tor}(N)\right), \quad 0\leq i\leq \frac{g(g+1)}{2}
\end{equation*}
the complex vector space of holomorphic $i$-forms on ${\mathcal A}_{g,0}^{\rm tor}(N)$.
The Jacobian variety ${\rm Jac}\left( {\mathcal A}_{g,0}^{\rm tor}(N) \right)$ of
${\mathcal A}_{g,0}^{\rm tor}(N)$
is defined to be the abelian variety
\begin{equation}\label{(2.1)}
 {{\rm Jac}}({\mathcal A}_{g,0}^{\rm tor}(N)):=\Omega^{\nu}
 \left( {\mathcal A}_{g,0}^{\rm tor}(N) \right)^{\vee}
 / H_\nu \left( {\mathcal A}_{g,0}^{\rm tor}(N),\BZ \right),\qquad \nu=\frac{g(g+1)}{2}.
\end{equation}
The geometric genus of ${\mathcal A}_{g,0}^{\rm tor}(N)$ is the dimension of
the Jacobian variety ${\rm Jac}\left( {\mathcal A}_{g,0}^{\rm tor}(N) \right)$.
It is known that the following two vector spaces are isomprphic\,:
\begin{equation}\label{(2.2)}
 [\G_{g,0}(N),g+1]_0\cong \Omega^{\nu}
 \left( {\mathcal A}_{g,0}^{\rm tor}(N) \right),\qquad \nu=\frac{g(g+1)}{2}.
\end{equation}
More precisely, for a coordinate $\Omega=(\omega_{ij})\in \BH_g$, we let
\begin{equation*}
  \omega_0:=d\omega_{11}\wedge d\omega_{12}\wedge d\omega_{13}\wedge \cdots \wedge
  d\omega_{gg}
\end{equation*}
be a holomorphic $\nu$-form on $\BH_g$. If $\omega=F(\Om)\,\omega_0$ is a $\G_{g,0}(N)$-invariant
holomorphic form on $\BH_g$,
then
\begin{equation*}
F(\g\cdot \Om)=\det (C\Om+D)^{g+1}F(\Om)
\end{equation*}
for all $\g=\begin{pmatrix}
              A & B \\
              C & D
            \end{pmatrix}\in \G_{g,0}(N)$ and $\Om\in\BH_g.$ Thus
$F\in [\G_{g,0}(N),g+1].$ It was shown by Freitag\,\cite{Fr} that $\om$ can be extended
to a holomorphic $\nu$-form on ${\mathcal A}_{g,0}^{\rm tor}(N)$ if and only if $F$ is
a cusp form in $[\G_{g,0}(N),g+1]_0.$ Indeed, the mapping
\begin{equation*}
[\G_{g,0}(N),g+1]_0 \lrt \Omega^{\nu}
 \left( {\mathcal A}_{g,0}^{\rm tor}(N) \right),\qquad F\mapsto F\,\omega_0
\end{equation*}
is an isomorphism as complex vector spaces. We observe that if
$\om_k:=G(\Om)\, \om_0^{\otimes k}$ is a $\G_{g,0}(N)$-invariant
holomorphic form on $\BH_g$ of degree $k\nu$, then $G(\Om)\in [\G_{g,0}(N),k(g+1)]_0$
is a cusp form of weight $k(g+1).$
\vskip 3mm
Therefore according to \eqref{(2.1)} and \eqref{(2.2)}, we have
\begin{equation}\label{(2.3)}
{{\rm Jac}}({\mathcal A}_{g,0}^{\rm tor}(N))
 \cong [\G_{g,0}(N),g+1]_0^{\vee}
 / H_\nu \left( {\mathcal A}_{g,0}^{\rm tor}(N),\BZ \right),\qquad \nu=\frac{g(g+1)}{2}.
\end{equation}
If there is no confusion, we simply set
$$
J_{g,0}(N):= {{\rm Jac}}({\mathcal A}_{g,0}^{\rm tor}(N)).
$$

\vskip 3mm
We propose the following conjectures.
\vskip 2mm \noindent
\begin{conjecture}\label{conj:2.7}
%{\bf Conjecture 2.1.}
A simple abelian variety of dimesion $g$ defined over $\BQ$ is modular.
\end{conjecture}
\vskip 3mm
\noindent
\begin{conjecture}\label{conj:2.8}
%{\bf Conjecture 2.2.}
Let A be a simple abelian variety of dimesion $g$ defined over $\BQ$.
The following statements are equivalent\,:
\vskip 2mm
{\sf{(MAV1)}} $A$ is modular.
\vskip 1mm
{\sf{(MAV2)}} There exists a non-constant holomorphic mapping
${\mathcal A}_{g,0}^{\rm tor}(N)\lrt A$ for some
\par
\ \ \ \ \ \ \ \ \ \ \ positive integer $N$.
\vskip 1mm
{\sf{(MAV3)}} There exists a non-constant holomorphic mapping
$J_{g,0}(N)\lrt A$ for some
\par
\ \ \ \ \ \ \ \ \ \ \ positive integer $N$.
\vskip 1mm
{\sf{(MAV4)}} $\rho_{A,\ell}$ is modular for any prime $\ell$.
\end{conjecture}

\vskip 3mm
We propose the following problems.
\vskip 2mm \noindent
\begin{problem}\label{prob:2.9}
%{\bf Problem 2.1.}
Let $F\in [\G_{g,0}(N),g+1]_0$ be a Siegel Hecke eigenform of weight $g+1$.
Associate to $F$ a $2g$-dimensional continuous irreducible Galois representation
of $G_\BQ$.
\end{problem}

\vskip 2mm \noindent
\begin{problem}\label{prob:2.10}
%{\bf Problem 2.2.}
Let $k$ be a positive integer.
Let $F\in [\G_{g,0}(N),k]_0$ be a Siegel Hecke eigenform of weight $k$.
Associate to $F$ a $2g$-dimensional continuous irreducible Galois representation
of $G_\BQ$.
\end{problem}

\begin{remark}\label{rk:2.11}
As mentioned in Section 1, in the case $g=1$, to a Hecke eigenform of weight $k\geq 1$, Shimura, Deligne and Serre associated a two-dimensional continuous irreducible Galois representation. For the case
$g=2$, Taylor\,\cite{T1,T2} tried to associate the four dimensional continuous Galois representation
of $G_\BQ$ to a Siegel Hecke eigenform of small weight. But he did not specify the precise weight.
\end{remark}

\end{section}

\vskip 5mm
\begin{thebibliography}{99}

\bibitem{AMRT} A. Ash, D. Mumford, M. Rapoport and Y.-S. Tai, {\em Smooth Compactifications of Locally Symmetric Varieties. With the Collaboration of Peter Scholze}, 2nd ed. Cambridge Math. Library, Cambridge Univ. Press, Cambridge (2010).

\bibitem{BCDT} C.\,Breuil, B.\,Conrad, F.\,Diamond and R.\,Taylor,
{\em On the modularity of elliptic curves over $\mathbb Q$: wild 3-adic exercises},
J. Amer. Math. Soc., {\bf 14}\,(2001), no.\,2, 843--939.

\bibitem{C} H. Carayol, {\em Sur les repr{\' e}sentations $\ell$-adiques
associ{\'e}es aux formes modulaires de Hilbert}, Ann. Sci. {\'E}c. Norm. Sup.
{\bf 19}\,(1986), 409--468.

\bibitem{De} P. Deligne, {\em Formes modulaires et repr{\' e}sentations $\ell$-adiques},
Lecture Notes in Math. {\bf 179}, Springer-Verlag (1971), 89--138.

\bibitem{D-S} P. Deligne and J.-P. Serre, {\em Formes modulaires de poids 1},
Ann. Sci. Ec. Norm. Sup. {\bf 7}\,(1974), 507--530.

\bibitem{Di-S} F.\,Diamond and J.\,Shurman, {\em A First Course in Modular Forms},
Graduate Texts in Mathematics {\bf 228}, Springer (2005).

\bibitem{F} G. Faltings, {\em Endlichkeitss{\"a}tze f{\"u}r abelsche Variet{\"a}ten
{\" u}ber Zahlk{\"o}rpern}, Inv. Math. {\bf 73}\,(1983), 349--366.

\bibitem{F-C} G. Faltings and C.-L. Chai, {\em Degeneration of abelian varieties}, Ergebnisse der Math.
{\bf 22}, Springer-Verlag, Berlin-Heidelberg-New York (1990).

\bibitem{F-M} J.-M. Fontaine and B. Mazur, {\em Geometric Galois representations},
in Elliptic Curves, Modular Forms and Fermat's Last Theorem (Hong Kong,\,1993),
International Press\,(1995), 41--78.

\bibitem{Fr} E. Freitag, {\em Siegelsche Modulfunktionen}, Grundlehren de mathematischen Wissenschaften {\bf 55}, Springer-Verlag, Berlin-Heidelberg-New York (1983).

\bibitem{K-W1} C. Khare and J.-P. Wintenberger, {\em Serre's modularity conjecture. I.},
Inv. Math. {\bf 178}\,(2009), no.\,3, 485--504.

\bibitem{K-W2} C. Khare and J.-P. Wintenberger, {\em Serre's modularity conjecture. II.},
Inv. Math. {\bf 178}\,(2009), no.\,3, 505--586.

\bibitem{M} H. Maass, {\em Siegel modular forms and Dirichlet series,} Lecture Notes in Math. {\bf 216}, Springer-Verlag, Berlin-Heidelberg-New York (1971).

\bibitem{MA} B. Mazur, {\em Number theory as gadfly}, Amer. Math. Monthly
{\bf 98}\,(1991), 593--610.

\bibitem{MU} D. Mumford, {\em Abelian Varieties}, Published for TIFR\,(Bombay), Oxford
University Press, Reprinted in 1985.

\bibitem{S} J.-P. Serre, {\em Sur les repr{\' e}sentations modulaires de degr{\'e} de
${\rm Gal}(\bar{\BQ} /\BQ)$}, Duke Math. J. {\bf 54}\,(1987), 179--230.

%\bibitem{Sh1} G. Shimura, {\em On elliptic curves with complex multiplication as factors
%of the Jacobians of modular function fields}, Nagoya Math. J. {\bf 43}\,(1971), 199--208.

\bibitem{Sh2} G. Shimura, {\em Introduction to the Arithmetic Theory of Automorphic
Functions}, Princeton Univ. Press, Princeton (1971).

\bibitem{Sh3} G. Shimura, {\em Yutaka Taniyama and his time: Very personal recollections,}
Bull. London Math. Soc., {\bf 21} (1989), 186--196.

\bibitem{T1} R. Taylor, {\em Galois Representations Associated to Siegel Modular Forms
of Low Weight}, Duke Math. J. {\bf 63}\,(1991), 281--332.

\bibitem{T2} R. Taylor, {\em On the $\ell$-adic cohomology of Siegel threefolds},
Inv. math. {\bf 114}\,(1993), 289--310.

\bibitem{TW} R. Taylor and A. Wiles, {\em Ring-theoretic properties of certain Hecke
algebras}, Ann. Math., {\bf 141}\,(1995), no.\,3, 553--572.

\bibitem{W} A. Wiles, {\em Modular elliptic curves and Fermat's Last Theorem},
Ann. Math., {\bf 141}\,(1995), no.\,3, 443--551.

\bibitem{Y} J.-H. Yang, {\em Theory of the Siegel modular variety}, Number Theory and
Applications, Proc. International Conferences on Number Theory and Cryptography, edited by
S.\,D. Adhikari and B. Ramakrishnan, Hindustan Book Agency, New Delhi (2009), 219--278.
ISBN: 978-81-85931-97-5..

\end{thebibliography}

\end{document} 