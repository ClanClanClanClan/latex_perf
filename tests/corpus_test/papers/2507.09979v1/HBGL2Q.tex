\documentclass[12pt]{article}
\usepackage{amsfonts,amsmath,amsxtra}
\usepackage{latexsym}
\usepackage{amssymb}
\usepackage{color}
\usepackage[matrix,arrow,curve]{xy}
%\usepackage[active]{srcltx}

\synctex=1


\makeatletter
\def\@seccntformat#1{\csname
the#1\endcsname\enspace} \makeatother

%%%%%%%%%%%%%%%%%%%%%%%%%%%%%%%%%%%%%%%%%%%%%%%%%%%%%%%%%%%%%%%%%%%%
%%%%%%%%           DEFINITIONS FOR "DRAFT" STYLE           %%%%%%%%%
%%%%%%%%%%%%%%%%%%%%%%%%%%%%%%%%%%%%%%%%%%%%%%%%%%%%%%%%%%%%%%%%%%%%
\def\hybrid{\topmargin 0pt      \oddsidemargin 0pt
        \headheight 0pt \headsep 0pt
        \textwidth 16.5cm
        \textheight 23cm
        \marginparwidth 0.0in
        \parskip 5pt plus 1pt   \jot = 1.5ex}
\catcode`\@=11
\def\marginnote#1{}
\newcount\hour
\newcount\minute
\newtoks\amorpm
\hour=\time\divide\hour by60 \minute=\time{\multiply\hour by60
\global\advance\minute by-\hour}
\edef\standardtime{{\ifnum\hour<12 \global\amorpm={am}%
        \else\global\amorpm={pm}\advance\hour by-12 \fi
        \ifnum\hour=0 \hour=12 \fi
      \number\hour:\ifnum\minute<10 0\fi\number\minute\the\amorpm}}
\edef\militarytime{\number\hour:\ifnum\minute<10 0\fi\number\minute}

\def\draftlabel#1{{\@bsphack\if@filesw {\let\thepage\relax
   \xdef\@gtempa{\write\@auxout{\string
      \newlabel{#1}{{\@currentlabel}{\thepage}}}}}\@gtempa
   \if@nobreak \ifvmode\nobreak\fi\fi\fi\@esphack}
        \gdef\@eqnlabel{#1}}
\def\@eqnlabel{}
\def\@vacuum{}
\def\draftmarginnote#1{\marginpar{\raggedright\scriptsize\tt#1}}

\def\draft{\oddsidemargin -0.1truein
        \def\@oddfoot{\sl WLimit.tex \hfil
        \rm\thepage\hfil\sl\today\quad\militarytime}
        \let\@evenfoot\@oddfoot \overfullrule 3pt
        \let\label=\draftlabel
        \let\marginnote=\draftmarginnote
\def\@eqnnum{{\rm (\theequation)}
\rlap{\kern\marginparsep\tt\@eqnlabel}%
\global\let\@eqnlabel\@vacuum}  }
%%%%%%%%%%%%%%%%%%%%%%%%%%%%%%%%%%%%%%%%%%%%%%%%%%%%%%%%%%%%%%%%%%%%
%%%%%%%%%               END "DRAFT" DEFINITION          %%%%%%%%%%%%
%%%%%%%%%%%%%%%%%%%%%%%%%%%%%%%%%%%%%%%%%%%%%%%%%%%%%%%%%%%%%%%%%%%%



%%%%%%%%%%%%%%%%%%%%%%%%%%%%%%%%%%%%%%%%%%%%%%%%%%%%%%%%%%%%%%
%                          EXTRA MATH FONTS                  %
%%%%%%%%%%%%%%%%%%%%%%%%%%%%%%%%%%%%%%%%%%%%%%%%%%%%%%%%%%%%%%

\newfont{\Bbbb}{msbm7 scaled 1\@ptsize00}
\newcommand{\zs}{\raise-1pt\hbox{$\mbox{\Bbbb Z}$}}
\newcommand{\rs}{\hbox{$\mbox{\Bbbb R}$}}

\font\teneufm=cmmib10 scaled 1\@ptsize00 \font\seveneufm=cmmib7
scaled 1\@ptsize00
\font\fiveeufm=cmmib5  %scaled 1\@ptsize00
\def\bfit#1{{\textfont1=\teneufm\scriptfont1=\seveneufm
\scriptscriptfont1=\fiveeufm \mathchoice{
\hbox{$\mathsurround=0pt\displaystyle#1$}}
{\mathsurround=0pt\hbox{$\textstyle#1$}}
{\hbox{$\mathsurround=0pt\scriptstyle#1$}}
{\hbox{$\mathsurround=0pt\scriptscriptstyle#1$}}}}

\def\rit#1{{\textfont1=\ten\scriptfont1=\sev
\scriptscriptfont1=\fiv \mathchoice{
\hbox{$\mathsurround=0pt\displaystyle#1$}}
{\mathsurround=0pt\hbox{$\textstyle#1$}}
{\hbox{$\mathsurround=0pt\scriptstyle#1$}}
{\hbox{$\mathsurround=0pt\scriptscriptstyle#1$}}}}

%%%%%%%%%THIS CAUSES EQUATIONS TO BE NUMBERED BY SECTIONS%%%%%%%%%%
\def\numberbysection{\@addtoreset{equation}{section}
        \def\theequation{\thesection.\arabic{equation}}}
\numberbysection
\newcommand{\sect}[1]{\setcounter{equation}{0}\section{#1}}
\renewcommand{\theequation}{\thesection.\arabic{equation}}

%%%%%%%%%%%%%%%%%%%%%%%%%%%%%%%%%%%%%%%%%%%%%%%%%%%%%%%%%%%%%%%%
\def\titlepage{\@restonecolfalse\if@twocolumn\@restonecoltrue\onecolumn
     \else \newpage \fi \thispagestyle{empty}\c@page\z@
\def\thefootnote{\fnsymbol{footnote}} }
\def\endtitlepage{\if@restonecol\twocolumn \else  \fi
        \def\thefootnote{\arabic{footnote}}
        \setcounter{footnote}{0}}  %\c@footnote\z@ }
\relax \hybrid
%%%%%%%%%%%%%%%%%%%%%%%%%%%%%%%%%%%%%%%%%%%%%%%%%%%%%%%%%%%%%%%
\parskip=0.4em
%
\makeatletter
\newdimen\normalarrayskip            % skip between lines
\newdimen\minarrayskip               % minimal skip between lines
\normalarrayskip\baselineskip \minarrayskip\jot
\newif\ifold             \oldtrue            \def\new{\oldfalse}
%
\def\arraymode{\ifold\relax\else\displaystyle\fi}%mode of array enrties
\def\eqnumphantom{\phantom{(\theequation)}} % ight phantom in eqnarray
\def\@arrayskip{\ifold\baselineskip\z@\lineskip\z@
     \else
     \baselineskip\minarrayskip\lineskip1\baselineskip\fi}
%%%%%%%%%%%%%%%%%%%%%%%%%%%%%%%%%%%%%%%%%%%%%%%%%%%%%%%%%%%%%%%%%
\def\@arrayclassz{\ifcase \@lastchclass \@acolampacol \or
\@ampacol \or \or \or \@addamp \or
   \@acolampacol \or \@firstampfalse \@acol \fi
\edef\@preamble{\@preamble
  \ifcase \@chnum
     \hfil$\relax\arraymode\@sharp$\hfil
     \or $\relax\arraymode\@sharp$\hfil
     \or \hfil$\relax\arraymode\@sharp$\fi}}
%%%%%%%%%%%%%%%%%%%%%%%%%%%%%%%%%%%%%%%%%%%%%%%%%%%%%%%%%%%%%%%%%%
\def\@array[#1]#2{\setbox\@arstrutbox=\hbox{\vrule
     height\arraystretch \ht\strutbox
     depth\arraystretch \dp\strutbox
width\z@}\@mkpream{#2}\edef\@preamble{\halign \noexpand\@halignto
\bgroup \tabskip\z@ \@arstrut \@preamble \tabskip\z@ \cr}%
\let\@startpbox\@@startpbox \let\@endpbox\@@endpbox
  \if #1t\vtop \else \if#1b\vbox \else \vcenter \fi\fi
  \bgroup \let\par\relax
  \let\@sharp##\let\protect\relax
  \@arrayskip\@preamble}
%
%
%  \eqnarray -> middle element in \displaystyle
%
%
\def\eqnarray{\stepcounter{equation}%
              \let\@currentlabel=\theequation
              \global\@eqnswtrue
              \global\@eqcnt\z@
              \tabskip\@centering              %formulae  centering
              \let\\=\@eqncr
              $$%
            \halign to \displaywidth  \bgroup
             \eqnumphantom \@eqnsel
      \hskip\@centering                               %right tab%
    $\displaystyle  \tabskip\z@ {##}$%
    &\global\@eqcnt\@ne \hskip 2\arraycolsep
         $ \displaystyle  \arraymode{##}$\hfil
    &\global\@eqcnt\tw@ \hskip 2\arraycolsep
         $\displaystyle\tabskip\z@{##}$\hfil
         \tabskip\@centering
    &{##}\tabskip\z@\cr}
\makeatother
%%%%%%%%%%%%%%%%%%%%%  Mathbb font   %%%%%%%%%%%%%%%%%%%%%%%%
\def\IA{\mathbb{A}}
\def\IB{\mathbb{B}}
\def\IC{\mathbb{C}}
\def\ID{\mathbb{D}}
\def\IE{\mathbb{E}}
\def\IF{\mathbb{F}}
\def\IG{\mathbb{G}}
\def\IH{\mathbb{H}}
\def\IK{\mathbb{K}}
\def\IL{\mathbb{L}}
\def\IM{\mathbb{M}}
\def\IP{\mathbb{P}}
\def\IQ{\mathbb{Q}}
\def\IR{\mathbb{R}}
\def\IZ{\mathbb{Z}}
%%%%%%%%%%%%%%%%%%%%% Calligraphic letters  %%%%%%%%%%%%%%%%%%%%%
\def\CA {\mathcal{A}}
\def\CB {\mathcal{B}}
\def\CC {\mathcal{C}}
\def\CD {\mathcal{D}}
\def\CE {\mathcal{E}}
\def\CF {\mathcal{F}}
\def\CG {\mathcal{G}}
\def\CH {\mathcal{H}}
\def\CI {\mathcal{I}}
\def\CJ {\mathcal{J}}
\def\CK {\mathcal{K}}
\def\CL {\mathcal{L}}
\def\CM {\mathcal{M}}
\def\CN {\mathcal{N}}
\def\CO {\mathcal{O}}
\def\CP {\mathcal{P}}
\def\CQ {\mathcal{Q}}
\def\CR {\mathcal{R}}
\def\CS {\mathcal{S}}
\def\CT {\mathcal{T}}
\def\CU {\mathcal{U}}
\def\CV {\mathcal{V}}
\def\CW {\mathcal{W}}
\def\CX {\mathcal{X}}
\def\CY {\mathcal{Y}}
\def\CZ {\mathcal{Z}}
%%%%%%%%%%%%%%%%%%%%%%%%%%%%%%%%%    Frakture %%%%%%%%%%%%%%%%%%%%%%%%
\def\fg{{\frak g}}
\def\fb{{\frak b}}
\def\fh{{\frak h}}
\def\fH{{\frak H}}
\def\fK{{\frak K}}
\def\fk{{\frak k}}
\def\fL{{\frak L}}
\def\fn{{\frak n}}
\def\fu{{\frak u}}

%%%%%%%%%%%%%%%%%%%%%%%%% Greek letters %%%%%%%%%%%%%%%%%%%%%%%
\def\t {{\theta}}
\def\T {{\Theta}}
\def\w {{\omega}}
\def\a {{\alpha}}
\def\b {{\beta}}
\def\g {{\gamma}}
\def\s {{\sigma}}
\def\Si{{\Sigma}}
\def\la{\lambda}
\def\ve{\varepsilon}
\def\e{\epsilon}
%%%%%%%%%%%%%%%%%%%%%%%%%%%%%%%%%%%%%%%%%%%%%%%%%%%%%%%%%%%%%%%
\def\balpha{{\bfit\alpha}}
\def\bbeta{{\bfit\beta}}
\def\bgamma{{\bfit\gamma}}
\def\bnu{{\bfit\nu}}                      %Boldface Greek letters
\def\bmu{{\bfit\mu}}
\def\bo{{\bfit\omega}}
\def\bphi{{\bfit\phi}}
\def\blambda{{\bfit\lambda}}
\def\brho{{\bfit\rho}}
%%%%%%%%%%%%%%%%%%%%%%%%%% Derivatives  %%%%%%%%%%%%%%%%%%%%%%%%
\def\pr {\partial}
\def\apr {\overline {\partial }}
%%%%%%%%%%%%%%%%%%%% Letters with bar %%%%%%%%%%%%%%%%%%%%%%%%%%
\def\ib{\bar{i}}
\def\jb{\bar{j}}
\def\ub{{\bar{u}}}
\def\wb {\bar{w}}
\def\zb {\bar{z}}
\def\kb {\bar{k}}
\def\Ab {\overline{A}}
%%%%%%%%%%%%%%%%%%%% Letters with tilde %%%%%%%%%%%%%%%%%%%%%%%%%%
\def\wt{\widetilde}
\def\wh{\widehat}
\def\rhot {{\widetilde{\rho}}}
\def\at{\tilde{a}}
\def\bt{\tilde{b}}
\def\lt{\tilde{l}}
%%%%%%%%%%%%%%%%%%%%%%%%%% Underlined letters  %%%%%%%%%%%%%%%%%%%%%%%%
\def\un{\underline{n}}
\def\um{\underline{m}}

%%%%%%%%%%%%%%%%%%%%%%%%%%% Math symbols %%%%%%%%%%%%%%%%%%%%%%%
\def\Ad{{\mathop{\rm Ad}}}
\def\Arg{{\mathop{\rm Arg}}}
\def\Aut{{\mathop{\rm Aut}}}
\def\codim{{\mathop{\rm codim}}}
\def\cok{{\rm cok}}
\def\coker{{\mathop {\rm coker}}}
\def\Ch{{\rm Ch}}
\def\ch{{\rm ch}}
\def\Det{{\rm Det}}
\def\DET{{\rm DET}}
\def\diag{{\rm diag}}
\def\diff{{\rm diff}}
\def\Diff{{\rm Diff}}
\def\End{{\rm End}}
\def\Id{{\rm Id}}
\def\Fun{{\rm Fun}}
\def\Hom{{\rm Hom}}
\def\c{\cdot}
\def\np{\nabla_{\partial}}
\def\npb{\nabla_{\bar {\partial}}}
\def\Ker{{\rm Ker}}
\def\Lie{{\rm Lie}}
\def\Mat{{\rm Mat}}
\def\noi{\noindent}
\def\nn{\nonumber}
\def\pt{{\rm pt}}
\def\rank{{\rm rank}}
\def\Res{{\mathop{\rm Res}}}
\def\Span{{\mathop{\rm span}}}
\def\Sym{{\mathop{\rm Sym}}}
\def\Td{{\rm Td}}
\def\vol{{\rm vol}}
\def\Vol{{\rm Vol}}
\def\frak{\mathfrak}
\def\ov {{\overline}}
\def\tr{{\rm tr}\,}
\def\Tr{{\rm Tr}\,}

\def\gl{\mathfrak{gl}}
\def\hgl{\mathfrak{hgl}}
\def\ssl{\mathfrak{sl}}
\def\ssp{\mathfrak{sp}}
\def\hsp{\mathfrak{hsp}}
\def\su{\mathfrak{su}}
\def\<{\langle}
\def\>{\rangle}
\def\ov{\overline}

\makeatletter
\DeclareRobustCommand{\loplus}{\mathbin{\mathpalette\dog@lsemi{+}}}
\DeclareRobustCommand{\roplus}{\mathbin{\mathpalette\dog@rsemi{+}}}

%%%%%%%%%%%%%%%%%%%%%%%%%%%%%%%%%%%%%%%%%%%%%%%%%%%%%%%%%%%%%%%%%%%
\newtheorem{te}{Theorem}[section]%Usage:\begin{te}Statement\end{te}
\newtheorem{de}{Definition}[section]
\newtheorem{prop}{Proposition}[section]           %  ETC ...
\newtheorem{cor}{Corollary}[section]
\newtheorem{lem}{Lemma}[section]
\newtheorem{ex}{Example}[section]
\newtheorem{rem}{Remark}[section]
\newtheorem{conj}{Conjecture}[section]
\newtheorem{prob}{Problem}[section]
\newtheorem{quest}{Question}[section]
%%%%%%%%%%%%%%%%%%%%%%%%%%%%%%%%%%%%%%%%%%%%%%%%%%%%%%%%%%%%%%%
\newcommand\cod{\operatorname{codim}}
\newcommand\im{\operatorname{im}}
\newcommand\id{\operatorname{id}}
\newcommand\coim{\operatorname{coim}}
\newcommand\rk{\operatorname{rank}}
\newcommand\ann{\operatorname{ann}}
\newcommand{\proof}{\noindent {\it Proof}. }
%%%%%%%%%%%%%%%%%%%%%%SOME DEFINITIONS%%%%%%%%%%%%%%%
\newcommand\bqa{\begin{eqnarray}}
\newcommand\eqa{\end{eqnarray}}
\def\be{\begin{eqnarray}\new\begin{array}{cc}}
\def\ee{\end{array}\end{eqnarray}}
\def\beq{\begin{equation}}
\def\eeq{\end{equation}}
\def\bse{\begin{subequations}}                %%%SUBEQUATIONS
\def\ese{\end{subequations}}
\def\bp{\begin{pmatrix}}
\def\ep{\end{pmatrix}}
\def\bel{\be\label}
\def\h{\hbar}
\def\i{\imath}
\def\square{\hfill{\vrule height6pt width6pt            %Black
depth1pt} \break \vspace{.01cm}}                        %square
%%%%%%%%%%%%%%% New counters %%%%%%%%%%%%%%%%%%%%%%%%%%%%%%%%%%%%
\newcounter{pac}[section]
\newcommand{\npa}{\addtocounter{pac}{1} \noindent {\bf
\arabic{section}.\arabic{pac}}\,\,\,}
\newcounter{pacc}[subsection]
\newcommand{\npaa}{\addtocounter{pacc}{1} \noindent {\bf
\arabic{section}.\arabic{subsection}.\arabic{pacc}}\,\,\,}
%%%%%%%%%%%%%%%%%%%%%%%%%%%%%%%%%%%%%%%%%%%%%%%%%%%%%%%%%%%%%%%%%
%\mathsurround=2pt
%\draft                             %SWITCH ON/OFF DRAFT VERSION%

\setcounter{pac}{0} \setcounter{footnote}0
%%%%%%%%%%%% Title page %%%%%%%%%%%%%%%%%%%%%

\begin{document}

\title{\bf Global $GL_2$ Hecke-Baxter operator}
\author{A.A. Gerasimov, D.R. Lebedev and S.V. Oblezin}
\date{\today}
\maketitle

\renewcommand{\abstractname}{}

\begin{abstract}

  \noindent {\bf Abstract}. We construct a global Hecke-Baxter
  operator for integrable systems of arithmetic type associated with
  the group $GL_2$. It is an
  element of a global Hecke algebra associated with the double coset
  space   $GL_2(\IZ)\backslash GL_2(\IR)/O_2$.   Eigenvalues
  of the global Hecke-Baxter
  operator  acting on the $GL_2$-Eisenstein series
  are given by the   corresponding global $L$-factors. This construction generalizes
  our previous construction of the Hecke-Baxter operators  over
  local completions $\IR$ and $\IQ_p$ of the number field $\IQ$.
  Presumably, zeroes of the corresponding global $L$-factors
  should be subjected to an arithmetic version of the Bethe  ansatz
  equations.
\end{abstract}

 \vspace{5mm}

%\tableofcontents

\section{Introduction}

An interpretation of a wide class of integrable systems in terms of
representation theory provides  important insights both into  theory
of integrable systems and into representations theory
allowing  transferring various techniques from one  area of research
into the other. Among various examples we would like to mention the
formalism of the Baxter operator \cite{Ba} which was properly placed in
representation theory perspective using  Hecke
algebras formalism in \cite{GLO08} (see also \cite{G}).
We coin the term the Hecke-Baxter operator for a
one-parameter family of element of the Hecke algebra reproducing Baxter operator
of various quantum integrable systems.
Starting with the case of spherical principles series
representations of $GL_{\ell+1}(\IR)$ the construction of the
Hecke-Baxter operator  is recently extended
to general principle series representations of $GL_{\ell+1}(\IR)$
  (for details see \cite{GLO25} and reference therein).  A remarkable fact is
that the Hecke-Baxter operators  are directly related to
the Archimedean $L$-factors attached to the corresponding
representations of $GL_{\ell+1}(\IR)$. Precisely the Archimedean $L$-factors appear
as eigenvalues of the Hecke-Baxter operators  acting on  the
spherical and Whittaker functions given by specific matrix elements in the
spherical principle series  representations.
As a direct consequence  the local $L$-factors enter the integral
representations of the Whittaker functions expressed via a version of the
Gelfand-Tsetlin construction of irreducible representations of
$GL_{\ell+1}(\IR)$ \cite{GKL}. Note that the $GL_{\ell+1}(\IR)$-Whittaker
functions are eigenfunctions  of the
quantum $GL_{\ell+1}(\IR)$-Toda chains,  one of the most well-studied
finite-dimensional integrable systems. Actually  the Hecke-Baxter
operator (depending on auxiliary parameter) provides an alternative
formulation of the quantum Toda chain.
Notice in this regard that an advantage of the Hecke algebra
formulation of integrable systems  is in a unified treatment
of both continues and discrete  symmetries of the systems.


Not surprisingly,  proper counterpart of the Hecke-Baxter operator exists  in the case
of representations theory over non-Archimedean fields. The case of
spherical principle series representations of $GL_{\ell+1}(\IQ_p)$ was
considered in \cite{GLO08} together with the
corresponding integrable systems governing spherical and Whittaker
functions over $\IQ_p$. Connection with local $L$-factors
still holds in this case, and local non-Archimedean $L$-factors show up as eigenvalues
of the non-Archimedean Hecke-Baxter operators  acting on
$GL_{\ell+1}(\IQ_p)$-Whittaker functions.


It is natural to expect that the Hecke-Baxter operator formalism may
be further generalized to the case of (global) number fields.
This is indeed so, and in this short note we consider one
representative example  of the global Hecke-Baxter operator: we introduce
a $GL_2$  Hecke-Baxter operator over compactification  $\overline{{\rm
    Spec}(\IZ)}$ of ${\rm  Spec}(\IZ)$ acting on the non-ramified $GL_2$-automorphic
functions (i.e. functions on the double coset $GL_2(\IZ)\backslash
GL_2(\IR)/O_2$). The following result is proven in   Theorem \ref{TH1} in Section 4.
The automorphic functions represented by  matrix elements
of the spherical principle series $GL_2(\IR)$-representations are
eigenfunctions of the proposed global Hecke-Baxter operator with the
eigenvalues given by the corresponding global
$L$-functions  generalizing the completed form of the Riemann zeta function $\zeta(s)$:
 \be
  \hat{\zeta}(s)\,=\,\zeta(s)\, \pi^{-\frac{s}{2}}\,\Gamma\Big(\frac{s}{2}\Big)\,.
 \ee
 This result complements the constructions of \cite{JL}.


 Let us stress that in terms of quantum integrable systems we basically consider
a hyperbolic billiard on upper half-plane modulo action of the
modular group $PSL_2(\IZ)$. These  quantum systems are integrable and are
deeply connected with the quantum
$SL_2(\IR)$-Toda chains: harmonics of the quantum billiard
eigenfunctions are given by solutions of the quantum Toda chain for
integer coupling constants. On the other hand this quantum billiard
is a generalization of the Euclidean billiard arsing in the tropical
limit of the $SL_2(\IR)$-Toda chain  \cite{GL}. This provides an interesting
number theoretic perspective on an interpretation of the tropical limit proposed in
\cite{GL}.



One curious  point worth mentioning is as follows.  In the case of
the integrable systems with discrete spectrum,  the Baxter operator is instrumental in
finding the spectrum given by common eigenvalues of quantum
Hamiltonians. Precisely the eigenvalues of quantum Hamiltonian are
expressed in a simple way through zeroes of the eigenvalues of the Baxter operators (considered as functions of an auxiliary parameter). In turn,
 zeroes of the eigenvalues of the Baxter operator satisfy a set of
 equations called the Baxter equations. Our interpretation of the global
 $L$-functions as eigenvalues of the Hecke-Baxter operators points to
 the problem of finding an  analog of the Baxter  equations in the
 arithmetic setup. This might extend a traditional optics
for looking at analogs of Riemann hypotheses for global $L$-functions as well as various conjectures on
 the special values of global $L$-functions. This suggestion seems
 close to  the Faddeev-Pavlov approach \cite{FP} to studying
 analytic properties of the Riemann zeta-functions  via scattering
 theory. This line of research seems still  worth to pursue.

 Let us also note that
 the   Hecke-Baxter operators (and more general elements of Hecke algebras) are
examples of averaging operators that are ubiquitous in various areas of Mathematics
and Physics. One interesting example of the averaging operator appears
in  the  Kadanov approach to the renormalization (semi)group  in lattice
quantum field theories  (see e.g. \cite{Ka}). Fixed points of the
renormalization group flow corresponding to the eigenvalues of the Kadanov
operators describe continuum limit of the lattice
theory. The analogy between the constructions of \cite{Ka}  and this
paper is very fruitful and will be considered in details  elsewhere.   As an
obvious next step  we  are  going to generalize the results of
this note to the global ramified  case for the groups $GL_{\ell+1}$  of higher ranks.



{\it  Acknowledgements:} The research of S.V.O. is partially
supported by the Beijing Natural Science Foundation grant IS24004.



\section{Automorphic forms and global Hecke-Baxter operator:  $GL_1$ }


In this Section we consider the almost trivial case of the Lie group $GL_1$. Our
goal is to introduce  basic  elements of the construction to proceed in the following
Sections with a more involved  case of $GL_2$.

Let us define the non-ramified $GL_1$ automorphic functions  as
functions on  $GL_1(\IR)$  invariant under the left action of
$GL_1(\IZ)$ and right action of the orthogonal subgroup $O_1\subset
GL_1(\IR)$ i.e. these functions may be considered as functions on
the double coset space
\be\label{Mone}
\CM_1=GL_1(\IZ)\backslash GL_1(\IR)/O_1\,.
\ee
Note that $\CM_1$  is a $GL_1(\IZ)$-orbifold as $GL_1(\IZ)$
acts trivially on the coset space $GL_1(\IR)/O_1$. We however are
interested in the space of functions on $\CM_1$ and thus  might ignore this
subtlety by  considering  functions on
$GL_1(\IR)/O_1$ (invariant under trivial action of $GL_1(\IZ)$)
as functions on $\CM_1$. Thus taking into account the identifications
\be
GL_1(\IR)\simeq \IR^*, \qquad GL_1(\IZ)\simeq \mu_2, \qquad O_1\simeq
\mu_2\,,\qquad \mu_2=\{\pm 1\}\,.
\ee
the $GL_1$-automorphic functions may be identified with functions on
$\IR_+=\IR^*/\mu_2$.

The  double cosets space  \eqref{Mone} allows an interpretation as   a
moduli  space of circles $S^1$ supplied with  $S^1$-invariant
metrics. Indeed $\CM_1$ may be presented in the following factorized form
\be\label{Mone1}
\CM_1=GL_1(\IZ)\backslash GL_1(\IR)\times_{GL_1(\IR)} GL_1(\IR)/O_1\,.
\ee
The first factor $GL_1(\IZ)\backslash GL_1(\IR)$ should be identified with
the space of lattices $L$ in $\IR$
\be
L=\{  nv|n\in \IZ\}, \qquad v\in \IR-\{0\}\,,
\ee
characterized by  non-zero real numbers $v$ with group
$GL_1(\IZ)=\mu_2$ of automorphisms of $L$ acting by effectively
changing the sign of $v$. By action of $GL_1(\IR)\simeq \IR^*$ any
lattice may transformed   into the standard one $L=\IZ\subset \IR$.
The second factor $GL_1(\IR)/O_1$ in \eqref{Mone1}
is identified with the  space of constant
metrics on $\IR$ with  $O_1$ being
stabilizer of a reference metric.  From this description we  infer that
the space \eqref{Mone} is  the moduli space of
 $GL_1(\IR)$-equivalence classes of  pairs of lattices  and constant metrics
on $\IR$ or equivalently as the moduli space of circles $S^1=\IR/\IZ$ supplied with
constant metrics. Algebraically $\CM_1$ may be understood as a moduli
space of rank one $\IZ$-modules $L$ supplied with a metric on its real
extension $L\otimes_\IZ \IR$.  The interpretation of $\CM_1$ as moduli
space metricized circles  provides us with a canonical
coordinate on $\CM_1$ , the volume of the corresponding circle
\be\label{VolC}
|x|={\rm Vol}_h(\IR/L)\,.
\ee
Here $x$ is the canonical coordinate on $GL_1(\IR)=\IR^*$ identified
with the moduli space of the oriented circles $S^1_{or}$ supplied with a constant
metric $h$. Given  a pair $(S^1_{or},h)$
we might consider corresponding volume one-form $\omega$ so that
the coordinate $x$ would be a period of this form
\be\label{VolC1}
x=\int_{\IR/L}\omega\,.
\ee

We are interested in a particular bases in the
space of $GL_1$-automorphic functions given by $GL_1$ analog of the Eisenstein
functions. This bases may be defined precisely in various ways
but having in mind  subsequent
generalizations to the case of $GL_2$ we  construct these functions
using  representation theory approach. Precisely we define $GL_1$
Eisenstein functions   as  matrix elements of
$GL_1(\IZ)$ and $O_1$ invariant vectors in unitary spherical principle series
 representations $GL_1(\IR)$.
Let $(\pi_\gamma,\CV_\gamma)$, $\gamma\in \IR$ be a  one-dimensional unitary spherical
unitary representation of
$GL_1(\IR)$, $\<\,,\,\>$ be the corresponding
 Hermitian pairing  and  $v\in \CV_{\gamma}$ be such that
$\<v,v\>=1$. By definition   spherical representations of
$GL_1(\IR)\simeq \IR^*$ are  factored through homomorphysm $\IR^* \to
\IR_+$ and are  given  by
\be\label{RepGL1}
\pi_\gamma:\,\,x\longrightarrow |x|^{\imath \gamma}\,, \qquad x\in \IR^*\,.
\ee
Consider the following matrix elements  in representation $(\pi_\gamma,\CV_\gamma)$
\be \label{ME1}
\psi_\gamma(x)=\<v,\pi_\gamma(x)\,v\>=|x|^{\imath \gamma}\,, \qquad
x\in GL_1(\IR)\simeq \IR^*\,.
\ee
It is a $\mu_2$-invariant  function  on $GL_1(\IR)$ and
thus $\psi_\gamma(x)$ is a lift of a function on $\CM_1=\IR_+$. Corresponding function
on $\CM_1$ will be called  the $GL_1(\IR)$-Eisenstein series
associated with the representation $(\pi_\gamma,\CV_\gamma)$.
In the following we will consider
interchangeably automorphic eigenfunctions as functions on $\CM_1$
depending on $|x|$ or as $\mu_2$-invariant functions on $GL_1(\IR)$
depending on $x$.

The  $GL_1$-Eisenstein functions  may be defined also as
eigenfunctions  of some operators. In the following we will be interested
in characterization of the $GL_1$-Eisenstein functions  as common eigenfunctions
of elements of the Hecke algebra associated with the space of double
cosets (this formulation especially useful as
takes into account the invariance under both  discrete and
continues groups). Define the Hecke
algebra associated with the double coset space  \eqref{Mone} as
 a tensor product of two convolution algebras
 $\CH(GL_1(\IQ),GL_1(\IZ))$ and $\CH(GL_1(\IR),O_1)$.
Recall that Hecke algebra $\CH(G,K)$ associated with the pair
$K\subset G$ is an associative algebra of the proper subset of
$K$-biinvariant functions on $G$  under convolution. It is natural to
consider the maximal subset of functions on $G$ such that the
convolution operation is defined. In the case when $(G,K)$ is a
Gelfand pair i.e. $K$ is a fixed point of an involution of $G$ the
corresponding algebra is commutative. The power of the Hecke algebra
formalism is in the fact that $\CH(G,K)$ in general is not a group
algebra but replaces it in various representation theory
constructions.

The Hecke algebra $\CH(GL_1(\IR),O_1)$,
the  algebra of $O_1$-biinvraint functions on $GL_1(\IR)$,  acts naturally
on the functions on $GL_1(\IR)/O_1$   and in particular
on the functions on double coset $\CM_1$ via convolution.   Note that
it  does not take into account $\IZ$ (and thus $\IQ$) structure
responsible for the lattice moduli space interpretation of $\CM_1$. To
take into account this arithmetic structure we consider another Hecke
algebra $\CH(GL_1(\IQ),GL_1(\IZ))$ which we identify with the
convolution algebra of
$GL_1(\IZ)$-biinvariant generalized function on $GL_1(\IR)$ with the support at
$GL_1(\IQ)\subset GL_1(\IR)$. It is easy to verify that the algebras
$\CH(GL_1(\IQ),GL_1(\IZ))$ and $\CH(GL_1(\IR),O_1)$
 are (mutually) commutative associative algebras acting on
functions on $\CM_1$ from the right and the left
correspondingly. Note that in the considered case of $GL_1$ the Hecke algebras are
actually groups algebras of the quotient groups $GL_1(\IQ)/GL_1(\IZ)$ and
$GL_1(\IR)/O_1$.

It is instructive to describe the $\CH(GL_1(\IQ),GL_1(\IZ))$-action
 considering $\CM_1$ as a  moduli space of
metricized circles. Given a lattice $L\subset\IR$, we can consider a
sublattice $L_p\subset L$ of index $[L:L_p]=p$, and also we consider
$L_{1/q}\subset\IR$, such that $L\subset L_{1/q}$ with
$[L_{1/q}:L]=q$. Now we define, for any
$p/q\in \IQ_+^*$, an operation $T_{p/q}$ on lattices by first taking
$L\subset L_{1/q}$ such that
$[L_{1/q}:L]=q$, and then considering a sublattice $L_{p/q}\subset
L_{1/q}$ of index $[L_{1/q}:L_{p/q}]=p$. Thus we have the following
operator acting on functions on the space of lattices:
  \be
   (T_{p/q}\cdot f)(L)\,
   =\sum_{L\subset L_{1/q}\supset L_{p/q} }\,f(L_{p/q}),
   \qquad
   [L_{1/q}:L]=q, \quad [L_{1/q}:L_{p/q}]=p\,.
  \ee
 In terms of the functions of the coordinate $|x|\in \IR_+$ this
reduces to a simple mupltiplication operation
 \be
  (T_{p/q}\cdot f)(|x|)=f(p/q\cdot |x|)\,.
 \ee
These operators obviously belongs to the algebra
$\CH(GL_1(\IQ),GL_1(\IZ)$ and satisfy the following relations:
 \be
  T_{p_1/q_1}\circ T_{p_2/q_2}=T_{(p_1p_2)/(q_1q_2)}\,.
 \ee
Let us remark that the collection of operators $T_{p/q},\,p/q\in\IQ^*_+$
provides a $GL_1$-analog of the modular tower structure arising in the case of
$GL_2$.

To construct  a meaningful generating function   we consider a
multiplicative semigroup $\IZ_+\subset \IQ_+^*$. Corresponding
elements of the Hecke algebra  $\CH(GL_1(\IQ),GL_1(\IZ)$
act as follows
 \be\label{Tn}
  (T_n\cdot f)(L)=f(nL)\,,\qquad n\in \IZ_+\,,
 \ee
or equivalently, in terms of functions on $GL_1(\IR)$
 \be\label{Tnx}
  (T_n\cdot f)(x)=f(nx)\,,\qquad n\in \IZ_+\,, \qquad x\in \IR^*\,.
 \ee
These operators may be conveniently combined into  the generating series
 \be\label{GenZ}
  Q^{GL_1(\IZ)}_s\,=\sum_{n=1}^{\infty}\frac{1}{n^s} T_n\,.
 \ee
Its action on the functions of $x$ is given by
 \be
  (Q^{GL_1(\IZ)}_s\cdot f)(x)\,=\sum_{n=1}^{\infty} n^{-s}\,f(nx)\,.
 \ee
Now we consider a kind of generating functions for the elements of the
Hecke algebra $\CH(GL_1(\IR),O_1)$ providing a
proper analog for the generating function \eqref{GenZ}. Such
generating functions  were introduced
(for more general case of $GL_{\ell+1}(\IR)$) in \cite{GLO08}
 under the name the Hecke-Baxter operator.  Precisely
the $GL_1(\IR)$ Hecke-Baxter operator is the integral operator,
 \be
  (Q^{GL_1(\IR)}_s\cdot f)(x)\,
  =\int\limits_{\IR^*}\!d\mu_{\IR^*}(y)\,\, |y|^s\,f(y^{-1}x)\,,\quad
  d\mu_{\IR^*}(y)=e^{-\pi y^2}\,\frac{dy}{y}\,,
 \ee
acting by convolution with $O_1$-biinvariant function on $GL_1(\IR)$
\be\label{ArKernel}
Q^{GL_1(\IR)}_s(y)=|y|^s\,e^{-\pi y^2}\,.
\ee
In the following,  for brevity, we identify suitable functions on Lie groups, operators
 obtained by the actions of these functions via convolution and
 the corresponding integral kernels.



\begin{prop} The matrix elements  \eqref{ME1}
 \be\label{ME11}
  \psi_{\gamma}(x)=\<v,\pi_{\gamma}(x)\,v\>=|x|^{\imath \gamma}\,,
 \ee
are common  eigen-functions of the operators $Q^{GL_1(\IZ)}_s$ and\,
 $Q^{GL_1(\IR)}_s$:
 \be
  \bigl(Q^{GL_1(\IZ)}_s\cdot  \psi_{\gamma}\bigr)(x)\,
  =\,\zeta(s-\imath \gamma)\, \psi_{\gamma}(x)\,,
 \ee
 \be
  \bigl(Q^{GL_1(\IR)}_s\cdot  \psi_{\gamma}\bigr(x)\,
  =\,L^{\IR}(s-\imath \gamma)\,\psi_{\gamma}(x)\,.
 \ee
The  eigenvalues are given by
 \be
  \zeta(s)=\sum_{n\in \IZ_+}\,\frac{1}{n^s}\,,\qquad
  L^{\IR}(s)=\pi^{-s/2}\Gamma\left(\frac{s}{2}\right)\,,
 \ee
where we impose ${\rm Re}(s)>1$ for convergence.
\end{prop}

\proof Using
 \be
  (T_n\cdot \psi_{\gamma})(x)\,
  =\,\psi_{\gamma}(nx)\,
  =\,n^{\imath \gamma}\,
  \psi_{\gamma}(x)\,,
 \ee
we indeed have
 \be
  \bigl(Q^{GL_1(\IZ)}_s\cdot  \psi_{\gamma}\bigr)(x)\,
  =\left(\sum_{n\in\IZ_+}\frac{1}{n^{s-\imath \gamma}}\right)\,
  \psi_{\gamma}(x)\,
  =\,\zeta(s-\imath \gamma)\, \psi_{\gamma}(x)\,.
 \ee
The analogous statement for the  Archimedean Hecke-Baxter
operator  $Q^{GL_1(\IR)}_s$ acting on matrix element \eqref{ME1} via
 \be
  \bigl(Q^{GL_1(\IR)}_s\cdot \psi_{\gamma}\bigr)(x)\,
  =\int\limits_{\IR^*}\!\frac{dy}{y}\,\,|y|^s\,e^{-\pi y^2}\,
  \psi_{\gamma}(y^{-1}x)\,
  =\,\pi^{-\frac{s-\imath \gamma}{2}}\,
  \Gamma\left(\frac{s-\imath \gamma}{2}\right)\,\psi_{\gamma}(x)\,,
 \ee
reduces basically to the integral representation of the
Gamma-function.  $\Box$


Now we introduce  main object of our considerations in this Section,
global Hecke-Baxter operator $\wh{Q}_s$.

\begin{de} The $GL_1$ global Hecke-Baxter operator is the operator
  acting in the space of functions on
  $GL_1(\IR)\simeq \IR^*$ via convolution with the   function
 \be\label{GLB}
  \wh{Q}^{GL_1}_s(x)\,
  =\,\frac{1}{2}\,|x|^s\Big(\Theta(0|\imath
  x^2)-1\Big)\,,\qquad x\in \IR^*\,,
\ee
where the theta-constant is given by
 \be
  \Theta(0|\tau)=\sum_{n\in \IZ} e^{\imath \pi \tau n^2}\,.
 \ee
\end{de}


\begin{prop}  Consider the  matrix element \eqref{ME1}
 \be \label{ME12}
  \psi_{\gamma}(x)=\<v,\pi_{\gamma}(x)\,v\>=|x|^{\imath \gamma}\,,
 \ee
in the unitary spherical principle series representation
$(\pi_\gamma,\CV_\gamma)$ of
$GL_1(\IR)$. Define  completed zeta-function as follows
 \be\label{ExtRM}
  \hat{\zeta}(s)\,=\,\zeta(s)\,\pi^{-\frac{s}{2}}\,\Gamma\Big(\frac{s}{2}\Big)\,.
 \ee
Then the global Hecke-Baxter operator \eqref{GLB} acts on \eqref{ME12}
via multiplication by a shifted completed zeta-function
 \be\label{GL1act}
  \bigl(\wh{Q}^{GL_1}_s\cdot \psi_\gamma\bigr)(x)\,
  =\hat{\zeta}(s-\imath \gamma)\,\psi_\gamma(x)\,,\qquad{\rm Re}(s)>1\,.
 \ee
\end{prop}

\proof We have
 \be
  \bigl(\wh{Q}^{GL_1}_s*\psi_\gamma\bigr)(x)\,
  =\!\int\limits_{\IR^*}\frac{dy}{y}\,\, |y|^{s}\,\,\,\frac{1}{2}\Big(
  \Theta(0|\imath  y^2)-1\Big)\,\psi_\gamma(y^{-1}x)\\
  =\sum_{n\in \IZ_+}\,\int\limits_{\IR^*}\!\frac{dy}{y}\,|y|^{s}\,e^{-\pi |ny|^2}\,
  |y|^{-\imath \gamma}\,\psi_\gamma(x)\,\\
  =\,\pi^{-\frac{s-\imath \gamma}{2}}\,\Gamma\Big(\frac{s-\imath \gamma}{2}\Big)\,
  \left(\sum_{n\in \IZ_+} \frac{1}{n^{s-\imath
        \gamma}}\right)\,\,\psi_\gamma(x)\\
        =\,\pi^{-\frac{s-\imath \gamma}{2}}\,\Gamma\Big(\frac{s-\imath \gamma}{2}\Big)\,
  \zeta(s-\imath \gamma)\,\,\psi_\gamma(x)\,,
\ee
thus arriving at the required identity. $\Box$

Let us notice that in the simple case of the trivial representation
$(\pi_{\g=0},\CV_{\gamma=0})$ the identity
\eqref{GL1act} reduces to the standard
integral expression for the completed zeta-function:
 \be
  \hat{\zeta}(s)\,
  =\sum_{n\in \IZ_+}\,\,\int\limits_{\IR_+}\frac{dt}{t}\,\,
  t^{\frac{s}{2}}\,\,e^{-\pi tn^2}\,.
 \ee

The fundamental property of the completed  Riemann zeta-function
\eqref{ExtRM} is the functional relation
\be\label{FR1}
\hat{\zeta}(1-s)=\hat{\zeta}(s)\,.
\ee
The  matrix elements \eqref{ME12} also respect an analogous
reflection symmetry
\be
\psi_{-\gamma}(x^{\tau})=\psi_\gamma(x)\,,
\ee
where $x^{\tau}:=x^{-1}$ is the involution on the group
$GL_1(\IR)$. Taking into account that matrix elements \eqref{ME12} are
eigenfunctions of the Hecke-Baxter operator $\hat{Q}_s^{GL_1}$ with
the eigenvalues expressed through completed zeta-function  one
expects that the kernel of the Hecke-Baxter integral operator
should also  satisfy a form of functional equation. Indeed we have the
following relation
\be\label{FR2}
\wh{Q}^{GL_1}_{1-s}(x^\tau)+\frac{1}{2}|x^\tau|^{1-s}=
\wh{Q}^{GL_1}_{s}(x)+\frac{1}{2}|x|^{s}\,,
\ee
where the terms $|x|^s$ compensate the correction terms
entering the expression \eqref{GLB} of the kernel via theta-constant.
The functional relation  \eqref{FR2} is a direct consequence of the
modular properties of the theta-constant verified using the Poisson
summation formula. Thus we have a deep connection between properties
of the global Hecke-Baxter  operator and analytic properties of the
completed  Riemann zeta-function.


It is possible to interpolate between the global and Archimedean Hecke-Baxter
operators via considering a $GL_1$-analog of the
congruence (semi)groups. Let us introduce
the following semigroup $\IZ_+^{(N)}\subset\IQ^*_+$:
 \be
  \IZ_+^{(N)}\,=\,\{\eta \in \IQ_+^*|\eta=1+Nm, \,m\in \IZ_+\}\,.
 \ee
Then the generating function of the elements of the Hecke
algebra $\CH(GL_1(\IQ),GL_1(\IZ))$ reads
 \be
  Q^{GL_1(\IZ)}_{s,N}=\sum_{n=1}^{\infty} \frac{1}{(1+nN)^s}\,T_{1+nN}\,.
 \ee
Therefore the  modified kernel of the global Hecke-Baxter operator is
given by
 \be\label{GLBN}
  \wh{Q}^{GL_1}_{s,N}(x)\,
  =\,\frac{1}{2}\,|x|^s\,\,\Theta^{(N)}(0|\imath x^2)\,,\qquad x\in
  \IR^*\,,\qquad N>1\,.
 \ee
Here
\be
\Theta^{(N)}(0|\tau)=\sum_{n\in \IZ} e^{\imath \pi \tau (1+Nn)^2}=
e^{\imath \pi \tau }\,
\sum_{n\in \IZ} e^{\imath \pi \tau N^2n^2+2\pi \imath Nn\tau}=
e^{\imath \pi \tau }\,\Theta_{N^2}\Big(\frac{\tau}{N}\Big|\tau\Big)\,,
\ee
where the level $k$ theta function is defined by:
 \be
  \Theta_k(z|\tau)\,
  =\sum_{n\in \IZ} e^{\imath \pi k\tau n^2+2\pi \imath knz}\,.
 \ee
Now it is easy to check that taking the limit $N\to+\infty$ the kernel \eqref{GLBN}
of the modified global Hecke-Baxter operator   turns into the kernel
\eqref{ArKernel} of the Archimedean Hecke-Baxter operator.  This
provides a kind of regularization of the Archimedean  Hecke-Baxter
operator.


\section{$GL_2$-automorphic forms }

In this Section we recall a  construction of the Eisenstein functions
for $GL_2$ (for a review see e.g. \cite{ILP}).
Let us start with  the  double coset space
\be\label{Mtwo}
\CM_2=GL_2(\IZ)\backslash GL_2(\IR)/O_2\,.
\ee
The space \eqref{Mtwo}, similar to  \eqref{Mone} in the previous Section
 is an orbifold. We will define the space of functions on
\eqref{Mtwo} as functions on $GL_2(\IR)/O_2$ invariant under left
action of $GL_2(\IZ)$. The double coset space  $\CM_2$ allows
interpretation as a moduli space of two-tori $T^2$ supplied with
$T^2$-invariant metrics. Indeed it may  be identified with the space
of pairs of lattices $L$   and a constant metrics $h$ in $\IR^2$
modulo simulations action of $GL_2(\IR)$. Taking into account that
any lattice in $\IR^2$ may be transformed by linear transformations
into the standard one $\IZ^2\subset \IR^2$ we arrive at the
identification of  $\CM_2$ with the moduli space of metricizes tori.
Constant metric on $T^2$ defines a conformal metric, and therefore a
complex structure, supplying $T^2$ with a structure of elliptic
curve $E(\IC)$. As a result the space $\CM_2$ is naturally fibred
over the moduli space $\CM_2^c$ of elliptic curves. The fiber of the
projection $\CM_2\to \CM_2^c$ may be identified  with $\IR_+$
supplied with the natural coordinate, the volume of
$T^2=\IR^2/\IZ^2$ in the considered metric. In turn, the space
$\CM_2^c$ of complex structures  has the double coset description as
the upper complex half-plane \be\label{ISO12}
\CH_+=PSL_2(\IR)/SO(2)\,, \ee modulo action of the discrete group
$PSL_2(\IZ)$. In the standard linear coordinates on $\CH_+=\{\tau\in
\IC|{\rm
  Im}(\tau)>0\}$ the isomorphism \eqref{ISO12}
may be described as  $SO_2$-projection of the following lift
 \be
  \tau=(\tau_1+\imath \tau_2)\in\CH_+ \longrightarrow
  \frac{1}{\sqrt{\tau_2}}
  \begin{pmatrix}  \tau_2 & \tau_1 \\ 0 & 1\end{pmatrix}\in
  PSL_2(\IR)\,,
 \ee
while the left action of $PSL_2(\IZ)$ on $\CH_+$ is realized by  the
fractional linear transformations.

Now we introduce a special kind of $GL_2$-automorphic functions,
the Eisenstein functions. The $GL_2$-Eisenstein functions are
associated with  spherical principle
series representations entering  the
decomposition of the $GL_2(\IR)$-representation $L^2(GL_2(\IZ)\backslash GL_2(\IR))$
acting from the right:
 \be
  \bigl(\pi(g)\cdot f\bigr)(\tilde{g})=f(\tilde{g}\cdot g),\qquad f\in
  L^2(GL_2(\IZ)\backslash GL_2(\IR))\,.
 \ee
The irreducible components
corresponding to spherical principle series representations are in one
to one correspondence with the elements of
$L^2(GL_2(\IZ)\backslash GL_2(\IR))$ invariant under the subgroup
$O_2\subset GL_2(\IR)$.  This  correspondence   follows by  the uniqueness of
spherical vectors in spherical principle series
representations (see e.g. \cite{GGPS}).

 Let $(\pi_\gamma,\CV_\gamma)$ be a unitary spherical principle
series
 representation  of $GL_2(\IR)$
realized via induction from the Borel subgroup $B\subset GL_2(\IR)$
(identified with the subgroup of  lower triangular matrices) via the
spherical character of $B$
 \be
  \pi_{\g}={\rm Ind}_B^{GL_2(\IR)}\chi^B_\gamma,\quad
  \chi^B_\gamma(b)=\prod_{j=1}^2 |b_{jj}|^{\imath \g_j-\rho_j}\,,\quad
  \rho=\Big(\frac{1}{2},\,-\frac{1}{2}\Big)\,.
 \ee
Therefore the representation space $\CV_\gamma$
 \be
  \CV_{\g}=\bigl\{f\in\Fun(GL_2(\IR))\,:\quad
  f(bg)=\chi^B_{\g}(b)\,f(g),\,\,b\in B\bigr\}\,,
 \ee
supports the right $GL_2(\IR)$-action. Using the Bruhat decomposition,
the representation $(\pi_\gamma,\CV_\gamma)$ may be realized in the
space of functions on $B\backslash GL_2(\IR)=\IP^1(\IR)$, which in
turn can be identified with the compactification the (opposite)
unipotent subgroup $N_+\subset GL_2(\IR)$:
 \be
  N_+\,=\,\Big\{n_x=\Big(
  \begin{smallmatrix} 1&&x\\&&\\0 &&1\end{smallmatrix}\Big)\,\Big|
                                                 \quad x\in\IR\Big\}\,.
 \ee
Explicitly, the $GL_2(\IR)$ action in $\CV_\gamma\subset
L^2(B\backslash GL_2(\IR))$, for
$g=\Big(\begin{smallmatrix}a&&b\\&&\\c&&d\end{smallmatrix}\Big)$, is
given by
 \be\label{TrRools}
  [\pi_\gamma(g)\cdot f](x)\,=\,f(n_xg)\,
 =\,|\det g|^{\imath\gamma_2+\frac{1}{2}}\,|a+xc|^{\imath (\gamma_1-\gamma_2)-1}\,
  f\Big(\frac{b+xd}{a+xc}\Big)\,,
 \ee
providing the following $GL_2(\IR)$-action on $\IR P^1$:
 \be\label{RLaction}
  g\cdot x\,=\,\frac{b+xd}{a+xc}\,,\qquad
  g=\Big(\begin{smallmatrix}a&&b\\&&\\c&&d\end{smallmatrix}\Big)\,.
  \ee
  The Hilbert space structure on the space of functions on $N_+$
is defined via  the pairing
 \be\label{PAIR}
  \<\phi_1\,,\phi_2\>\,=\int\limits_{N_+}\!dn_x\,\ov{\phi_1(n_x)}\,\,\phi_2(n_x)\,.
 \ee
We  supply the  Hilbert space  $\CV_{\gamma}$ with a structure of
the rigged Hilbert spaces $\CV_{\gamma}^{(t)}\subset
\CV_{\gamma}\subset \CV_{\gamma}^{(g)}$ (the Gelfand triple) where
$\CV_{\gamma}^{(t)}$ is the subspace of smooth test functions and
$\CV_{\gamma}^{(g)}$ is the  space of generalized functions. The
pairing \eqref{PAIR} extends to the duality between
$\CV_{\gamma}^{(t)}$ and $\CV_{\gamma}^{(g)}$. We would like to
represent the $GL_2$-Eisenstein functions in terms of matrix
elements of the spherical principle series representations. As we
will see the corresponding matrix elements are not well-defined for
the unitary principle series and requires analytic continuation of
the representation parameters $\gamma=(\gamma_1,\gamma_2)\in \IR^2$.
In turn this implies a replacement of the structure of Hilbert space
$\CV_\gamma$ by a pair of dual spaces. Thus we consider
$\CV_{\gamma}^{(t)}$ and $\CV_{\gamma}^{(g)}$ to be the dual
$GL_2(\IR)$-modules  with representation parameters having a small
imaginary part. To construct matrix element representation of the
Eisenstein functions we start with the  explicit construction
$GL_2(\IZ)$- and $O_2$-invariant vectors. Let
$\phi_{O_2}\in\CV_{\g}^{(t)}$ be the unique (up to normalization)
spherical vector, i.e. vector invariant under the action of
$O_2\subset GL_2(\IR)$ and let $\phi_{GL_2(\IZ)}$ be the unique (up
to normalization) $GL_2(\IZ)$-invariant vector in $\CV_{\g}^{(g)}$.


\begin{lem} In the representation $(\pi_\gamma,\CV_\gamma)$ given by \eqref{TrRools}
  the $O_2$-invariant vector $\phi_{O_2}\in\CV_{\g}^{(t)}$ and
the  $GL_2(\IZ)$-invariant vector $\phi_{GL_2(\IZ)}\in\CV_{\g}^{(g)}$
may be chosen in the following form
 \be\label{SphVect1}
  \phi_{O_2}(x)\,
  =\,(1+x^2)^{\frac{\imath(\gamma_1-\gamma_2)-1}{2}}\,,
 \ee
 \be \label{SphVect2}
  \phi_{GL_2(\IZ)}(x)=\sum_{(m,n)\in \CP}\!\!
  |n|^{-\imath(\gamma_1-\gamma_2)}\, \delta(m+nx)\,,
  \ee
  where
 \be\label{Range}
  \CP=\{ (m,n)\in \IZ^2-\{0\}|\,\, {\rm gcd}(m,n)=1,\, (m,n)\sim
  (-m,-n)\}\,.
 \ee
\end{lem}

\proof Elements of $O_2\subset GL_2(\IR)$
may be written in the following form
 \be
  k=\begin{pmatrix} \cos \theta & (-1)^\epsilon \sin \theta
    \\-\sin\theta &(-1)^\epsilon\,   \cos\theta
  \end{pmatrix} \in O_2\,,\quad0\leq\theta<2\pi\,, \qquad \epsilon\in \{0,1\}\,.
 \ee
  By \eqref{RLaction}, a direct calculation gives
  \be
  1+(k\cdot x)^2\,
  =\,1+\Big(\frac{(-1)^{\e}\sin\theta+x(-1)^{\e}\cos\theta}
  {\cos\theta-x\sin\theta}\Big)^2\,.
  \ee
Taking into account $|\det k|=1$ we infer from \eqref{TrRools}
  the $O_2$-invariance of the vector \eqref{SphVect1}.

  Next,   we find a $GL_2(\IZ)$-invariant vector $\phi_{GL_2(\IZ)}$ by
  solving the   following equation,
 \be\nonumber
  [\pi_\gamma(g)\cdot \phi_{GL_2(\IZ)}](x)\,
  =\,\phi_{GL_2(\IZ)}(n_xg)\,=\,\phi_{GL_2(\IZ)}(n_x)\,,\qquad
  g=\Big(\begin{smallmatrix}a&&b\\&&\\c&&d\end{smallmatrix}\Big)\in GL_2(\IZ)\,.
 \ee
The group $GL_2(\IZ)$ is generated by
 \be
  T=\Big(\begin{smallmatrix}1&&1\\&&\\0&&1\end{smallmatrix}\Big),\qquad
  S=\Big(\begin{smallmatrix}0&&1\\&&\\1&&0\end{smallmatrix}\Big),\qquad
  R=\Big(\begin{smallmatrix}-1&&0\\&&\\0&&1\end{smallmatrix}\Big)\,,
 \ee
so by \eqref{RLaction} each generator is acting in
$\CV_\gamma\subset L^2(B\backslash GL_2(\IR))$ via
 \be\label{GL2Zrel}
  [\pi_{\g}(T)\,f](x)\,=\,f(x+1),\qquad
  [\pi_{\g}(R)\,f](x)\,=\,f(-x),\\
  \,[\pi_{\g}(S)\,f](x)\,=\,|x|^{\i(\g_1-\g_2)-1}\,f(x^{-1})\,.
 \ee
Considering the expression \eqref{SphVect2}, its $R$-invariance
reduces to change $n\to -n$ of the summation variable, and the
invariance under $T$ is compensated the change of the variable $m\to
m-n$ (which does not spoil the condition $(m,n)=1$). To check the
invariance under $S$ we take into account the following identity
 \be
  |x|^{\imath (\gamma_1-\gamma_2)-1}\, \delta(m+x^{-1}n)\,
  =\,|x|^{\imath (\gamma_1-\gamma_2)}\, \delta(xm+n)\,
  =\,\frac{|n|^{\imath (\gamma_1-\gamma_2)}}
  {|m|^{\imath (\gamma_1-\gamma_2)}}\, \delta(xm+n)\,.
 \ee
This completes a verification of the required properties of
\eqref{SphVect1} and \eqref{SphVect2}. $\Box$

Now the Eisenstein automorphic function
associated with $(\pi_\gamma,\CV_\gamma)$ is defined as a matrix  element
\be \label{Autform}
\Phi_{\gamma}(g)=\<\phi_{GL_2(\IZ)},\pi_\gamma(g) \phi_{O(2)}\>\,.
\ee
Note that $\Phi_{\gamma}(g)$ obviously  defines a function on the double coset $\CM_2$.
Explicit realization  of the  principle series representation allows
to obtain  explicit expression for the $GL_2$-automorphic
form  \eqref{Autform}.


\begin{prop} For the $GL_2$-Eisenstein function given by the matrix element \eqref{Autform}, the following series representation holds,
  \be\label{Eisen}
   \Phi_\gamma(g)=\<\phi_{GL_2(\IZ)}, \pi_\gamma(g)\, \phi_{O_2}\>\\
   =\,|\det g|^{\imath \gamma_2+\frac{1}{2}}\!
   \sum_{(n,m)\in\CP}\!
  \bigl|(na+mc)^2+(nb+md))^2\bigr|^{\frac{\imath(\gamma_1-\gamma_2)-1}{2}}\,, \quad
  g=\Big(\begin{smallmatrix}a&&b\\&&\\c&&d\end{smallmatrix}\Big)\,.
 \ee
provided ${\rm Im}(\g_1-\g_2)>\frac{1}{2}$ for convergence.
\end{prop}

\proof Using \eqref{SphVect1}, \eqref{SphVect2} and \eqref{TrRools} we
have
 \be
  \Phi_{\gamma}(g)\,
  =\,\<\phi_{GL_2(\IZ)},\,\pi_\gamma(g)\,\phi_{O_2}\>\,
  =\!\int\limits_{N_+}\!dn_x\,\,\ov{\varphi_{GL_2(\IZ)}(n_x)}\,\,\phi_{O_2}(n_xg)\\
  =\,|\det(g)|^{\i\g_2+\frac{1}{2}}\!\sum_{(n,m)\in\CP}
  |n|^{\i(\g_1-\g_2)}\!
  \int\limits_{\IR}\!dx\,\,\delta(m+xn)\\
  \times\,\,\bigl|(a+xc)^2+(b+xd)^2\bigr|^{\frac{\i(\g_1-\g_2)-1}{2}}\\
  =\,|\det(g)|^{\i\g_2+\frac{1}{2}}\!\sum_{(n,m)\in\CP}
  |n|^{\i(\g_1-\g_2)-1}\,
  \Big|\Big(a-\frac{mc}{n}\Big)^2+\Big(b-\frac{md}{n}\Big)^2
  \Big|^{\frac{\i(\g_1-\g_2)-1}{2}}\,,
%  =\,|\det(g)|^{\i\g_2+\frac{1}{2}}\!\sum_{(n,m)\in\CP}
%  \bigl|(na-mc)^2+(nb-md)^2\bigr|^{\frac{\i(\g_1-\g_2)-1}{2}}

 \ee
which gives \eqref{Eisen} after changing the summation variable $m\mapsto-m$. $\Box$

The Eisenstein functions are right $O_2$-invariant and thus are
defined as functions of $GL_2(\IR)/O_2$ via a choice of a section of
the projection $GL_2(\IR)\to GL_2(\IR)/O_2$. We chose the following
section
 \be\label{Param123}
  g(x,y,t)=t^{\frac{1}{2}}y^{-\frac{1}{2}}\,\begin{pmatrix} y & x\\0&1\end{pmatrix}\in
  GL_2(\IR)\,, \qquad x\in \IR, \quad t,y\in \IR_+\,.
 \ee
Evaluation  of the Eisenstein function of elements \eqref{Param123}
gives
 \be\label{EisenRed}
  \Phi_\gamma(\tau,t)=t^{\imath(\gamma_1+\gamma_2)}\,\,
  \sum_{(n,m)\in\CP}
  \frac{({\rm Im}(\tau))^{\frac{\i(\g_2-\g_1)+1}{2}}}
  {|m+n\tau|^{\i(\g_2-\g_1)+1}}\,,\qquad \tau=x+\imath y\,.
 \ee
The Eisenstein series may be written in a form that makes its
$GL_2(\IZ)$-invariance obvious.

\begin{lem} Let us define a subgroup $B(\IZ)\subset GL_2(\IZ)$ as
  follows
 \be
  B(\IZ)=\Big\{g=\Big(\begin{smallmatrix}
  (-1)^{\epsilon_1} && r \\&&\\
  0 && (-1)^{\epsilon_2}\end{smallmatrix}\Big)\,\Big|\,\,
  \epsilon_{1,2}\in\{0,1\}, r\in
  \IZ\Big\}\,.
 \ee
Then we have the following coset decomposition
 \be
  GL_2(\IZ)=\bigsqcup_{(m,n)\in \CP}\,\,B(\IZ)\cdot \g_{(m,n)}\,,
 \ee
where
 \be\label{gmn}
  \g_{(m,n)}=\begin{pmatrix} k & l \\m & n\end{pmatrix}, \qquad
  {\rm gcd}(m,n)=1\,\qquad m,n,k,l \in \IZ \,,
 \ee
where we take $k,l$ to be unique solutions of the equation
 \be
  |kn-ml|=1\,, \qquad 0\leq k<m\,.
 \ee
In the expression \eqref{gmn} we imply a fixed lift along  the projection
$(\IZ^2-\{0\})\to (\IZ^2-\{0\})/\IZ_2$  of $(m,n)\in \CP$.
 \end{lem}

 \proof Taking into account the explicit form of the left action of
 elements of $B(\IZ)$
\be
\begin{pmatrix} (-1)^{\epsilon_1} & r \\ 0
  & (-1)^{\epsilon_2}\end{pmatrix}\cdot
\begin{pmatrix} k & l \\ m & n \end{pmatrix}
=\begin{pmatrix} (-1)^{\epsilon_1}k+rm  & (-1)^{\epsilon_1}l+rn \\
 (-1)^{\epsilon_2}m & (-1)^{\epsilon_2} n \end{pmatrix}\,,
\ee
and determinant condition
\be
|kn-ml|=1\,,
\ee
we infer that the set of $B(\IZ)$-orbits is projected to $\CP$. Let us
now check that  fibers of this projection allow transitive action of
$B(\IZ)$. For  $m\neq 0$ one might  chose $m>0$ and $n\in \IZ-\{0\}$.
Moreover we can chose $0<k<m$. Then for fixed $m$ and $n$ the
elements $l$ and $k$ are uniquely defined as
solution of the equation
\be
|kn-ml|=1\,, \qquad 0\leq k<m\,.
\ee
For $m=0$ we take $n\in \IZ_+$ and then from $|nk|=1$ derive that
$n=1$. Thus we can take
$k=1$ and $l=0$. This exhausts the set of representatives in the right
$B(\IZ)$-cosets.  $\Box$


\begin{prop} The Eisenstein function \eqref{EisenRed} allows the
  following presentation
 \be \label{EisenInv}
  \Phi_\gamma(\tau,t)=t^{\imath(\gamma_1+\gamma_2)}\,\,
  \sum_{\gamma\in B(\IZ)\backslash GL_2(\IZ)}
  {\rm Im}(\gamma\cdot\tau)^{\frac{\i(\g_2-\g_1)+1}{2}}\,,
 \ee
where  the action of $GL_2(\IZ)$ is given by
 \be
  \gamma\cdot\tau\,
  =\,\frac{a+b\tau}{c+d\tau},\qquad
  \gamma=\Big(\begin{smallmatrix}
  a&&b\\&&\\c&&d
  \end{smallmatrix}\Big)\in GL_2(\IZ)\,.
 \ee
\end{prop}

\proof In the summation we may chose the representatives
\eqref{gmn}. Then the expression \eqref{EisenInv} follows from  the
following simple identity
 \be
  {\rm Im}(\gamma_{(m,n)}\cdot \tau)\,
  =\,\frac{{\rm Im}(\tau)}{|m+n\tau|^2}\,.
 \ee
This completes our proof. $\Box$



\section{ Global $GL_2$ Hecke-Baxter operator}

To construct global $GL_2$ Hecke-Baxter operator we start with the
description of the elements of the Hecke  algebras
$\CH(GL_2(\IQ),GL_2(\IZ))$  and $\CH(GL_2(\IR),O_2)$ acting on the
space of the $GL_2$-automorphic forms, functions on $\CM_2$ (by
convolutions form the right and form the left).  Taking into account
the interpretation of  $\CM_2$ as space equivalence classes of
lattices $L$ in $\IR^2$ we introduce the following  averaging
operators $T_n$ as analogs of \eqref{Tn}
 \be\label{HA1}
  (T_n\cdot f)(L)=\sum_{[L:L']=n}\,f(L')\,,
 \ee
where the sum goes over sub-lattices $L'\subset L$ of index $n$.
The double coset description of $\CM_2$ allows to rewrite
the operators \eqref{HA1} as follows
 \be\label{HA2}
  (T_n\cdot f)(g)=\sum_{\gamma\in {SL_2(\IZ)\backslash\Mat}^{(n)}_2(\IZ)}\!
  f(\gamma\cdot g)\,,
 \ee
where
 \be\label{Matn}
  \Mat^{(n)}_2(\IZ)\,
  =\,\{\gamma\in {\rm Mat}_2(\IZ)|\det \gamma=n\}\,.
 \ee
This action may be written more explicitly using a specific choice of
representatives of the coset space
$SL_2(\IZ)\backslash\Mat^{(n)}_2(\IZ)$.

\begin{lem} For $n\in\IZ_+$, the space $\Mat^{(n)}_2(\IZ)$ defined in \eqref{Matn} allows the following coset decomposition:
 \be\label{Coset1}
  \Mat^{(n)}_2(\IZ)\,
  =\bigsqcup_{i=1}^{\s(n)}\,\,SL_2(\IZ)\,\alpha_i\,,
 \ee
where
 \be\label{Cosetrep}
  \alpha_i=\begin{pmatrix} a_i& b_i \\0 & d_i\end{pmatrix}, \qquad
  a_i,b_i,d_i>0,\quad a_id_i=n, \quad 0\leq b_i<d_i\,,
\ee
and the number $\s(n)$ of the coset representatives
$SL_2(\IZ)\backslash\Mat^{(n)}_2(\IZ)$ equals the positive divisors sum:
 \be
  \s(n)\,=\sum_{d|n}d\,.
 \ee
\end{lem}
\proof See e.g. \cite{L}, Chapter II, \S1. $\Box$

As a consequence of the previous Lemma we obtain the following
presentation for the operators $T_n$
\be
  (T_n\cdot f)(g)\,
  =\sum_{a,d>0\atop ad=n}\sum_{b=0}^{d-1}\!
  f\Big(\,\Big(\begin{smallmatrix} a&&b\\&&\\0&&d\end{smallmatrix} \Big)g\Big)\,.
 \ee
Let us combine the operators $T_n$, $n\in \IZ_+$ into the  generating series
 \be
  Q^{GL_2(\IZ)}_s=\sum_{n=1}^\infty \frac{1}{n^{s+\frac{1}{2}}} T_n\,,
 \ee
acting on a function on the space of lattices $L\subset\IR^2$ in the
following way
 \be
  (Q^{GL_2(\IZ)}_s\cdot f)(L)\,
  =\sum_{L'\subset L}^\infty \frac{1}{[L:L']^{s+\frac{1}{2}}}\,f(L')\,,
  \ee
(note that  the shift $s\to s+\frac{1}{2}$ is a special case of the
general expression $s+\frac{\ell}{2}$ for $GL_{\ell+1}$).
Equivalently,
 \be\label{HBZ0}
  \bigl(Q^{SL_2(\IZ)}_s\cdot f\bigr)(g)\,
  =\!\!\sum_{\a\in SL_2(\IZ)\backslash\Mat^*_2(\IZ)}\,\,
  \frac{1}{|\det\a|^{s+\frac{1}{2}}} \,f(\alpha \cdot g)\,,
 \ee
where $\a$ runs through the set of $SL_2(\IZ)$-coset representatives
\eqref{Cosetrep} of the space
 \be
  \Mat^*_2(\IZ)\,={\rm
    Mat}_2(\IZ)\cap GL_2^+(\IQ)=\bigsqcup_{n\in \IZ_+} \Mat^{(n)}_2(\IZ)\,.
 \ee
One has an  analog of the Riemann zeta-function for $GL_2$:
 \be
\zeta^{GL_2}(s)=\sum_{\alpha\in  SL_2(\IZ)\backslash {\rm Mat}^*_2(\IZ)}\,
\frac{1}{|\det  \alpha|^{s+\frac{1}{2}}}\,.
\ee
\begin{lem} The following identity holds
\be\label{2Riem}
\zeta^{GL_2}(s)=\sum_{\alpha\in SL_2(\IZ)\backslash {\rm Mat}^*_2(\IZ)}\,
\frac{1}{|\det  \alpha|^{s+\frac{1}{2}}}=\zeta\left(s+\frac{1}{2}\right)\cdot
\zeta\left(s-\frac{1}{2}\right)\,,
\ee
where $\zeta(s)$ is the Riemann zeta-function given by
by
\be
\zeta(s)=\sum_{n=1}^{\infty} \,\frac{1}{n^s}\,, \qquad {\rm Re}(s)>1\,.
\ee
\end{lem}

\proof Using the set of representatives for the coset we have
\be
\sum_{\alpha \in SL_2(\IZ)\backslash {\rm Mat}^*_2(\IZ)}\,
\frac{1}{|\det  \alpha |^{s+\frac{1}{2}}}=\sum_{a,d>0 \atop 0\leq b<d}
\frac{1}{a^{s+\frac{1}{2}}}\cdot
\frac{1}{d^{s+\frac{1}{2}}}\,.
\ee
Summing over $b$ we obtain
\be
\sum_{\alpha \in SL_2(\IZ)\backslash {\rm Mat}^*_2(\IZ)}\,
\frac{1}{|\det  \alpha|^{s+\frac{1}{2}}}=\sum_{a,d>0}\frac{1}{a^{s+\frac{1}{2}}}\cdot
\frac{1}{d^{s-\frac{1}{2}}}=
\zeta\left(s+\frac{1}{2}\right)\cdot \zeta\left(s-\frac{1}{2}\right)\,.
\ee
$\Box$

Below we encounter a generalization of \eqref{2Riem}
associated with   spherical principle series representations.  The
following result is a direct consequence of the well-known expressions
for action of the Hecke algebra generators $T_n$ \eqref{HA1} on the
Eisenstein functions.

 \begin{prop}\label{HBZ}
The action of the operator $Q_s^{GL_2(\IZ)}$ on
the  $GL_2$-Eisenstein functions \eqref{Eisen} associated with a
spherical principle series representation $(\pi_\gamma,\CV_\gamma)$
is given by
 \be \label{Act456}
  \bigl(Q_s^{GL_2(\IZ)}\cdot \Phi_\gamma\bigr)(g)\,
  =\,\zeta^{GL_2}(s|\gamma)\,\,\Phi_\gamma(g)\,,
 \ee
 where
 \be\label{CompGLzeta}
 \zeta^{GL_2}(s|\gamma)=\zeta(s-\imath \gamma_1)\,
 \zeta(s-\imath \gamma_2)\,.
\ee
\end{prop}

\proof  Application of the Hecke-Baxter operator \eqref{HBZ0}  gives
\be
\bigl(Q_s^{GL_2(\IZ)}\cdot \Phi_\gamma\bigr)(g)
=\sum_{\alpha \in SL_2(\IZ)
  \backslash {\rm Mat}^*_2(\IZ)}\,
\frac{1} {|\det \alpha |^{s+\frac{1}{2}}}\,\,
\Phi_\gamma(\alpha \cdot g)\,.
\ee
Using the presentation  \eqref{Param123}, \eqref{EisenRed} we have
 \be\nonumber
  \bigl(Q_s^{GL_2(\IZ)}\cdot \Phi_\gamma\bigr)(\tau,t)\\
  =\,t^{\imath(\gamma_1+\gamma_2)}\!\!\sum_{\alpha \in SL_2(\IZ)
  \backslash {\rm Mat}^*_2(\IZ)} \,\,
  \frac{|\det \alpha|^{\frac{\imath(\gamma_1+\gamma_2)}{2}}}
  {|\det\alpha |^{s+\frac{1}{2}}}\!\!
  \sum_{\gamma\in B(\IZ)\backslash GL_2(\IZ)}\,\,
  ({\rm Im}(\gamma\cdot\alpha\cdot\tau))^{\frac{\imath(\gamma_2-\gamma_1)+1}{2}}\,.
 \ee
We formally
extend the summation domain in the first  sum
 \be\nonumber
  \bigl(Q_s^{GL_2(\IZ)}\cdot \Phi_\gamma\bigr)(\tau,t)\\
  =\,\frac{t^{\imath(\gamma_1+\gamma_2)}}{|SL_2(\IZ)|\cdot |B(\IZ)|}
  \!\!\sum_{\alpha \in {\rm Mat}^*_2(\IZ)} \,
  \frac{|\det\alpha|^{\frac{\imath(\gamma_1+\gamma_2)}{2}}}{|\det\alpha|^{s+\frac{1}{2}}}\!\!
  \sum_{\gamma\in  GL_2(\IZ)}\,\,
  ({\rm Im}(\gamma\cdot\alpha\cdot\tau))^{\frac{\imath(\gamma_2-\gamma_1)+1}{2}}\\
  =\,\frac{t^{\imath(\gamma_1+\gamma_2)}}{|SL_2(\IZ)|\cdot |B(\IZ)|}\!\!
  \sum_{\alpha \in {\rm Mat}^*_2(\IZ)} \,
  \frac{|\det\alpha|^{\frac{\imath(\gamma_1+\gamma_2)}{2}}}{|\det \alpha|^{s+\frac{1}{2}}}\!\!
  \sum_{\gamma\in  GL_2(\IZ)}\,\,
  ({\rm Im}(\alpha\cdot \gamma\cdot\tau))^{\frac{\imath(\gamma_2-\gamma_1)+1}{2}}\,.
 \ee
Let us split the summation domain using the right hand side analog
of the decomposition \eqref{Coset1}
  \be
   {\rm Mat}^*_2(\IZ)\,=\,\bigsqcup_{n>0}\bigsqcup_i\,\,\tilde{\alpha}_i\,SL_2(\IZ)\,,
 \ee
to obtain \be\nonumber \bigl(Q_s^{GL_2(\IZ)}\cdot
\Phi_\gamma\bigr)(\tau,t)\\=
t^{\imath(\gamma_1+\gamma_2)}\,\,\sum_{\alpha \in
  {\rm Mat}^*_2(\IZ)/SL_2(\IZ)}\,
\frac{|\det \alpha|^{\frac{\imath(\gamma_1+\gamma_2)}{2}}}
{|\det \alpha |^{s+\frac{1}{2}}}\,\,
\sum_{\gamma\in  B(\IZ)\backslash GL_2(\IZ)}\,\,
({\rm Im}(\alpha \cdot \gamma\cdot
\tau))^{\frac{\imath(\gamma_2-\gamma_1)+1}{2}}\,.
\ee
Now using the identity
\be
{\rm Im}(\alpha \cdot \gamma\cdot  z)=\frac{a}{d}\cdot
{\rm Im}(\gamma\cdot  z)\,,
\ee
and the fact that $\det \alpha =ad$ for the considered representatives
of the coset space  we obtain
\be\nonumber
\bigl(Q_s^{GL_2(\IZ)}\cdot \Phi_\gamma\bigr)(\tau,t)\\=
t^{\imath(\gamma_1+\gamma_2)}\,\,
\left(\sum_{a,d>0}\sum_{b=0}^{d-1}\,\,
  \frac{a^{\frac{\imath(\gamma_1+\gamma_2)}{2}}}
  {a^{s-\frac{\imath(\gamma_2-\gamma_1)}{2}}}
\frac{d^{\frac{\imath(\gamma_1+\gamma_2)}{2}}}
  {d^{s+\frac{\imath(\gamma_2-\gamma_1)}{2}+1}}
\right)
\sum_{\gamma\in  B(\IZ)\backslash GL_2(\IZ)}\,\,
({\rm Im}(\gamma\cdot\tau))^{\frac{\imath(\gamma_2-\gamma_1)+1}{2}}\,\\=
\left(\sum_{a,d>0}\sum_{b=0}^{d-1}\,\,
\frac{1}{a^{s-\imath\gamma_2}}
\frac{1}{d^{s-\imath\gamma_1+1}}\right)\cdot
\Phi_\gamma(\tau,t)\,.
\ee
Thus for the eigenvalue we have
\be
\sum_{a,d>0}\sum_{b=0}^{d-1}\,\,
\frac{1}{a^{s-\imath\gamma_1}}
\frac{1}{d^{s-\imath\gamma_2+1}}=
\sum_{a,d>0}\,\,
\frac{1}{a^{s-\imath\gamma_1}}
\frac{1}{d^{s-\imath\gamma_2}}=
\zeta(s-\imath \gamma_1)\, \zeta(s-\imath
\gamma_2)\,.
\ee
This completes the proof.
$\Box$

The Archimedean counterpart of the one-parameter family
$Q^{GL_2(\IZ)}_s$   of elements in \\$\CH(GL_2(\IQ),GL_2(\IZ)$
given by \eqref{HBZ0} is the
 $GL_2(\IR)$ Hecke-Baxter operator \cite{GLO08}:
 \be
  \bigl(Q^{GL_2(\IR)}_s\cdot f\bigr(g)\,
  =\!\int\limits_{GL_2(\IR)}\!\! d\mu^G(\tilde{g})\,\, |\det
  \tilde{g}|^{s+\frac{1}{2}} f(\tilde{g}^{-1}g)\,,\\
  d\mu^G(\tilde{g})=e^{-\pi \Tr \tilde{g}^{\top}\tilde{g}}\,d\mu(\tilde{g})\,,
 \ee
acting by convolution with $O_2$-biinvariant function on $GL_2(\IR)$
 \be\label{ArKernel2}
  Q^{GL_2(\IR)}_s(g)=|\det g|^{s+\frac{1}{2}}\,e^{-\pi \Tr g^{\top}g}\,.
 \ee
Its action on the matrix element \eqref{Autform} (actually on any
matrix element with the right $O_2$-invariant vector) was calculated in
\cite{GLO08} and is given by
 \be\label{Act123}
  \bigl(Q^{GL_2(\IR)}_s\cdot \Phi_\gamma\bigr)(g)\,
  =\,L^{GL_2(\IR)}(s|\gamma)\,
  \Phi_\gamma(g)
 \ee
where
 \be\label{GammaF}
  L^{GL_2(\IR)}(s|\gamma)=
  \prod_{j=1}^2 \pi^{-(s-\imath \gamma_j)/2}\,
  \Gamma((s-\imath \gamma_j)/2)\,.
 \ee

Let us note that while the operator $Q_s^{GL_2(\IZ)}$ acts by
convolution on the functions on the double coset space
$GL_2(\IZ)\backslash GL_2(\IR)/O_2$ from the left the operator
$Q_s^{O_2}$ acts by convolution on the functions on the double coset space
$GL_2(\IZ)\backslash GL_2(\IR)/O_2$ from  the right. Still it is
reasonable to consider its combination acting by simultaneous via
left/right convolution. Let us  define  the following
integral operator acting on $GL_2$-automorphic functions
 \be\label{GHBO1}
  \bigl(\wh{Q}^{GL_2}_s \bullet  \Phi\bigr)(g)\,
  =\!\int\limits_{GL_2(\IR)}\!\!d\mu(h)\,\,\wh{Q}^{GL_2}_s(g,h)\,\,\Phi(h^{-1})\,,
 \ee
where the kernel of the integral operator is given by
 \be\label{GHBO}
  \wh{Q}^{GL_2}_s(g,h)=\,\,|\det g|^{s+\frac{1}{2}}\!\!
  \sum_{\a\in SL_2(\IZ)\backslash {\rm Mat}^*_2(\IZ)}\,|\det h|^{s+\frac{1}{2}} \,
  e^{-\pi\Tr(h^{\top}h
  \,\alpha gg^{\top}\,\alpha^{\top})}\,,
 \ee
where sum goes over representatives \eqref{Cosetrep}.
This operator is a global analog  (in the sense of arithmetic geometry
of $\overline{{\rm Spec}(\IZ)}$) of the local
 spherical Hecke-Baxter  operators considered above. Our previous
 considerations may be summarized in the following form.

\begin{te}\label{TH1}  The global Hecke-Baxter operator \eqref{GHBO1}
  acts on the Eisenstein functions represented by matrix elements
\eqref{Eisen} of the spherical principle series representation
$(\pi_\gamma,\CV_\gamma)$ of $GL_2(\IR)$
by multiplication on the corresponding global completed zeta-functions
 \be\label{GlobalL}
  \hat{\zeta}^{GL_2}(s|\gamma)\,
  =\,\pi^{-\frac{s-\imath \gamma_j}{2}}\,
  \Gamma\Big(\frac{s-\imath \gamma_j}{2}\Big)
  \prod_{j=1}^2 \zeta(s-\imath \gamma_j)
  \,.
 \ee
\end{te}

\proof The action of $\wh{Q}^{GL_2}_s$ on \eqref{Eisen} is given by
the following integral
 \be\nonumber
  \bigl(\wh{Q}^{GL_2}_s \bullet  \Phi_\gamma\bigr)(g)\\
  =\,|\det g|^{s+\frac{1}{2}}\!\!\!\!
  \sum_{\a\in SL_2(\IZ)\backslash {\rm Mat}^*_2(\IZ)}\,\,
  \int\limits_{GL_2(\IR)}\!\!
  d\mu(h)\,\,|\det h|^{s+\frac{1}{2}} \,
  e^{-\pi\Tr(h^{\top}h\,\alpha gg^{\top}\,\alpha^{\top})} \,\,
  \Phi_\gamma(h^{-1})\,.
 \ee
Using the change of integration variable $h\to h\alpha^{-1}g^{-1}$
we obtain
 \be
  \bigl(\wh{Q}^{GL_2}_s \bullet  \Phi_\gamma\bigr)(g)\\
  =\!\!\sum_{\a\in SL_2(\IZ)\backslash {\rm
    Mat}^*_2(\IZ)}\,\,\int\limits_{GL_2(\IR)}\!\!d\mu(h)\,\,
  \frac{|\det h|^{s+\frac{1}{2}}}{\det \alpha|^{s+\frac{1}{2}}}\,\,
  e^{-\pi\Tr(h^{\top}h)}\,\,\Phi_\gamma(\alpha g h^{-1})\,.
 \ee
This is  a combination of
$Q_s^{GL_2(\IZ)}$ and $Q_s^{GL_2(\IR)}$.  Therefore the required result
follows from \eqref{Act456} and \eqref{Act123}. $\Box$

There is global analog of representation \eqref{2Riem} that may be
described as follows. Let us define  the $GL_2$-analog of the  theta
constant as follows
 \be\label{Theta2}
  \Theta_{\alpha}(0|\imath gg^{\top})\,
  =\,1\,+\!\sum_{\alpha \in  SL_2(\IZ)\backslash {\rm
    Mat}_2^*(\IZ)}
    e^{-\pi\Tr  gg^\top \alpha^\top\alpha}\,.
 \ee
Then the following expression for the global zeta-function \eqref{GlobalL}
 for $\gamma=0$  holds
 \be\label{ALL}
  \hat{\zeta}^{GL_2}(s|0)\,=\!\int\limits_{GL_2(\IR)}\!\! d\mu(g)\,\,|\det g|^s\,
  \Big(\Theta_{\alpha}(0|\imath gg^{\top})-1\Big)\,.
 \ee
Indeed substituting  \eqref{Theta2} into \eqref{ALL} and making the
change of integration variable $g\to \alpha^{-1}g$ leads to
factorization summation and integration. Thus  \eqref{ALL} reduces to the product of
zeta-functions and Gamma-factor \eqref{GammaF} with $\gamma=0$. Note
that the essential part of the integral kernel \eqref{GHBO} is
expressed in terms of the following generalization of the classical theta
constant
 \be
  \hat{\Theta}(0|A,B)\,
  =\!\sum_{\a\in SL_2(\IZ)\backslash {\rm
      Mat}^*_2(\IZ)}\!\!\!\!e^{-\pi\Tr\,A\alpha
    B\alpha^{\top}}\,.
 \ee
where $A$ and $B$ are symmetric positive $(2\times 2)$-matrices. This
kind of theta series is instrumental for the verification
of the analog of the functional equation for the global $GL_2$
Hecke-Baxter operator extending the relations \eqref{FR1} for $GL_1$.
The functional equations for the global Hecke-Baxter operator are
compatible with the  functional equations for the completed  $GL_2$ zeta-function
\eqref{GlobalL}
\be
\hat{\zeta}^{GL_2}(1-s|-\gamma)=\hat{\zeta}^{GL_2}(s|\gamma)\,.
\ee
Similar functional equations hold (when supplied with the Cartan
involution $g\to g^{\tau}=(g^{\top})^{-1}$ of $GL_2(\IR)$) for
$GL_2$-Eisenstein function.

As a final remark let us  note that in the case of $GL_2$ (as well
as in the case of $GL_1$, see  Section 2) one might construct
interpolation of the global and Archimedean Hecke-Baxter operators
by considering congruence semigroups
 \be
  \Gamma(N)\,
  =\,\Id\,+\,N\Mat_2(\IZ)\subset GL_2(\IZ)\,.
 \ee
This might be useful to compare this approach with the results by D.
Kazhdan  \cite{Kaj}.






\vspace{5mm}



\begin{thebibliography}{15}
\bibitem[Ba]{Ba}  R.J.~Baxter, Exactly solved models in statistical mechanics,
Academic Press, 1982.


\bibitem[FP]{FP} L.D. Faddeev, B.S. Pavlov,
{\it Scattering theory and automorphic functions}, J. Soviet Math. 3 (1975) 522-548.


\bibitem[GGPS]{GGPS} I.M. Gelfand, M.I. Graev, I.I. Pyatetskii-Shapiro,
Representation theory and automorphic functions. Generalized functions,
Vol. 6, AMS 2016.


\bibitem[G]{G} A. Gerasimov, {\it Archimedean Langlands duality
and exactly solvable quantum systems}, in Proc. ICM Seoul 2014, Vol.
3, 1097--1121.

\bibitem[GL]{GL} A.~Gerasimov, D.~Lebedev,
  {\it Representation Theory over Tropical Semifield and Langlands
    Duality},   Commun. Math. Phys., 320 (2013) 301--346;
{\tt[arXiv:1011.2462]}.


\bibitem[GKL]{GKL} A. Gerasimov, S. Kharchev, D. Lebedev,
{\it Representation theory and quantum inverse scattering method:
open Toda chain and hyperbolic Sutherland model}, Int. Math. Res.
Notices 17 (2004) 823-854; {\tt[arXiv:math.QA/0204206]}.


\bibitem[GLO08]{GLO08} A.~Gerasimov, D.~Lebedev, S.~Oblezin, {\it Baxter operator
and Archimedean Hecke algebra}, Commun. Math. Phys. 284 (2008)
867--896; {\tt [arXiv:0706.3476]}.



\bibitem[GLO25]{GLO25} A. Gerasimov, D. Lebedev, S. Oblezin,
  {\it The $GL_{\ell+1}(\mathbb{R})$ Hecke-Baxter operator: principal
    series     representations}, {\tt[arXiv:2506.16708]}.


\bibitem[ILP]{ILP} D.~Bump, J.W.~Cogdell, E.~de Shalit,
D.~Gaitsgory, E.~Kowalski, S.S.~Kudla, An introduction to the
Langlands program, Eds. J.~Bernstein, S.~Gelbart, Birkhauser 2004.


\bibitem[JL]{JL} H.~Jacquet, R.~Langlands,
 Automorphic forms on $GL(2)$, Springer Lect. Notes Math. 114,
Springer, 1970.

\bibitem[Ka]{Ka} L.P. Kadanoff, Statistical physics. Statics,
    dynamics and renormalization, World Scientific, 2000.

\bibitem[Kaj]{Kaj} D. Kajdan, {\it Arithmetic varieties and their
fields of quasi-definition}, Actes Congr\'{e}s Intern. Math., 1970,
Tome 2, 321--325.


\bibitem[L]{L} S.Lang, Introduction to modular forms, Springer, 1976.

\bibitem[W]{W} A.Weil, Basic number theory, Springer, 1995.



\end{thebibliography}
\noindent {\small {\bf A.A.G.} {\sl Laboratory for Quantum Field
Theory
and Information},\\
\hphantom{xxxx} {\sl Institute for Information
Transmission Problems, RAS, 127994, Moscow, Russia};\\
\hphantom{xxxx} {\it E-mail address}: {\tt anton.a.gerasimov@gmail.com}}\\
\noindent{\small {\bf D.R.L.} {\sl Laboratory for Quantum Field
Theory
and Information},\\
\hphantom{xxxx}  {\sl Institute for Information
Transmission Problems, RAS, 127994, Moscow, Russia};\\
\hphantom{xxxx} {\it E-mail address}: {\tt lebedev.dm@gmail.com}}\\
\noindent{\small {\bf S.V.O.} {\sl
 Beijing Institute of Mathematical Sciences and Applications\,,\\
\hphantom{xxxx} Huairou District, Beijing 101408, China};\\
\hphantom{xxxx} {\it E-mail address}: {\tt oblezin@gmail.com}}


\end{document}
