\subsection{Arithmetic manifolds}\label{arithmfd}

This section will deal with the general setup of $G$ being a reductive algebraic group defined over $\Q$ with $\Q$-split zenter $Z_G$. See (\cite{Arthur0}, Section I.$2$) for a partial reference.  Let $K_f\subset G(\A_f)$ be any open compact subgroup. We write $G(\A) = G(\A)^1\times A_G$ for $A_G$ the identity component of $Z_G$, and we set $G(\R)^1\coloneqq G(\A)^1\cap G(\R)$, which is then a semisimple real Lie group. We fix a Cartan involution $\theta$ of $G(\R)$ and let $K$ denote its fixpoints. We set $\Tilde{X}\coloneqq G(\R)^1/K$. The group $G(\R)$ has a natural action on the finite double coset space,

\begin{align*}
    A_G(\R)^0G(\Q)\backslash G(\A)/G(\R)K_f.
\end{align*}
Taking a set of representatives $z_1,\dots,z_m$ for the double cosets in $G(\A_f)$ and defining 
\begin{align*}
    \Gamma_j \coloneqq (G(\R)\times z_jK_fz_j^{-1})\cap G(\Q)
\end{align*}
for all $1\leq j\leq m$, the action induces the decomposition
\begin{align}\label{decomparithm}
    A_G(\R)^0 G(\Q)\backslash G(\A)/K_f \cong \bigsqcup_{j=1}^m (\Gamma_j\backslash G(\R)^1).
\end{align}
Now we define the arithmetic manifold associated to $K_f$. Set
\begin{align}\label{adelicsym}
    X(K_f)\coloneqq G(\Q)\backslash (\Tilde{X} \times G(\A_f))/K_f.
\end{align}
Using (\ref{decomparithm}), we get
\begin{align}\label{decompsym}
    X(K_f)\cong \bigsqcup_{j=1}^m (\Gamma_j\backslash \Tilde{X}).
\end{align}
Here, $\Gamma_j\backslash \Tilde{X}$ is a locally symmetric space. We will assume that $K_f$ is neat such that $X(K_f)$ is a locally symmetric manifold of finite volume.

\subsection{The heat kernel}\label{heatker}

\noindent In the setup above, given a finite-dimensional unitary representation of $K$, we let $\Tilde{E}_\nu$ be the associated homogeneous Hermitian vector bundle over $\Tilde{X}$. Using homogeneity, one can push down this vector bundle to give locally homogeneous Hermitian vector bundles over each component $\Gamma_j\backslash \Tilde{X}$. Taking the disjoint union of these vector bundles, we get a locally homogeneous vector bundle $E_\nu$ over $X(K_f)$. 

As it is sufficient to work on each component in (\ref{decompsym}), we continue this section with the simplifying assumption that $G$ is a connected semisimple algebraic group. Let $K\subset G(\R)$ be a maximal compact subgroup, and $\Gamma\subset G(\R)$ a torsion-free lattice. We let $\Tilde{X}=G(\R)/K$ and $X=\Gamma\backslash \Tilde{X}$. Take a finite-dimensional unitary representation $(\nu,V_\nu)$ of $K$ with inner product $\langle\cdot,\cdot\rangle_\nu$, and define
\begin{align*}
    \Tilde{E_\nu} \coloneqq G(\R)\times_\nu V_\nu
\end{align*}
as the associated homogenous vector bundle over $\Tilde{X}$. The action of $K$ on $G(\R)$ is by right multiplication. The inner product $\langle\cdot,\cdot\rangle_\nu$ induces a $G(\R)$-invariant metric $\Tilde{h}_\nu$ on this vector bundle. 
Let $E_\nu\coloneqq \Gamma\backslash\Tilde{E_\nu}$ be the associated locally homogenous vector bundle over $X$, with metric $h_\nu$ induced by $\Tilde{h}_\nu$ using $G(\R)$-invariance. 
Let $C^\infty(\Tilde{X},\Tilde{E}_\nu)$ denote the space of smooth sections of $\Tilde{E}_\nu$. Now, set
\begin{align*}
    C^\infty(G(\R),\nu) \coloneqq \lbrace &f:G(\R)\to V_\nu \mid f\in C^\infty,\\
    &f(gk) = \nu(k)^{-1}f(g) \:\forall k\in K,g\in G(\R)\rbrace.
\end{align*}
The following isomorphism (\cite{Miatello}, p. 4) allows us to understand this geometric setup through representation theory.
\begin{align}\label{c-inf spaces}
    C^\infty(\Tilde{X},\Tilde{E}_\nu) \xrightarrow{\sim} C^\infty(G(\R),\nu).
\end{align}
\noindent This extends to an isometry of corresponding $L^2$-spaces. Let $\mathcal{C}(G(\R))$ denote Harish-Chandra's Schwartz space, and $\mathcal{C}^q(G(\R))$ Harish-Chandra's $L^q$-Schwartz space. 

We now specify the above to our setting. Let $(\tau,V_\tau)$ be an irreducible finite-dimensional representation of $G(\R)$, $E_\tau\coloneqq E_{\tau|_K}$, and $F_\tau$ the flat vector bundle over $X$ associated to the restriction of $\tau$ to $\Gamma$. Then we have a canonical isomorphism (see \cite{MM}, Proposition $3.1$)
\begin{align*}
    E_\tau\cong F_\tau.
\end{align*}
As $K$ is compact there exists an inner product on $V_\tau$ with respect to which $\tau|_K$ is unitary. Fix such an inner product. From this we induce a metric on $E_\tau$, and hence on $F_\tau$ as well. Define $\Lambda^p(X,F_\tau)\coloneqq \Lambda^pT^*(X)\otimes F_\tau$. Under the isomorphism above, this is isomorphic to the vector bundle associated to the representation $\Lambda^p\text{Ad}^*\otimes \tau$ of $K$ on $\Lambda^p\mf{p}^*\otimes V_\tau$. We will denote this representation by $\nu_{\tau,p}$. 

Let $\Delta_p(\tau)$ be the Laplace operator on $\Lambda^p(X,F_\tau)$, and let $\Tilde{\Delta}_p(\tau)$ be its lift to the universal covering $\Tilde{X}$. Let also $\Tilde{F}_\tau$ be the pullback of $F_\tau$ to $\Tilde{X}$. Then $\Tilde{\Delta}_p(\tau)$ is an operator on the space $\Lambda^p(\Tilde{X},\Tilde{F}_\tau)$ of $\Tilde{F}_\tau$-valued $p$-forms on $\Tilde{X}$. By (\ref{c-inf spaces}), we get an isomorphism
\begin{align}\label{pformsrep}
    \Lambda^p(\Tilde{X},\Tilde{F}_\tau)\cong C^\infty(G(\R),\nu_{\tau,p}).
\end{align}
Let $R$ be the right regular representation of $G(\R)$ on $C^\infty(G(\R),\nu_{\tau,p})$, and $\Omega$ the Casimir element of $G(\R)$. With respect to the above isomorphism, Kuga's lemma implies
\begin{align}\label{casimirglobal}
    \Tilde{\Delta}_p(\tau)=\tau(\Omega)-R(\Omega).
\end{align}
The operator $\Tilde{\Delta}_p(\tau)$ is formally self-adjoint and non-negative. Regarded as an operator with domain the space of compactly supported smooth $p$-forms, it has a unique self-adjoint extension to $L^2(\Tilde{X},\Tilde{F}_\tau)$, the $L^2$-sections of $\Tilde{F}_\tau$, which we by abuse of notation will also denote $\Tilde{\Delta}_p(\tau)$. This extension inherits non-negativity. We denote by $e^{-t\Tilde{\Delta}_p(\tau)}$, with $t>0$, the heat semigroup associated to $\Tilde{\Delta}_p(\tau)$. Considered as a bounded operator on $L^2(G(\R),\nu_{\tau,p})$ under the extension of the isomorphism (\ref{pformsrep}), it is a convolution operator, and thus it has a kernel
\begin{align*}
    H_t^{\tau,p}:G(\R)\to \End(\Lambda^p\mf{p}^*\otimes V_\tau)
\end{align*}
called the heat kernel. It satisfies the following covariance property
\begin{align}\label{covar}
    H_t^{\tau,p}(k^{-1}gk')=\nu_{\tau,p}(k)^{-1}\circ H_t^{\tau,p}(g)\circ \nu_{\tau,p}(k'), \quad \forall k,k'\in K, g\in G(\R).
\end{align}
By analogy of the proof of (\cite{BM}, Proposition $2.4$), we have that 
\begin{align}\label{heatkernelspace}
    H_t^{\tau,p}\in \mathcal{C}^q(G(\R))\otimes \End(\Lambda^p\mf{p}^*\otimes V_\tau)
\end{align} 
for any $q>0$. We may define
\begin{align}\label{trheat}
    h_t^{\tau,p}(g)\coloneqq \tr H_t^{\tau,p}(g), \quad g\in G(\R),
\end{align}
for $\tr$ being the trace over $\Lambda^p\mf{p}^*\otimes V_\tau$. By (\ref{covar}) and (\ref{heatkernelspace}), we have that $h_t^{\tau,p}(g)\in \mathcal{C}^q(G(\R))_{K\times K}$ for any $q>0$, with $\mathcal{C}^q(G(\R))_{K\times K}$ the subspace of $\mathcal{C}^q(G(\R))$ consisting of left and right $K$-finite functions. 





