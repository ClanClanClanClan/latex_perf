
\noindent To prove Theorem \ref{glmytheorem}, we first turn to an analysis of the terms (\ref{mellin}). We truncate the integral in this expression with respect to a parameter $T>0$, leaving a remainder term.
\begin{align}\label{trunc}
    \int_0^\infty \Trreg(e^{-t\Delta_p(\tau)})t^{s-1}dt = &\int_0^T \Trreg(e^{-t\Delta_p(\tau)})t^{s-1}dt \\
    + &\int_T^\infty \Trreg(e^{-t\Delta_p(\tau)})t^{s-1}dt.
\end{align}
We focus first on the second integral on the right hand side, what we call the remainder. It is convergent for all $s$, in particular it is holomorphic at $s=0$. For any meromorphic function $f$ holomorphic at $0$ it holds that $\FP_{s=0}(f(s))=f(0)$, thus by linearity, we need only analyze
\begin{align*}
    E_0(0,T)\coloneqq \int_T^\infty \Trreg(e^{-t\Delta_p(\tau)})t^{-1}dt.
\end{align*}
By applying our large $t$ asymptotic for the spectral side (\ref{larget}) and its following remark, and using the trace formula, i.e. the equality of the geometric side and spectral side, we get the following lemma.

\begin{lem}\label{error0}
There exists a $C>0$ only depending on $G$ such that for any $\epsilon>0$,
    \begin{align*}
    \vert E_0(0,T)\vert\leq Ce^{-\lambda(1-\epsilon)T}\vol(Y(N)).
\end{align*}
\end{lem}

\medskip

\subsection{Restricting to unipotent contribution}

\noindent We return to the first integral on the right-hand side of (\ref{trunc}). We want to substitute the test function with its compactified version. By Proposition \ref{compactdiff}, we may write
\begin{align*}
    \int_0^T J_{\geo}(h^{\tau,p}_t\otimes\chi_{K(N)})t^{s-1}dt = \int_0^T J_{\geo}(h^{\tau,p}_{t,R}\otimes\chi_{K(N)})t^{s-1}dt+E_1(s,R,T),
\end{align*}
with the error term $E_1(s,R,T)$ given by an integral convergent for all $s\in\C$ and satisfying
\begin{align}\label{error1}
    |E_1(0,R,T)| &\leq C_3\int_0^Te^{-C_1R^2/t+C_2t}t^{-1}dt\vol(Y(N)) \\
    &\leq C_3e^{-C_4R^2/T+C_2T}\int_0^{T/R^2}e^{-C_4/t}t^{-1}dt\vol(Y(N)).
\end{align}
Again, $C_2,C_4>0$ both only depend on the group $G$. We will return to this estimate in a moment. The reward for compactifying our test function is that by Proposition \ref{compactunip}, only the unipotent part of the coarse geometric expansion (\ref{coarsegeom}) contribute if we keep the radius of the support small enough. Assume $R\leq C_n\log N$. Then
\begin{align*}
    \int_0^T J_{\geo}(h^{\tau,p}_{t,R}\otimes\chi_{K(N)})t^{s-1}dt = \int_0^T J_{\unip}(h^{\tau,p}_{t,R}\otimes\chi_{K(N)})t^{s-1}dt
\end{align*}

\bigskip

\subsection{Applying the fine geometric expansion}\label{applyingfine}

To deal with the truncated Mellin transform of $J_{\unip}$, we recall the fine geometric expansion (\ref{finegeom}). We isolate the term for $(M,\mf{u})=(G,\lbrace 1\rbrace)$, which is exactly $h^{\tau,p}_{t,R}(1)\vol(Y(N))$, and denote the remaining sum by $J_{\unip-1}$. This allows us to write
\begin{align}\label{unip-1}
    \int_0^T J_{\unip}(h^{\tau,p}_{t,R}\otimes\chi_{K(N)})t^{s-1}dt &= \int_0^T h_{t,R}^{\tau,p}(1)t^{s-1}dt \vol(Y(N)) \\
    &+\int_0^T J_{\unip-1}(h^{\tau,p}_{t,R}\otimes\chi_{K(N)})t^{s-1}dt .
\end{align}
The first integral on the right-hand side is the truncated Mellin transform of $h_{t,R}^{\tau,p}(1)$. As we are evaluating at $1\in G(\R)$, the compactification has no effect, and we can ignore $R$. By (\cite{MzM1}, ($5.11$)) we have an asymptotic expansion
\begin{align*}
    h_t^{\tau,p}(1)\sim \sum_{i=0}^\infty a_i t^{-d/2+j}, \quad t\to 0.
\end{align*}
Furthermore, as a special case of Theorem \ref{traceheatdecay} we have for $t\geq 1$,
\begin{align*}
    |h_t^{\tau,p}(1)|\leq D_{\tau,p}e^{-\lambda t}.
\end{align*}
Thus, the integral is convergent for $s$ in some half-plane and has a meromorphic extension to all of $\C$. We may write
\begin{align*}
    \int_0^T h_{t,R}^{\tau,p}(1)t^{s-1}dt \vol(Y(N)) = \int_0^\infty h_t^{\tau,p}(1)t^{s-1}dt \vol(Y(N)) + E_2(s,T),
\end{align*}
with the error term $E_2(s,T)$ given by an integral convergent for all $s\in\C$ and satisfying
\begin{align}\label{error2}
    |E_2(0,T)|&\leq C\int_T^\infty e^{-\lambda t}t^{-1}dt \vol(Y(N)) \\
    &\leq C'e^{-\lambda T}\vol(Y(N))
\end{align}
for $T\geq 1$. We return to the second integral on the right-hand side of (\ref{unip-1}). Using the splitting of the orbital integrals into local parts, i.e. equation (\ref{localparts}), we may express this integral as
\begin{align}\label{expansioncalc}
    \sum_{(M,\mf{u})\neq(G,\lbrace 1\rbrace)}a^M(S(N),\mf{u})\sum_{L_1,L_2\in \mathcal{L}(M)}d_M(L_1,L_2)A_M^{L_1}(h^{\tau,p}_{t,R},\mf{u}_\infty,T)J_M^{L_2}(\chi_{K(N)},\mf{u}_{\text{fin}}),
\end{align}
with the archimedean part defined as 
\begin{align*}
    A_M^{L}(h^{\tau,p}_{t,R},\mf{u}_\infty,T) = \int_0^T J^L_M((h^{\tau,p}_{t,R})_Q,\mf{u}_\infty)t^{s-1}dt.
\end{align*}
We will treat this term like we did the integral of $h^{\tau,p}_{t,R}(1)$ above. By the asymptotics given in Proposition \ref{orbitsmallt}, this is convergent for $s$ in some half plane. By Proposition \ref{orbitasymp}, we may write
\begin{align}\label{archterm}
    A_M^{L}(h^{\tau,p}_{t,R},\mf{u}_\infty,T) = \int_0^\infty J^L_M((h^{\tau,p}_{t,R})_Q,\mf{u}_\infty)t^{s-1}dt+E_3(s,T),
\end{align}
with $E_3(s,T)$ some error term, given by an integral convergent for all $s\in\C$, analogously to $E_2$ satisfying
\begin{align}\label{error3}
    |E_3(0,T)|\leq D_{\tau,p,k} e^{-\lambda T} R^a(\log R)^b
\end{align}
for $T\geq 1$ and some $a,b>0$. We will write $A_M^{L}(h^{\tau,p}_{t,R},\mf{u}_\infty)$ for the integral on the right-hand side of (\ref{archterm}), which is convergent in some half plane with a meromorphic extension to all of $s\in\C$. This means that
\begin{align*}
    \text{FP}_{s=0}\left(\frac{1}{s\Gamma(s)}A_M^{L}(h^{\tau,p}_{t,R},\mf{u}_\infty)\right)
\end{align*}
is well defined. 
\begin{lem}\label{unip-1bound}
    There exists $C>0$ only depending on $\tau,p$ and $\lambda$ and $a,b,c>0$ only depending on the group such that for $T\geq 1$, we have
    \begin{align*}
    &\Bigg\vert\FP_{s=0}\left(\frac{1}{s\Gamma(s)}\int_0^T J_{\unip-1}(h^{\tau,p}_{t,R}\otimes\chi_{K(N)})t^{s-1}dt\right)\Bigg\vert \\
    \leq\:\: &C N^{-(n-1)}(1+\log N)^c R^a (\log R)^b\vol(Y(N)).
\end{align*}
\end{lem}

\begin{proof}
    Considering the expression (\ref{expansioncalc}) and the discussion above, it suffices to appropriately bound the three following terms:
    \begin{align*}
        &a^M(S(N),\mf{u}), \\
        & J_M^{L_2}(\chi_{K(N)},\mf{u}_{\text{fin}}) \:\: \text{for} \:\:(M,\mf{u})\neq (G,\lbrace1\rbrace), \\
        &\FP_{s=0}\left(\frac{1}{s\Gamma(s)}\left(A_M^{L}(h^{\tau,p}_{t,R},\mf{u}_\infty)+E_3(s,T)\right)\right).
    \end{align*}

    \noindent The global coefficients $a^M(S(N),\mf{u})$ were bounded in Lemma \ref{coeffbound}, and the local orbital integrals $J_M^{L_2}(\chi_{K(N)},\mf{u}_{\text{fin}})$ were bounded in Lemma \ref{localbound} and its remark. Note that $\vol(Y(N))=c\vol(K(N))^{-1}$ for some constant $c$ depending only on normalisations of measures (see the appendix).

    By Proposition \ref{orbitsmallt} and its remark, alongside Proposition \ref{orbitasymp}, the term 
    $$\FP_{s=0}\left(\frac{1}{s\Gamma(s)}A_M^{L}(h^{\tau,p}_{t,R},\mf{u}_\infty)\right)$$
    is uniformly bounded by a constant times $R^a(\log R)^b$, following the theory of Mellin transforms. The bound can be chosen to be uniform over all $(L,M,\mf{u}_\infty)$, as there are only finitely many such triples. Finally, the term
    $$\FP_{s=0}\left(\frac{1}{s\Gamma(s)}E_3(s,T)\right) = E_3(0,T)$$ 
    was bounded in a similar way in (\ref{error3}). Collecting all these bounds and applying to (\ref{expansioncalc}), we arrive at the desired bound.    
\end{proof}


\subsection{Asymptotics of analytic torsion}

Recall the definition of analytic torsion (\ref{torsion}). By collecting our initial analysis in the previous subsections, we may write
\begin{align}\label{analbound}
    \log T_{Y(N)}(\tau) &= \FP_{s=0}\left(\frac12\frac{1}{s\Gamma(s)}\int_0^\infty\sum_{p=1}^d (-1)^p p \, h^{\tau,p}_t(1)t^{s-1}dt\right) \vol(Y(N)) \\\label{FPunip-1}
    &+ \FP_{s=0}\left(\frac12\sum_{p=1}^d (-1)^p p\,\frac{1}{s\Gamma(s)} \int_0^T J_{\unip-1}(h^{\tau,p}_{t,R}\otimes\chi_{K(N)})t^{s-1}dt\right) \\
    &+ \frac12\sum_{p=1}^d (-1)^p p\, \left(E_0(0,T) + E_1(0,R,T)+E_2(0,T)\right)
\end{align}
The first term on the right-hand side is the $L^2$-torsion of $Y(N)$ (see \cite{Lott}). We set
\begin{align*}
    t^{(2)}_{\Tilde{X}}(\tau)\coloneqq  \frac12\frac{d}{ds}\left( \frac{1}{\Gamma(s)}\int_0^\infty\sum_{p=1}^d (-1)^p p \, h^{\tau,p}_t(1)t^{s-1}dt\right)\Bigg\vert_{s=0}.
\end{align*}
Note that although $\frac{1}{\Gamma(s)}\int_0^\infty h^{\tau,p}_t(1)t^{s-1}dt$ may not be holomorphic at $s=0$, swapping $h^{\tau,p}_t(1)$ with this alternating sum over $p$ seen above turns the meromorphic extension holomorphic at $s=0$, from which the above definition is well defined (see \cite{BV}, §$4.4$). Then the first term is exactly $\log T^{(2)}_{Y(N)} = t^{(2)}_{\Tilde{X}}(\tau)\vol(Y(N))$. All that is left to prove Theorem \ref{glmytheorem} is to show that the second and third term above both have the form $O(\vol(Y(N)N^{-(n-1)}\log(N)^a)$.

This was accomplished for the second term in Lemma \ref{unip-1bound}, as we assume $R\leq C_n\log N$. Note that the sum over $p$ contributes at most a constant multiple, as the bound was independent of $p$. To arrive at our desired bound for the third and final term, we wish to pick our parameters $R$ and $T$ such that the error terms $E_0,E_1,E_2$ are all as small as possible. This is done in the following section.

\subsection{A dance of error terms}\label{dance}

We pick $T=\beta \log N$ and $R=C_n\log N$, with $C_n$ from Proposition \ref{compactunip} and $\beta>0$ still to be determined. We see that the integral $\int_0^{T/R^2}e^{-C_4/t}t^{s-1}dt$ is then vanishing in $N$, in particular bounded by a constant for $N\geq 3$. Thus, by Lemma \ref{error0}, (\ref{error1}), and (\ref{error2}) respectively we have
\begin{align*}
    E_0(0,\beta\log N) &= O( N^{-\lambda(1-\epsilon)\beta}\vol(Y(N))),\\
    E_1(0,\beta\log N) &= O( N^{-C_4C_n^2/\beta +\beta}\vol(Y(N))),\\
    E_2(0,\beta\log N) &= O( N^{-\lambda\beta}\vol(Y(N))).
\end{align*}
The implied constants may depend on $\tau$. The choice of $R$ also implies that our bound on (\ref{FPunip-1}) has the form
\begin{align}\label{secondordersize}
    O(\vol(Y(N)N^{-(n-1)}(\log N)^a)
\end{align}
for some $a>0$ and $N$ large enough. We now ensure that the error terms above are bounded by a similar expression. Here is where the independence of the constants $C_2,C_4$ on $\tau$ becomes important. 

Pick $\beta$ such that $C_4C_n/\beta-C_2\beta>n-1$. This choice of $\beta$ then only depends on $n$, i.e. on the group $G$. Then, with $\beta$ fixed, let $\lambda$ be chosen such that $\lambda(1-\epsilon)\beta>n-1$. As $\epsilon$ can be chosen arbitrarily close to $0$, it is in fact sufficient to satisfy $\lambda>\frac{n-1}{\beta}$. Note that this choice of $\lambda$ also only depends on $C_2$ and $C_4$, which only depend on $G$. This implies that for $\tau$ being $\lambda$-strongly acyclic for the chosen $\lambda$, all the error terms are of the form (\ref{secondordersize}) as desired.

We have now proven Theorem \ref{glmytheorem}, from which Theorem \ref{mytheorem} follows.

\vspace{15em}