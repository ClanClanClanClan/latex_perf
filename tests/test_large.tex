\documentclass[12pt]{article}
\usepackage{amsmath}
\usepackage{amssymb}
\usepackage{amsthm}
\usepackage{geometry}
\usepackage{graphicx}
\usepackage{hyperref}

\geometry{margin=1in}

\title{Large Test Document for VSNA Performance Testing}
\author{LaTeX Perfectionist v24 - VSNA Architecture}
\date{\today}

\newtheorem{theorem}{Theorem}
\newtheorem{lemma}{Lemma}
\newtheorem{definition}{Definition}

\begin{document}

\maketitle

\tableofcontents

\section{Introduction}

This is a large test document designed to stress-test the VSNA (Verified Streaming Nested Automaton) architecture for LaTeX validation. The document contains numerous mathematical expressions, nested environments, and complex LaTeX constructs to thoroughly exercise the O(bytes + nesting) algorithm.

The VSNA architecture combines three key components:
\begin{itemize}
\item DFA (Deterministic Finite Automaton) for regular pattern recognition
\item VPA (Visibly Pushdown Automaton) for balanced construct tracking  
\item SymTab (Symbol Table) for context-sensitive validation
\end{itemize}

\section{Mathematical Content}

\subsection{Basic Algebra}

Consider the fundamental algebraic equation:
\begin{equation}
ax^2 + bx + c = 0
\end{equation}

The solutions are given by the quadratic formula:
\begin{equation}
x = \frac{-b \pm \sqrt{b^2 - 4ac}}{2a}
\end{equation}

For complex analysis, we often work with expressions like:
\begin{align}
f(z) &= \sum_{n=0}^{\infty} a_n z^n \\
&= a_0 + a_1 z + a_2 z^2 + \cdots \\
&= \lim_{N \to \infty} \sum_{n=0}^{N} a_n z^n
\end{align}

\subsection{Linear Algebra}

Matrix operations are fundamental to many mathematical computations:
\begin{equation}
\begin{pmatrix}
a_{11} & a_{12} & \cdots & a_{1n} \\
a_{21} & a_{22} & \cdots & a_{2n} \\
\vdots & \vdots & \ddots & \vdots \\
a_{m1} & a_{m2} & \cdots & a_{mn}
\end{pmatrix}
\begin{pmatrix}
x_1 \\ x_2 \\ \vdots \\ x_n
\end{pmatrix}
=
\begin{pmatrix}
b_1 \\ b_2 \\ \vdots \\ b_m
\end{pmatrix}
\end{equation}

The determinant of a 2x2 matrix is:
\begin{equation}
\det\begin{pmatrix} a & b \\ c & d \end{pmatrix} = ad - bc
\end{equation}

For eigenvalue problems, we solve:
\begin{equation}
\det(A - \lambda I) = 0
\end{equation}

where $\lambda$ represents the eigenvalues of matrix $A$.

\subsection{Calculus}

The fundamental theorem of calculus states:
\begin{theorem}
If $f$ is continuous on $[a,b]$ and $F$ is an antiderivative of $f$ on $[a,b]$, then:
\begin{equation}
\int_a^b f(x) \, dx = F(b) - F(a)
\end{equation}
\end{theorem}

For multivariable calculus, we have partial derivatives:
\begin{equation}
\frac{\partial^2 f}{\partial x \partial y} = \frac{\partial}{\partial x}\left(\frac{\partial f}{\partial y}\right)
\end{equation}

The gradient vector is:
\begin{equation}
\nabla f = \left(\frac{\partial f}{\partial x}, \frac{\partial f}{\partial y}, \frac{\partial f}{\partial z}\right)
\end{equation}

\section{Complex Nested Structures}

\subsection{Deeply Nested Environments}

\begin{theorem}[Nested Structure Test]
Consider the following nested structure:
\begin{proof}
We proceed by induction on $n$.
\begin{itemize}
\item Base case: For $n = 1$, we have:
\begin{equation}
\sum_{k=1}^{1} k = 1 = \frac{1 \cdot 2}{2}
\end{equation}
\item Inductive step: Assume the formula holds for $n$. Then for $n+1$:
\begin{align}
\sum_{k=1}^{n+1} k &= \sum_{k=1}^{n} k + (n+1) \\
&= \frac{n(n+1)}{2} + (n+1) \\
&= \frac{n(n+1) + 2(n+1)}{2} \\
&= \frac{(n+1)(n+2)}{2}
\end{align}
\end{itemize}
This completes the proof.
\end{proof}
\end{theorem}

\subsection{Complex Mathematical Expressions}

Here are some complex expressions to test the parser:

\begin{equation}
\lim_{n \to \infty} \left(1 + \frac{1}{n}\right)^n = e
\end{equation}

\begin{equation}
\int_{-\infty}^{\infty} e^{-x^2} \, dx = \sqrt{\pi}
\end{equation}

\begin{equation}
\sum_{n=1}^{\infty} \frac{1}{n^2} = \frac{\pi^2}{6}
\end{equation}

\begin{equation}
\prod_{p \text{ prime}} \frac{1}{1-p^{-s}} = \zeta(s) \quad \text{for } \Re(s) > 1
\end{equation}

\section{Advanced Mathematical Structures}

\subsection{Set Theory}

\begin{definition}
A set $S$ is called \emph{countable} if there exists a bijection $f: \mathbb{N} \to S$.
\end{definition}

The power set of a set $A$ is denoted $\mathcal{P}(A)$ and defined as:
\begin{equation}
\mathcal{P}(A) = \{B : B \subseteq A\}
\end{equation}

For any sets $A$ and $B$:
\begin{align}
A \cup B &= \{x : x \in A \text{ or } x \in B\} \\
A \cap B &= \{x : x \in A \text{ and } x \in B\} \\
A \setminus B &= \{x : x \in A \text{ and } x \notin B\}
\end{align}

\subsection{Logic and Proof Theory}

\begin{lemma}
For any propositions $P$ and $Q$:
\begin{equation}
\neg(P \land Q) \equiv (\neg P) \lor (\neg Q)
\end{equation}
\end{lemma}

The logical equivalences include:
\begin{align}
P \land Q &\equiv Q \land P \\
P \lor Q &\equiv Q \lor P \\
(P \land Q) \land R &\equiv P \land (Q \land R) \\
(P \lor Q) \lor R &\equiv P \lor (Q \lor R)
\end{align}

\section{Stress Testing Constructs}

This section contains various constructs designed to stress-test the VSNA parser:

\subsection{Balanced Construct Stress Test}

Multiple levels of nesting:
\begin{equation}
\left\{\left[\left(\frac{a+b}{c+d}\right)^2 + \left(\frac{e+f}{g+h}\right)^2\right]^{1/2}\right\}
\end{equation}

Complex fraction with multiple levels:
\begin{equation}
\frac{\frac{a}{b} + \frac{c}{d}}{\frac{e}{f} - \frac{g}{h}} = \frac{ad + bc}{eh - fg} \cdot \frac{fh}{bd}
\end{equation}

Nested sum and product expressions:
\begin{equation}
\sum_{i=1}^{n} \sum_{j=1}^{m} \prod_{k=1}^{p} a_{ijk} = \prod_{k=1}^{p} \sum_{i=1}^{n} \sum_{j=1}^{m} a_{ijk}
\end{equation}

\subsection{Large Tables and Matrices}

\begin{equation}
\begin{pmatrix}
\cos\theta & -\sin\theta & 0 \\
\sin\theta & \cos\theta & 0 \\
0 & 0 & 1
\end{pmatrix}
\begin{pmatrix}
1 & 0 & t_x \\
0 & 1 & t_y \\
0 & 0 & 1
\end{pmatrix}
=
\begin{pmatrix}
\cos\theta & -\sin\theta & t_x \\
\sin\theta & \cos\theta & t_y \\
0 & 0 & 1
\end{pmatrix}
\end{equation}

\section{Performance Testing Content}

This section contains repetitive content to increase document size for performance testing:

\subsection{Repeated Mathematical Expressions}

For $i = 1, 2, \ldots, 100$, consider the expression:
\begin{align}
f_i(x) &= \sum_{n=0}^{\infty} \frac{(-1)^n x^{2n+1}}{(2n+1)!} \\
&= x - \frac{x^3}{3!} + \frac{x^5}{5!} - \frac{x^7}{7!} + \cdots \\
&= \sin(x)
\end{align}

Similarly, for $j = 1, 2, \ldots, 100$:
\begin{align}
g_j(x) &= \sum_{n=0}^{\infty} \frac{(-1)^n x^{2n}}{(2n)!} \\
&= 1 - \frac{x^2}{2!} + \frac{x^4}{4!} - \frac{x^6}{6!} + \cdots \\
&= \cos(x)
\end{align}

\subsection{Repeated Theorem Structures}

\begin{theorem}[Test Theorem 1]
For any real number $x$, we have $\sin^2(x) + \cos^2(x) = 1$.
\end{theorem}

\begin{theorem}[Test Theorem 2]
For any complex number $z = x + iy$, we have $|e^{iz}| = 1$.
\end{theorem}

\begin{theorem}[Test Theorem 3]
The function $f(x) = e^x$ satisfies $f'(x) = f(x)$.
\end{theorem}

\section{Complex Environments}

\subsection{Algorithmic Structures}

Consider the following algorithm structure:
\begin{enumerate}
\item Initialize variables: $x_0 = 0$, $y_0 = 1$, $n = 0$
\item While $|x_n - y_n| > \epsilon$:
\begin{enumerate}
\item Set $x_{n+1} = \frac{x_n + y_n}{2}$
\item Set $y_{n+1} = \sqrt{x_n \cdot y_n}$
\item Increment $n$
\end{enumerate}
\item Return $(x_n + y_n)/2$ as approximation to $\pi/2$
\end{enumerate}

\subsection{Case Analysis}

Consider the function defined piecewise:
\begin{equation}
f(x) = \begin{cases}
x^2 & \text{if } x \geq 0 \\
-x^2 & \text{if } x < 0
\end{cases}
\end{equation}

For complex analysis:
\begin{equation}
\log(z) = \begin{cases}
\ln|z| + i\arg(z) & \text{if } z \neq 0 \\
\text{undefined} & \text{if } z = 0
\end{cases}
\end{equation}

\section{Final Performance Content}

This final section adds more content to reach the target size while maintaining realistic LaTeX structure.

\subsection{Additional Mathematical Content}

The Fourier transform of a function $f(t)$ is:
\begin{equation}
F(\omega) = \int_{-\infty}^{\infty} f(t) e^{-i\omega t} \, dt
\end{equation}

The inverse transform is:
\begin{equation}
f(t) = \frac{1}{2\pi} \int_{-\infty}^{\infty} F(\omega) e^{i\omega t} \, d\omega
\end{equation}

For discrete signals, we have the DFT:
\begin{equation}
X[k] = \sum_{n=0}^{N-1} x[n] e^{-j2\pi kn/N}
\end{equation}

\subsection{Probability and Statistics}

The probability density function of a normal distribution is:
\begin{equation}
f(x) = \frac{1}{\sigma\sqrt{2\pi}} e^{-\frac{(x-\mu)^2}{2\sigma^2}}
\end{equation}

The cumulative distribution function is:
\begin{equation}
F(x) = \int_{-\infty}^{x} f(t) \, dt = \frac{1}{2}\left[1 + \text{erf}\left(\frac{x-\mu}{\sigma\sqrt{2}}\right)\right]
\end{equation}

For Bayesian inference:
\begin{equation}
P(H|E) = \frac{P(E|H) \cdot P(H)}{P(E)}
\end{equation}

\subsection{Final Test Expressions}

To conclude this large test document, we include several final complex expressions:

\begin{align}
\oint_C f(z) \, dz &= 2\pi i \sum_k \text{Res}(f, z_k) \\
\nabla \times \mathbf{F} &= \left(\frac{\partial F_z}{\partial y} - \frac{\partial F_y}{\partial z}\right)\mathbf{i} + \left(\frac{\partial F_x}{\partial z} - \frac{\partial F_z}{\partial x}\right)\mathbf{j} + \left(\frac{\partial F_y}{\partial x} - \frac{\partial F_x}{\partial y}\right)\mathbf{k} \\
\mathcal{L}\{f(t)\} &= \int_0^{\infty} f(t) e^{-st} \, dt
\end{align}

\section{Extended Content for Size Requirements}

\subsection{Additional Repeated Mathematical Content}

To reach the target document size for proper performance testing, we include additional mathematical content:

For $k = 1, 2, \ldots, 50$, consider the Bessel function expansion:
\begin{equation}
J_n(x) = \sum_{m=0}^{\infty} \frac{(-1)^m}{m! \Gamma(n+m+1)} \left(\frac{x}{2}\right)^{n+2m}
\end{equation}

The modified Bessel functions satisfy:
\begin{align}
I_n(x) &= i^{-n} J_n(ix) = \sum_{m=0}^{\infty} \frac{1}{m! \Gamma(n+m+1)} \left(\frac{x}{2}\right)^{n+2m} \\
K_n(x) &= \frac{\pi}{2} \frac{I_{-n}(x) - I_n(x)}{\sin(n\pi)}
\end{align}

For hypergeometric functions:
\begin{equation}
_2F_1(a,b;c;z) = \sum_{n=0}^{\infty} \frac{(a)_n (b)_n}{(c)_n} \frac{z^n}{n!}
\end{equation}

where $(a)_n = a(a+1)(a+2)\cdots(a+n-1)$ is the Pochhammer symbol.

\subsection{Extensive Matrix Operations}

Consider the following large matrix operations for stress testing:

\begin{equation}
\begin{pmatrix}
a_{11} & a_{12} & a_{13} & a_{14} & a_{15} \\
a_{21} & a_{22} & a_{23} & a_{24} & a_{25} \\
a_{31} & a_{32} & a_{33} & a_{34} & a_{35} \\
a_{41} & a_{42} & a_{43} & a_{44} & a_{45} \\
a_{51} & a_{52} & a_{53} & a_{54} & a_{55}
\end{pmatrix}
\begin{pmatrix}
b_{11} & b_{12} & b_{13} & b_{14} & b_{15} \\
b_{21} & b_{22} & b_{23} & b_{24} & b_{25} \\
b_{31} & b_{32} & b_{33} & b_{34} & b_{35} \\
b_{41} & b_{42} & b_{43} & b_{44} & b_{45} \\
b_{51} & b_{52} & b_{53} & b_{54} & b_{55}
\end{pmatrix}
\end{equation}

The characteristic polynomial is:
\begin{equation}
p(\lambda) = \det(\lambda I - A) = \lambda^5 + c_4\lambda^4 + c_3\lambda^3 + c_2\lambda^2 + c_1\lambda + c_0
\end{equation}

For eigenvalue decomposition:
\begin{equation}
A = P \Lambda P^{-1} = \sum_{i=1}^{n} \lambda_i \mathbf{v}_i \mathbf{v}_i^T
\end{equation}

\subsection{Advanced Calculus Examples}

Multiple integrals with complex bounds:
\begin{align}
\iiint_D f(x,y,z) \, dx \, dy \, dz &= \int_a^b \int_{g_1(x)}^{g_2(x)} \int_{h_1(x,y)}^{h_2(x,y)} f(x,y,z) \, dz \, dy \, dx \\
&= \int_c^d \int_{p_1(y)}^{p_2(y)} \int_{q_1(x,y)}^{q_2(x,y)} f(x,y,z) \, dz \, dx \, dy
\end{align}

Vector calculus identities:
\begin{align}
\nabla \cdot (\nabla \times \mathbf{F}) &= 0 \\
\nabla \times (\nabla f) &= \mathbf{0} \\
\nabla \cdot (f\mathbf{F}) &= f(\nabla \cdot \mathbf{F}) + \mathbf{F} \cdot (\nabla f) \\
\nabla \times (f\mathbf{F}) &= f(\nabla \times \mathbf{F}) + (\nabla f) \times \mathbf{F}
\end{align}

Green's theorem:
\begin{equation}
\oint_C (P \, dx + Q \, dy) = \iint_D \left(\frac{\partial Q}{\partial x} - \frac{\partial P}{\partial y}\right) dx \, dy
\end{equation}

Stokes' theorem:
\begin{equation}
\oint_C \mathbf{F} \cdot d\mathbf{r} = \iint_S (\nabla \times \mathbf{F}) \cdot d\mathbf{S}
\end{equation}

Divergence theorem:
\begin{equation}
\iiint_V (\nabla \cdot \mathbf{F}) \, dV = \oiint_S \mathbf{F} \cdot d\mathbf{S}
\end{equation}

\subsection{Complex Analysis Extensions}

Laurent series expansion around singularities:
\begin{equation}
f(z) = \sum_{n=-\infty}^{\infty} a_n (z-z_0)^n = \sum_{n=-\infty}^{-1} a_n (z-z_0)^n + \sum_{n=0}^{\infty} a_n (z-z_0)^n
\end{equation}

Residue calculations:
\begin{align}
\text{Res}(f,z_0) &= a_{-1} = \frac{1}{2\pi i} \oint_{|z-z_0|=r} f(z) \, dz \\
&= \lim_{z \to z_0} (z-z_0) f(z) \quad \text{(simple pole)} \\
&= \lim_{z \to z_0} \frac{d^{m-1}}{dz^{m-1}} \left[(z-z_0)^m f(z)\right] \frac{1}{(m-1)!} \quad \text{(pole of order } m \text{)}
\end{align}

Cauchy's integral formula:
\begin{equation}
f^{(n)}(z_0) = \frac{n!}{2\pi i} \oint_C \frac{f(z)}{(z-z_0)^{n+1}} \, dz
\end{equation}

\subsection{Additional Theorem Examples}

\begin{theorem}[Fundamental Theorem of Algebra]
Every non-constant single-variable polynomial with complex coefficients has at least one complex root.
\end{theorem}

\begin{proof}
The proof follows from Liouville's theorem. Suppose $p(z)$ is a polynomial of degree $n \geq 1$ with no zeros. Then $1/p(z)$ is entire and bounded, hence constant by Liouville's theorem. This contradicts the assumption that $p(z)$ is non-constant.
\end{proof}

\begin{theorem}[Intermediate Value Theorem]
If $f$ is continuous on $[a,b]$ and $k$ is any value between $f(a)$ and $f(b)$, then there exists $c \in (a,b)$ such that $f(c) = k$.
\end{theorem}

\begin{theorem}[Mean Value Theorem]
If $f$ is continuous on $[a,b]$ and differentiable on $(a,b)$, then there exists $c \in (a,b)$ such that:
\begin{equation}
f'(c) = \frac{f(b) - f(a)}{b - a}
\end{equation}
\end{theorem}

\subsection{Differential Equations}

First-order linear ODEs:
\begin{equation}
\frac{dy}{dx} + P(x)y = Q(x)
\end{equation}

Solution using integrating factor $\mu(x) = e^{\int P(x) dx}$:
\begin{equation}
y = \frac{1}{\mu(x)} \left( \int \mu(x) Q(x) \, dx + C \right)
\end{equation}

Second-order linear homogeneous ODEs:
\begin{equation}
ay'' + by' + cy = 0
\end{equation}

Characteristic equation: $ar^2 + br + c = 0$ with discriminant $\Delta = b^2 - 4ac$.

Solutions depend on $\Delta$:
\begin{align}
\Delta > 0: \quad y &= c_1 e^{r_1 x} + c_2 e^{r_2 x} \\
\Delta = 0: \quad y &= (c_1 + c_2 x) e^{rx} \\
\Delta < 0: \quad y &= e^{\alpha x}(c_1 \cos(\beta x) + c_2 \sin(\beta x))
\end{align}

where $r = \alpha \pm i\beta$.

\subsection{Partial Differential Equations}

Heat equation:
\begin{equation}
\frac{\partial u}{\partial t} = \alpha^2 \frac{\partial^2 u}{\partial x^2}
\end{equation}

Wave equation:
\begin{equation}
\frac{\partial^2 u}{\partial t^2} = c^2 \frac{\partial^2 u}{\partial x^2}
\end{equation}

Laplace equation:
\begin{equation}
\nabla^2 u = \frac{\partial^2 u}{\partial x^2} + \frac{\partial^2 u}{\partial y^2} = 0
\end{equation}

Poisson equation:
\begin{equation}
\nabla^2 u = f(x,y)
\end{equation}

\subsection{Series and Sequences}

Convergence tests for infinite series:

\textbf{Ratio Test:} For $\sum a_n$, let $L = \lim_{n \to \infty} \left|\frac{a_{n+1}}{a_n}\right|$.
\begin{itemize}
\item If $L < 1$, the series converges absolutely
\item If $L > 1$, the series diverges  
\item If $L = 1$, the test is inconclusive
\end{itemize}

\textbf{Root Test:} For $\sum a_n$, let $L = \lim_{n \to \infty} \sqrt[n]{|a_n|}$.
\begin{itemize}
\item If $L < 1$, the series converges absolutely
\item If $L > 1$, the series diverges
\item If $L = 1$, the test is inconclusive  
\end{itemize}

\textbf{Integral Test:} If $f(x)$ is positive, continuous, and decreasing for $x \geq 1$, then $\sum_{n=1}^{\infty} f(n)$ and $\int_1^{\infty} f(x) dx$ both converge or both diverge.

Power series representation:
\begin{align}
e^x &= \sum_{n=0}^{\infty} \frac{x^n}{n!} = 1 + x + \frac{x^2}{2!} + \frac{x^3}{3!} + \cdots \\
\sin x &= \sum_{n=0}^{\infty} \frac{(-1)^n x^{2n+1}}{(2n+1)!} = x - \frac{x^3}{3!} + \frac{x^5}{5!} - \cdots \\
\cos x &= \sum_{n=0}^{\infty} \frac{(-1)^n x^{2n}}{(2n)!} = 1 - \frac{x^2}{2!} + \frac{x^4}{4!} - \cdots \\
\ln(1+x) &= \sum_{n=1}^{\infty} \frac{(-1)^{n+1} x^n}{n} = x - \frac{x^2}{2} + \frac{x^3}{3} - \cdots
\end{align}

\subsection{Repetitive Content for Size Target}

To reach the 30KB target for performance testing, we include additional repetitive mathematical content:

\subsubsection{Integration Formulas}

For $i = 1, 2, \ldots, 25$, consider the following integrals:
\begin{align}
\int x^n \, dx &= \frac{x^{n+1}}{n+1} + C \quad (n \neq -1) \\
\int \frac{1}{x} \, dx &= \ln|x| + C \\
\int e^x \, dx &= e^x + C \\
\int a^x \, dx &= \frac{a^x}{\ln a} + C \quad (a > 0, a \neq 1) \\
\int \sin x \, dx &= -\cos x + C \\
\int \cos x \, dx &= \sin x + C \\
\int \tan x \, dx &= -\ln|\cos x| + C \\
\int \cot x \, dx &= \ln|\sin x| + C \\
\int \sec x \, dx &= \ln|\sec x + \tan x| + C \\
\int \csc x \, dx &= -\ln|\csc x + \cot x| + C
\end{align}

\subsubsection{Trigonometric Identities}

For $j = 1, 2, \ldots, 20$, we have the following identities:
\begin{align}
\sin^2 \theta + \cos^2 \theta &= 1 \\
1 + \tan^2 \theta &= \sec^2 \theta \\
1 + \cot^2 \theta &= \csc^2 \theta \\
\sin(A \pm B) &= \sin A \cos B \pm \cos A \sin B \\
\cos(A \pm B) &= \cos A \cos B \mp \sin A \sin B \\
\tan(A \pm B) &= \frac{\tan A \pm \tan B}{1 \mp \tan A \tan B} \\
\sin 2\theta &= 2 \sin \theta \cos \theta \\
\cos 2\theta &= \cos^2 \theta - \sin^2 \theta = 2\cos^2 \theta - 1 = 1 - 2\sin^2 \theta \\
\tan 2\theta &= \frac{2 \tan \theta}{1 - \tan^2 \theta}
\end{align}

\subsubsection{Logarithmic Identities}

For $k = 1, 2, \ldots, 15$, consider these logarithmic properties:
\begin{align}
\log_a(xy) &= \log_a x + \log_a y \\
\log_a\left(\frac{x}{y}\right) &= \log_a x - \log_a y \\
\log_a(x^r) &= r \log_a x \\
\log_a 1 &= 0 \\
\log_a a &= 1 \\
a^{\log_a x} &= x \\
\log_a x &= \frac{\log_b x}{\log_b a} \quad \text{(change of base)} \\
\ln e &= 1 \\
e^{\ln x} &= x
\end{align}

\subsubsection{Matrix Algebra Repetition}

For stress testing purposes, we repeat matrix operations. Consider the following $n \times n$ matrix operations:

\begin{equation}
A + B = \begin{pmatrix}
a_{11} + b_{11} & a_{12} + b_{12} & \cdots & a_{1n} + b_{1n} \\
a_{21} + b_{21} & a_{22} + b_{22} & \cdots & a_{2n} + b_{2n} \\
\vdots & \vdots & \ddots & \vdots \\
a_{n1} + b_{n1} & a_{n2} + b_{n2} & \cdots & a_{nn} + b_{nn}
\end{pmatrix}
\end{equation}

Matrix multiplication:
\begin{equation}
(AB)_{ij} = \sum_{k=1}^{n} a_{ik} b_{kj}
\end{equation}

For the transpose:
\begin{equation}
(A^T)_{ij} = a_{ji}
\end{equation}

The inverse of a 2×2 matrix:
\begin{equation}
\begin{pmatrix} a & b \\ c & d \end{pmatrix}^{-1} = \frac{1}{ad - bc} \begin{pmatrix} d & -b \\ -c & a \end{pmatrix}
\end{equation}

\subsubsection{Complex Numbers}

For complex number $z = x + iy$:
\begin{align}
|z| &= \sqrt{x^2 + y^2} \\
\arg(z) &= \arctan\left(\frac{y}{x}\right) \\
z^* &= x - iy \\
z + z^* &= 2x = 2\Re(z) \\
z - z^* &= 2iy = 2i\Im(z) \\
zz^* &= |z|^2 = x^2 + y^2
\end{align}

De Moivre's theorem:
\begin{equation}
z^n = r^n(\cos(n\theta) + i\sin(n\theta))
\end{equation}

where $z = r(\cos\theta + i\sin\theta) = re^{i\theta}$.

\subsubsection{Limits and Continuity}

Standard limits for reference:
\begin{align}
\lim_{x \to 0} \frac{\sin x}{x} &= 1 \\
\lim_{x \to 0} \frac{1 - \cos x}{x} &= 0 \\
\lim_{x \to 0} \frac{1 - \cos x}{x^2} &= \frac{1}{2} \\
\lim_{x \to 0} \frac{\tan x}{x} &= 1 \\
\lim_{x \to 0} \frac{e^x - 1}{x} &= 1 \\
\lim_{x \to 0} \frac{\ln(1 + x)}{x} &= 1 \\
\lim_{x \to \infty} \left(1 + \frac{1}{x}\right)^x &= e \\
\lim_{x \to 0} (1 + x)^{1/x} &= e
\end{align}

L'Hôpital's rule: If $\lim_{x \to a} f(x) = \lim_{x \to a} g(x) = 0$ or $\pm\infty$, then:
\begin{equation}
\lim_{x \to a} \frac{f(x)}{g(x)} = \lim_{x \to a} \frac{f'(x)}{g'(x)}
\end{equation}

provided the limit on the right exists.

\subsubsection{Additional Theorem Statements}

\begin{theorem}[Bolzano-Weierstrass Theorem]
Every bounded sequence in $\mathbb{R}^n$ has a convergent subsequence.
\end{theorem}

\begin{theorem}[Extreme Value Theorem]
If $f$ is continuous on the closed interval $[a,b]$, then $f$ attains its maximum and minimum values on $[a,b]$.
\end{theorem}

\begin{theorem}[Rolle's Theorem]
If $f$ is continuous on $[a,b]$, differentiable on $(a,b)$, and $f(a) = f(b)$, then there exists $c \in (a,b)$ such that $f'(c) = 0$.
\end{theorem}

\begin{theorem}[Taylor's Theorem]
If $f$ is $(n+1)$ times differentiable in a neighborhood of $a$, then:
\begin{equation}
f(x) = \sum_{k=0}^{n} \frac{f^{(k)}(a)}{k!}(x-a)^k + R_n(x)
\end{equation}
where $R_n(x)$ is the remainder term.
\end{theorem}

\subsubsection{Statistics and Probability}

Discrete probability distributions:
\begin{align}
\text{Binomial: } P(X = k) &= \binom{n}{k} p^k (1-p)^{n-k} \\
\text{Poisson: } P(X = k) &= \frac{\lambda^k e^{-\lambda}}{k!} \\
\text{Geometric: } P(X = k) &= (1-p)^{k-1} p
\end{align}

Continuous probability distributions:
\begin{align}
\text{Normal: } f(x) &= \frac{1}{\sigma\sqrt{2\pi}} e^{-\frac{(x-\mu)^2}{2\sigma^2}} \\
\text{Exponential: } f(x) &= \lambda e^{-\lambda x} \quad (x \geq 0) \\
\text{Uniform: } f(x) &= \frac{1}{b-a} \quad (a \leq x \leq b)
\end{align}

Central moments:
\begin{align}
\text{Mean: } \mu &= E[X] \\
\text{Variance: } \sigma^2 &= E[(X-\mu)^2] = E[X^2] - \mu^2 \\
\text{Skewness: } \gamma_1 &= E\left[\left(\frac{X-\mu}{\sigma}\right)^3\right] \\
\text{Kurtosis: } \gamma_2 &= E\left[\left(\frac{X-\mu}{\sigma}\right)^4\right]
\end{align}

\subsubsection{Final Size Padding Content}

To ensure we reach the 30KB target, we add more mathematical content for comprehensive VSNA testing:

\paragraph{Advanced Calculus of Variations}

The Euler-Lagrange equation for functionals of the form $I[y] = \int_a^b F(x, y, y') \, dx$ is:
\begin{equation}
\frac{\partial F}{\partial y} - \frac{d}{dx}\left(\frac{\partial F}{\partial y'}\right) = 0
\end{equation}

For the brachistochrone problem, we minimize:
\begin{equation}
T = \int_0^a \sqrt{\frac{1 + (y')^2}{2gy}} \, dx
\end{equation}

The solution is a cycloid curve given parametrically by:
\begin{align}
x &= \frac{a}{2}(\phi - \sin\phi) \\
y &= \frac{a}{2}(1 - \cos\phi)
\end{align}

\paragraph{Fourier Analysis Extensions}

The Fourier series of a periodic function $f(x)$ with period $2L$ is:
\begin{equation}
f(x) = \frac{a_0}{2} + \sum_{n=1}^{\infty} \left(a_n \cos\frac{n\pi x}{L} + b_n \sin\frac{n\pi x}{L}\right)
\end{equation}

where the coefficients are:
\begin{align}
a_0 &= \frac{1}{L} \int_{-L}^{L} f(x) \, dx \\
a_n &= \frac{1}{L} \int_{-L}^{L} f(x) \cos\frac{n\pi x}{L} \, dx \\
b_n &= \frac{1}{L} \int_{-L}^{L} f(x) \sin\frac{n\pi x}{L} \, dx
\end{align}

The complex form is:
\begin{equation}
f(x) = \sum_{n=-\infty}^{\infty} c_n e^{in\pi x/L}
\end{equation}

where $c_n = \frac{1}{2L} \int_{-L}^{L} f(x) e^{-in\pi x/L} \, dx$.

\paragraph{Numerical Methods}

Newton's method for finding roots of $f(x) = 0$:
\begin{equation}
x_{n+1} = x_n - \frac{f(x_n)}{f'(x_n)}
\end{equation}

The secant method:
\begin{equation}
x_{n+1} = x_n - f(x_n) \frac{x_n - x_{n-1}}{f(x_n) - f(x_{n-1})}
\end{equation}

For numerical integration, the trapezoidal rule:
\begin{equation}
\int_a^b f(x) \, dx \approx \frac{b-a}{2n} \left[f(a) + 2\sum_{i=1}^{n-1} f(a + ih) + f(b)\right]
\end{equation}

where $h = \frac{b-a}{n}$.

Simpson's rule:
\begin{equation}
\int_a^b f(x) \, dx \approx \frac{h}{3} \left[f(a) + 4\sum_{i=1,3,5,...}^{n-1} f(a + ih) + 2\sum_{i=2,4,6,...}^{n-2} f(a + ih) + f(b)\right]
\end{equation}

\paragraph{Additional Matrix Operations for Testing}

The QR decomposition of matrix $A$ gives $A = QR$ where $Q$ is orthogonal and $R$ is upper triangular.

The singular value decomposition (SVD) gives $A = U\Sigma V^T$ where:
\begin{itemize}
\item $U$ is an $m \times m$ orthogonal matrix
\item $\Sigma$ is an $m \times n$ diagonal matrix with non-negative real numbers
\item $V^T$ is an $n \times n$ orthogonal matrix
\end{itemize}

For eigenvalue problems with the characteristic polynomial:
\begin{equation}
\det(A - \lambda I) = \sum_{k=0}^{n} (-1)^k s_k \lambda^{n-k}
\end{equation}

where $s_k$ are the elementary symmetric polynomials in the eigenvalues.

\paragraph{Complex Analysis Advanced Topics}

The residue theorem states that for a simple closed contour $C$ and function $f$ analytic inside and on $C$ except for isolated singularities:
\begin{equation}
\oint_C f(z) \, dz = 2\pi i \sum_{j} \text{Res}(f, z_j)
\end{equation}

The argument principle: if $f$ is meromorphic inside and on a simple closed contour $C$ with no zeros or poles on $C$, then:
\begin{equation}
\frac{1}{2\pi i} \oint_C \frac{f'(z)}{f(z)} \, dz = N - P
\end{equation}

where $N$ is the number of zeros and $P$ is the number of poles inside $C$, counted with multiplicity.

\paragraph{Advanced Differential Equations}

For the Bessel differential equation:
\begin{equation}
x^2 \frac{d^2y}{dx^2} + x \frac{dy}{dx} + (x^2 - \nu^2) y = 0
\end{equation}

The general solution is:
\begin{equation}
y = c_1 J_\nu(x) + c_2 Y_\nu(x)
\end{equation}

where $J_\nu$ and $Y_\nu$ are Bessel functions of the first and second kind.

For Legendre's equation:
\begin{equation}
(1-x^2) \frac{d^2y}{dx^2} - 2x \frac{dy}{dx} + n(n+1) y = 0
\end{equation}

The solution is given by Legendre polynomials $P_n(x)$.

\paragraph{Final Mathematical Content}

To complete our 30KB+ test document, we include these final expressions:

The gamma function:
\begin{equation}
\Gamma(z) = \int_0^{\infty} t^{z-1} e^{-t} \, dt
\end{equation}

The beta function:
\begin{equation}
B(x,y) = \int_0^1 t^{x-1} (1-t)^{y-1} \, dt = \frac{\Gamma(x)\Gamma(y)}{\Gamma(x+y)}
\end{equation}

The Riemann zeta function:
\begin{equation}
\zeta(s) = \sum_{n=1}^{\infty} \frac{1}{n^s} = \prod_{p \text{ prime}} \frac{1}{1-p^{-s}}
\end{equation}

Stirling's approximation:
\begin{equation}
n! \approx \sqrt{2\pi n} \left(\frac{n}{e}\right)^n
\end{equation}

The error function:
\begin{equation}
\text{erf}(x) = \frac{2}{\sqrt{\pi}} \int_0^x e^{-t^2} \, dt
\end{equation}

This concludes our comprehensive test content designed to stress-test the VSNA implementation with realistic mathematical LaTeX content exceeding 30KB in size.

\section{Conclusion}

This large test document exercises various aspects of the VSNA architecture including:
\begin{itemize}
\item Complex nested mathematical expressions with deep nesting levels
\item Multiple environment types (equation, align, theorem, proof, etc.)
\item Balanced construct tracking with various bracket types: (), [], \{\}
\item Performance testing with substantial content (target 30KB+)
\item Stress testing of parser with complex mathematical notation
\item Real-world LaTeX document structure and complexity
\end{itemize}

The document should be sufficient to test the O(bytes + nesting) performance characteristics of the VSNA implementation and verify that it meets the SLA requirements for processing speed while maintaining accuracy in detecting unbalanced constructs and other LaTeX validation issues.

The VSNA architecture combines:
\begin{enumerate}
\item \textbf{DFA component} for recognizing regular patterns like backslash sequences
\item \textbf{VPA component} for tracking balanced constructs and nested environments  
\item \textbf{SymTab component} for context-sensitive validation rules
\end{enumerate}

This unified approach ensures O(bytes + nesting) complexity while providing comprehensive LaTeX validation capabilities for production use.

\end{document}